%%%  Chapter 11: Files and Streams: Input/Output Techniques
%%%  3rd Edition
\setcounter{SSTUDYcount}{1}
\setcounter{chapter}{10}
\chapter{Files and Streams: Input/Output Techniques}
\label{chapter-files}


\CObegin

\secCOBH{Objectives}
\label{objectives}

\noindent After studying this chapter, you will

\begin{COBL}
\item  Be able to read and write text files.
\item  Know how to read and write binary files.
\item  Understand the  use of {\tt InputStream}s and {\tt OutputStream}s.
\item  Be able to design methods for performing
 input and output.
\item  Know how to use the {\tt File} class.
\item  Be able to use the {\tt JFileChooser} class.
\end{COBL}

\secCOLH{Outline}
\label{outline}

\begin{COL}
\item {Introduction}
\item {Streams and Files}
\item {Case Study: Reading and Writing Text Files}
\item {The {\tt File} Class}
\item {Example: Reading and Writing Binary Files}
\item {Object Serialization: Reading and Writing Objects}
\item {From the Java Library: {\tt javax.swing.JFileChooser}}
\item {Using File Data in Programs}    %%%REW new section
\item[] {\color{cyan}Special Topic: Databases and Personal Privacy}
\par\small\item[] {Chapter Summary}
\par\small\item[] {Solutions to Self-Study Exercises}
\par\small\item[] {Exercises}
\end{COL}
\COend

\section{Introduction}
\label{introduction}
\noindent We have been using input and output in our programs 
since the very first chapters of the book. In this chapter we
will take a closer look at Java's input and output elements.

{\bf Input} refers to information or data read from some external
source into a running program. We introduced you to working with input
in Chapter~4, when we developed the KeyboardReader class with methods
for reading data from the keyboard into the console window. We also
discussed reading data from the keyboard into a {\tt JTextField} in a
GUI interface, as well as \marginnote{input and output} reading data
from a text file using methods in the {\tt Scanner} class during that
chapter.

{\bf Output} refers to information or data written from the running
program to some external destination.  Up to this point, whenever our
programs have produced output, it has been sent to either the Java
console, to a text area, or to some other GUI component.  These
destinations are transitory, in the sense that they reside in the
computer's primary memory and exist only so long as the program is
running.

A {\bf file} is a collection of data that's stored on a disk or on
some other relatively permanent storage medium.  A file's existence
does not depend on a running program.  In this chapter, we will learn
how to create files and how to perform input and output operations on
their data using the Java classes designed specifically for this
purpose.  Methods from these classes allow us to write data to files
and provide greater flexibility in the way we read data from files 
than the {\tt Scanner} class offers.

%\begin{figure}
\begin{figure}[b]
\figaleft{chptr11/sysinout.eps}{The {\tt System.out} output stream connects your program 
 to the screen and the {\tt System.in} input stream connects it to the keyboard.
} {fig-systemout}

\end{figure}
%\end{figure}

\section{Streams and Files}
\label{streamsand-files}
\noindent As was noted in Chapter~4, all input and output (I/O) in
Java is accomplished through the use of input streams and output
streams.  You are already familiar with input and output streams
because we \marginnote{I/O streams} have routinely used the {\tt
System.out} output stream and and the {\tt System.in} input stream
(Fig.~\ref{fig-systemout}) in this text's examples. Recall that {\tt
System.out} usually connects your program (source) to the screen
(destination) and {\tt System.in} usually connects the keyboard
(source) to the running program (destination). What you have learned
about streams will also be a key for connecting files to a program.

\subsection{The Data Hierarchy}
\noindent Data, or information, are the contents that flow through
Java streams or stored in files.  All data are comprised of binary
digits or {\it bits}. A bit is simply a 0 or a 1, the electronic
states that correspond to these values.  As we learned in Chapter~5, a
bit is the smallest unit of data.

However, it would be tedious if a program had to work with data in units
as small as bits.  Therefore, most operations involve various-sized
aggregates of data such as an 8-bit {\tt byte}, a 16-bit {\tt short},
a 16-bit {\tt char}, a 32-bit {\tt int}, a 64-bit {\tt long}, a 32-bit
{\tt float}, or a 64-bit {\tt double}. As we know, these are Java's
primitive numeric types.   In addition to these aggregates, we can
group together a sequence of {\tt char} to form a {\tt String}.

It is also possible to group data of different types into objects.   A
{\bf record}\index{record}, which corresponds closely to a Java object, can have
{\bf fields}\index{fields} that contain different types of data.   For example, a
student record might contain fields for the student's name and address
%\begin{marginalnote}\it The data hierarchy\end{marginalnote}
%%\marginnote{The data hierarchy}
represented by ({\tt String}s), expected year of graduation ({\tt int}), and current
grade point average represented by ({\tt double}).  Collections of these records are
typically grouped into {\em files}. For example, your
registrar's office may have a separate file for each of its graduating
classes.  These are typically organized into a collection of related
files, which is called a {\bf database}\index{database}.
\marginfig{chptr11/datahier.eps}{The data hierarchy.}
{fig-datahier}

Taken together, the different kinds of data that are processed by a
computer or stored in a file can be organized into a {\bf data
hierarchy}\index{data hierarchy} (Fig.~11\mbox{.}2).

It's important to recognize that while we, the programmers, may group
data into various types of abstract entities, the information flowing
through an input or output stream is just a sequence of bits.  There
are no natural boundaries that mark where one byte (or one {\tt int}
or one record) ends and the next one begins.   Therefore, it will be up
to us to provide the boundaries as we process the data.


\subsection{Binary Files and Text Files}
\noindent As we noted in chapter~4, there are two types of files in
 Java: binary files and text
files.  Both kinds store data as a sequence of bits---that is, a
sequence of 0's and 1's.  Thus, the difference between the two types of
files lies in the way they are interpreted by the programs that read
and write them.   A {\em binary file} is processed 
as a sequence of bytes, whereas a {\em text file} is 
processed as a sequence of characters.



Text editors and other programs that process text files interpret the
file's sequence of bits as a sequence of characters---that is, as a
string.   Your Java source programs ({\tt *.java}) are text
files, and so are the HTML files that populate the World Wide Web.
The big advantage of text files is their portability.  Because their data
%\begin{marginalnote}\it Text files are portable\end{marginalnote}
\marginnote{Text files are portable}
are represented in the ASCII code \mbox{(Table~5.13),} they can
be read and written by just about any text-processing program.  Thus, a
text file created by a program on a Windows/Intel computer can be read
by a Macintosh program.

In non-Java environments, data in binary files are stored as bytes,
and the representation used varies from computer to computer.  The
manner in which a computer's memory stores binary data determines how
it is represented in a file. Thus, binary data are not very portable.
For example, a binary file of integers created on a Macintosh cannot
be read by a Windows/Intel program.

One reason for the lack of portability is that each type of computer
uses its own definition for how an integer is defined.  On some systems
%\begin{marginalnote}\it Binary files are platform dependent\end{marginalnote}
\marginnote{Binary files are platform dependent} an integer might be
16 bits, and on others it might be 32 bits, so even if you know that a
Macintosh binary file contains integers, that still won't make it
readable by Windows/Intel programs.  Another problem is that even if
two computers use the same number of bits to represent an integer,
they might use different representation schemes.  For example, some
computers might use 10000101 as the 8-bit representation of the number
133, whereas other computers might use the reverse, 10100001, to
represent 133.

The good news for us is that Java's designers have made its binary
files {\em platform independent} by carefully defining the exact size
and representation that must be used for integers and all other
primitive types.  Thus, binary files created by Java programs can be
interpreted by Java programs on any platform.

\JavaRule{Platform Independence.}{Java binary files are
platform independent.  They can be interpreted by any computer that
supports Java.}

\subsection{Input and Output Streams}
\noindent Java has a wide variety of streams for performing I/O.~They
are defined in the {\tt java.io} package, which must be imported by
any program that does I/O.  They are generally organized into the
hierarchy illustrated in Figure~\ref{fig-streamhier}.  We will cover
only a small portion of the hierarchy in this text.  Generally
speaking, binary files are processed by subclasses of {\tt
InputStream} and {\tt OutputStream}.  \marginnote{I/O streams} Text
files are processed by subclasses of {\tt Reader} and {\tt Writer},
both of which are streams, despite their names.  

\begin{figure}[t!]
\figaright{chptr11/streamhier.eps}{Java's stream hierarchy.
} {fig-streamhier}
\end{figure}

{\tt InputStream} and {\tt OutputStream} are abstract classes that
serve as the root classes for reading and writing binary data.  Their
most commonly used subclasses are {\tt DataInputStream} and
{\tt DataOutputStream}, which are used for processing {\tt String} data and
data of any of Java's primitive types---{\tt char},
{\tt boolean}, {\tt int}, {\tt double}, and so on.  The analogues of these
classes for processing text data are the {\tt Reader} and {\tt Writer}
classes, which serve as the root classes for all text I/O.

\JavaTIP{PROGRAMMING TIP}{Choosing a Stream.}{In choosing an appropriate
stream for an I/O operation, {\tt DataInputStream}s and
{\tt Data\-OutputStream}s normally are used for binary I/O.~{\tt Reader} and
{\tt Writer} streams normally are used for text I/O.}

\noindent The various subclasses of these root classes perform various
specialized I/O operations.  For example, {\tt FileInputStream} and
{\tt File\-Output\-Stream} are used for performing binary input and
output on files.  The {\tt PrintStream} class contains methods for
outputting various primitive data---integers, floats, and so
forth---as text.  The {\tt System.out} stream, one of the most widely
used output streams, is an object of this type.  The {\tt PrintWriter}
class, which was introduced in JDK 1.1 contains the same methods as
{\tt PrintStream} but the methods are designed to support platform
independence and internationalized I/O---that is, I/O that works in
different languages and alphabets.

%%%However, the {\tt PrintStream}
%%%class has been deprecated in JDK 1.2 and has been superseded by 
%%% RAM NOTE 4/5/2005: I find no evidence that PrintStream is deprecated.

\pagebreak
The various methods defined in {\tt PrintWriter} are designed 
\marginfig{chptr11/writer.eps}{{\tt PrintWriter}
methods print data of various types.}
{fig-writer}
to output a particular type of primitive data (Fig.~11.4).
As you would expect, there is both a {\tt print()} and
{\tt println()} method for each kind of data that the programmer wants
to output.

Table~11.1 briefly describes Java's most commonly used input and
output streams.  In addition to the ones we've already mentioned, you
are already familiar with methods from the {\tt BufferedReader} and
{\tt File} classes, which were used in Chapter~4.

{\bf Filtering}\index{Filtering} refers to performing operations on
data while the data are being input or output.  Methods in the {\tt
FilterInputStream} and {\tt FilterReader} classes can be used to
filter binary and text data during input.  Methods in the {\tt
FilterOutputStream} and {\tt FilterWriter} can be used to filter
output data. These classes serve as the root classes for various
filtering subclasses.  They can also be subclassed to perform
customized data filtering.

\begin{table}[b!]
%\hphantom{\caption{Description of some of Java's important stream classes.}}
\TBT{-5pc}{Description of some of Java's important stream classes.}
\hspace*{-5.5pc}\begin{tabular}{ll}
\multicolumn{2}{l}{\color{cyan}\rule{32pc}{1pt}}\\[2pt]
%%%REW\TBCH{\bf Class} & \TBCH{Description}
{\bf Class} & {\bf Description}
\\[-4pt]\multicolumn{2}{l}{\color{cyan}\rule{32pc}{0.5pt}}\\[2pt]
{\tt InputStream}             & Abstract root class of all binary input streams \cr
{\tt FileInputStream}         & Provides methods for reading bytes from a binary file \cr
{\tt FilterInputStream}       & Provides methods required to filter data \cr
{\tt BufferedInputStream}     & Provides input data buffering for reading large files \cr
{\tt ByteArrayInputStream}    & Provides methods for reading an array as if it were a stream \cr
{\tt DataInputStream}         & Provides methods for reading Java's primitive data types \cr
{\tt PipedInputStream}        & Provides methods for reading piped data from another thread \cr

{\tt OutputStream}            & Abstract root class of all binary output streams \cr
{\tt FileOutputStream}        & Provides methods for writing bytes to a binary file \cr
{\tt FilterOutputStream}      & Provides methods required to filter data \cr
{\tt BufferedOutputStream}    & Provides output data buffering for writing large files \cr
{\tt ByteArrayOutputStream}   & Provides methods for writing an array as if it were a stream \cr
{\tt DataOutputStream}        & Provides methods for writing Java's primitive data types \cr
{\tt PipedOutputStream}       & Provides methods for writing piped data to another thread \cr
{\tt PrintStream}             & Provides methods for writing primitive data as text \cr

{\tt Reader}                  & Abstract root class for all text input streams \cr
{\tt BufferedReader}          & Provides buffering for character input streams  \cr
{\tt CharArrayReader}         & Provides input operations on {\tt char} arrays  \cr
{\tt FileReader}              & Provides methods for character input on files \cr
{\tt FilterReader}            & Provides methods to filter character input \cr
{\tt StringReader}            & Provides input operations on {\tt String}s  \cr

{\tt Writer}                  & Abstract root class for all text output streams \cr
{\tt BufferedWriter}          & Provides buffering for character output streams \cr
{\tt CharArrayWriter}         & Provides output operations to {\tt char} arrays \cr
{\tt FileWriter}              & Provides methods for output to text files \cr
{\tt FilterWriter}            & Provides methods to filter character output \cr
{\tt PrintWriter}             & Provides methods for printing binary data as characters \cr
{\tt StringWriter}            & Provides output operations to {\tt String}s 
\\[-4pt]\multicolumn{2}{l}{\color{cyan}\rule{32pc}{1pt}}
\end{tabular}
\endTB
\end{table}



One type of filtering is {\bf buffering}\index{buffering}, which is
provided by several buffered streams, including {\tt
BufferedInputStream} and {\tt BufferedReader}, for performing binary
and text input, and {\tt BufferedOutputStream} and {\tt
Buffered\-Writer}, for buffered output operations.  As was discussed
in chapter~4, a {\em buffer} is a relatively large region of memory
used to temporarily store data while they are being input or
output. When buffering is used, a program will transfer a large number
of bytes into the buffer from the relatively slow input device and
then transfer these to the program as each read operation is
performed.  The transfer from the buffer to the program's memory is
very fast.

Similarly, when buffering is used during output, data are transferred
directly to the buffer and then written to the disk when the buffer
fills up or when the {\tt flush()} method is called.

\JavaTIP{PROGRAMMING TIP}{Buffering.}{Buffered streams can 
improve a program's overall efficiency by reducing the amount of time
it spends accessing relatively slow input or output devices.}

\noindent You can also define your own data filtering subclasses to perform
%\begin{marginalnote}\it Buffering\end{marginalnote}
\marginnote{Buffering}
customized filtering.   For example, suppose you want to add line
numbers to a text editor's printed output.  To perform this task, you
could define a {\tt FilterWriter} subclass and override its
%\begin{marginalnote}\it Filtering data\end{marginalnote}
\marginnote{Filtering data}
{\tt write()} methods to perform the desired filtering operation.
Similarly, to remove the line numbers from such a file during input,
you could define a {\tt FilterReader} subclass.  In that case, you would
override its {\tt read()} methods to suit your goals for the program.

There are several classes that provide I/O-like operations on various
internal memory structures.  {\tt ByteArrayInputStream}, {\tt
ByteArray\-Output\-Stream}, {\tt CharArrayReader}, and {\tt
CharArrayWriter} are four classes that take input from or send output
to arrays in the program's memory.  Methods in these classes can be
useful for performing various operations on data during input or
output.  For example, suppose a program reads an entire line of
integer data from a binary file into a {\tt ByteArray}. It might then
transform the data by, say, computing the remainder modulo {\tt N} of
each value.  The program now can read these transformed data by
treating the byte array as an input stream.  A similar example would
apply for some kind of output transformation.

The {\tt StringReader} and {\tt StringWriter} classes provide methods
for treating {\tt String}s and {\tt StringBuffer}s as I/O
streams.  These methods can be useful for performing certain
data conversions.

\JavaTIP{PROGRAMMING TIP}{Integer/String Conversion.}{An integer can
be converted to a {\tt String} by writing it to a {\tt StringBuffer},
which can then be output as an entire line of text.  {\tt
StringReader} methods can be used to read integer data from an
ordinary {\tt String} object.}


%\section{CASE STUDY}\begin{casestudy}{Reading and Writing Text Files}
\label{casestud-ybegin-readingand-writing-text-files}
%%%\secCSFHleft{CASE STUDY: Reading and Writing Text Files}
\label{casestudy-readingand-writing-text-files}
\section{CASE STUDY: Reading and Writing Text Files}
\label{casestudy-readingand-writing-text-files}
\noindent  Let's write a GUI application that will be able to 
read and write data to and from a text file.  To do this, we will need
to develop a set of methods to perform I/O on text files. 

The GUI for this application will contain a {\tt JTextArea}, where text
%\begin{marginalnote}\it GUI design\end{marginalnote}
\marginnote{GUI design}
file data can be input and displayed, and a {\tt JTextField}, where the
user can enter the file's name.  It will also contain two
{\tt JButton}s, one for reading a file into the {\tt JTextArea}, and the
other for writing the data in the {\tt JTextArea} into a file
(Fig.~\ref{fig-guitextio}). Note that even this simple interface
will let the user create new files and rename existing files.
%\begin{figure}
\begin{figure}[h]
\figaleft{chptr11/guitextio.eps}{The GUI design for a program that reads
and writes text files.
} {fig-guitextio}
\end{figure}
%\end{figure}


\subsection{Text File Format}
\noindent A text file consists of a sequence of characters divided into zero or
more lines and ending with a special {\bf end-of-file character}\index{end-of-file character}.
%\begin{marginalnote}\it The end-of-file character\end{marginalnote}
\marginnote{The end-of-file character}
When you open a new file in a text editor, it contains zero lines and
zero characters.  After typing a single character, it would contain one
character and one line.  The following would be an example of a file with
four lines of text:

%%%REW
\begin{jjjlisting}
\begin{lstlisting}
one\ntwo\nthree\nfour\n\eof
\end{lstlisting}
\end{jjjlisting}


\noindent Note the use of the end-of-line character,
 \verb|\n|, to mark the
end of each line, and the use of the end-of-file character,
 \verb|\eof|,
to mark the end of the file.   As we'll see, the I/O methods for
text files use these special characters to control reading and writing
loops.  Thus, when the file is read by appropriate Java methods, such
as the {\tt BufferedReader.readLine()} and {\tt Buffered\-Reader.read()}
methods, one or more characters will be read until either an
end-of-line or end-of-file character is encountered.   When a line of
characters is written using {\tt println()}, the end-of-line character
is appended to the characters themselves.

\subsection{Writing to a Text File}
\noindent Let's see how to write to a text file.  In this program we write the
entire contents of the {\tt JTextArea()} to the text file.  In general,
writing data to a file requires three steps:

%%%REW\clearpage

%\begin{list}{}{}
\begin{NL}
\item  Connect an output stream to the file.
\item  Write text data into the stream, possibly using a loop.
\item  Close the stream.
\end{NL}
%\end{list}

\noindent As Figure~\ref{fig-systemout} shows, connecting
a stream to a file looks like doing a bit of plumbing.   The first step is
to connect an output stream to the file.   The output stream serves as
\marginnote{Output stream}
a conduit between the program and a named file.  The output stream
opens the file and gets it ready to accept data from the program.  If
the file already exists, then opening the file will destroy any data
it previously contained.  If the file doesn't yet exist, then it will
be created from scratch.

Once the file is open, the next step is to write the text to the
stream, which passes the text on to the file.   This step might require a loop
that outputs one line of data on each iteration.  Finally, once all
the data have been written to the file, the stream should be closed.
This also has the effect of closing the file itself.

\JavaTIP{EFFECTIVE DESIGN}{Writing a File.}{Writing data to a file
requires a three-step algorithm: (1) Connect an output stream to the
file, (2) write the data, and (3) close the file.}

\subsubsection*{Code Reuse: Designing an Output Method}
\noindent Now let's see how these three steps are done in Java.  Suppose the text
we want to write is contained in a {\tt JTextArea}. Thus, we want a
method that will write the contents of a {\tt JTextArea} to a named
file.

What output stream should we use for the task of writing a {\tt String}
%\begin{marginalnote}\it Choosing an output stream\end{marginalnote}
\marginnote{\raggedright \it Choosing an output stream}
to a named file? To decide this, we need to use the information in
Figure~\ref{fig-streamhier} and Table~11.1. As we pointed
out earlier, because we're writing a text file, we would use a {\tt
Writer} subclass.  But which subclass should we use? The only way to
decide this is to consult the Java API documentation, using links at

\begin{jjjlisting}
\begin{lstlisting}[commentstyle=\color{black}]
http://java.sun.com/j2se/1.5.0/docs/api/
\end{lstlisting}
\end{jjjlisting}

%%%REW http://java.sun.com/products/jdk/1.3/docs/api/
%%% This old url no longer worked for me
%%%http://java.sun.com/reference/api/index.html

\noindent to see what methods are available in the various subclasses.
For I/O operations you want to consult the classes in the {\tt
java.io} package. Ideally, we would like to be able to create an
output stream to a named file, and we would like to be able to write a
{\tt String} to the file.

One likely candidate is the {\tt FileWriter} class
(Fig.~\ref{fig-filewriter}).  Its name and description
(Table~11.1) suggest that it's designed for writing text
files.  And indeed it contains the kind of constructor we need---that is, one 
that takes the file name as a parameter.  Note that by
taking a {\tt boolean} parameter, the second constructor allows us to
append data to a file rather than rewrite the entire file, which is
the default case.

However, {\tt FileWriter} doesn't define a {\tt write()}
method.  This doesn't necessarily mean that it doesn't contain such a
method.  It might have inherited one from its superclasses, 
%\begin{marginalnote}\it Inheritance\end{marginalnote}
\marginnote{Inheritance}
{\tt OutputStream\-Writer} and {\tt Writer}.  Indeed, the {\tt Writer}
class contains a method, {\tt write()}, whose signature suggests that
it is ideally suited for our task (Fig.~\ref{fig-filewriter}).


%\begin{figure}
\begin{figure}[tb]
\figaleft{chptr11/filewriter.eps}{To find the right I/O method, it is sometimes
necessary to search the Java class hierarchy. This is
easy to do with the online documentation.
} {fig-filewriter}
\end{figure}
%\end{figure}


Having decided on a {\tt FileWriter} stream, the rest of the
task of designing our method is simply a matter of using
{\tt FileWriter} methods in an appropriate way:

\begin{jjjlisting}
\begin{lstlisting}
private void writeTextFile(JTextArea display,
                                        String fileName) {
    // Create stream & open file
    FileWriter outStream = new FileWriter(fileName); 
    //  Write the entire display text and close the stream
    outStream.write(display.getText());              
    outStream.close();   //  Close the output stream
}
\end{lstlisting}
\end{jjjlisting}

\noindent We use the {\tt FileWriter()} constructor to create an 
output stream to the file whose name is stored in {\tt fileName}.
In this case, the task of writing data to the file is handled by a
single {\tt write()} statement, which writes the entire contents of
the {\tt JTextArea} in one operation.

Finally, once we have finished writing the data, we {\tt close()} the
output stream.  This also has the effect of closing the file.  The
overall effect of this method is that the text contained in
{\tt display} has been output to a file, named {\tt fileName}, which is
stored on the disk.

\JavaTIP{PROGRAMMING TIP}{Closing a File.}{Even though Java will close
any apen files and streams when a program terminates normally, it is
good programming practice to close the file yourself with a {\tt
close()} statement.  It also reduces the chances of damaging the file
if the program terminates abnormally.}

\noindent Because so many different things can go wrong during an I/O operation,
most I/O operations generate some kind of {\it checked exception}.
Therefore, it is necessary to embed the I/O operations within a {\tt try/catch}
statement.  In this example, the {\tt FileWriter()} constructor, the
{\tt write()} method, and the {\tt close()} method may each throw an
{\tt IOException}. Therefore, the entire body of this method should be
embedded within a {\tt try/catch} block that catches the
{\tt IOException} (Fig.~\ref{fig-writetextmethod}).

\begin{figure}[tb]
\jjjprogstart
\begin{jjjlisting}[26.5pc]
\begin{lstlisting}
private void writeTextFile(JTextArea display, 
                                      String fileName) {
  try {
      FileWriter outStream =  new FileWriter (fileName);
      outStream.write (display.getText());
      outStream.close();
  } catch (IOException e) {
      display.setText("IOERROR: " + e.getMessage() + "\n");
      e.printStackTrace();
  }
} // writeTextFile()
\end{lstlisting}
\end{jjjlisting}
\jjjprogstop{A method to write a text file.}
{fig-writetextmethod}
\end{figure}


\subsection{Code Reuse: Designing Text File Output}
\noindent The {\tt writeTextFile()} method provides a simple example of how to
write data to a text file.   More importantly, its development
illustrates the kinds of choices necessary to design effective
I/O methods.  Two important design questions we asked and answered
were

\begin{BL}
\item  What methods do we need to perform the desired task?
\item  What streams contain the desired methods?
\end{BL}

\noindent As in so many other examples we've considered,
\marginnote{Method design}
designing a method to perform a task is often a matter of finding the
appropriate methods in the Java class hierarchy.

\JavaTIP{EFFECTIVE DESIGN}{Code Reuse.}{Developing effective I/O routines
is primarily a matter of choosing the right library methods.  Start
by asking yourself, ``What methods do I need?'' and then find a
stream class that contains the appropriate methods.}

\noindent As you might expect, there is more than one way to write data to a
text file.  Suppose we decided that writing text to a file is like
printing data to {\tt System.out}.  And suppose we chose to use a {\tt
PrintWriter} object as our first candidate for an output stream
(Fig.~\ref{fig-streamhier} and Table~11.1). This class
(Fig.~11\mbox{.}4) contains a wide range of {\tt print()} methods
for writing different types of data as text.  So it has exactly the
kind of method we need: {\tt print(String)}.  However, this stream
does not contain a constructor method that allows us to create a
stream from the name of a file.  Its constructors require either a
{\tt Writer} object or an {\tt OutputStream} object.

This means that we can use a {\tt PrintWriter} to print to a file, but
only if we can first construct either an {\tt OutputStream} or a {\tt
Writer} object to the file.  So we must go back to searching
Figure~\ref{fig-streamhier} and Table~11.1 for an
appropriate candidate.  Fortunately, the {\tt FileOutputStream} class
(Fig.~\ref{fig-fileoutput}) has just the constructors we want.
We now have an alternative way of coding the {\tt writeTextFile()}
method, this time using a combination of {\tt PrintWriter} and
{\tt FileOutputStream}:
%\begin{figure}
\begin{figure}[t]
\figaleft{chptr11/fileoutput.eps}{The {\tt File\-Output\-Stream} class.
} {fig-fileoutput}
\end{figure}
%\end{figure}


\begin{jjjlisting}
\begin{lstlisting}
//  Create an output stream and open the file
PrintWriter outStream = 
    new PrintWriter(new FileOutputStream(fileName)); 
//  Write the display's text and close the stream
outStream.print ( display.getText() );       
outStream.close(); 
\end{lstlisting}
\end{jjjlisting}

\noindent Note how the output stream is created in this case.  First,
we create a {\tt FileOutputStream} using the file name as its
argument.  Then we create a {\tt PrintWriter} using the
{\tt FileOutputStream} as its argument.  The reason we can do this is
%\begin{marginalnote}\it Parameter agreement\end{marginalnote}
\marginnote{Parameter agreement}
because the {\tt PrintWriter()} constructor takes a
{\tt FileOutputStream} parameter.  This is what makes the connection
possible.

%\begin{figure}
%\begin{graphic}
%\includegraphics{./art/11/p778f2.eps}
%\caption{\bf Figure: MacDraft to two pipes connected.
%\label{fig-p778f2}\label{pg-fig-p778f2}}
%\end{graphic}
%\end{figure}

%(Figure~\ref{fig-p778f2}).  
To use the plumbing analogy again, this is like connecting two
sections of pipe between the program and the file.
The data will flow from the program through {\tt PrintWriter}, through
the {\tt OutputStream}, to the file.  Of course, you can't just
arbitrarily connect one stream to another.  They have to ``fit
together,'' which means that their parameters have to match.

\JavaTIP{EFFECTIVE DESIGN}{Stream/Stream Connections.}{Two different kinds
of streams can be connected if a constructor for one stream
takes the second kind of stream as a parameter.  This is often an
effective way to create the kind of object you need to perform an I/O
task.}
\WWWjava

\noindent The important lesson here is that we found what we wanted by searching
through the {\tt java.io.*} hierarchy.  This same approach can be used
to help you to design I/O methods for other tasks.

\secEXRHone{Self-Study Exercise}
\label{self-study-exercise}
\begin{SSTUDY}
\item  Is it possible to perform output to a text file using a
{\tt PrintWriter} and a {\tt FileWriter} stream in combination? If so,
write the Java code.
\end{SSTUDY}

\subsection{Reading from a Text File}
\noindent Let's now look at the problem of inputting data from an
existing text file, a common operation that occurs whenever your email
program opens an email message or your word processor opens a
document.  In general, there are three steps to reading data from a
file:

\begin{NL}
\item  Connect an input stream to the file.

\item  Read the text data using a loop.

\item  Close the stream.
\end{NL}
%\end{list}

\noindent As Figure~\ref{fig-readfromstream} shows, the
input stream serves as a kind of pipe between the file and the
\begin{figure}[h]
\figa{chptr11/filestream.eps}{A stream serves as a pipe through which
data flow.}
{fig-readfromstream}
\end{figure}
program.  The first step is to connect an input stream to the file.
Of course, in order to read a file, the file must exist.  The input
stream serves as a conduit between the program and the named file.  It
opens the file and gets it ready for reading.  Once the file is open,
the next step is to read the file's data.  This will usually require a
loop that reads data until the end of the file is reached.  Finally,
once all the data are read, the stream should be closed.


\JavaTIP{EFFECTIVE DESIGN}{Reading Data}{Reading data from a file
requires a three-step algorithm: (1) Connect an input stream to the
file, (2) read the data, and (3) close the file.}

\noindent Now let's see how these three steps are done in Java.  Suppose
that we want to put the file's data into a {\tt JTextArea}. Thus, we
want a method that will be given the name of a file and a
reference to a {\tt JTextArea}, and it will read the data
from the file into the {\tt JTextArea}.

What input stream should we use for this task?  Here again we
%\begin{marginalnote}\it Choosing an input stream\end{marginalnote}
\marginnote{\raggedright \it Choosing an input stream}
need to use the information in  Figure~\ref{fig-streamhier} and
Table~11.1. Because we're reading a text file,
we should use a {\tt Reader} subclass.  A good candidate is
the {\tt File\-Reader}, whose name and description suggest
that it might contain useful methods.

What methods do we need? As in the previous example, we need a
constructor method that connects an input stream to a file when the
constructor is given the \marginnote{What methods should we use?}
name of the file.  And, ideally, we'd like to have a method that will
read one line at a time from the text file.  

%\begin{figure}
\begin{figure}[th]
\figaleft{chptr11/filereader.eps}{{\tt FileReader}'s superclasses contain
{\tt read()} methods but no {\tt readLine()} methods.
} {fig-filereader}
\end{figure}
%\end{figure}
The {\tt FileReader} class (Fig.~\ref{fig-filereader}) has the right
kind of constructor.  However, it contains no {\tt readLine()} methods
itself, which would be necessary for our purposes.  Searching upward
through its superclasses, we find that \mbox{\tt InputStreamReader},
its immediate parent class, has a method that reads {\tt int}s:

\begin{jjjlisting}
\begin{lstlisting}
public int read() throws IOException();
\end{lstlisting}
\end{jjjlisting}

\noindent As shown in Figure~\ref{fig-filereader}, this {\tt read()}
method is an override of the {\tt read()} method defined in the {\tt
Reader} class, the root class for text file input streams.  Thus,
there are no {\tt readLine()} methods in the {\tt Reader} branch of
the hierarchy. We have to look elsewhere for an appropriate class.


One class that does contain a {\tt readLine()} method is {\tt
BufferedReader} (Fig.~\ref{fig-filereader}). Can we somehow use it?
Fortunately, the answer is yes.  {\tt BufferedReader}'s constructor
takes a {\tt Reader} object as a parameter.  But a {\tt FileReader}
{\it is} a {\tt Reader}---that is, it is a descendant of the {\tt
Reader} class.  So, to use our plumbing analogy again, to build an
input stream to the file, we can join a {\tt BufferedReader}
and a {\tt FileReader} 
%(Figure~\ref{fig-p780f2}):

\begin{jjjlisting}
\begin{lstlisting}
BufferedReader inStream
    = new BufferedReader(new FileReader(fileName));
\end{lstlisting}
\end{jjjlisting}

\noindent Given this sort of connection to the file, the
program can use {\tt Buffered\-Reader.readLine()} to read one line at a
time from the file.

%\begin{figure}
%\begin{graphic}
%\includegraphics{./art/14/p780f2.eps}
%\caption{\bf Another plumbing picture.
%\label{fig-p780f2}\label{pg-fig-p780f2}}
%\end{graphic}
%\end{figure}

So, we have found a method that reads one line at a time.  Now we need an
algorithm that will read the entire file.  Of course, this will involve
a loop, and the key will be to make sure we get the loop's termination
condition correct.  

An important fact about {\tt readLine()} 
%\begin{marginalnote}\it Using the end-of-file character\end{marginalnote}
\marginnote{\raggedright \it Using the end-of-file character}
is that it
will return {\tt null} as its value when it reaches the end of the
file.   Recall that text files have a special end-of-file character.
When {\tt readLine()} encounters this character, it will return
{\tt null}. Therefore, we can specify the following {\tt while} loop:

\begin{jjjlisting}
\begin{lstlisting}
String line = inStream.readLine();
while (line != null) {
    display.append(line + "\n");
    line = inStream.readLine();
}
\end{lstlisting}
\end{jjjlisting}

\noindent We begin outside the loop by attempting to
read a line from the file.  If the file happens to be empty (which it
might be), then {\tt line} will be set to {\tt null}; otherwise it will
contain the {\tt String} that was read.  In this case, we append the
line to a {\tt JTextArea}. Note that {\tt readLine()} {\it does not}
return the end-of-line character with its return value.  That's why we
add a \verb|\n| before we append the line to the {\tt JTextArea}.


\JavaTIP[false]{PROGRAMMING TIP}{End of Line.}{Remember that {\tt
readLine()} does not return the end-of-line character as part of the
text it returns.  If you want to print the text on separate lines, you
must append $\backslash$n.}

\noindent The last statement in the body of the loop attempts to read the next
line from the input stream.  If the end of file has been reached, this
attempt will return {\tt null} and the loop will terminate.  Otherwise,
the loop will continue reading and displaying lines until the end of
file is reached.  Taken together, these various design decisions lead to
the definition for {\tt readTextFile()} shown in Figure~\ref{fig-readtextmethod}.
\begin{figure}[h!]
\jjjprogstart
\begin{jjjlisting}
\begin{lstlisting}
private void readTextFile(JTextArea display, 
                                    String fileName) {
 try {
  BufferedReader inStream  // Create and open the stream
      = new BufferedReader (new FileReader(fileName));
  String line = inStream.readLine(); // Read one line
  while (line != null) {            // While more text
      display.append(line + "\n");  // Display a line
      line = inStream.readLine();   // Read next line
  }
   inStream.close();                // Close the stream
  } catch (FileNotFoundException e) {
   display.setText("IOERROR: "+ fileName +" NOT found\n");
   e.printStackTrace();
  } catch ( IOException e ) {
   display.setText("IOERROR: " + e.getMessage() + "\n");
   e.printStackTrace();
  }
} // readTextFile()
\end{lstlisting}
\end{jjjlisting}
\jjjprogstop{A method for reading a text file.}
{fig-readtextmethod}
\end{figure}

Note that we must catch both the {\tt IOException}, thrown by
%\begin{marginalnote}\tt IOException\end{marginalnote}
\marginnote{\tt IOException}
{\tt readLine()} and {\tt close()}, and the {\tt
FileNotFoundException}, thrown by the {\tt FileRead\-er()} constructor.
It's important to see that the read loop has the following form:

\begin{jjjlisting}
\begin{lstlisting}
try to read one line of data 
     and store it in line     // Loop initializer
while ( line is not null ) { // Loop entry condition
    process the data
    try to read one line of data 
        and store it in line  // Loop updater
}
\end{lstlisting}
\end{jjjlisting}

\noindent When it attempts to read the {\it end-of-file}
character, {\tt readLine()} will return {\tt null}.

\JavaTIP[false]{EFFECTIVE DESIGN}{Reading Text.}{In reading text files, the
{\tt readLine()} method will return {\tt null} when it tries to read the
end-of-file character.   This provides a convenient way of testing for
the end of file.}

\JavaTIP[false]{EFFECTIVE DESIGN}{Reading an Empty File.}{Loops for
reading text files are designed to work even if the file is
empty.  Therefore, the loop should attempt to read a line {\it before}
testing the loop-entry condition.  If the initial read returns
{\tt null}, that means the file is empty and the loop body will be skipped.}

\secEXRHone{Self-Study Exercise}
\label{self-study-exercise}
\begin{SSTUDY}

\item  What's wrong with the following loop for reading
a text file and printing its output on the screen?

\begin{jjjlisting}
\begin{lstlisting}
    String line = null;
    do {
        line = inStream.readLine();
        System.out.println ( line );
    } while (line != null);
\end{lstlisting}
\end{jjjlisting}
\end{SSTUDY}

\subsection{Code Reuse: Designing Text File Input}
\noindent Our last example used {\tt BufferedReader.readLine()} to read an entire
line from the file in one operation.  But this isn't the only way
to do things.  For example, we could use the {\tt FileReader}
stream directly if we were willing to do without the {\tt readLine()}
method.  Let's design an algorithm that works in this case.

As we saw earlier, if you use a {\tt FileReader} stream, then you must
use the {\tt InputStreamReader.read()} method.  This method reads bytes
from an input stream and translates them into Java Unicode characters.
The {\tt read()} method, for example, returns a single Unicode character
as an {\tt int}:

\begin{jjjlisting}
\begin{lstlisting}
public int read() throws IOException();
\end{lstlisting}
\end{jjjlisting}

\noindent Of course, we can always convert this to a {\tt char} and
concatenate it to a {\tt JTextArea}, as the following algorithm
illustrates:

\begin{jjjlisting}
\begin{lstlisting}
int ch = inStream.read(); // Init: Try to read a character
while (ch != -1) { // Entry-condition: while more chars
    display.append((char)ch + ""); // Append the character
    ch = inStream.read();   // Updater: try to read
}
\end{lstlisting}
\end{jjjlisting}

\noindent Although the details are different, the structure of this
loop is the same as if we were reading one line at a time.

The loop variable in this case is an {\tt int} because {\tt
InputStreamReader.\-read()} returns the next character as an {\tt int},
or it returns $-1$ if it encounters the {\tt end-of-file}
%\begin{marginalnote}\it Data conversion\end{marginalnote}
\marginnote{Data conversion}
character.  Because {\tt ch} is an {\tt int}, we must convert it to a
{\tt char} and then to a {\tt String} in order to {\tt append()} it to the
display.

A loop to read data from a file has the following basic form:

\begin{jjjlisting}
\begin{lstlisting}
try to read data into a variable // Initializer
while ( read was successful ) {  // Entry condition
    process the data
    try to read data into a variable // Updater
}
\end{lstlisting}
\end{jjjlisting}

\JavaTIP{EFFECTIVE DESIGN}{Read Loop Structure.}{The {\tt read()} and
{\tt readLine()} methods have different ways to indicate when a read
attempt fails.  These differences affect how the loop-entry condition is
specified, but the structure of the read loop is the same.}

\JavaTIP{PROGRAMMING TIP}{Read Versus Readline.}{Unless it is necessary to
manipulate each character in the text file, reading a line at a time
is more efficient and, therefore, preferable.}

\noindent It is worth noting again the point we made earlier: Designing
effective I/O routines is largely a matter of searching the
{\tt java.io} package for appropriate classes and methods.  The methods
we've developed can serve as suitable models for a wide variety of text
I/O tasks, but if you find that they aren't suitable for a particular
task, you can design your own method.  Just think about what it
is you want the program to accomplish, then find the stream classes that
contain methods you can use to perform the desired task.  Basic
\marginnote{Reusing existing code}
reading and writing algorithms will be pretty much the same no matter
which particular read or write method you use.

\secEXRHone{Self-Study Exercise}
\label{self-study-exercise}
\begin{SSTUDY}

\item  What's wrong with the following loop for reading
a text file and printing its output on the screen?

\begin{jjjlisting}
\begin{lstlisting}
int ch;
do {
    ch = inStream.read();
    System.out.print((char)ch);
}  while (ch != -1) 
\end{lstlisting}
\end{jjjlisting}
\end{SSTUDY}

\subsection{The {\tt TextIO} Application}
\noindent Given the text I/O methods we wrote in the previous sections, we can
now specify the overall design of our {\tt TextIO} class
(Fig.~\ref{fig-textiouml}).  In order to complete this application, we
need only set up its GUI and write its {\tt action\-Performed()} method.

\begin{figure}[h!]
\figaleft{chptr11/textiouml.eps}{The {\tt TextIO} class.
} {fig-textiouml}
\end{figure}

\noindent Setting up the GUI for this application is straightforward.
Figure~11.13 shows how the finished product will look.  The code is
given in Figure~11.14. Pay particular attention to the {\tt
actionPerformed()} method, which uses the methods we defined in the
previous section.

\begin{figure}[h!]
\figaleft{chptr11/textioapp.eps}{An application that performs simple text I/O.
} {fig-textioapp}
\end{figure}

\begin{figure}[p]
\jjjprogstart
\begin{jjjlisting}[30pc]
\begin{lstlisting}
import javax.swing.*;         // Swing components
import java.awt.*;
import java.io.*;
import java.awt.event.*;

public class TextIO extends JFrame implements ActionListener{
 private JTextArea display = new JTextArea();
 private JButton read = new JButton("Read From File"),
                 write = new JButton("Write to File");
 private JTextField nameField = new JTextField(20);
 private JLabel prompt = new JLabel("Filename:",JLabel.RIGHT);
 private JPanel commands = new JPanel();

 public TextIO() {                     // Constructor
  super("TextIO Demo");             // Set window title
  read.addActionListener(this);
  write.addActionListener(this);
  commands.setLayout( new GridLayout(2,2,1,1));  // Control panel
  commands.add(prompt);
  commands.add(nameField);
  commands.add(read);
  commands.add(write);
  display.setLineWrap(true);
  this.getContentPane().setLayout(new BorderLayout());
  this.getContentPane().add("North", commands);
  this.getContentPane().add( new JScrollPane(display));
  this.getContentPane().add("Center", display);
 } // TextIO
 private void writeTextFile(JTextArea display, String fileName) {
   try {
      FileWriter outStream =  new FileWriter (fileName);
      outStream.write (display.getText());
      outStream.close();
  } catch (IOException e) {
      display.setText("IOERROR: " + e.getMessage() + "\n");
      e.printStackTrace();
  }
 } // writeTextFile()
 private void readTextFile(JTextArea display, String fileName) {
   try {
      BufferedReader inStream  // Create and open the stream
          = new BufferedReader (new FileReader(fileName));
      String line = inStream.readLine(); // Read one line
      while (line != null) {         // While more text
        display.append(line + "\n"); // Display a line
        line = inStream.readLine();      // Read next line
      }
        inStream.close();        // Close the stream
      } catch (FileNotFoundException e) {
        display.setText("IOERROR: "+ fileName +" NOT found\n");
        e.printStackTrace();
      } catch (IOException e) {
        display.setText("IOERROR: " + e.getMessage() + "\n");
        e.printStackTrace();
      }
 } // readTextFile
\end{lstlisting}
\end{jjjlisting}
\jjjprogstop{Part I of the {\tt TextIO} class. }
{fig-textioclass}
\end{figure}

\begin{figure}[tb]
\addtocounter{figure}{-1}
\jjjprogstart
\begin{jjjlisting}
\begin{lstlisting}
  public void actionPerformed(ActionEvent evt) {
    String fileName = nameField.getText();
    if (evt.getSource()  == read) {
        display.setText("");
        readTextFile(display, fileName);
    }
    else writeTextFile(display, fileName);
  } // actionPerformed()

  public static void main(String args[]) {
    TextIO tio = new TextIO();
    tio.setSize(400, 200);
    tio.setVisible(true);
    tio.addWindowListener(new WindowAdapter() {      
        public void windowClosing(WindowEvent e) {
            System.exit(0); // Quit the application
        }
    });
  } // main()
}//TextIO
\end{lstlisting}
\end{jjjlisting}
\jjjprogstop{({\it continued}) The {\tt TextIO} class, Part II.}
{fig:textIOpart2}

\end{figure}

\pagebreak
\section{The {\tt File} Class}
\label{the-class}
\noindent As we've seen, an attempt to create a {\tt FileReader}
stream may throw a {\tt FileNotFoundException}. The way this happens
is if the user provides a name for a file that either doesn't exist or 
isn't located where its name says it should be located.  The question that
needs to be considered: Is there any
way we can detect these kinds of errors before attempting to read the
file?

The {\tt java.io.File} class provides methods that we can use for this
\marginfig{chptr11/hier.eps}{A simple hierarchy of directories and files.}
{fig-filehier}
task.  The {\tt File} class provides a representation of the computer's
file and directory information in a platform-independent manner.  As you
know, a file is a collection of data, whereas a {\bf directory}\index{directory}
is a collection of files.  (To be exact, a directory is a file that stores
its files' names and attributes, not the files themselves.) In this section,
we will provide details about the {\tt File} class and how to
use the methods available in the class.

\subsection{Names and Paths}
\noindent In order to correctly specify a file's location, it is
necessary to know a little about how files are stored on your
computer's disk drive.  File systems are organized into a hierarchy.
A {\bf path}\index{path} is a description of a file's location in the
hierarchy.  For example, consider the hierarchy of files in
Figure~11.15.  Assume that your Java program is named {\tt
MyClass.class}. When a program is running, the program's directory is
considered the {\it current directory}. Any files located in the
current directory can be referred to by name alone---for example, {\tt
MyClass.java}. To refer to a file located in a subdirectory of the
\marginnote{The file hierarchy} 
current directory, you need to provide the name of the subdirectory
and the file: {\tt datafiles/data.txt}.  In this case, we are assuming
a Unix file system, so we are using the \verb|/| as the separator
between the name of the directory ({\tt datafiles}) and the name of
the file ({\tt data.txt}).  This is an example of a {\bf relative path
name}, because we are specifying a file in relation to the current
directory.

Alternatively, a file can be specified by its {\bf absolute path
name}\index{absolute path name}. This would be a name whose path
%\marginnote{File path names}
starts at the root directory of the file system.  For example,

\begin{jjjlisting}
\begin{lstlisting}
/root/java/examples/datafiles/data.txt 
\end{lstlisting}
\end{jjjlisting}

\noindent would be the
absolute path name for the file named {\tt data.txt} on a Unix system.
When you supply the name of a file to one of the stream constructors,
you are actually providing a {\it path} name.  If the path consists of
just a name, such as {\tt data.txt}, Java assumes that the file is
located in the same directory as the program itself.



\subsection{Validating File Names}
\noindent Before reading a file it is often necessary to determine
that the file's name is a valid one and that the file can
be read. The {\tt File} class (Fig.~11.16) provides
platform-independent methods for dealing with files and directories.
It contains methods that list the contents of directories, determine
a file's attributes, and rename and delete files.
%\begin{figure}
%\begin{graphic}
\marginfig{chptr11/fileuml.eps}{The {\tt java.io.File} class.}
{fig-fileuml-files}
%\end{graphic}
%\end{figure}
Note the several {\tt static} constants provided.  These allow
path names to be specified in a platform-independent way.  For example,
on a Unix system, the {\tt File.separator} character will be the
%%%REW\verb@/@ and on a Windows system it will be the \verb@\,backslash@. {\tt File.separator}
\verb|/| and on a Windows system it will be the \verb|\,backslash|. {\tt File.separator}
will be initialized to the appropriate separator for the particular
system being used.

\JavaTIP{PROGRAMMING TIP}{File Separators.}{To help make your programs platform
independent, use the {\tt File.separator} constant instead of a
literal value whenever you are specifying a path name.}

\noindent As an example of how you might use some of {\tt File}'s methods, let's
write a method that tests whether the file name entered by the user is
the name of a valid, readable file.

A file might be unreadable for a number of reasons.  It might be owned by
%\begin{marginalnote}\it Method design\end{marginalnote}
\marginnote{Method design}
another user and readable only by that user.  Or it might be designated
as not readable by its owner.   We'll pass the method the name of the
file (a {\tt String}), and the method will return {\tt true} if a
readable file with that name exists.   Otherwise, the method will throw
an exception and return {\tt false}:

\begin{jjjlistingleft}[27pc]{-1pc}
\begin{lstlisting}
private boolean isReadableFile(String fileName) {
  try {
    File file = new File(fileName);
    if (!file.exists())
        throw (new FileNotFoundException("No such File:" 
                                     + fileName));
    if (!file.canRead())
        throw (new IOException("File not readable: " 
                                     + fileName));
    return true;                                    
  } catch (FileNotFoundException e) {
    System.out.println("IOERROR: File NOT Found: " 
                             + fileName + "\n");
    return false;
  } catch (IOException e) {
    System.out.println("IOERROR: " + e.getMessage() + "\n");
    return false;
  }
} // isReadableFile
\end{lstlisting}
\end{jjjlistingleft}

\noindent The method simply creates a {\tt File} instance and
uses its {\tt exists()} and {\tt canRead()} methods to
check whether its name is valid.  If either condition
fails, an exception is thrown.  The method handles its
own exceptions, printing an error message and returning false
in each case.

Before attempting to write data to a file, we might want to check that
the file has been given an appropriate name.  For example, if the user
leaves the file name blank, we should not write data to the file.
Also, a file might be designated as unwriteable in order to protect it
from being inadvertently overwritten.  We should check that the file is
writeable before attempting to write to it:

\begin{jjjlisting}
\begin{lstlisting}
private boolean isWriteableFile(String fileName) {
  try {
    File file = new File (fileName);
    if (fileName.length() == 0)
        throw (new IOException("Invalid file name: " 
                                     + fileName));
    if (file.exists() && !file.canWrite())
         throw (new IOException(
            "IOERROR: File not writeable: " + fileName));
    return true;
  } catch (IOException e) {
    display.setText("IOERROR: " + e.getMessage() + "\n");
    return false;
  }
} // isWriteableFile()
\end{lstlisting}
\end{jjjlisting}

\noindent The first check in this code tests that the user has
not forgotten to provide a name for the output file.  It is unlikely
that the user wants to name the file with the empty string.   We use
the {\tt exists()} method to test whether the user is attempting to
write to an existing file.  If so, we use the {\tt canWrite()} method
to test whether the file is writeable.  Both kinds of errors result in
{\tt IOException}s.

\secEXRHone{Self-Study Exercise}
\label{self-study-exercise}
\begin{SSTUDY}

\item  The other methods of the {\tt File} class are 
just as easy to use as the ones we have illustrated in this section.
Write a method that takes the name of a file as its single parameter and
prints the following information about the file: its absolute path,
its length, and whether it is a directory or a file.
\end{SSTUDY}

\section{Example: Reading and Writing Binary Files}
\label{example-readingand-writing-binary-files}
\noindent Although text files are extremely useful and often employed, they can't 
and shouldn't be used for every data-processing application.
For example, your college's administrative data system undoubtedly
uses files to store student records.  Because your student record
contains a variety of different types of data---{\tt String}s,
{\tt int}s, {\tt double}s---it cannot be processed as text.   Similarly, a
company's inventory files, which also include data of a wide variety of
types, cannot be processed as text.  Files such as these must be
processed as binary data.

Suppose you are asked to write an application that involves the
use of a company's employee records.  Recall that a record
is a structure that combines different types of data into a
single entity.  It's like an object with no methods, just instance
variables.

A binary file is a sequence of bytes.  Unlike a text file, which is
terminated by a special end-of-file marker, a binary file consists of
nothing but data.  A binary file doesn't have an end-of-file character
because any such character would be indistinguishable from a binary
datum.

\JavaTIP{DEBUGGING TIP}{End of Binary File.}{Because a binary file does
not have an end-of-file character, it would be an error to use the
same loop-entry conditions we used in the loops we designed for
reading text files.}

\noindent Generally speaking, the steps involved in reading and writing binary
files are the same as for text files:

%\begin{list}{}{}
\begin{NL}
\item  Connect a stream to the file.
\item  Read or write the data, possibly using a loop.
\item  Close the stream.
\end{NL}
%\end{list}

\noindent The difference between text and binary file I/O resides
in the Java streams that we use.

\subsection{Writing Binary Data}
\noindent Let's begin by designing a method that will output employee data to a
binary file.  As the developer of this program, one thing you'll have
%\begin{marginalnote}\it Generating binary data\end{marginalnote}
\marginnote{Generating binary data}
to do is build some sample data files.  These can't easily be built
by hand---remember you can't use a text editor to create them---so
you'll want to develop a method that can generate some random data
of the sort your application will have to process.

\JavaTIP[false]{EFFECTIVE DESIGN}{I/O Design.}{When designing file I/O
applications, it is good to design the input and the output methods
together.   This is especially important for binary I/O.}

\noindent The first thing we need to know is exactly what the data look
like.  Let's assume that each record contains three individual pieces
of data---the employee's name, age, and pay rate.  For example, the
data in a file containing four records might look like this, once the
data are interpreted:

\begin{jjjlisting}
\begin{lstlisting}
Name0 24 15.06
Name1 25 5.09
Name2 40 11.45
Name3 52 9.25
\end{lstlisting}
\end{jjjlisting}

\noindent As you can see, these data look as if they were randomly
generated, but they resemble the real data in the important respects:
They are of the right type---{\tt String}, {\tt int}, {\tt double}---and have 
the right kind of values.   Of course, when these data are stored in the
file, or in the program's memory, they just look like one long string
of 0's and 1's.

Our approach to designing this output method will be the same
as the approach we used in designing methods for text I/O.~That is,
we start with two questions:

\begin{BL}
\item  What stream classes should I use?
\item  What methods can I use?
\end{BL}

\noindent And we find the answers to these by searching through
the {\tt java.io} package (Fig.~\ref{fig-streamhier} and Table~11.1).

%\begin{figure}
\begin{figure}[b]
\figaleft{chptr11/fostream.eps}{The {\tt File\-Output\-Stream} class.
} {fig-fostream}
\end{figure}
%\end{figure}

Because we are performing binary output, we need to use some subclass
%%\marginnote{\raggedright \it Choosing an output stream}
of {\tt OutputStream}. Because we're outputting to a file, one likely
candidate is {\tt FileOutputStream} (Fig.~\ref{fig-fostream}). This
class has the right kind of constructors, but it only contains {\tt
write()} methods for writing {\tt int}s and {\tt byte}s, and
we need to be able to write {\tt String}s and {\tt double}s
as well as {\tt int}s.  

\pagebreak
These kinds of methods are found in {\tt DataOutputStream}
\marginfig{chptr11/dostream.eps}{The {\tt java.io.Data\-Output\-Stream} class contains methods
for writing all types of data.}
{fig-dostream}
(Fig.~11.18), which contains a {\tt write()} method for each
different type of data. As you can see, there's one method for each
primitive type.  However, note that the {\tt writeChar()} takes an
{\tt int} parameter, which indicates that the character is written in
binary format rather than as a ASCII or Unicode character.  Although
you can't tell by just reading its method signature, the {\tt
writeChars(String)} method also writes its data in binary format
rather than as a sequence of characters.  This is the main difference
between these {\tt write()} methods and the ones defined in the {\tt
Writer} branch of Java's I/O hierarchy.


Now that we've found the appropriate classes and methods,  we need to create
a pipeline to write data to the file and develop an output algorithm.
To construct a stream
to use in writing employee records, we want to join together a
%\begin{marginalnote}\it Connecting two streams\end{marginalnote}
%´´´\marginnote{\vspace{12pt}Connecting two streams}
{\tt DataOutputStream} and a {\tt FileOutputStream}. The
{\tt DataOutputStream} gives us the output methods we need, and the
{\tt FileOutputStream} lets us use the file's name to create the stream:

\begin{jjjlisting}
\begin{lstlisting}
DataOutputStream outStream
  = new DataOutputStream(new FileOutputStream (fileName));
\end{lstlisting}
\end{jjjlisting}

\noindent This enables the program to write data to the {\tt DataOutputStream},
which will pass them through the {\tt FileOutputStream} to the
file itself.  That settles the first question.

To develop the output algorithm, we need some kind of loop that
involves calls to the appropriate methods.  In this case, because we are
generating random data, we can use a simple for loop to generate, say,
five records of employee data.  We need one {\tt write()} statement for
each of the elements in the employee record: The name ({\tt String}), age
({\tt int}), and pay rate ({\tt double}):

\begin{jjjlisting}
\begin{lstlisting}
for (int k = 0; k < 5; k++) {  // Output 5 data records
 outStream.writeUTF("Name" + k);             // Name
 outStream.writeInt((int)(20 + Math.random() * 25)); //Age
 outStream.writeDouble(Math.random() * 500); // Payrate
}
\end{lstlisting}
\end{jjjlisting}

\noindent Within the loop body we have one output statement for
each data element in the record.  The names of the methods reflect
the type of data they write.  Thus, we use {\tt writeInt()}
to write an {\tt int} and {\tt writeDouble()} to write a {\tt double}.
But why do we use {\tt writeUTF} to write the employee's
name, a {\tt String}?

\subsubsection*{The Unicode Text Format (UTF)}
\noindent There is no {\tt DataOutputStream.writeString()} method.  Instead,
{\tt String}s are written using the {\tt writeUTF()} method.   {\bf UTF}\index{UTF}
stands for {\bf Unicode Text Format}, a coding scheme for Java's
Unicode character set.  Recall that Java uses the Unicode character set
instead of the ASCII set.  As a 16-bit code, Unicode can represent
\marginnote{ASCII vs.  Unicode}
8-bit ASCII characters plus a wide variety of Asian and other
international characters.  However, Unicode is not a very efficient
%\begin{marginalnote}\it ASCII vs.  Unicode\end{marginalnote}
coding scheme if you aren't writing an international program.   If your
program just uses the standard ASCII characters, which can be stored
in 1 byte, you would be wasting 1 byte per character if you stored
them as straight Unicode characters.  Therefore, for efficiency
purposes, Java uses the UTF format.  UTF encoding can
still represent all of the Unicode characters, but it provides a more
efficient way of representing the ASCII subset.

It's now time to combine these separate elements into a single method
(Fig.~\ref{fig-writerecordsmethod}).  The {\tt writeRecords()} method
takes a single {\tt String} parameter that specifies the name of the
file.  This is a {\tt void} method.  It will output data to a file,
but it will not return anything to the calling method.  The method
follows the standard output algorithm: Create an output stream, write
the data, close the stream.  Note also that the method includes a {\tt
try/catch} block to handle any {\tt IOException}s that might be thrown.
\begin{figure}[h]
\jjjprogstart
\begin{jjjlistingleft}[28.5pc]{-1.5pc}
\begin{lstlisting}
private void writeRecords( String fileName )  {
 try {
   DataOutputStream outStream   // Open stream
      = new DataOutputStream(new FileOutputStream(fileName)); 
   for (int k = 0; k < 5; k++) { // Output 5 data records
      String name = "Name" + k;  // of name, age, payrate
      outStream.writeUTF("Name" + k);          
      outStream.writeInt((int)(20 + Math.random() * 25)); 
      outStream.writeDouble(5.00 + Math.random() * 10);   
   } // for
   outStream.close();          // Close the stream
 } catch (IOException e) {
   display.setText("IOERROR: " + e.getMessage() + "\n");
 }
} // writeRecords()
\end{lstlisting}
\end{jjjlistingleft}
\jjjprogstop{A method to write a binary file consisting of five
randomly constructed records.}
{fig-writerecordsmethod}
\end{figure}


\subsection{Reading Binary Data}
\noindent The steps involved in reading data from a binary file are the
same as for reading data from a text file: Create an input
stream and open the file, read the data, close the file.  The main
difference lies in the way you check for the end-of-file marker
in a binary file.

Let's design a method to read the binary data that were output by the
{\tt writeRecords()} method.  We'll call this method
{\tt readRecords()}. It, too, will consist of a single {\tt String}
\marginfig{chptr11/fistream.eps}{The {\tt java.io.File\-InputStream} class.}
{fig-fistream}
parameter that provides the name of the file  to be read, and
it will be a void method.  It will just display the data on
{\tt System.out}.

The next questions we need to address are: What stream classes should
we use, and what methods should we use?  For binary input, we need an
{\tt InputStream} subclass (Fig.~\ref{fig-streamhier} and
Table~11.1). As you've probably come to expect, the {\tt
FileInputStream} class contains constructors that let us create a
stream from a file name (Fig.~11\mbox{.}20). However, it does not
contain useful {\tt read()} methods.  Fortunately, the {\tt
DataInputStream} class contains the input counterparts of the methods
we found in {\tt DataOutput\-Stream} (Fig.~11.21). Therefore, our
input stream for this method will be a combination of {\tt
DataInputStream} and {\tt FileInputStream}:

\begin{jjjlisting}
\begin{lstlisting}
DataInputStream inStream
   = new DataInputStream(new FileInputStream(file));
\end{lstlisting}
\end{jjjlisting}

\noindent Now that we have identified the classes and methods we'll use 
to read the data, the most important remaining issue is designing a
read loop that will terminate correctly.  Unlike text files, binary
files do not contain a special end-of-file marker.  Therefore, the read
methods can't see anything in the file that tells them they're at the
end of the file.  Instead, when a binary read method attempts to read
past the
%\begin{marginalnote}\it Designing the read loop\end{marginalnote}
%\marginnote{Designing the read loop}
end of the file, an end-of-file exception {\tt EOFException} is
thrown.  Thus, the binary loop is coded as an infinite loop that's
exited when the {\tt EOFException} is raised:
%\begin{figure}
%\begin{graphic}
\marginfigvspace{-120pt}{chptr11/distream.eps}{The {\tt java.io.Data\-Input\-Stream} 
class contains methods for reading all types of data.
} {fig-distream}
%\end{graphic}
%\end{figure}


\begin{jjjlisting}
\begin{lstlisting}
try {
  while (true) {  // Infinite loop
    String name = inStream.readUTF();  // Read a record
    int age = inStream.readInt();
    double pay = inStream.readDouble();
    display.append(name + " " + age + " " + pay + "\n");
  } // while
} catch (EOFException e) {} // Until EOF exception
\end{lstlisting}
\end{jjjlisting}

\noindent The read loop is embedded within a {\tt try/catch}
statement.   Note that the {\tt catch} clause for the
{\tt EOFException} does nothing.  Recall that when an exception is thrown in
a {\tt try} block, the block is exited for good, which is precisely
the action we want to take.  That's why we needn't do anything when we
catch the {\tt EOFException}. We have to catch the exception or else
Java will catch it and terminate the program.  This is an
%\begin{marginalnote}\it An expected exception\end{marginalnote}
\marginnote{An expected exception}
example of an expected exception.

\JavaTIP{EFFECTIVE DESIGN}{{\tt EOFException}.}{An attempt to read past
the end of a binary file will cause an {\tt EOFException} to be
thrown.  Catching this exception is the standard way of terminating a
binary input loop.}

\noindent Note also the {\tt read()} statements within the loop are mirror
opposites of the {\tt write()} statements in the method that created
the data.  This will generally be true for binary I/O routines: The
statements that read data from a file should ``match'' those that
wrote the data in the first place.

\JavaTIP{EFFECTIVE DESIGN}{Matching Input to Output.}{The statements used
to read binary data should match those that wrote the data.   If a
{\tt writeX()} method were used to write the data, a {\tt readX()} should be
used to read~it.}

\noindent To complete the method, the only remaining task is to {\tt close()}
the stream after the data are read.  The complete definition is shown
in Figure~\ref{fig-readrecordsmethod}.

\begin{figure}[tb]
\jjjprogstart
\begin{jjjlistingleft}[29pc]{-3pc}
\begin{lstlisting}
private void readRecords( String fileName ) {
 try {
   DataInputStream inStream     // Open stream
     = new DataInputStream(new FileInputStream(fileName)); 
   display.setText("Name   Age Pay\n");
   try {
     while (true) {               // Infinite loop
        String name = inStream.readUTF();  // Read a record
        int age = inStream.readInt();
        double pay = inStream.readDouble();
        display.append(name + " " + age + " " + pay + "\n");
     } // while
   } catch (EOFException e) { // Until EOF exception
   } finally {
       inStream.close();             // Close the stream
   }
   } catch (FileNotFoundException e) {
      display.setText("IOERROR: "+ fileName + " NOT Found: \n");
   } catch (IOException e) {
      display.setText("IOERROR: " + e.getMessage() + "\n");
   }
} // readRecords()
\end{lstlisting}
\end{jjjlistingleft}
\jjjprogstop{A method for reading binary data.}
{fig-readrecordsmethod}
\end{figure}

It's important that a {\tt close()} statement be placed after the
{\tt catch EOF\-Exception} clause.  If it were placed in the {\tt try} block,
it would never get executed.   Note also that the entire method is
embedded in an outer {\tt try} block that catches the
{\tt IOException}, thrown by the various {\tt read()} methods, and the
{\tt FileNotFoundException}, thrown by the {\tt FileInputStream()}
constructor.  These make the method a bit longer, but conceptually
they belong in this method.

\JavaTIP{PROGRAMMING TIP}{The {\tt finally} Block.}{In coding a binary
read loop, the try block is exited as soon as the {\tt EOFException}
is raised.   Therefore, the {\tt close()} statement must be placed in
the {\tt finally} clause, which is executed after the {\tt catch}
clause.}

\JavaTIP{EFFECTIVE DESIGN}{Nested Try/Catch.}{Nested try blocks must be
used to perform binary I/O correctly.  The outer block encapsulates
statements that throw {\tt IOException}s.  The inner block encapsulates
the read loop and catches the {\tt EOFException}. No particular action
need be taken when the {\tt EOFException} is caught.}

\pagebreak
\secEXRHone{Self-Study Exercise}
\label{self-study-exercise}
\begin{SSTUDY}

\item  Identify the error in the following method, which is supposed to read
a binary file of {\tt int}s from a {\tt DataInputStream}:

\begin{jjjlisting}
\begin{lstlisting}
public void readIntegers(DataInputStream inStream) {
    try {
        while (true) {
            int num = inStream.readInt();
            System.out.println(num);
        }
        inStream.close();
    } catch (EOFException e) {
    } catch (IOException e) {
    }
} // readIntegers
\end{lstlisting}
\end{jjjlisting}
\end{SSTUDY}


%%%REW\subsection{The {\ttHtwo BinaryIO} Application}
\subsection{The {\tt BinaryIO} Application}
\noindent Given the methods we wrote in the previous section, we can now
specify the overall design of the {\tt BinaryIO} class
(Fig.~11.23).  The program sets up the same interface we used
in the text file example (Fig.~11.24). It allows~the 
user to specify the name of a data file to read or
write.  One button allows the user to write random employee records to a binary
file, and the other allows the user to display the contents of a file
in a {\tt JTextArea}. The {\tt BinaryIO} program in Figure~11.25
%\begin{figure}
%\begin{graphic}
\marginfig{chptr11/biniouml.eps}{Design of the {\tt BinaryIO} class.}
{fig-biniouml}
%\end{graphic}
%\end{figure}
incorporates both {\tt readRecords()} and {\tt writeRecords()} into a
complete Java program. 

%\begin{figure}
%\begin{figure}[p]
\marginfigvspace{30pt}{chptr11/binioapp.eps}{A program to read and write binary files.
} {fig-binaryioscreen}
%\end{figure}
%\end{figure}

\begin{figure}[p]
\jjjprogstart
\begin{jjjlistingleft}[36pc]{-10pc}
\begin{lstlisting}
import javax.swing.*;         // Swing components
import java.awt.*;
import java.io.*;
import java.awt.event.*;

public class BinaryIO extends JFrame implements ActionListener{
    private JTextArea display = new JTextArea();
    private JButton read = new JButton("Read Records From File"),
                    write = new JButton("Generate Random Records");
    private JTextField nameField = new JTextField(10);
    private JLabel prompt = new JLabel("Filename:", JLabel.RIGHT);
    private JPanel commands = new JPanel();

    public BinaryIO() {
        super("BinaryIO Demo");                    // Set window title
        read.addActionListener(this);
        write.addActionListener(this);
        commands.setLayout(new GridLayout(2,2,1,1)); // Control panel
        commands.add(prompt);
        commands.add(nameField);
        commands.add(read);
        commands.add(write);
        display.setLineWrap(true);
        this.getContentPane().setLayout(new BorderLayout () );
        this.getContentPane().add("North", commands);
        this.getContentPane().add( new JScrollPane(display));
        this.getContentPane().add("Center", display);
    } // BinaryIO()

    private void readRecords( String fileName ) {
        try {
            DataInputStream inStream     // Open stream
               = new DataInputStream(new FileInputStream(fileName)); 
            display.setText("Name   Age Pay\n");
            try {
                while (true) {                // Infinite loop
                    String name = inStream.readUTF(); // Read a record
                    int age = inStream.readInt();
                    double pay = inStream.readDouble();
                    display.append(name + "   " + age + "   " + pay + "\n");
                } // while
            } catch (EOFException e) {  // Until EOF exception
            } finally {
                inStream.close();                  // Close the stream
            }
        } catch (FileNotFoundException e) {
            display.setText("IOERROR: File NOT Found: " + fileName + "\n");
        } catch (IOException e) {
            display.setText("IOERROR: " + e.getMessage() + "\n");
        }
    } // readRecords()
\end{lstlisting}
\end{jjjlistingleft}
\jjjprogstop{Part I of the {\tt BinaryIO} class, which illustrates
simple input and output from a binary file.}
{fig-binaryioclass}
\end{figure}


\begin{figure}[tb]
\addtocounter{figure}{-1}
\jjjprogstart
\begin{jjjlisting}[28pc]
\begin{lstlisting}
 private void writeRecords( String fileName )  {
   try {
     DataOutputStream outStream   // Open stream
       = new DataOutputStream(new FileOutputStream(fileName)); 
     for (int k = 0; k < 5; k++) { // Output 5 data records
       String name = "Name" + k;   // of name, age, payrate
       outStream.writeUTF("Name" + k);    
       outStream.writeInt((int)(20 + Math.random() * 25));
       outStream.writeDouble(5.00 + Math.random() * 10);  
     } // for
     outStream.close();             // Close the stream
   } catch (IOException e) {
      display.setText("IOERROR: " + e.getMessage() + "\n");
   }
 } // writeRecords()

 public void actionPerformed(ActionEvent evt) {
     String fileName = nameField.getText();
     if (evt.getSource()  == read)
         readRecords(fileName);
      else
         writeRecords(fileName);
 } // actionPerformed()

 public static void main(String args[]) {
    BinaryIO bio = new BinaryIO();
    bio.setSize(400, 200);
    bio.setVisible(true);
    bio.addWindowListener(new WindowAdapter() { // Quit
       public void windowClosing(WindowEvent e) {
            System.exit(0);
       }
    });
 } // main()
} // BinaryIO
\end{lstlisting}
\end{jjjlisting}
\jjjprogstop{({\it continued}) The {\tt BinaryIO} class, Part II.}
{fig:binaryIOpart2}
\end{figure}

\subsection{Abstracting Data from Files}
\noindent It's important to recognize that the method to read a binary
file must exactly match the order of the write and read statements of
the method that wrote the binary file. For example, if the file contains
records that consist of a {\tt String} followed by an {\tt int}
followed by a {\tt double}, then they must be written by a sequence
consisting of

\begin{jjjlisting}
\begin{lstlisting}
writeUTF();
writeInt():
writeDouble();
\end{lstlisting}
\end{jjjlisting}

\noindent And they must thereafter be read by the sequence of

\begin{jjjlisting}
\begin{lstlisting}
readUTF();
readInt():
readDouble();
\end{lstlisting}
\end{jjjlisting}

\noindent Attempting to do otherwise would make it impossible to interpret
the data in the file.

This point should make it evident why (non-Java) binary files are not portable
%\begin{marginalnote}\it Portability\end{marginalnote}
\marginnote{Portability}
whereas text files are.  With text files, each character consists of 8
bits, and each 8-bit chunk can be interpreted as an ASCII character.
So even though a text file consists of a long sequence of 0's and
1's, we know how to find the boundaries between each character.  That's
why any text editor can read a text file, no matter what program
created it.

On the other hand, binary files are also just a long sequence of 0's
and 1's, but we can't tell where one data element begins and another
one ends.  For example, the  64-bit sequence

\begin{jjjlisting}
\begin{lstlisting}
    010100110011001001010100110011000
    010100110011001011010100110011000
\end{lstlisting}
\end{jjjlisting}

\noindent could represent two 32-bit {\tt int}s or two 32-bit {\tt float}s or
one 64-bit \mbox{\tt double} or four 16-bit {\tt char}s or a single
{\tt String} of 8 ASCII characters.  We can't tell what data we have unless
we know exactly how the data were written.

\JavaTIP[false]{DEBUGGING TIP}{Interpreting Binary Data.}{The fact that you can
read the data in a binary file is no guarantee that you are
interpreting it correctly.  To interpret it correctly, you must read it
the same way it was written.}

\JavaTIP[false]{EFFECTIVE DESIGN}{Data Abstraction.}{Binary data are ``raw.''
They have no inherent structure.  It is only the programs that read and
write the data that provide them with structure.  A string of 64 0's
and 1's can be interpreted as two {\tt int}s or one {\tt long} or even
as some kind of object, so an {\tt int}, {\tt long} or an object is an
abstraction imposed upon the data by the program.}

\markright{{\color{cyan}SECTION\,\,\arabic{section}\,\,$\bullet$}\,\, Object Serialization}

\section{Object Serialization: Reading and Writing Objects}
\label{object-serialization-readingand-writing-objects}

\markright{{\color{cyan}SECTION\,\,\arabic{section}\,\,$\bullet$}\,\, Object Serialization}

\noindent The examples in the previous sections showed how to perform I/O operations
on simple binary data or text.  The {\tt java.io} package also
provides methods for reading and writing objects, a process known as
{\bf object serialization}\index{object serialization}. Objects can be 
converted into a sequence of bytes, or {\it serialized}, by  using the
{\tt ObjectOutputStream} class, and they can be {\it deserialized}, or
converted from bytes into a structured object, by 
using the {\tt ObjectInputStream} class
(Fig.~11.26). Despite the complexity of the
serialization/deserialization processes, the methods in these classes
make the task just as easy as reading and writing primitive data.
%\begin{figure}
%\begin{graphic}
\marginfig{chptr11/objiouml.eps}{Classes used for performing I/O on objects.}
{fig-objiouml}
%\end{graphic}
%\end{figure}


To illustrate object serialization, let's begin by defining a
{\tt Student} class (Fig.~\ref{fig-student}).  In order to serialize an
%\begin{marginalnote}\it Object serialization\end{marginalnote}
object, it must be a member of a class that implements the
{\tt Serializable} interface.  The {\tt Serializable} interface is a {\it
marker interface}, an interface that doesn't define any methods or
constants but just serves to designate whether an object can be
serialized or not.

\begin{figure}[tb]
\jjjprogstart
\begin{jjjlisting}[31pc]
\begin{lstlisting}
import java.io.*;

public class Student implements Serializable {
  private String name;
  private int year;
  private double gpa;

  public Student() {}

  public Student (String nameIn, int yr, double gpaIn) {
    name = nameIn;
    year = yr;
    gpa = gpaIn;
  }

  public void writeToFile(FileOutputStream outStream) 
                                     throws IOException{
    ObjectOutputStream ooStream = new ObjectOutputStream(outStream);
    ooStream.writeObject(this);
    ooStream.flush();
  } // writeToFile()

  public void readFromFile(FileInputStream inStream)  
           throws IOException, ClassNotFoundException {
    ObjectInputStream oiStream = new ObjectInputStream(inStream);
    Student s = (Student)oiStream.readObject();
    this.name = s.name;
    this.year = s.year;
    this.gpa = s.gpa;
  } // readFromFile()

  public String toString() {
    return name + "\t" + year + "\t" + gpa;
  }
} // Student
\end{lstlisting}
\end{jjjlisting}
\jjjprogstop{The {\it serializable} {\tt Student} class.}
{fig-student}
\end{figure}

The {\tt Student} class contains its own I/O methods,
{\tt readFromFile()} and {\tt writeToFile()}. This is an appropriate
object-oriented design.  The {\tt Student} class encapsulates
all the relevant information needed to read and write
its data.

%%
\JavaTIP{EFFECTIVE DESIGN}{I/O Design.}{If an object is going to be input
and output to and from files, it should define its own I/O methods.
An object contains all the relevant information needed to perform I/O
correctly.}

\noindent Note the definition of the {\tt writeToFile()} method, which performs
the output task.  This method's {\tt FileOutputStream} parameter is
used to create an {\tt ObjectOutputStream}, whose {\tt writeObject()}
\marginnote{\vspace{24pt}Object serialization}
method writes the object into the file.   To output a
{\tt Student} object, we merely invoke the {\tt write\-Object()} method.  This
method writes out the current values of all the object's public and
private fields.  In this case, the method would write a {\tt String} for
the object's {\tt name}, an {\tt int} for the object's {\tt year}, and
a {\tt double} for the object's {\tt gpa}.

Although our example doesn't require it, the {\tt writeObject()}
method can also handle fields that refer to other objects.  For
example, suppose our {\tt Student} object provided a field for
{\tt courses} that contained a reference to an array of objects, each of
which described a course the student has taken.  In that case, the
{\tt writeObject()} method would serialize the array and all its
objects (assuming they are serializable). Thus, when a complex
object is serialized, the result would be a complex structure that
contains all the data linked to that root object.

Object deserialization, as shown in the {\tt readFromFile()} method,
%\begin{marginalnote}\it Object deserialization\end{marginalnote}
\marginnote{\vspace{-12pt}Object deserialization}
is simply the reverse of the serialization process.   The
{\tt readObject()} method reads one serialized object from the
{\tt ObjectInputStream}. Its result type is {\tt Object}, so it is
necessary to cast the result into the proper type.   In our example we
use a local {\tt Student} variable to store the object as it is
input.  We then copy each field of the local object to {\tt this}
object.

Note that the {\tt readFromFile()} method throws both the
{\tt IOException} and {\tt ClassNotFoundException}.  An {\tt IOException}
will be generated if the file you are attempting to read does not
contain serialized objects of the correct type.  Objects that can be
input by {\tt readObject()} are those that were output by
{\tt writeObject()}. Thus, just as in the case of binary I/O, it is best to
design an object's input and output routines together so that they are
compatible.  The {\tt ClassNotFoundException} will be thrown if the
{\tt Student} class cannot be found.  This is needed to determine how
to deserialize the object.

\JavaTIP{PROGRAMMING TIP}{Object Serialization.}{Java's serialization
classes, {\tt ObjectOutputStream} and {\tt ObjectInputStream}, should
be used whenever an object needs to be input or output from a stream.}

\subsection{The {\tt ObjectIO} Class}
\noindent Given the {\tt Student} class, let's now write a user interface that
can read and write {\tt Student} objects.   We can use the same
interface we used in the {\tt Binary\-IO} program.  The only things we
need to change are the {\tt write\-Records()} and {\tt readRecords()}
methods.  Everything else about this program will be exactly the same
as in {\tt BinaryIO}.

Figure~\ref{fig-objectioclass} provides the full implementation of the
{\tt ObjectIO} class. Note that the {\tt writeRecords()} method will
still write five random records to the data file.  The difference in
this case is that we will call the {\tt Student.write\-To\-File()}
method to take care of the actual output operations.  The revised
algorithm will create a new {\tt Student} object, using randomly
generated data for its name, year, and GPA and then invoke its {\tt
writeToFile()} to output its data.  Note how a {\tt FileOutputStream}
is created and passed to the {\tt Student.\-writeToFile()} method.

\begin{figure}[h!]
\jjjprogstart
\begin{jjjlisting}[29pc]
\begin{lstlisting}
import javax.swing.*;         // Swing components
import java.awt.*;
import java.io.*;
import java.awt.event.*;

public class ObjectIO extends JFrame implements ActionListener{
  private JTextArea display = new JTextArea();
  private JButton read = new JButton("Read From File"),
                  write = new JButton("Write to File");
  private JTextField nameField = new JTextField(10);
  private JLabel prompt = new JLabel("Filename:",JLabel.RIGHT);
  private JPanel commands = new JPanel();

  public ObjectIO () {
    super("ObjectIO Demo");                // Set window title
    read.addActionListener(this);
    write.addActionListener(this);
    commands.setLayout(new GridLayout(2,2,1,1)); 
    commands.add(prompt);                 // Control panel
    commands.add(nameField);
    commands.add(read);
    commands.add(write);
    display.setLineWrap(true);
    this.getContentPane().setLayout(new BorderLayout () );
    this.getContentPane().add("North",commands);
    this.getContentPane().add( new JScrollPane(display));
    this.getContentPane().add("Center", display);
  } // ObjectIO

  public void actionPerformed(ActionEvent evt) {
    String fileName = nameField.getText();
    if (evt.getSource() == read)
        readRecords(fileName);
    else
        writeRecords(fileName);
  } // actionPerformed()
\end{lstlisting}
\end{jjjlisting}
\jjjprogstop{Part I of the {\tt ObjectIO} class, which provides 
an interface to reading and writing files of {\tt Student}s.}
{fig-objectioclass}
\end{figure}


\begin{figure}[p]
\addtocounter{figure}{-1}
\jjjprogstart
\begin{jjjlistingleft}[37pc]{-11pc}
\begin{lstlisting}
   private void readRecords(String fileName) {
     try {
       FileInputStream inStream = new FileInputStream(fileName);  // Open a stream
       display.setText("Name\tYear\tGPA\n");
       try {
         while (true) {                   // Infinite loop
           Student student = new Student();    // Create a student instance
           student.readFromFile(inStream);     //  and have it read an object
           display.append(student.toString() +  "\n"); //  and display it
         }
       } catch (IOException e) {     // Until IOException
       }
       inStream.close();                           // Close the stream
     } catch (FileNotFoundException e) {
         display.append("IOERROR: File NOT Found: " + fileName + "\n");
     } catch (IOException e) {
         display.append("IOERROR: " + e.getMessage() + "\n");
     } catch (ClassNotFoundException e) {
         display.append("ERROR: Class NOT found " + e.getMessage() + "\n");
     }
  } // readRecords()

  private void writeRecords(String fileName) {
    try {
      FileOutputStream outStream = new FileOutputStream( fileName );// Open stream
      for (int k = 0; k < 5 ; k++) {               // Generate 5 random objects
        String name = "name" + k;                     // Name
        int year = (int)(2000 + Math.random() * 4);   // Class year
        double gpa = Math.random() * 12;              // GPA
        Student student = new Student(name, year, gpa); // Create the object
        display.append("Output: "+ student.toString() +"\n"); // and display it
        student.writeToFile(outStream) ;    //  and tell it to write data
      } //for
      outStream.close();
    } catch (IOException e) {
       display.append("IOERROR: " + e.getMessage() + "\n");
    }
  } // writeRecords()

  public static void main(String args[]) {
    ObjectIO io = new ObjectIO();
    io.setSize( 400,200);
    io.setVisible(true);
    io.addWindowListener(new WindowAdapter() {  
        public void windowClosing(WindowEvent e) {
            System.exit(0);  // Quit the application
            }
    });
  } // main()
} // ObjectIO
\end{lstlisting}
\end{jjjlistingleft}
\jjjprogstop{({\it continued}) The {\tt ObjectIO} class, Part II.}
{fig:objectIOpart2cont}
\end{figure}


The {\tt readRecords()} method (Fig.~\ref{fig-objectioclass}, Part
II) will read data from a file containing serialized {\tt Student}
objects.  To do so, it first creates a \mbox{\tt Student} object and
then invokes its {\tt readFromFile()} method, passing it a {\tt
FileInputStream}. Note how the {\tt FileInputStream} is created and,
unlike in {\tt BinaryIO}, the inner try block is exited by an
\mbox{\tt IOException} rather than an {\tt EOFException}.

\pagebreak
\secEXRHone{Self-Study Exercise}
\label{self-study-exercise}
\begin{SSTUDY}

\item  Given the following definition, would a binary file consisting of
several {\tt SomeObject}s be readable by either the {\tt BinaryIO} or
the {\tt ObjectIO} programs?   Explain.

\begin{jjjlisting}
\begin{lstlisting}
public class SomeObject {
    private String str;
    private short n1;
    private short n2;
    private long  n3;
}
\end{lstlisting}
\end{jjjlisting}
\end{SSTUDY}

\markright{{\color{cyan}SECTION\,\,\arabic{section}\,\,$\bullet$}\,\, Java Library : {\tt JFileChooser}}

\section{From the Java Library \\ {\tt javax.swing.JFileChooser}}
\label{fromthe-java-library}

\markright{{\color{cyan}SECTION\,\,\arabic{section}\,\,$\bullet$}\,\, Java Library : {\tt JFileChooser}}

\WWWjava

\tBOXseven{The} {\tt javax.swing.JFileChooser} class is useful for dealing
with files and directories in a GUI environment.  You are probably
already familiar with {\tt JFileChooser}s, although you may not have
known them by that name.   A {\tt JFileChooser} provides a dialog box
that enables the user to select a file and a directory when opening or
saving a file.  Figure~\ref{fig-opendialog} shows an example.

A {\tt JFileChooser} is designed primarily to be used in conjunction
with menu-based programs.  The {\tt JFileChooser} class
(Fig.~11.29) contains methods that support the {\it Open
File} and {\it Save As} options which often appear in GUI applications 
either in a menu or attached to buttons.  In this section we 
\marginfig{chptr11/chooseruml.eps}{The {\tt javax.swing.J\-File\-Chooser} class.}
{fig-chooseruml}
provide the basics for using a {\tt JFileChooser}.  Options for opening
a file or saving a file can be added to the kind of GUI applications that
we encountered earlier in the text by using buttons. 
In  Chapter~13, we will discuss the use of {\tt JMenu}s which will
provide a more natural means of using the {\tt JFileChooser} dialogs.


A {\tt JFileChooser} is not itself the dialog window, but rather the
object that manages the dialog.  After creating a {\tt JFileChooser}
instance, its {\tt showOpenDialog()} or {\tt showSaveDialog()} methods
are used to open a dialog window.  Note that these methods require a
{\tt Component} parameter, usually a {\tt JFrame} or a
{\tt JApplet}. Thus, {\tt JFileChooser}s can be used only in GUI
applications and applets.

To illustrate how to use a {\tt JFileChooser}, let's consider the case
where the user has selected an {\it Open File} menu item or clicked a 
button labeled {\it Open File}.  In this case, executing the 
following code  will cause an ``Open File'' dialog to appear:

\begin{jjjlisting}
\begin{lstlisting}
JFileChooser chooser = new JFileChooser();
int result = chooser.showOpenDialog(this);

if (result == JFileChooser.APPROVE_OPTION) {
    File file = chooser.getSelectedFile();
    // Insert code here to read data from file
    String fileName = file.getName();
    display.setText("You selected " + fileName);
} else
    display.setText("You cancelled the file dialog");
\end{lstlisting}
\end{jjjlisting}


\noindent We begin by creating a {\tt JFileChooser} and then
\marginnote{Opening a file}
telling it to {\tt showOpen\-Dia\-log()}. If we were saving a file rather
than opening one, we would tell it to {\tt showSaveDialog()}.  In
either case, a dialog window will pop up on the screen.  The dialog
assists the user in navigating through the file system and selecting a
file (Fig.~\ref{fig-opendialog}).

%\begin{figure}
\begin{figure}[h!]
\figaleft{chptr11/opendialog.eps}{The {\it Open File} dialog window.
} {fig-opendialog}
\end{figure}
%\end{figure}

The dialog contains two buttons, one labeled Open and
%\begin{marginalnote}\it Opening a file\end{marginalnote}
the other labeled Cancel.  If the user selects a file, that choice
will correspond to {\tt APPROVE\_OPTION}. If the user cancels the
dialog, that will correspond to {\tt CANCEL\_OPTION}. After
opening a dialog, the code should check which option resulted.  In
this case, if the user opened a file, the code gets a reference to the
file and then simply uses that to print the file's path name to a
text area named {\tt display}.  In an  actual application, code would be
inserted at that spot which uses the file reference to read data from
the file.

\section{Using File Data in Programs} 

This chapter's examples have provided explicit details for several
ways of writing data to files and reading data from files.  In actual
programs, deciding if and how files might be useful in the program are
worked out as part of the design process.  Choosing between text
files, binary files, and reading and writing objects is part of this
process.

To illustrate how we can apply what we've learned about file I/O,
let's modify the {\tt WordGuess} class (which is listed in
Fig.~8.27) so that it reads a list of possible words 
for players to guess from a file.  The Chapter~8 
\marginnote{modifying {\tt WordGuess}} 
version of the class contains a method, {\tt getSecretWord()}, which
uses a {\tt switch} statement to randomly choose and return a word
from a fixed list of ten words.  Reading the words from a text file
would allow a user to modify the list of possible words by adding
or changing words without needing to recompile the
program. 

\noindent Let's modify the {\tt WordGuess} class in three ways:

\begin{enumerate}

\item adding two new instance variables, an array of type {\tt String}
and a variable to store the size of the array;

\item adding code at the beginning of the class's constructor to read
words from the file and store them in the array;

\item rewrite the {\tt getSecretWord()} method so that it randomly
chooses a word from the array.
\end{enumerate}

\noindent Let us first choose descriptive names for
\marginnote{New instance variables}
declaring the two new instance variables:

\begin{jjjlisting}
\begin{lstlisting}
    private String[] wordArray;
    private int arraySize;
\end{lstlisting}
\end{jjjlisting}

\noindent Note that it will be useful to store the number of
words in the file in its first line so that this information can be 
used to allocate memory for the array.  For example, let us assume
the text file will be named {\tt secretwords.txt}, it will be located 
in the same directory as the {\tt WordGuess} class, it will have
the number of words in the file as its first line, and it will have
\marginnote{Format for the text file}
a single word per line after that.  Thus, a  small file 
might look like:
\begin{jjjlisting}
\begin{lstlisting}
    3
    STREAMS
    CONDUIT
    DIALOGS
\end{lstlisting}
\end{jjjlisting}

We can use the body of the {\tt readTextFile()} method of the {\tt TextIO} 
class as a model for the Java code that needs to be added to the {\tt WordGuess} 
\marginnote{Code to add to constructor}
constructor. Pseudocode for this code will look like:
\begin{jjjlisting}
\begin{lstlisting}
    Use file name to open a  BufferedReader stream
    Read first line and convert to an integer
    Store the integer as the size of the array 
    Allocate memory for the array
    Read second line of file
    While a word is read
       Store the word in the next array element
       Read next line of file
    Close the BufferedReader stream   
\end{lstlisting}
\end{jjjlisting}

\noindent When this pseudocode is translated into Java and inserted into a
{\tt try-catch} statement we get the code fragment in Figure~\ref{fig-wordguessfile}.
\begin{figure}[h!]
\jjjprogstart
\begin{jjjlisting}
\begin{lstlisting}
    try{
        FileReader fr = new FileReader("secretwords.txt");
        BufferedReader inStream
            = new BufferedReader(fr);
        String line = inStream.readLine();
        arraySize = Integer.parseInt(line);
        wordArray = new String[arraySize];
        line = inStream.readLine();
        int k = 0;
        while((line != null) && (k < arraySize)){
            wordArray[k] = line;
            line = inStream.readLine();
            k++;
        }//while
        inStream.close();
    } catch(FileNotFoundException e){
        e.printStackTrace();
    } catch(IOException e){
        e.printStackTrace();
    } //catch
\end{lstlisting}
\end{jjjlisting}
\jjjprogstop{Code added at beginning of the {\tt WordGuess} constructor.}
{fig-wordguessfile}
\end{figure}

The new {\tt getSecretWord()} method merely needs to generate a
\marginnote{New code for {\tt getSecretWord}}
random array index and return the corresponding array element:
\begin{jjjlisting}
\begin{lstlisting}
    private String getSecretWord(){
        int num = (int)(Math.random()*arraySize);
        return wordArray[num];
    } //getSecretWord()
\end{lstlisting}
\end{jjjlisting}

\noindent The only other modification that is needed for to complete new
{\tt WordGuess} class is to add an initial {\tt import java.io.*;} statement
so that the file IO classes can be accessed. 

The earlier examples in this chapter can be used as models
to enhance numerous practical applications.  GUI applications that involve
a user's choice to load data from a file or save data in a file should make 
use of the {\tt JFileChooser} dialogs to initiate the file operations. 

\section*{Special Topic: Databases and Personal Privacy}

\noindent During a typical day we all come in contact with lots of electronic
databases that store information about us.  If you use a supermarket
discount card, every purchase you make is logged against your name in
the supermarket's database.  When you use your bank card at the ATM
machine, your financial transaction is logged against your account.
When you charge gasoline or buy dinner, those transactions are logged
against your credit card account.  If you visit the doctor or dentist,
a detailed record of your visit is transmitted to your medical
insurance company's database.   If you receive a college loan, detailed
financial information about you is entered into several different
credit service bureaus.   And so on. 

Should we be worried about how this information is used? Many privacy
advocates say yes.  With the computerization of medical records, phone
records, financial transactions, driving records, and many other
records, there is an enormous amount of personal information held in
databases.  At the same time, there are pressures from a number of
sources for access to this information.  Law-enforcement agencies want
to use this information to monitor individuals.   Corporations want to
use it to help them market their products.   Political organizations
want to use it to help them market their candidates.

Recently there has been pressure from government and industry in the
United States to use the social security number (SSN) as a unique
identifier.   Such an identifier would make it easy to match personal
information across different databases.  Right now, the only thing your
bank records, medical records, and supermarket records may have in
common is your name, which is not a unique identifier.   If all online
databases were based on your SSN, it would be much simpler to create a
complete profile.  While this might improve services and reduce
fraud and crime, it might also pose a significant threat to our
privacy.

The development of online databases serve many useful
purposes.  They help fight crime and reduce the cost of doing
business.  They help improve government and commercial services on which we
have come to depend.  On the other hand, databases can be and
have been misused.  They can be used by unauthorized individuals or
agencies or in unauthorized ways.  When they contain inaccurate
information, they can cause personal inconvenience or even harm.\\
\hspace*{1pc}There are a number of organizations that have sprung up to address
the privacy issues raised by online databases.  If you're interested
in learning more about this issue, a good place to start would be the
Web site maintained by the Electronic Privacy Information Center
(EPIC) at
\begin{jjjlisting}
\begin{lstlisting}[commentstyle=\color{black}]
http://www.epic.org/
\end{lstlisting}
\end{jjjlisting}
\rule[0pt]{26pc}{2pt}\\



\secSMH{Chapter Summary}
\label{chapter-summary}

\secKTH{Technical Terms}
\label{technical-terms}

%\parindent=0pt
\begin{KT}
absolute path name

binary file

%%%REW buffer

buffering

%%%REW command-line argument

database

data hierarchy

directory

end-of-file \nobreak{character}

field

file

filtering

%%%REW input stream   %%%was introduced in chapter 4

input  %%new

object serialization

output %%new

%%%REW output stream

path

%%REW platform independence

record

relative path name

%%%REW stream

%%%REW text file

Unicode Text Format (UTF)

\end{KT}

%%%REW\clearpage

\secSMHtwo{Summary of Important Points}
\label{summaryof-important-points}
\begin{SMBLlarge}
\item  A {\it file} is a collection of data stored on a disk.
A {\it stream} is an object that delivers data to and from other
objects.  

\item An {\tt InputStream} is a stream that delivers data to a
program from an external source---such as the keyboard or a file.
{\tt System.in} is an example of an {\tt InputStream}. An
{\tt OutputStream} is a stream that delivers data from a program to an
external destination---such as the screen or a file.   {\tt System.out} is
an example of an {\tt OutputStream}.

\item  Data can be viewed as a hierarchy.  From highest to
lowest, a {\it database} is a collection of files.  A {\it file} is a
collection of records.  A {\it record} is a collection of fields.   A
{\it field} is a collection of bytes.  A {\it byte} is a collection of
8 bits.   A {\it bit} is one binary digit, either 0 or 1.

\item  A {\it binary file} is a sequence of 0s and 1s that
is interpreted as a sequence of bytes.  A {\it text file} is a sequence
of 0s and 1s that is interpreted as a sequence of characters.  A text
file can be read by any text editor, but a binary file cannot.
{\tt InputStream} and {\tt OutputStream} are abstract classes that
serve as the root classes for reading and writing binary data.   {\tt Reader}
and {\tt Writer} serve as root classes for text I/O.

\item  {\it Buffering} is a technique in which a {\it buffer}, a temporary
region of memory, is used to store data while they are being input or
output.

\item  A text file contains a sequence of characters divided into
%%%REWlines by the \verb@\n@ character and ending with a special
lines by the \verb|\n| character and ending with a special
{\it end-of-file} character.

\item  The standard algorithm for performing I/O on a file consists
of three steps: (1) Open a stream to the file, (2) perform the
I/O, and (3) close the stream.


\item  Designing effective I/O routines answers
two questions: (1) What streams should I use to perform
the I/O? (2) What methods should I use to do the reading
or writing?

\item  To prevent damage to files when a program terminates
abnormally, streams should be closed when they are no longer
needed.

\item  Most I/O operations generate an {\tt IOException} that
should be caught in the I/O methods.

\item  Text input uses a different technique to determine
when the end of a file has been reached.  Text input methods return
{\tt null} or {\tt -1} when they attempt to read the special
end-of-file character.   Binary files don't contain an end-of-file
character, so binary read methods throw an {\tt EOFException} when they
attempt to read past the end of the file.

\item  The {\tt java.io.File} class provides methods that
enable a program to interact with a file system.  Its methods can be
used to check a file's attributes, including its name, directory,
and path.

\item  Streams can be joined together to perform
I/O.~For example, a {\tt DataOutputStream} and a {\tt FileOutputStream}
can be joined to perform output to a binary file.

\item  A binary file is ``raw'' in the sense that it contains
no markers within it that allow you to tell where one data element
ends and another begins.  The interpretation of binary data
is up to the program that reads or writes the file.

\item  Object serialization is the process of writing an
object to an output stream.  Object deserialization is the reverse
process of reading a serialized object from an input stream.   These
processes use the {\tt java.io.\-ObjectOutputStream} and
{\tt java.io.ObjectInputStream} classes.

\item  The {\tt JFileChooser} class provides a dialog box that
enables the user to select a file and directory when
opening or saving a file.
\end{SMBLlarge}

%\section*{}{Answers to Self-Study Exercises}
\secANSHleft
%\addcontentsline{toc}{section}{\S~~~ Answers to Self-Study Exercises}
\begin{ANS}
\item 
Because {\tt FileWriter} contains a constructor that takes a file name
argument, {\tt FileWriter(String)}, it can be used with
{\tt Print\-Writer} to perform output to a text file:

\begin{jjjlisting}
\begin{lstlisting}
PrintWriter outStream =     //  Create output stream
    new PrintWriter(new FileWriter(fileName)); // Open file
outStream.print (display.getText());// Display text
outStream.close();                  // Close output stream
\end{lstlisting}
\end{jjjlisting}

\item An empty file doesn't affect this loop.  If the file is empty,
it will print a {\tt null} line.  The test {\tt line != null}, should
come right after the {\tt readLine()}, as it does in the {\tt while}
loop.


\item  This loop won't work on an empty text file.  In that case, {\tt ch}
would be set to $-1$, and the attempt to cast it into a {\tt char}
would cause an error.

\item  \mbox{ }
\begin{jjjlisting}
\begin{lstlisting}
public void getFileAttributes(String fileName) {
    File file = new File (fileName);
    System.out.println(filename);
    System.out.println("absolute path:" 
           + file.getAbsolutePath());
    System.out.println("length:" + file.length());
    if (file.isDirectory())
        System.out.println("Directory");
    else
        System.out.println("Not a Directory");
} // getFileAttributes()
\end{lstlisting}
\end{jjjlisting}

\item  The {\tt inStream.close()} statement is misplaced in {\tt read\-Integers()}.
By placing it inside the same {\tt try/catch} block as the read loop,
it will get skipped and the stream will not be closed.  The {\tt EOFException}
should be caught in a separate {\tt try/catch} block from other
exceptions, and it should just cause the read loop to exit.

\item  Yes, a binary file containing several {\tt SomeObjects} would be
``readable'' by the {\tt BinaryIO} program because the program will
read a {\tt String} followed by 64 bytes.  However, {\tt BinaryIO}
would misinterpret the data, because it will assume that {\tt n1} and
{\tt n2} together comprise a single {\tt int}, and {\tt n3} (64 bits)
will be interpreted as a {\tt double}. A file of {\tt SomeObject}s
could not be read by the {\tt ObjectIO} program, because
{\tt SomeObject} does not implement the {\tt Serializable} interface.

\end{ANS}


%\section*{}{Exercises}
%%\secEXRHtwoleft{Exercises}
\label{exercises}
\secEXRHtwo{Exercises}
\label{exercises}
\marginnote{\raggedright\vspace{9pt}{\bf Note:} For programming exercises, {\bf first} draw 
a UML class diagram describing all classes and
their inheritance relationships and/or associations.}
%\addcontentsline{toc}{section}{\S~~~ Exercises}
%%%REW\begin{EXRtwolarge}
\begin{EXRtwo}

\item  Explain the difference between each of the following pairs of terms:

\begin{EXRtwoLL}
\begin{multicols}{2}
\item  {\tt System.in} and {\tt System.out}.
\item  {\it File} and {\it directory}.
\item  {\it Buffering} and {\it filtering}.
\item  {\it Absolute} and {\it relative path name}.
\item  {\it Input stream} and {\it output stream}.
\item  {\it File} and {\it database}.
\item  {\it Record} and {\it field}.
\item  {\it Binary file} and {\it text file}.
\item  {\it Directory} and {\it database}.
\end{multicols}
\end{EXRtwoLL}

\item  Fill in the blanks.
\begin{EXRtwoLL}\baselineskip=15pt
\item  Unlike text files, binary files do not have a
special \rule{60pt}{0.5pt} character.
\item  In Java, the {\tt String} array parameter in the
{\tt main()} method is used for \rule{60pt}{0.5pt}\,.
\item  \rule{60pt}{0.5pt} files are portable and platform
independent.
\item  A  \rule{70pt}{0.5pt}  file  created on one
computer can't be read by another \mbox{computer.}
\end{EXRtwoLL}\baselineskip=11pt

\item  Arrange the following kinds of data into their
correct hierarchical relationships: {\tt bit},
{\tt field}, {\tt byte}, {\tt record}, {\tt database},
{\tt file}, {\tt String}, {\tt char}.

\item  In what different ways can the following string
of 32 bits be interpreted?
\begin{jjjlisting}
\begin{lstlisting}
00010101111000110100000110011110
\end{lstlisting}
\end{jjjlisting}

\item  When reading a binary file, why is it necessary
to use an infinite loop that's exited only when an
exception occurs?

\item  Is it possible to have a text file with 10 characters
and 0 lines? Explain.

\item  In reading a file, why is it necessary to attempt
to read from the file before entering the read loop?

\item  When designing binary I/O, why is it especially
important to design the input and output routines
together?

\item  What's the difference between ASCII code and
UTF code?

\item  Could the following string of bits possibly be
a Java object?  Explain.

\begin{jjjlisting}
\begin{lstlisting}
00010111000111101010101010000111001000100
11010010010101010010101001000001000000111
\end{lstlisting}
\end{jjjlisting}

\item  Write a method that could be added to the {\tt TextIO}
program to read a text file and print all lines containing a
certain word.  This should be a {\tt void} method that takes two
parameters: The name of the file and the word to search for.  Lines not
containing the word should not be printed.

\item  Write a program that reads a text file and reports the
number of characters and lines contained in the file.

\item  Modify the program in the previous exercise so that it
also counts the number of words in the file.   ({\it Hint}: The
{\tt StringTokenizer} class might be useful for this task.)

\item  Modify the {\tt ObjectIO} program so that it allows the
user to designate a file and then input {\tt Student} data with
the help of a GUI.  As the user inputs data, each record should
be written to the file.

\item  Write a program that will read a file of {\tt int}s into
memory, sort them in ascending order, and output the sorted
data to a second file.

\item  Write a program that will read two files of {\tt int}s,
which are already sorted into ascending order, and merge their
data.  For example, if one file contains 1, 3, 5, 7, 9, and the
other contains 2, 4, 6, 8, 10, then the merged file should
contain 1, 2, 3, 4, 5, 6, 7, 8, 9, 10.

\item Suppose you have a file of data for a geological survey. Each
record consists of a longitude, a latitude, and an amount of rainfall,
all represented by {\tt double}s.  Write a method to read this file's
data and print them on the screen, one record per line.  The method
should be {\tt void} and it should take the name of the file as its
only parameter.

\item  Suppose you have the same data as in the previous exercise.
Write a method that will generate 1,000 records of random data and
write them to a file.  The method should be {\tt void} and should take the
file's name as its parameter.  Assume that longitudes have values in
the range $+/-$ 0 to 180 degrees, latitudes have values in the range $+/-$
0 to 90 degrees, and rainfalls have values in the range 0 to 20 inches.

\item  Design and write a file copy program that will work for either
text files or binary files.  The program should prompt the user for the
names of each file and copy the data from the source file into the
destination file.  It should not overwrite an existing file,
however.  ({\it Hint}: Read and write the file as a file of {\tt byte}.)

\item  Design a class, similar to {\tt Student}, to represent an
{\tt Address}. It should consist of street, city, state, and zip code
and should contain its own {\tt readFromFile()} and
{\tt writeToFile()} methods.

\item  Using the class designed in the previous exercise, modify
the {\tt Student} class so that it contains an {\tt Address}
field.  Modify the {\tt ObjectIO} program to accommodate this
new definition of {\tt Student} and test your program.

\item  Write a program called {\tt Directory}, which provides
a listing of any directory contained in the current directory.  
This program should prompt the user for the name of the directory.
It should then print a listing of that directory.  The
listing should contain the following information: The full path name
of the directory, and then include the file name, length, and
last modified date, and a read/write code for each file.  The read/write code should
be an {\it r} if the file is readable and a {\it w} if the file is writeable,
in that order.  Use a ``-'' to indicate not readable or not
writeable.  For example, a file that is readable but not writable will
have the code {\it r}-.  Here's an example listing:

\begin{jjjlisting}
\begin{lstlisting}
Listing for directory: myfiles
  name          length modified   code
  index.html    548    129098     rw
  index.gif     78     129190     rw
  me.html       682    128001     r-
  private.txt   1001   129000     --
\end{lstlisting}
\end{jjjlisting}

\noindent  Note that the {\tt File.lastModified()} returns a {\tt long},
which gives the modification time of the file.  This number can't
easily be converted into a date, so just report its value.

\item  {\bf Challenge:} Modify the {\tt OneRowNimGUI} class 
that is listed in Chapter~4's Figure~ 4-25 so that
the user can save the position of the game to a file or open and 
read a game position from a file.  You should add two new {\tt JButton}s
to the GUI interface. Use the object serialization example as a model
for your input and output streams.

\item  {\bf Challenge:} In Unix systems, there's a program named
{\tt grep} that can list the lines in a text file
containing a certain string.  Write a Java version of this program
that prompts the user for the name of the file and the string to search for.

\item  {\bf Challenge:} Write a program in Java named {\tt Copy} 
to copy one file into another.  The program should prompt the user
for two file names, {\tt filename1} and {\tt filename2}. 
Both {\tt filename1} and {\tt filename2} must exist or
the program should throw a {\tt FileNotFoundException}.  Although
{\tt filename1} must be the name of a file (not a directory),
{\tt filename2} may be either a file or a directory.  If {\tt filename2} is
a file, then the program should copy {\tt filename1} to
{\tt filename2}.  If {\tt filename2} is a directory, then the program
should simply copy {\tt filename1} into {\tt filename2}. That is, it
should create a new file with the name {\tt filename1} inside the
{\tt filename2} directory, copy the old file to the new file, and then
delete the old file.
%%%REW\end{EXRtwolarge}
\end{EXRtwo}
%
