%%%  Chapter 9: Arrays and Array Processing
%%%  3rd Edition

\setcounter{SSTUDYcount}{1}
\setcounter{chapter}{8}
\chapter{Arrays and Array Processing}
\label{chapter-arrays}


\CObegin
\secCOBH{Objectives}
\noindent After studying this chapter, you will

\begin{COBL}
\item  Know how to use array data structures.
\item  Be able to solve problems that require collections of data.
\item  Know how to sort an array of data.
\item  Be familiar with sequential and binary search algorithms.
\item  Gain a better understanding of inheritance and polymorphism.
\end{COBL}

\secCOLH{Outline}
\begin{COL}
\item {Introduction}
\item {One-Dimensional Arrays}
\item {Simple Array Examples}
\item {Example: Counting Frequencies of Letters}
%%RAM\item[] {\color{cyan}Data Compression}
\item {Array Algorithms: Sorting}
\item {Array Algorithms: Searching}
\item {Two-Dimensional Arrays}
\item {Multidimensional Arrays (Optional)}
\item {Object-Oriented Design: Polymorphic Sorting (Optional)}
\item {From the Java Library: {\tt java.util.Vector}}
\item {Case Study: An N-Player Computer Game}
\item {A GUI-Based Game (Optional Graphics)}
\par\small\item[] {Chapter Summary}
\par\small\item[] {Solutions to Self-Study Exercises}
\par\small\item[] {Exercises}
\end{COL}
\COend

\section{Introduction}
\label{introduction}

\noindent In this chapter we will learn about arrays.  An {\bf array} 
is a named collection of contiguous storage locations---storage
locations that are next to each other---that contain data items of the
same type.

Arrays offer a more streamlined way to store data than using
individual data items for each variable.  Arrays also allow you to
work with their data more efficiently than with data stored in
individual variables.

Let's see why.  Suppose you want to create a GUI that has 26 buttons
on it, one for each letter of the alphabet.  Given our present
knowledge of Java, our only alternative would be to declare a separate
{\tt JButton} variable for each letter of the alphabet:

\begin{jjjlisting}
\begin{lstlisting}
 JButton button1;
 JButton button2;
 ...
 JButton button26;
\end{lstlisting}
\end{jjjlisting}

\noindent Obviously, requiring 26 separate variables for this
problem is tedious and inconvenient. Similarly, to instantiate and
assign a label to each button would require 26 statements:

\begin{jjjlisting}
\begin{lstlisting}
 button1 = new JButton("A");
 button2 = new JButton("B");
 ...
 button26 = new JButton("Z");
\end{lstlisting}
\end{jjjlisting}

\noindent This approach is also tedious.  What we need is some way to
use a loop to process each button, using a loop counter, {\it k}, to
refer to the {\it k}th button on each iteration of the loop.  An array
lets us do that.  For example, the following code will declare an
array for storing 26 {\tt JButton}s and then instantiate and label
each button:

\begin{jjjlisting}\begin{lstlisting}
 JButton letter[]  = new JButton[26];
 for (int k = 0; k < 26; k++)
     letter[k] = new JButton("A" + k);
\end{lstlisting}
\end{jjjlisting}

\noindent You don't yet understand the code in this segment, but you can
see how economical it is. It uses just three lines of code to do what
would have required 50 or 60 lines of code without arrays. 

Our discussion of arrays will show how to store and retrieve data from
one-, two-, and three-dimensional arrays.  We also study sorting and
searching algorithms to process arrays. Finally, we illustrate how
arrays can be used in a variety of applications, including an
animation problem, a sorting class, and a card-playing \mbox{program.}

\section{One-Dimensional Arrays}
\label{one-dimensional-arrays}

\noindent An array is considered a {\em data structure}. A {\bf data
structure} is an organized collection of data. In an array, data are
arranged \marginnote{The array data structure} in a linear or
sequential structure, with one element following another.  When
referencing elements in an array, we refer to the position of the
particular element within the array. For example, if the array is
named {\tt arr}, then the elements are named
\verb|arr[0], arr[1], arr[2], ... arr[n-1]|, where {\it n} gives the
number of elements in the array.  This naming also reflects the fact
that the array's data are contained in storage locations that are next
to each other. In Java, as in C, C++, and some other programming
languages, the first element of an array has index 0.
\marginnote{Zero indexing}
(This is the same convention we used for {\tt String}s.)

\begin{figure}[h!]
\figa{chptr09/8f1.eps}{An array of 15 integers named {\tt arr}.}
{fig-intarray}
\end{figure}
Figure~\ref{fig-intarray} shows an array named {\tt arr} that
contains 15 {\tt int} elements.  
The syntax for referring to elements of an array is

\BOXDT{{\tt {\it arrayname} [ {\it subscript} ] }}

\noindent where {\it arrayname} is the name of the array---any valid
identifier will do---and {\it subscript} is the position of the
element within the array. As Figure~\ref{fig-intarray} shows, the
first element in the array has subscript 0, the second has subscript
1, and so on.


A {\bf subscript} is an integer quantity contained in square brackets
that is used to identify an array element. An subscript must be
either an integer value or an integer
\marginnote{Subscript expressions}
expression.  Using Figure~\ref{fig-intarray} to illustrate an example,
suppose that {\it j} and {\it k} are integer variables equaling 5 and
7, respectively.  Each of the following then would be valid references
to elements of the array {\it arr}:

\begin{jjjlisting}[26.5pc]
\begin{lstlisting}
arr[4]     //Refers to 16
arr[j]     //Is arr[5] which refers to 20
arr[j + k] //Is arr[5+7] which is arr[12] which refers to 45
arr[k % j] //Is arr[7%5] which is arr[2] which refers to -1
\end{lstlisting}
\end{jjjlisting}


\noindent These examples show that when an expression, such as {\tt j
 + k}, is used as a subscript, it is evaluated (to 12 in this case)
before the reference is made.

It is a syntax error to use a noninteger type as an array subscript.
Each of the following expressions would be invalid:

\begin{jjjlisting}[26.5pc]
\begin{lstlisting}
arr[5.0]  // 5.0 is a float and can't be an array subscript
arr["5"]  // "5" is a string not an integer
\end{lstlisting}
\end{jjjlisting}

\noindent For a given array, a valid array subscript must be in the
range \verb|0 ... N-1|, where {\tt N} is the number of elements in the
array or it is considered out-of-bounds. An out-of-bounds subscript
creates a run-time error---that is, an error that occurs when the
program is running---rather than a syntax error, which can be detected
when the program is compiled.  For the array {\tt arr}, each of the
following expressions contain out-of-bounds subscripts:

\begin{jjjlisting}
\begin{lstlisting}
arr[-1]  // Arrays cannot have negative subscripts
arr['5'] // Char '5' promoted to its Unicode value, 53
arr[15]  // The last element of arr has subscript 14
arr[j*k] // Since j*k equals 35
\end{lstlisting}
\end{jjjlisting}

\noindent Each of these references would lead to an {\tt
IndexOutOfBoundsException}.  (Exceptions are covered in detail in
Chapter~10.)

\JavaRule{Array Subscripts.}{Array subscripts must be
integer values in the range {\tt 0...(N-1)}, where {\it N} is the number of
elements in the array.}

\JavaTIP{DEBUGGING TIP}{Array Subscripts.}{In developing array algorithms,
it's important to design test data that show that array subscripts
do not cause run-time errors.}


\subsection{Declaring and Creating Arrays}
\noindent For the most part, arrays in Java are treated as objects.  
Like objects, they are instantiated with the {\tt new} operator 
\marginnote{Are arrays objects?}
and they have instance variables (for example, {\tt length}). Like
variables for objects, array variables are considered {\it reference}
variables.  When arrays are used as parameters, a reference to the
array is passed rather than a copy of the entire array.  The primary
difference between arrays and full-fledged objects is that arrays
aren't defined in terms of an {\tt Array} class.  Thus, arrays don't
fit into Java's {\tt Object} hierarchy.  They don't inherit any
properties from {\tt Object} and they cannot be subclassed.

You can think of an array as a container that contains a number of
variables.  As we've seen, the variables contained in an array object
are not referenced by name but by their relative position in the
array.  The variables are called {\it components}. If an array object
has {\it N} components\index{array component}, then we say that the
{\bf array length}\index{array length} is {\it N}. Each of the
components of the array has the same type, which is called the array's
{\it component\index{array component type} type}.  An {\it empty}
array is one that contains zero variables.

A {\bf one-dimensional} array\index{one-dimensional array} has
%\begin{marginalnote}\it Components and elements\end{marginalnote}
\marginnote{Components and elements}
components that are called the array's {\bf elements\index{array
element}}. Their type is the array's {\bf element\index{array element
type} type}. An array's elements may be of any type, including
primitive and reference types.  This means  you can have arrays of {\tt int,
char, boolean, String, Object, Image, TextField, TwoPlayerGame}, and so on.

When declaring a one-dimensional array, you have to indicate both the
array's element type and its length.  Just as in declaring and creating
other kinds of objects, creating an array object requires that we
create both a name for the array and then the array\index{array
declaration} itself.   The following statements create the array shown
in Figure~\ref{fig-intarray}:

\begin{jjjlisting}
\begin{lstlisting}
int arr[];          // Declare a name for the array
arr = new int[15];  // Create the array itself
\end{lstlisting}
\end{jjjlisting}

\noindent These two steps can be combined into a single statement
as follows:

\begin{jjjlisting}
\begin{lstlisting}
int arr[] = new int[15];
\end{lstlisting}
\end{jjjlisting}

\noindent In this example, the array's element type is {\tt int} and
its {\tt length} is 15, which is fixed and cannot be changed.  This
means that the array contains 15 variables of type {\tt int}, which
will be referred to as {\tt arr[0], arr[1], \dots arr[14]}.
%%It is important to note that once an array
%%has been instantiated, its length is fixed and cannot be changed.

\subsection{Array Allocation}
\noindent Creating the array in Figure~\ref{fig-intarray}  means allocating
\marginnote{Allocating memory}
15 storage locations that can store integers.  Note that one
difference between declaring an array and declaring some other kind of
object is that square brackets ({\tt []}) are used to declare an array
type.  The brackets can be attached either to the array's name or to
its type, as in the following examples:

\begin{jjjlisting}
\begin{lstlisting}
int arr[];     // The brackets may follow the array's name
int[] arr;     // The brackets may follow the array's type
\end{lstlisting}
\end{jjjlisting}

\noindent The following example creates an array of five {\tt String}s
and then uses a for loop to assign the strings {\tt "hello1"}, {\tt
"hello2"}, {\tt "hello3"}, {\tt "hello4"}, and {\tt "hello5"} to the
five array locations:

\begin{jjjlisting}[28pc]
\begin{lstlisting}
String strarr[];                // Declare a name for the array
strarr = new String[5];              // Create the array itself
                                 // Assign strings to the array
for (int k = 0; k < strarr.length; k++)     // For each element
  strarr[k] = new String("hello" + (k + 1)); // Assign a string
\end{lstlisting}
\end{jjjlisting}

\noindent Note that the expression \verb|k < strarr.length| specifies
the loop bound.  Every array has a {\tt length}\index{array length}
instance variable, which refers to the number of elements contained in
the array.  As we mentioned, arrays, like {\tt String}s, are zero
\marginnote{{\tt length} vs.  {\tt length()}}
indexed, so the last element of the array is always given by its {\tt
length-1}.  However, {\tt length} is an instance variable for arrays,
whereas {\tt length()} is an instance method for {\tt String}s.
Therefore, it would be a syntax error in this example to refer to {\tt
strarr.length()}.

\JavaTIP{DEBUGGING TIP}{Array Length.}{A common syntax error involves
forgetting that for arrays {\tt length} is an instance variable,
not an instance method, as it is for {\tt String}s.}

In the example, we first use the {\tt new} operator to create {\tt
strarr}, an array of type {\tt String} of length five.  We then use a
{\tt String} constructor to create the five {\tt String}s that are
stored in the array.  It is important to realize that creating an
array to store five {\tt Object}s (as opposed to five primitive data
elements) does not also create the {\tt Object}s themselves that will
be stored in the array.

When an {\tt array} of objects is created, the array's elements are
\marginnote{Arrays of objects}
references to those objects (Fig.~\ref{fig-arrayobjs}). Their initial
values, like all reference variables, are {\tt null}.  So to create
and initialize the array {\tt strarr}, we need to create {\it six}
objects---the array itself, which will contain five {\tt String}s, and
then the five {\tt String}s that are stored in {\tt strarr}.

%\begin{figure}
\begin{figure}[tb]
%%\figaleft{chptr09/p406f1.eps}{Creating an array of five {\tt String}s involves
\figa{chptr09/p406f1.eps}{Creating an array of five {\tt String}s involves
six objects, because the array itself is a separate object.
In (a), the array variable is declared.  In (b), the array
is instantiated, creating an array of five {\tt null}
references.  In (c), the five {\tt Strings} are created
and assigned to the array.}
{fig-arrayobjs}
\end{figure}
%\end{figure}

One more example will help underscore this point.   The following
statements create four {\it new} {\tt Object}s, an array to store
three {\tt Student}s plus the three {\tt Student}s themselves:

\begin{jjjlisting}
\begin{lstlisting}
Student school[] = new Student[3];   // A 3 Student array
school[0] = new Student("Socrates"); // The first Student
school[1] = new Student("Plato");    // The second Student
school[2] = new Student("Aristotle");// The third Student
\end{lstlisting}
\end{jjjlisting}

\noindent The first statement creates an array named {\tt school}
to store three \mbox{\tt Student}s, and the next three statements create
the individual {\tt Student}s and assign them to the array
(Fig.~9.3).  Thus, creating the array and initializing
its elements require four {\tt new} statements.

%\begin{figure}
%\begin{graphic}
\marginfig{chptr09/p407f1.eps}{An array of {\tt Student}s.}
{fig-studarray}
%\end{graphic}
%\end{figure}
The following sequence of statements would lead to a null pointer
exception because the array's elements have not been instantiated:

\begin{jjjlisting}
\begin{lstlisting}
Student students[] = new Student[3]; // A 3 Student array
System.out.println(students[0].getName());
\end{lstlisting}
\end{jjjlisting}

\noindent In this case, {\tt students[0]} is a null reference, thus
causing the exception.

%%RAM false argument flips icon to other side
\JavaTIP[false]{DEBUGGING TIP}{Array Instantiation.}{Creating a new {\tt array}
does not also create the objects that are stored in the array.  They
must be instantiated separately.   It is a semantic error to refer to
an uninstantiated ({\tt null}) array element.}

\noindent Now that we've assigned the three {\tt Student}s to
the array, we can refer to them by means of subscripted references.  A
reference to the \mbox{\tt Student} named ``Socrates'' is now {\tt
school[0]}, and a reference to the \mbox{\tt Student} named ``Plato''
is {\tt school[1]}. In other words, to refer to the three individual
students we must refer to their locations within {\tt school}. Of
course, we can also use variables, such as loop counters, to refer to
a {\tt Student}'s location within {\tt school}. The following for loop
invokes each {\tt Student}'s {\tt getState()} method to print out its
current state:

\begin{jjjlisting}
\begin{lstlisting}
for (int k = 0; k < school.length; k++)
    System.out.println(school[k].getState());
\end{lstlisting}
\end{jjjlisting}

What if the three {\tt Student}s already existed before the array was
created? In that case, we could just assign their references to the
array elements, as in the following example:

\begin{jjjlisting}
\begin{lstlisting}
Student student1 = new Student("Socrates"); 
Student student2 = new Student("Plato");
Student student3 = new Student("Aristotle");
Student school = new Student[3]; // A 3 Student array
school[0] = student1;
school[1] = student2;
school[2] = student3;
\end{lstlisting}
\end{jjjlisting}
%\begin{figure}
%\begin{graphic}
\marginfig{chptr09/p408f1.eps}{Arrays of objects store references to
the objects, not the objects themselves.}
{fig-arrayalloc}
%\end{graphic}
%\end{figure}

\noindent In this case, each of the three {\tt Student} objects
can be referenced by two different references---its variable
identifier (such as {\tt student1}) and its array location (such as
{\tt school[0]}).  For arrays of objects, Java stores just the reference to
the object in the array itself, rather than the entire object.  This
conserves memory, since references require only 4 bytes each whereas
each object may require hundreds of bytes (Fig.~9.4).

When an array\index{array allocation} of {\it N} elements is
%\begin{marginalnote}\it How much memory?\end{marginalnote}
created, the compiler {\it allocates} storage for {\it N} variables of
the element's type.  In the case of {\tt arr} that we discussed
earlier, the compiler would allocate storage for 15 {\tt int}s---60
contiguous bytes of storage, because each {\tt int} requires 4 bytes
(32 bits) of storage.  If we declare an array of 20 {\tt double}s,

\begin{jjjlisting}
\begin{lstlisting}
double arr[] = new double[20];
\end{lstlisting}
\end{jjjlisting}

\noindent the compiler will allocate 160 bytes of storage---20
variables of 8 bytes (64~bits) each.  In the case of the
\marginnote{How much memory?}  {\tt Student} examples and {\tt String}
examples, because these are objects (not primitive types), the
compiler will allocate space for {\it N} addresses, where {\it N} is
the length of the array and where each address requires 4 bytes.

\secEXRHone{Self-Study Exercise}
\begin{SSTUDY}
\item  How much space  (in bytes) would be allocated
for each of the following?

\begin{EXRLL}
\item {\tt int a[] = new int[5];}    
\item {\tt double b[] = new double[10];}    
\item {\tt char c[] = new char[30];}        
\item {\tt String s[] = new String[10];}    
\item {\tt Student p[] = new Student[5];}  
\end{EXRLL}

\end{SSTUDY}

\subsection{Initializing Arrays}

\noindent Array elements are automatically initialized
to default values that depend on the element type:
Boolean elements are initialized to {\tt false}, and integer and real
types are initialized to 0. Reference types---that is, arrays of
\marginnote{Default initialization}
objects---are initialized to {\tt null}.

Arrays can also be assigned initial values when they are created,
although this is feasible only for relatively small arrays.  An {\bf
array initializer}\index{array initializer} is written as a list of expressions separated by
commas and enclosed by braces.   For example, we can declare and
\marginnote{Array initializer}
initialize the array shown in Figure~\ref{fig-intarray} with the
following statement:

\begin{jjjlisting}
\begin{lstlisting}
int arr[] = {-2,8,-1,-3,16,20,25,16,16,8,18,19,45,21,-2};
\end{lstlisting}
\end{jjjlisting}

\noindent Similarly, to create and initialize an array of {\tt String}s,
we can use the following statement:

\begin{jjjlisting}
\begin{lstlisting}
String strings[] = {"hello", "world", "goodbye", "love"};
\end{lstlisting}
\end{jjjlisting}

\noindent This example creates and stores four {\tt String}s
in the array.  Subsequently, to refer to ``hello'', we would use the
reference {\tt strings[0]}, and to refer to ``love'', we would use the
reference {\tt strings[3]}. Note in these examples that when an array
declaration contains an initializer, it is not necessary to use
{\tt new} and it is not necessary to specify the number of elements in the
array.  The number of elements is determined from the number of values
in the initializer list.

\subsection{Assigning and Using Array Values}
\noindent Array elements can be used in the same way as other variables.  The
%\begin{marginalnote}\it Array assignment\end{marginalnote}
\marginnote{Array assignment}
only difference, of course, is that references to the elements are
subscripted.  For example, the following assignment\index{array
assignment} statements assign values to the elements of two arrays,
named {\tt arr} and {\tt strings}:

\begin{jjjlisting}
\begin{lstlisting}
arr[0] = 5;
arr[5] = 10;
arr[2] = 3;
strings[0] = "who";
strings[1] = "what";
strings[2] = strings[3] = "where";
\end{lstlisting}
\end{jjjlisting}

\noindent The following loop assigns the first 15 squares---{\tt 1, 4,
9 \dots }---to the array {\tt arr}:

\begin{jjjlisting}
\begin{lstlisting}
for (int k = 0; k < arr.length; k++)
    arr[k] = (k+1) * (k+1);
\end{lstlisting}
\end{jjjlisting}

\noindent The following loop prints the values of the array {\tt arr}:

\begin{jjjlisting}
\begin{lstlisting}
for (int k = 0; k < arr.length; k++)
    System.out.println(arr[k]);
\end{lstlisting}
\end{jjjlisting}

\secEXRHone{Self-Study Exercises}
\begin{SSTUDY}

\item  Declare an array named {\tt farr} that contains ten {\tt float}s
initialized to the values {\tt 1.0, 2.0, \dots , 10.0}.

\item  Write an expression that prints the first element of {\tt farr}.


\item  Write an assignment statement that assigns 100.0 to
the last element in {\tt farr}.


\item  Write a loop to print all of the elements of {\tt farr}.

\end{SSTUDY}


\section{Simple Array Examples}
\label{simple-array-examples}

\noindent The program in Figure~\ref{fig-initarrays} creates two arrays of ten
elements each and displays their values on the Java console.   In this
\begin{figure}[h!]
\jjjprogstart
\begin{jjjlisting}[27pc]
\begin{lstlisting}
public class PrintArrays { 
  static final int ARRSIZE = 10;         // The array's size
  static int intArr[] = new int[ARRSIZE];// Create int array
  static double realArr[] = { 1.1, 2.2, 3.3, 4.4,
     5.5, 6.6, 7.7, 8.8, 9.9, 10.10 }; // And a double array

  public static void main(String args[]) {
    System.out.println("Ints \t Reals");  // Print a heading
                          // For each int and double element
    for (int k = 0; k < intArr.length; k++) 
       System.out.println( intArr[k] + " \t " + 
                                 realArr[k]);  // Print them
    } // main()
} // PrintArrays
\end{lstlisting}
\end{jjjlisting}
\jjjprogstop{A program that displays two arrays. Its output
is shown in Figure~\hyperref[fig:printarraysout]{9.6}}
{fig-initarrays}
\end{figure} 
\noindent example, the elements of \mbox{\tt intArr} have not been
given initial values whereas the elements of \mbox{\tt realArr} have been
initialized.  Note the use of the integer constant
\mbox{\tt ARRSIZE} to store the arrays' size.  By using the constant
in this way, we do not have to use the literal value {\tt 10} anywhere
in the program, thereby making it easier to read and to modify the
program.  If we want to change the size of the array that the program
handles, we can just change the value of {\tt ARRSIZE}. This is an
example
\marginnote{Maintainability principle}
%\begin{marginalnote}Maintainability Principle\end{marginalnote}
of the maintainability
principle\index{Maintainability Principle}.



\JavaTIP{EFFECTIVE DESIGN}{Symbolic Constants.}{Using symbolic
constants (final variables) instead of literal values makes the
program easier to read and to maintain.}

\noindent Note the use of the {\tt static} qualifier throughout the
{\tt PrintArrays} class. This enables us to refer to the array
and the other variables from within the {\tt main()} method. 
\phantomsection 
\label{fig:printarraysout}
 If {\tt intArr} were not declared {\tt static}, we would get the
compiler error {\tt attempt to make static use of a non-static variable.}
This use of {\tt static} is justified mainly as a coding convenience
rather than a principle of object-oriented design. The only examples
we've seen so far in which static elements were a necessary design
element were the use of static elements in the {\tt Math} class---{\tt Math.PI} 
and {\tt Math.sqrt()}---and the use of static
final variables in {\tt TwoPlayerGame}---{\tt TwoPlayerGame.PLAYER\_ONE}.

\marginpar{
\vspace*{-5pc}
%\small
{\color{cyan}\rule{12.5pc}{1pt}}
\begin{tabular}{cc} 
 \hspace*{18pt}Ints\hspace*{18pt}    &      \hspace*{18pt}Reals\hspace*{18pt}  \\[4pt]
 0       &      \hphantom{0}1.1  \\ 
 0       &      \hphantom{0}2.2  \\ 
 0       &      \hphantom{0}3.3  \\ 
 0       &      \hphantom{0}4.4  \\ 
 0       &      \hphantom{0}5.5  \\ 
 0       &      \hphantom{0}6.6  \\ 
 0       &      \hphantom{0}7.7  \\ 
 0       &      \hphantom{0}8.8  \\ 
 0       &      \hphantom{0}9.9  \\ 
 0       &      10.1  \\ 
 \end{tabular}
 \addtocounter{figure}{1}
{\color{cyan}\rule[12pt]{12.5pc}{1pt}\hspace*{-12.6pc}
\noindent \bf \small\raggedright FIGURE \thefigure} \raggedright \quad
Output of the {\tt PrintArrays} program. 
}





For large arrays, it is not always feasible to initialize them in an
%\begin{marginalnote}\it Array initializers\end{marginalnote}
initializer statement.  Consider the problem of initializing an array
with the squares of the first 100 integers.  Not only would it be
tedious to set these values in an initializer statement, it would also
be error prone, since it is relatively easy to type in the wrong value
for one or more of the squares.

\JavaTIP{DEBUGGING TIP}{Array Initialization.}{Initializer statements
should be used only for relatively small arrays.}

\noindent The example in Figure~\ref{fig-squares100} creates an array of 50
integers and then fills the elements with the values 1, 4, 9, 16, and
so on.  It then prints the entire array.
\begin{figure}[h!]
\jjjprogstart
\begin{jjjlisting}
\begin{lstlisting}
public class Squares {
  static final int ARRSIZE = 50;     // The array's size
  static int intArr[] = new int[ARRSIZE]; // Instantiate
  public static void main(String args[]) {
    for (int k = 0; k < intArr.length; k++)// Initialize
      intArr[k] = (k+1) * (k+1);
    System.out.print("The first 50 squares are"); 
    for (int k = 0; k < intArr.length; k++) { // Print
      if (k % 5 == 0)             // For each 5th square
        System.out.println(" ");//   print a new line
      System.out.print( intArr[k] + " ");
    } // for
  } // main()
} // Squares
\end{lstlisting}
\end{jjjlisting}
\jjjprogstop{A program with an array that stores the squares of the first 50 integers.
Its output is shown in Figure~\hyperref[fig-squaresout]{9.8}.}
{fig-squares100}
\end{figure}


This example illustrates some important points about the use of
array variables.  The array's elements are individual
storage\index{array storage} locations.  In this example, {\tt intArr}
has 50 storage locations.  Storing a value in one of these variables
is done by an assignment statement:

\begin{jjjlisting}
\begin{lstlisting}
intArr[k] = (k+1) * (k+1);
\end{lstlisting}
\end{jjjlisting}

\phantomsection 
\label{fig-squaresout}
\noindent The use of the variable {\it k} in this assignment
statement allows us to vary the location that is assigned on each
iteration of the for loop. 
 Note that in this example, {\it k} occurs as
the array index on the left-hand side of this expression, while
{\it k+1} occurs on the right-hand side as the value to be squared.  The
\marginpar{
\vspace*{-10pc}
\small
{\color{cyan}\rule{12.5pc}{1pt}}\\[2pt]
\it 1 4 9 16 25 \\
36 49 64 81 100\\
121 144 169 196 225\\
256 289 324 361 400\\
441 484 529 576 625\\
676 729 784 841 900\\
961 1024 1089 1156 1225\\
1296 1369 1444 1521 1600\\
1681 1764 1849 1936 2025\\
2116 2209 2304 2401 2500 \rm\\[4pt]
 \addtocounter{figure}{1}
{\color{cyan}\rule[12pt]{12.5pc}{1pt}\hspace*{-12.6pc}
\noindent \bf \small\raggedright FIGURE \thefigure} \raggedright \quad
Output of the {\tt Squares} program. 
%\captionmargin{Output of the {\tt Squares} program.
%
}
reason for this is that arrays are indexed starting at 0 but we want
our table of squares to begin with the square of 1. So the square of
some number {\it n+1} will always be stored in the array whose index
is one less than the number itself---that 
\marginnote{Zero vs.  unit indexing}
is, {\it n}.

An array's {\tt length} variable can always be used as a loop bound
when iterating through all elements of the array:

\begin{jjjlisting}
\begin{lstlisting}
for (int k = 0; k < intArr.length; k++)
    intArr[k] = (k+1) * (k+1);
\end{lstlisting}
\end{jjjlisting}

\noindent However, it is important to note that the last element in
the array is always at location \verb|length-1|. Attempting to refer
to {\tt intArr[length]}
\marginnote{Off-by-one error}
would cause an {\tt IndexOutOfBoundsException} because no such element
exists.

\JavaTIP{DEBUGGING TIP}{Off-by-One Error.}{Because of zero
indexing, the last element in an array is always $length-1$.
Forgetting this fact can cause an off-by-one error.}

\secEXRHone{Self-Study Exercise}
\begin{SSTUDY}

\item  Declare an array of 100 {\tt double}s and write a loop to assign the
first 100 square roots to its elements.  [Use {\tt Math.sqrt(double)}.]

\end{SSTUDY}

\section{Example: Counting Frequencies of Letters}
\label{example-counting-frequenciesof-letters}
\noindent Suppose you wish to write a program to help break a text
message that has been encrypted with one of the historical ciphers
that we have discussed in the two previous chapters.  It is well known
that historical ciphers often can be broken, that is, the plaintext
can be found from the ciphertext, by examining the frequencies of
letters and comparing them to the average frequencies of typical
samples of plaintext.  For example, {\tt E} and {\tt T} are the two
most frequently used letters in the English language. So, in a
ciphertext encrypted with a Caesar cipher, {\tt E} and {\tt T} are
good guesses as the plaintext letter corresponding to the most
frequent letter in a ciphertext message.

Let's write a program that will count how many times each of the {\tt
26} letters of the English language appears in a given string.  There
are a number of ways to design such a program depending on how
flexible you wish the program to be.  Let us keep this example simple
by assuming that we will only be interested in counting occurrences of
the letters {\tt A} through {\tt Z} and not of occurrences of spaces
or punctuation marks.  Assume further that we will change lowercase
letters in our string sample to uppercase before counting letters and
that we will want to print out the frequencies of letters to the
console window.  Finally, assume that, later in the chapter after we
discuss sorting arrays, we will want to enhance our program so that it
can print out the letter frequencies in order of increasing frequency.

\subsection{A Class to Store the Frequency of One Letter}

\noindent It is clear that an array should be used for storing the
frequencies, but a decision must also be made as to what to store as the
array elements.  If we store letter frequencies as {\tt int} values,
with the frequency of {\em A} stored at index 0, and the frequency of
{\em B} at index 1, and so forth, we will not be able to rearrange the
frequencies into increasing order without losing track of which letter
corresponds to which frequency.  One way of solving this problem is to
create an array of objects, where each object stores both a letter and
its frequency.

So let us design a {\tt LetterFreq} class that stores a letter
in an instance variable of type {\tt char} and its frequency in an instance
variable of type {\tt int}. These instance variables can be declared as:

\begin{jjjlisting}
\begin{lstlisting}
 private char letter;    //A character being counted
 private int freq;       //The frequency of letter
\end{lstlisting}
\end{jjjlisting}

\noindent We will want a constructor that can initialize these two values
and two accessor methods to return these values.  We are familiar enough
with these kinds of methods that it will not be necessary to discuss them any
further.  We need one additional method to increment {\tt freq} whenever
we encounter the letter while processing the string:

\begin{jjjlisting}
\begin{lstlisting}
 public void incrFreq() {
     freq++;
 } //setFreq()
\end{lstlisting}
\end{jjjlisting}

A UML diagram for the {\tt LetterFreq} class is given in Figure~9.9 and
\marginfig{chptr09/letfrequml.eps}{UML for {\tt LetterFreq}.}
{fig-letfrequml}
the class definition is given
\begin{figure}[tb]
\jjjprogstart
\begin{jjjlisting}
\begin{lstlisting}
public class LetterFreq {
    private char letter;    //A character being counted
    private int freq;       //The frequency of letter

    public LetterFreq(char ch, int fre) {
        letter = ch;
        freq = fre;
    }
    public char getLetter() {
        return  letter;
    }
    public int getFreq() {
        return  freq;
    }
    public void incrFreq() {
        freq++;
    }
} //LetterFreq
\end{lstlisting}
\end{jjjlisting}
\jjjprogstop{The {\tt LetterFreq} class definition.}
{fig-letfreq}
\end{figure}
in Figure~\ref{fig-letfreq}. Note that we will have to make a minor
modification to this class later in this chapter to enable us to sort
an array of objects from this class.

\subsection{A Class to Count Letter Frequencies}

\noindent Now let us turn to designing a class named {\tt AnalyzeFreq}
that will use an array of objects of type {\tt LetterFreq} to count
the frequencies of the letters {\tt A} through {\tt Z} in a given
string.  The array, let's call it {\tt freqArr}, will be the only
instance variable of the class.  The class needs a constructor to
instantiate the array and to create the {\tt 26} array elements, each
with a different letter and an initial frequency of {\tt 0}.  This
class should also have two methods: a method to count the frequencies
of the {\tt 26} letters in a given string and a method that prints out
the frequency 
\marginfig{chptr09/anafrequml.eps}{UML for {\tt
AnalyzeFreq}.}
{fig-anafrequml}
of each letter to the console window.  The UML diagram for the class
is given in Figure~9.11.

The array instance variable can be declared by:

\begin{jjjlisting}
\begin{lstlisting}
 private LetterFreq[] freqArr; //An array of frequencies
\end{lstlisting}
\end{jjjlisting}

\noindent The constructor creates an array of {\tt 26} elements to
store references to {\tt LetterFreq} objects with the statement

\begin{jjjlisting}
\begin{lstlisting}
 freqArr = new LetterFreq[26];
\end{lstlisting}
\end{jjjlisting}

\noindent The indices of the array range from {\tt 0} to {\tt 25} and
the elements at these locations should store the letters {\tt A} to
{\tt Z}.  Recall that in Java, {\tt char} data are a form of {\tt int}
data and can be used in arithmetic.  If we let {\tt k} be an integer
that ranges between {\tt 0} and {\tt 25}, then the expression {\tt
(char)('A' + k)} will correspond to the letters {\tt A} to {\tt Z}.  .
Thus, the following loop will initialize the array correctly.  

\begin{jjjlisting}
\begin{lstlisting}
for (int k = 0; k < 26; k++) {
    freqArr[k] = new LetterFreq((char)('A' + k), 0);
} //for
\end{lstlisting}
\end{jjjlisting}

\noindent The {\tt countLetters()} method must identify the array index for {\tt
LetterFreq} object that stores a letter between {\tt A} and {\tt Z}.
If {\tt let} is a {\tt char} variable that stores such a letter, then
the expression {\tt (let - 'A')} will give the index of the array
element corresponding to {\tt let}.  Thus the following code will
calculate the frequencies the letters in the string parameter, {\tt
str}:

\begin{jjjlisting}
\begin{lstlisting}
 public void countLetters(String str) {
     char let;     //For use in the loop.
     str = str.toUpperCase();
     for (int k = 0; k < str.length(); k++) {
         let = str.charAt(k);
         if ((let >= 'A') && (let <= 'Z')) {
             freqArr[let - 'A'].incrFreq();
         } //if
     } //for
 } //countLetters()
\end{lstlisting}
\end{jjjlisting}

\noindent The definition of the {\tt printArray()} method is completely
straight forward:

\begin{jjjlisting}
\begin{lstlisting}
public void printArray() {
  for (int k = 0; k < 26; k++) {
    System.out.print("letter: " + freqArr[k].getLetter());
    System.out.println("   freq: " + freqArr[k].getFreq());
  } //for
} //printArray()
\end{lstlisting}
\end{jjjlisting}

\noindent The entire definition of {\tt AnalyzeFreq} is
\begin{figure}[tb]
\jjjprogstart
\begin{jjjlisting}[26.5pc]
\begin{lstlisting}
public class AnalyzeFreq {
  private LetterFreq[] freqArr; //An array of frequencies

  public AnalyzeFreq() {
    freqArr = new LetterFreq[26];
    for (int k = 0; k < 26; k++) {
      freqArr[k] = new LetterFreq((char)('A' + k), 0);
    } //for
  }
  public void countLetters(String str) {
    char let; //For use in the loop.
    str = str.toUpperCase();
    for (int k = 0; k < str.length(); k++) {
      let = str.charAt(k);
      if ((let >= 'A') && (let <= 'Z')) {
        freqArr[let - 'A'].incrFreq();
      } //if
    } //for
  }
  public void printArray() {
    for (int k = 0; k < 26; k++) {
      System.out.print("letter: " + freqArr[k].getLetter());
      System.out.println(" freq: " + freqArr[k].getFreq());
    } //for
  } 
} //AnalyzeFreq
\end{lstlisting}
\end{jjjlisting}
\jjjprogstop{The {\tt AnalyzeFreq} class definition.}
{fig-anafreq}
\end{figure}
\noindent given in Figure~\ref{fig-anafreq}. We will modify this class later in
the chapter to be able to sort the array after counting.  The
following {\tt main()} method, either in this class or in its own
class will demonstrate how the class methods are used.

\begin{jjjlisting}
\begin{lstlisting}
public static void main(String[] argv) {
  AnalyzeFreq af = new AnalyzeFreq();
  af.countLetters("Now is the time for all good students" +
     " to study computer related topics.");
  af.printArray();
} //main()
\end{lstlisting}
\end{jjjlisting}

\secEXRHone{Self-Study Exercises}
\begin{SSTUDY}

\item Rewrite the {\tt main()} of the {\tt AnalyzeFreq} class so that
it opens a file named {\tt freqtest.txt} and counts the frequencies of
the letters of the text stored in the file.  You will need to use the
{\tt Scanner} class to read from the file as was done in Chapter~4.
Create a file named {\tt freqtest.txt} that contains several hundred
characters of typical English text to test the new {\tt main()} method
\end{SSTUDY}


\section{Array Algorithms: Sorting}
\label{sec-sort}
\noindent {\bf Sorting} an array is the process of arranging its elements in
ascending or descending order.   Sorting algorithms are among the most
widely used algorithms.  Any time large amounts of data are maintained,
there is some need to arrange them in a particular order.   For example,
the telephone company needs to arrange its accounts by the last name
of the account holder as well as by phone number.


\subsection{Insertion Sort}
\noindent The first sorting algorithm we'll look at is known as {\bf
insertion sort}, so named because as it traverses through the array
from the first to the last element, it inserts each element
into its correct position in the partially sorted array. 

For an array of {\em N} elements, let's think of the array as divided
into two parts. The {\em sorted} part will be the left hand side of
the array.  And the {\em unsorted} part will be the right hand side of
the array.  Initially, the sorted part consists of the first element
in the array---the element at index 0.

Insertion sort moves through the unsorted portion of the
array---that is its loop variable, {\em k}, ranges from 1 through
{\em N-1}.  On each iteration it inserts the {\em k}th element into
its correct position in the sorted part of the array.  To insert an
element into the sorted part of the array, it may be necessary to move
elements greater than the one being inserted out of the way.

In pseudocode, insertion sort can be represented as follows:

\begin{jjjlistingleft}[32pc]{-6.5pc}
\begin{lstlisting}
Insertion Sort of an array, arr, of N elements into ascending order
1. For k assigned 1 through N-1
2.   Remove the element arr[k] and store it in x.
3.   For i starting at k-1 and for all preceding elements greater than x
4.     Move arr[i] one position to the right in the array.
5.   Insert x at its correct location.
\end{lstlisting}
\end{jjjlistingleft}

\noindent As is apparent from the pseudocode, we have a nested for
loops. The outer ({\em k}) loop, iterates through the array from 1 to
{\em N-1}. The inner loop iterates as many times as necessary,
starting with the element just to the left of the {\em k}th element
in order to insert the {\em k}th element into its correct position in
the sorted portion.  Note that the {\em k}th element is always removed
from the array (and stored in the variable {\em x}), to make room for
elements that have to be moved to the right.

To see how this works, consider an integer array containing the ages
of five friends:

\begin{jjjlisting}
\begin{lstlisting}
21 |  20  27  24  19      x = 20
      k 
\end{lstlisting}
\end{jjjlisting}

\noindent For this five-element array, insertion sort initially will
assume that the element at index 0 is in the correct position. The
vertical line marks the boundary between the sorted and unsorted
portions of the array.  The outer loop will look at each of the
remaining elements, one at a time, inserting it into its proper
position in the sorted portion of the array.  To insert 20, the number
at index 1, the inner loop will move 21 to the right by one
position. To do this, the algorithm will remove 20 from its location
and store it in x. It will then move 21 one space to the
right. Finally, it will insert 20, which is stored in x, at index 0,
where it belongs relative to the other elements in the sorted part of
the array.  At this point, the sorted portion of the array consists of
the first two elements, which are in the correct order, relative to
each other.

\begin{jjjlisting}
\begin{lstlisting}
20  21 |  27  24  19     x = 27
          k  
\end{lstlisting}
\end{jjjlisting}

\noindent For the next element, 27, none of elements
in the sorted portion need to be moved, so the inner for loop
will iterate zero times. This gives us:

\begin{jjjlisting}
\begin{lstlisting}
20  21  27 |  24  19    x = 24
              k 
\end{lstlisting}
\end{jjjlisting}

\noindent For the fourth element, 24, only the previous element, 27,
needs to be moved to the right, giving:

\begin{jjjlisting}
\begin{lstlisting}
20  21  24  27 | 19    x = 19
                 k 
\end{lstlisting}
\end{jjjlisting}

\noindent At this point, the sorted part of the array consists of the
first four elements, which are in the correct order relative to each
other.  Finally, for the last element, 19, all of the elements in the
sorted part of the array need to be moved one space to the right.
This will require four iterations of the inner loop. We show the state
of the array after each iteration of the inner for loop:

\begin{jjjlisting}
\begin{lstlisting}
              k
20 21 24 27 | 19   Remove 19 and store it x = 19  
20 21 24 27 | 27   Move 27 to the right
20 21 24 24 | 27   Move 24 to the right
20 21 21 24 | 27   Move 21 to the right
20 20 21 24 | 27   Move 20 to the right
19 20 21 24 27 |   Insert x=19 at index 0
\end{lstlisting}
\end{jjjlisting}

\noindent Clearly, the fact that so many elements may have to moved on
each iteration of the outer loop shows that insertion sort is not
a very efficient algorithm.

The {\tt Sort} class (Fig~\ref{fig-bubblemeth}) provides an
implementation of the {\tt insertionSort()} method.  There are several
points worth noting about this code. First, because it takes an {\tt
int} array as a parameter, the {\tt insertionSort()} method will sort
any array of integers, regardless of the array's length.

\begin{figure}[bth]
\jjjprogstart
\begin{jjjlisting}
\begin{lstlisting}
public class Sort {
  public void insertionSort(int arr[]) {
    int temp;  // Temporary variable for insertion
    for (int k = 1; k < arr.length; k++)  { 
      temp = arr[k]; // Remove element from array
      int i;         // For larger preceding elements
      for (i = k-1; i >= 0 && arr[i] > temp; i--) 
        arr[i+1] = arr[i]; // Move it right by one
      arr[i+1] = temp;       // Insert the element
    }
  } // insertionSort()
  public void print(int arr[]) {
    for (int k = 0; k < arr.length; k++)// For each integer
      System.out.print( arr[k] + " \t "); //  Print it
    System.out.println();
  } // print()
  public static void main(String args[]) {
    int intArr[] = { 21, 20, 27, 24, 19 };
    Sort sorter = new Sort();
    sorter.print(intArr);
    sorter.insertionSort(intArr); // Passing an array
    sorter.print(intArr);
  } // main()
} //Sort
\end{lstlisting}
\end{jjjlisting}
\jjjprogstop{Source code for  the {\tt insertionSort()} method.
Note in {\tt main()} how an integer array is passed to the method.}
{fig-bubblemeth}
\end{figure}

Second, note how empty brackets ({\tt []}) are used to
\marginnote{Array parameters} declare an array parameter.  If the
brackets were omitted, then {\tt arr} would be indistinguishable from
an ordinary {\tt int} parameter.  Using the brackets indicates that
this method takes an array of integers as its parameter.

\JavaTIP{DEBUGGING TIP}{Array Parameter.}{When declaring an array
parameter, empty brackets must be used either after the array name or
after the type name to distinguish it from a non-array parameter.}

Third, note how an array of integers is passed to the {\tt
insertionSort()} method in the {\tt main()} method:

\begin{jjjlisting}
\begin{lstlisting}
 sorter.insertionSort(intArr); // Pass intArr to the method
\end{lstlisting}
\end{jjjlisting}

\noindent That is, when passing an array to a method, you use just the
name of the array, without brackets. Both of the following statements
would cause syntax errors:

\begin{jjjlisting}
\begin{lstlisting}
sorter.insertionSort(intArr[]); // Err: Can't have brackets
sorter.insertionSort(intArr[5]);// Err: passing an integer
\end{lstlisting}
\end{jjjlisting}

\noindent In the first case, empty brackets are only used when you
declare an array variable, not when you are passing the array to a
method.  In the second case, {\tt intArr[5]} is an {\tt int}, not an
array, and cannot legally be passed to {\tt insertionSort()}.

\JavaTIP{DEBUGGING TIP}{Passing an Array Argument.}{It is a syntax error
to use empty brackets when passing an array argument to a method,
where the only the array's name should be used. Empty rackets are only
used when declaring an array variable.}

\noindent Finally, within the {\tt insertionSort()} method itself,
note that we declare the index for the inner {\tt for} loop outside of
the {\tt for} statement. This is so it can be used outside the scope
of the {\tt for} loop to insert {\tt temp} at location {\tt arr[i+1]}, its
correct location. Note also that the index of its correct location is
{\tt i+1}, rather than just {\tt i}. This is because the inner loop
might iterate past location 0, which would give {\em i} a value of -1
at that point.


\subsection{Selection Sort}
\label{sec-selectionsort}

\noindent There are a large variety of array sorting
algorithms. Selection sort is different from, but comparable to,
insertion sort in its overall performance.  To illustrate the {\bf
selection sort} algorithm, suppose you want to sort a deck of 25 index
cards, numbered from 1 to 25.  Lay the 25 cards out on a table, one
card next to the
%\begin{marginalnote}\it Selection sort algorithm\end{marginalnote}
\marginnote{Selection sort algorithm} other.  Starting with the
first card, look through the deck and find the smallest card, the
number 1 card, and exchange it with the card in the first location.
Then, go through the deck again starting at the second card, find the
next smallest card, the number 2 card, and exchange it with the card
in the second location.  Repeat this process 24 times.

Translating this strategy into pseudocode gives the following
\mbox{algorithm:}

\begin{jjjlisting}
\begin{lstlisting}
Selection sort of a 25-card deck from small to large
1. For count assigned 1 to 24  // Outer loop
2.   smallestCard = count
3.   For currentCard assigned count+1 to 25 // Inner loop
4.     If deck[currentCard] < deck[smallestCard]
5.        smallestCard = currentCard
6.   If smallestCard != count // You need to swap
7       Swap deck[count] and deck[smallestCard]
\end{lstlisting}
\end{jjjlisting}

\noindent For a deck of 25 cards, you need to repeat the outer loop
24 times.  In other words, you must select the smallest
card and insert it in its proper location 24 times.  The inner loop
takes care of finding the smallest remaining card.

On each iteration of this outer loop, the algorithm assumes that the
card specified by the outer loop variable, {\tt count}, is the
smallest card (line 2). It usually won't be, of course, but we have
to start somewhere.

The inner loop then iterates through the remaining cards (from {\tt
count+1} to 25) and compares each one with the card that is currently
the smallest (lines 4 and 5). Whenever it finds a card that is smaller
than the smallest card, it designates it as the smallest card (line
5). At the end of the loop, the {\tt smallestCard} variable will
remember where the smallest card is in the deck.

Finally, when the inner loop is finished, the algorithm swaps the
smallest card with the card in the location designated by {\tt count}.

\subsection{Algorithm: Swapping Memory Elements}

\noindent An important feature of the selection sort algorithm is its
need to swap two array elements, or cards, to continue our
example. Swapping two memory elements, whether they are array elements
or not, requires the use of a temporary variable. For example, to swap
the {\em j}th and {\em k}th elements in an {\tt int} array named {\tt
arr}, you would use the following algorithm: \marginnote{Swapping
algorithm}

\begin{jjjlisting}
\begin{lstlisting}
int temp = arr[j]; // Store the jth element in temp
arr[j] = arr[k];   // Move the kth element into j
arr[k] = temp;     // Move the jth element into k
\end{lstlisting}
\end{jjjlisting}

\noindent The {\tt temp} variable temporarily stores the {\em j}th
element so its value is not lost when its location is overwritten by
the {\em k}th element. The need for this variable is a subtlety that
beginning programmers frequently overlook. But consider what would
happen if we used the following erroneous algorithm:
\marginnote{Swapping blunder}

\begin{jjjlisting}
\begin{lstlisting}
arr[j] = arr[k]; // Erroneous swap code
arr[k] = arr[j];
\end{lstlisting}
\end{jjjlisting}

\noindent If {\tt arr[j]} refers to 4 and {\tt arr[k]} refers to 2 in
the array \mbox{{\tt 1 4 2 8}}, then the erroneous algorithm would
produce \mbox{{\tt 1 2 2 8}}, the wrong result.  


\JavaTIP[false]{PROGRAMMING TIP}{Swapping Variables.}{When 
swapping two memory elements, a temporary variable must be
used to store one of the elements while its memory location
is being overwritten.}

\noindent The following method implements the swap algorithm for two
elements, {\it el1} and {\it el2} of an {\tt int} array:

\begin{jjjlisting}
\begin{lstlisting}
void swap(int arr[], int el1, int el2) {
    int temp = arr[el1]; // Assign first element to temp
    arr[el1] = arr[el2]; // Overwrite first with second
    arr[el2] = temp;     // Overwrite second with first
} // swap()
\end{lstlisting}
\end{jjjlisting}

\secEXRHone{Self-Study Exercises}
\begin{SSTUDY}

\item Sort the array, 24 18 90 1 0 85 34 18, using the insertion sort
algorithm.  Show the order of the elements after each iteration of the
outer loop.

\item Sort the array, 24 18 90 1 0 85 34 18, using the selection sort
algorithm.  Show the order of the elements after each iteration of the
outer loop.

\item  Write a Java code segment to swap two {\tt Student} objects,
{\tt student1} and {\tt student2}.

\item Write a Java implementation of the {\tt selectionSort()} method to 
sort an array of {\tt int}.
\end{SSTUDY}

\subsection{Passing a Value and Passing a Reference}
\noindent Recall from Chapter~3 that when an {\tt Object} is
passed to a method, a copy of the reference to the {\tt Object} is
passed.  Because an array is an object, a reference to the array is
passed to {\tt insertionSort()}, rather than the whole array itself.
This is in contrast to how a value of a primitive type is passed.  In
that case, a copy of the actual value is passed.

\pagebreak
\JavaRule{Primitive vs. Object Parameters.}{When a value of a
primitive data type---{\tt int, double, char, boolean}--- is passed as
an argument to a method, a copy of the value is passed; when a
reference to an {\tt Object} is passed, a copy of the reference is
passed.}

\noindent One implication of this distinction is that for arguments of
primitive type, the original argument cannot be changed from within
the method because the method has only a copy of its value.  For
example, the following method takes an {\tt int} parameter {\it
n}, which is incremented within the method:

\begin{jjjlisting}
\begin{lstlisting}
public void add1(int n) {
  System.out.print("n = " + n);
  n = n + 1;
  System.out.println(",  n = " + n);
}
\end{lstlisting}
\end{jjjlisting}

\noindent But because {\it n} is a parameter of primitive type, incrementing
it within the method has no effect on its associated argument.  Thus,
in the following segment, the value of {\it Num}---{\it n}'s associated
argument---will not be affected by what goes on inside the {\tt add()}
\marginnote{Passing a primitive value}
method.  The output produced by the code segment is shown in the comments:

\begin{jjjlisting}
\begin{lstlisting}
int Num = 5;
System.out.println("Num = " + Num); // Prints Num = 5
add1(Num);                    // Prints n = 5,  n = 6            
System.out.println("Num = " + Num); // Prints Num = 5
\end{lstlisting}
\end{jjjlisting}

\noindent Note that while {\it n}'s value has changed inside the
method, {\it Num}'s value remains unaffected.

The case is much different when we pass a reference to an object.  In
that case, the object itself can be manipulated from within the
method.  The {\tt insertionSort()} method is a good illustration.  In
the following code
\marginnote{Passing an object}
segment, the array {\tt anArr} is printed, then sorted, and then printed
again:

\begin{jjjlisting}
\begin{lstlisting}
 Sort sorter = new Sorter();
 int anArr[] = { 5, 10, 16, -2, 4, 6, 1 }; 
 sorter.print(anArr);           // Prints 5 10 16 -2 4 6 1
 sorter.insertionSort(anArr);   // Sorts anArr
 sorter.print(anArr);           // Prints -2 1 4 5 6 10 16
\end{lstlisting}
\end{jjjlisting}

\noindent As you can see, the object itself (the array) has been
changed from within the method. This shows that changes within {\tt
insertionSort} to the array referenced by {\tt arr} are actually being
made to {\tt anArr} itself.  If fact, because {\tt insertionSort()} is
passed a copy of the reference variable {\tt anArr}, both {\tt arr}
and {\tt anArr} are references to the very same object---that is, to
the same array (Fig.~\ref{fig-arrayparam}).

\begin{figure}[tbh]
\figaleft{chptr09/8f11.eps}{When an array is passed to a method, both
the parameter and the corresponding argument refer to the same object.
} {fig-arrayparam}
\end{figure}

The justification for passing a reference to an object rather than the
\marginnote{Method call overhead}
entire object itself is a matter of efficiency.  A reference uses just
4 bytes of data, whereas an object may use thousands of bytes.   It
would just be too inefficient to copy hundreds of bytes each time an
object is passed to a method.  Instead, the method is passed a
reference to the object, thereby giving it access to the object
without incurring the expense of copying large amounts of data.
Indeed, Java provides no way to pass a copy of an object to a method.

\secEXRHone{Self-Study Exercise}
\begin{SSTUDY}

\item Give the values that will be stored in {\tt myArr} and {\tt k}
after you invoke {\tt mystery(myArr, k)}, where {\tt myArr}, {\tt k}
and {\tt mystery()} are declared as follows:

\begin{jjjlisting}
\begin{lstlisting}
int myArr[] = {1,2,3,4,5};    int k = 3;
void mystery(int a[], int m) {
    ++a[m];
    --m;
}
\end{lstlisting}
\end{jjjlisting}

\end{SSTUDY}

\section{Array Algorithms: Searching}
\label{array-algorithms-searching}
\noindent Suppose we have a large array and we need to find one of its
elements.  We need an algorithm to search the array for a particular
value, usually called the {\it key}. If the elements of the array are
not arranged in any particular order, the only way we can be sure to
find the key, assuming it is in the array, is to search every element,
beginning at the first element, until we find it.


\subsection{Sequential Search}
%<sec_seqsearch:\thisBheadno><pg_sec_seqsearch:\thispageno>
\noindent This approach is known as a {\bf sequential\index{sequential search}
search}, because each element of the array will be examined in
sequence until the key is found (or the end of the array is
reached). A pseudocode description of this algorithm is as follows:

\begin{jjjlisting}
\begin{lstlisting}
1. For each element of the array
2.    If the element equals the key
3.       Return its index
4. If the key is not found in the array
5.    Return -1 (to indicate failure)
\end{lstlisting}
\end{jjjlisting}

This algorithm can easily be implemented in a method that searches an
integer array, which is passed as the method's parameter.  If the key
is found in the array, its location is returned.  If it is not found,
then $-1$ is returned to indicate failure.

\marginfig{chptr09/p430f1.eps}{The {\tt Search} class.}
{fig-p430f1}


The {\tt Search} class (Figs.~9.15 and \ref{fig-search})
provides the Java implementation of the {\tt sequentialSearch()} method.
The method takes two parameters: the array to be searched and the key
to be searched for.  It uses a for statement to examine each element
of the array, checking whether it equals the key or not.  If an
element that equals the key is found, the method immediately returns
that element's index.  Note that the last statement in the method will
only be reached if no element matching the key is found.
%\clearpage

\begin{figure}[bht]
\jjjprogstart
\begin{jjjlisting}
\begin{lstlisting}
public class Search {

  public int sequentialSearch(int arr[], int key) {
    for (int k = 0; k < arr.length; k++)
      if (arr[k] == key)
        return k;
    return -1;           // Failure if this is reached
  } // sequentialSearch()

  public int binarySearch(int arr[], int key) {
    int low = 0;                // Initialize bounds
    int high = arr.length - 1;
    while (low <= high) {   // While not done
      int mid = (low + high) / 2;
      if (arr[mid] == key)
        return mid;            // Success
      else if (arr[mid] < key)
        low = mid + 1;        // Search top half
      else
        high = mid - 1;       // Search bottom half
    }  // while
    return -1;     // Post: if low > high search failed
  } // binarySearch()
}//Search
\end{lstlisting}
\end{jjjlisting}
\jjjprogstop{The {\tt Search} class contains both a {\tt sequentialSearch()}
and a {\tt binarySearch()}.}
{fig-search}
\end{figure}

\JavaTIP[false]{EFFECTIVE DESIGN}{Sentinel Return Value.}{Like Java's
{\tt indexOf()} method, the {\tt sequentialSearch()} returns
a sentinel value ($-1$) to indicate that the key was not found.
This is a common design for search methods.}

\subsection{Binary Search}
\noindent If the elements of an array have been sorted into ascending or
descending order, it is not necessary to search sequentially through
each element of the array in order to find the
key.  Instead, the search algorithm can make use of the knowledge that
the array is ordered and perform what's known as a {\bf
binary\index{binary search} search}, which is a
divide-and-conquer\index{divide-and-conquer} algorithm that divides
the array in half on each iteration and limits its search to just that
half that could contain the key.

To illustrate the binary search, recall the familiar guessing game in
which you try to guess a secret number between 1 and 100, being told
``too high'' or ``too low'' or ``just right'' on each guess.   A good
first guess should be 50. If this is too high, the next guess should
be 25, because if 50 is too high the number must be between 1 and 49.
If 50 was too low, the next guess should be 75, and so on.   After each
wrong guess, a good guesser should pick the midpoint of the sublist
that would contain the secret number.

Proceeding in this way, the correct number can be guessed in at most
%\begin{marginalnote}\it How many guesses?\end{marginalnote}
\marginnote{How many guesses?}
$log_2 N$ guesses, because the base-2 logarithm of {\it N} is the
number of times you can divide {\it N} in half.  For a list of 100
items, the search should take no more than seven guesses ($2^7 = 128
> 100$). For a list of $1,000$ items, a binary search would take at most
ten guesses (2$^{10}=1,024>1,000$).

So a binary search is a much more efficient way to search, provided
the array's elements are in order.  Note that ``order'' here needn't be
numeric order.  We could use binary search to look up a word in a
dictionary or a name in a phone book.

\noindent A pseudocode representation of the binary search is given
as follows:

\begin{jjjlisting}
\begin{lstlisting}
TO SEARCH AN ARRAY OF N ELEMENTS IN ASCENDING ORDER

1. Assign 0 low and assign N-1 to high initially
2. As long as low is not greater than high
3.    Assign (low + high) / 2 to mid
4.    If the element at mid equals the key
5.        then return its index
6.    Else if the element at mid is less than the key
7.        then assign mid + 1 to low
8.    Else assign mid - 1 to high
9. If this is reached return -1 to indicate failure
\end{lstlisting}
\end{jjjlisting}

\noindent Just as with the sequential search algorithm,
this algorithm can easily be implemented in a method that searches an
integer array that is passed as the method's parameter
(Fig.~\ref{fig-search}). If the key is found in the array, its
location is returned.  If it is not found, then $-1$ is returned to
indicate failure.  The {\tt binarySearch()} method takes the same type
of parameters as {\tt sequentialSearch()}. Its local variables, {\tt
low} and {\tt high}, are used as pointers, or references, to the
current low and high ends of the array, respectively.  Note the
loop-entry condition: \verb|low <= high|. If {\tt low} ever becomes
greater than {\tt high}, this indicates that {\tt key} is not
contained in the array.  In that case, the algorithm returns $-1$.

As a binary search progresses, the array is repeatedly cut in half and
{\tt low} and {\tt high} will be used to point to the low and high
index values in that portion of the array that is still being
searched.  The local variable {\tt mid} is used to point to the
approximate midpoint of the unsearched portion of the array.  If the
key is determined to be past the midpoint, then {\tt low} is adjusted
to {\tt mid+1}; if the key occurs before the midpoint, then {\tt high}
is set to \verb|mid-1|.  The updated values of {\tt low} and {\tt
high} limit the search to the unsearched portion of the original
array.

Unlike sequential search, binary search does not have to examine every
location in the array to determine that the key is not in the array.
It searches only that part of the array that could contain the key.
For example, suppose we are searching for $-5$ in the following array:

\begin{jjjlisting}
\begin{lstlisting}
int sortArr[] = 
   { 1,2,3,4,5,6,7,8,9,10,11,12,13,14,15,16,17,18,19,20};
\end{lstlisting}
\end{jjjlisting}

\noindent The $-5$ is
smaller than the smallest array element. Therefore, the algorithm will
repeatedly divide the low end of the array in half until the condition
\verb|low > high| becomes true.  We can see this by tracing the values
that low, mid, and high will take during the search:

\begin{jjjlisting}
\begin{lstlisting}
 Key  Iteration  Low   High    Mid
----------------------------------
 -5   0          0     19      9
 -5   1          0     8       4
 -5   2          0     3       1
 -5   3          0     0       0
 -5   4          0     -1      Failure
\end{lstlisting}
\end{jjjlisting}

\noindent As this trace shows, the algorithm examines only
four locations to determine that $-5$ is not in the array.
After checking location 0, the new value for {\tt high} will become
$-1$, which makes the condition \verb|low <= high | false.  So the
search will terminate.

\marginfig{chptr09/p432f1.eps}{The {\tt TestSearch} class. \\ }
{fig-p432f1}


The {\tt TestSearch} class (Figs.~9.17 and \ref{fig-testsearch})
provides a  program that can be used to test two search methods.
It creates an integer array, whose values are in ascending order.  It
then uses the {\tt getInput()} method to input an integer from the
keyboard and then performs both a {\tt sequentialSearch()} and a {\tt
binarySearch()} for the number.


\secEXRHone{Self-Study Exercise}
\begin{SSTUDY}

\item  For the array containing the elements 2, 4, 6, and so on up to 28 in
that order, draw a trace showing which elements are examined if you
search for 21 using a binary search.

\end{SSTUDY}

\begin{figure}[thb]
\jjjprogstart
\begin{jjjlistingleft}[28pc]{-2pc}
\begin{lstlisting}
import java.io.*;
public class TestSearch {
  public static int getInput() {
    KeyboardReader kb = new KeyboardReader();
    kb.prompt("This program searches for values in an array.");
    kb.prompt(
    "Input any positive integer (or any negative to quit) : ");
    return kb.getKeyboardInteger();
  } // getInput()

  public static void main(String args[]) throws IOException {
    int intArr[] = { 2,4,6,8,10,12,14,16,18,20,22,24,26,28};
    Search searcher = new Search();
    int key = 0, keyAt = 0;
    key = getInput();
    while (key >= 0) {
      keyAt = searcher.sequentialSearch( intArr, key );
      if (keyAt != -1)
        System.out.println("  Sequential: " + key + 
                              " is at intArr[" + keyAt + "]");
      else
        System.out.println("  Sequential: " + key 
                           + " is not contained in intArr[]");
      keyAt = searcher.binarySearch(intArr, key);
      if (keyAt != -1)
        System.out.println("  Binary: " + key + 
                              " is at intArr[" + keyAt + "]");
      else
        System.out.println("  Binary: " + key + 
                             " is not contained in intArr[]");
      key = getInput();
    } // while
  } // main()
} // TestSearch
\end{lstlisting}
\end{jjjlistingleft}
\jjjprogstop{The {\tt TestSearch} class.}
{fig-testsearch}
\end{figure}


\section{Two-Dimensional Arrays}
\label{two-dimensional-arrays}
\noindent A {\bf two-dimensional array}, an array whose components are
themselves arrays, is necessary or useful for certain kinds of
problems.  For example, you would use this type of array if you are
doing a scientific study in which you have to track the amount of
precipitation for every day of the year.

One way to organize these data would be to create a one-dimensional
array, consisting of 365 elements:

\begin{jjjlisting}
\begin{lstlisting}
double rainfall[] = new double[365];
\end{lstlisting}
\end{jjjlisting}

\noindent However, with this representation, it would make it very
difficult to calculate the average rainfall within a given month,
which might be an important part of your study.

A better representation for this problem would be to
\marginnote{What data do we need?}
use a two-dimensional array, one dimension for the months and one
for the days.  The following statement declares the array
variable {\tt rainfall} and creates a 12 by 31 array object
as its reference:

\begin{jjjlisting}
\begin{lstlisting}
double rainfall[][] = new double[12][31];
\end{lstlisting}
\end{jjjlisting}

\noindent Thus, {\tt rainfall} is an {\it array of arrays}.
You can think of the first array as the 12 months required for the
problem.  And you can think of each month as an array of 31 days.
The months will be indexed from 0 to 11, and the days will be
indexed from 0 to 30.

The problem with this representation is that when we want to refer to
\marginnote{Choosing an appropriate\hfill\break representation}
the rainfall for January 5, we would have to use
{\tt rainfall[0][4]}. This is awkward and misleading.  The problem is that
dates---1/5/1999---are unit indexed, while arrays are zero
indexed.  Because it will be difficult to remember this fact, our
representation of the rainfall data may cause us to make errors when
we start writing our algorithms.

We can easily remedy this problem by just defining our array to
have an extra month and an extra day each month:

\begin{jjjlisting}
\begin{lstlisting}
double rainfall[][] = new double[13][32];
\end{lstlisting}
\end{jjjlisting}

\noindent This representation creates an array with 13 months, indexed
from 0 to 12, with 32 days per month, indexed from 0 to 31. However,
we can simply ignore the 0 month and 0 day by using unit indexing in
all of the algorithms that process the array.  In other words,
if we view this array as a two-dimensional table, consisting of
13 rows and 32 columns, we can leave row 0 and column 0 unused
(Fig.~\ref{fig-rainfall}).

\begin{figure}[bth]
%%\figaleft{chptr09/8f14.eps}{A two-dimensional array with 13 rows
\figa{chptr09/8f14.eps}{A two-dimensional array with 13 rows
and 32 columns.  To represent 12 months of the year, we can simply
ignore row 0 and column 0.}
{fig-rainfall}
\end{figure}

As Figure~\ref{fig-rainfall} shows, the very first element of this
416-element array has subscripts (0,0) while the last location has
subscripts (12,31). The main advantages of this representation is that
the program as a whole will be much easier to read and understand and
much less prone to error.

\JavaTIP[false]{EFFECTIVE DESIGN}{Readability.}{To improve a program's robustness and 
readability, it may be preferable to use unit array indexing by
declaring extra array elements and ignoring those with index 0.}

In order to refer to an element in a two-dimensional array, you need
to use two subscripts.  For the {\tt rainfall} array, the first
subscript will specify the {\it month} and the second will specify the
{\it day} within the month.  Thus, the following statements assign 1.15 to
the {\tt rainfall} element representing January 5, and then print its
\marginnote{Referencing two-dimensional arrays}
value:

\begin{jjjlisting}
\begin{lstlisting}
rainfall[1][5] = 1.15;   // Rainfall for January 5
System.out.println( rainfall[1][5] );
\end{lstlisting}
\end{jjjlisting}

Just as in the case of one-dimensional arrays, it is an error to
attempt to reference an element that is not in the array.  Each of the
following examples would cause Java to raise an
{\tt IndexOutOfBoundsException}\index{IndexOutOfBoundsException}:

\begin{jjjlisting}
\begin{lstlisting}
rainfall[13][32] = 0.15 ;  // No such element
rainfall[11][33] = 1.3;    // No such column
rainfall[14][30] = 0.74;   // No such row
\end{lstlisting}
\end{jjjlisting}

If the initial values of an array's elements are supposed to be zero,
there is no need to initialize the elements.  Java will do it
automatically when you create the array with {\tt new}. However, for
\marginnote{Initializing two-dimensional arrays}
many array problems it is necessary to initialize the array elements
to some other value.  For a two-dimensional array, this would require a
nested loop.  To illustrate this algorithm, let's use a nested for loop
to initialize each element of the {\tt rainfall} array to 0:

\begin{jjjlisting}
\begin{lstlisting}
// Note that both loops are unit indexed.
for (int month = 1; month < rainfall.length; month++)
  for (int day = 1; day < rainfall[month].length; day++)
     rainfall[month][day] = 0.0;
\end{lstlisting}
\end{jjjlisting}

\noindent Note that both for loops use unit indexing.  This is
in keeping with our decision to leave month 0 and day 0 unused.

Remember that when you have a nested for loop, the inner loop iterates
%\begin{marginalnote}\it Nested for loops\end{marginalnote}
\marginnote{Nested for loops}
faster than the outer loop.  Thus, for each month, the inner loop will
iterate over 31 days.  This is equivalent to processing the array
as if you were going across each row and then down to the next
row in the representation shown in Figure~\ref{fig-rainfall}.

Note that for a two-dimensional array, both dimensions have an
associated {\tt length} variable, which is used in this example to
specify the upper bound of each for loop.  For the {\tt rainfall}
array, the first dimension (months) has a length of 13 and the second
dimension (days) has a length of 32.

Another way to view the {\tt rainfall} array is to remember that it is
an array of arrays. The length of the first array, which
\marginnote{Array of arrays}
corresponds to the number (13) of months, is given by
{\tt rainfall.length}. The length of each month's array, which corresponds
to the number of days (32) in a month, is given by
{\tt rainfall[month].length}.

The outer loop of the nested for loop iterates through months 1
through 12, and the inner for loop iterates through days 1 through
31.  In this way,  $372\; =\; 12\;\times\; 31$ elements of the array are set to 0.0.
In Table~9.1, the boldface numbers along the top
represent the day subscripts, while the boldface numbers along the
left represent the month subscripts.


\begin{table}[htb]
%\hphantom{\caption{The initialized {\tt rainfall} array.  The unused
%array elements are shown as dashes.}}
\TBT{1.5pc}{The initialized {\tt rainfall} array.  The unused array \\
\hspace*{1.5pc}elements are shown as dashes.}
\hspace*{1pc}\begin{tabular}{cccccccc}
\multicolumn{8}{l}{\color{cyan}\rule{24pc}{1pt}}\\[2pt]
 & {\bf0} & {\bf1}  & {\bf2} & {\bf3} &  {$\cdots$} & {\bf30} & {\bf31}
\\[-4pt]\multicolumn{8}{l}{\color{cyan}\rule{24pc}{0.5pt}}\\[2pt]
{\bf0} & \hspace*{10pt}--\hspace*{10pt}       & \hspace*{10pt}-- \hspace*{10pt}       &\hspace*{10pt} --
\hspace*{10pt}      & \hspace*{10pt}-- \hspace*{10pt}      &
\hspace*{10pt}$\cdots$ \hspace*{10pt}       &\hspace*{10pt} --\hspace*{10pt}        &\hspace*{10pt}
--\hspace*{10pt}       \\[2pt]
{\bf1} & --       & 0.0        & 0.0       & 0.0       & $\cdots$     & 0.0       & 0.0       \\[2pt]
{\bf2} & --       & 0.0        & 0.0       & 0.0       & $\cdots$     & 0.0       & 0.0       \\[0pt]
{$\vdots$}  &  $\vdots$   &  $\vdots$    &  $\vdots$   &  $\vdots$   &  $\vdots$    &  $\vdots$   & $\vdots$    \\[4pt]
{\bf10} & --       & 0.0        & 0.0       & 0.0       & $\cdots$     & 0.0       & 0.0       \\[2pt]
{\bf11} & --       & 0.0        & 0.0       & 0.0       & $\cdots$     & 0.0       & 0.0       \\[2pt]
{\bf12} & --       & 0.0        & 0.0       & 0.0       & $\cdots$     & 0.0       & 0.0       
\\[-4pt]\multicolumn{8}{l}{\color{cyan}\rule{24pc}{1pt}}
\end{tabular}
\endTB
\end{table}


\secEXRHone{Self-Study Exercises}
\begin{SSTUDY}

\item  Declare a two-dimensional array of {\tt int}, named {\tt int2d},
that contains five rows,  each of which  contains ten integers.


\item  Write a statement that prints the last integer in the third row of the
array that you created in the previous exercise.  Then write an assignment
statement that assigns 100 to the last element in the {\tt int2d}
array.

\item Write a loop to print all of the elements of {\tt int2d}, which
you declared in the previous exercise.  Print one row per line with a
space between each element on a line.

\end{SSTUDY}

\subsection{Two-Dimensional Array Methods}
\noindent Now that we have figured out how to represent the data for our
scientific experiment, let's develop methods to calculate some
results.   First, we want a method to initialize the array.  This method
will simply incorporate the nested loop algorithm we developed previously:

\begin{jjjlisting}
\begin{lstlisting}
public void initRain(double rain[][]) {
    for (int month = 1; month < rain.length; month++)
        for (int day = 1; day < rain[month].length; day++)
            rain[month][day] = 0.0;
} // initRain()
\end{lstlisting}
\end{jjjlisting}

\noindent Note how we declare the parameter for a multidimensional
array.  In addition to the element type ({\tt double}), and the name of
%\begin{marginalnote}\it Array parameters\end{marginalnote}
\marginnote{Array parameters}
the parameter ({\tt rain}), we must also include a set of brackets
for {\it each} dimension of the array.

Note also that we use the parameter name within the method to refer
to the array.  As with one-dimensional arrays, the parameter is a
reference to the array, which means that any changes made to the array
within the~method will persist when the method is exited.

\subsubsection*{The {\tt avgDailyRain()} Method}
\noindent One result that we need from our experiment is the average daily 
\marginnote{Algorithm design}
rainfall.  To calculate this result, we would add up all of the
rainfalls stored in the $12 \;\times\; 31$ array and divide by 365.
Of course, the array itself contains more than 365 elements.   It
contains 416 elements,  but we're not using the first month of the
array, and within some months---those with fewer than 31 days---we're not using some of 
the day elements.   For example, there's no
such day as {\tt rainfall[2][30]}, which would represent February 30.
However, because we initialized all of the array's elements to 0, the
rainfall recorded for the non-days will be 0, which won't affect our
overall average.

The method for calculating average daily rainfall should take
%\begin{marginalnote}\it Method design\end{marginalnote}
\marginnote{Method design}
our two-dimensional array of \mbox{\tt double} as a parameter, and it
should return a \mbox{\tt double}. Its algorithm will use a nested
for loop to iterate through the elements of the array, adding
each element to a running total.  When the loops exits, the
total will be divided by 365 and returned:

\begin{jjjlisting}
\begin{lstlisting}
public double avgDailyRain(double rain[][]) {
    double total = 0;
    for (int month = 1; month < rain.length; month++)
        for (int day = 1; day < rain[month].length; day++)
            total += rain[month][day];
    return total/365;
} // avgDailyRain()
\end{lstlisting}
\end{jjjlisting}

\subsubsection*{The {\tt avgRainForMonth()} Method}
\noindent One reason we used a two-dimensional array for this problem is so we
could calculate the average daily rainfall for a given month.  Let's
write a method to solve this problem.  The algorithm for this method
%\begin{marginalnote}\it Algorithm design\end{marginalnote}
\marginnote{Algorithm design}
will not require a nested for loop. We will just iterate
through the 31 elements of a given month, so the month subscript
will not vary.  For example, suppose we are calculating the average
for January, which is represented in our array as month 1:

\begin{jjjlisting}
\begin{lstlisting}
double total = 0;
for (int day = 1; day < rainfall[1].length; day++)
    total = total + rainfall[1][day];
\end{lstlisting}
\end{jjjlisting}

\noindent Thus, the month subscript is held constant (at 1)
while the day subscript iterates from 1 to 31. Of course,
in our method we would use a parameter to represent the month,
%\begin{marginalnote}\it Method design\end{marginalnote}
\marginnote{Method design}
thereby allowing us to calculate the average daily rainfall for any
given month.


Another problem that our method has to deal with is that months don't
all have 31 days, so we can't always divide by 31 to compute the
monthly average.  There are various ways to solve this problem, but
%\begin{marginalnote}\it Method design: What data do we need?\end{marginalnote}
\marginnote{Method design: What data do we need?}
perhaps the easiest is to let the number of days for that month be
specified as a third parameter.  That way, the month itself and the
number of days for the month are supplied by the user of the method:

\begin{jjjlisting}
\begin{lstlisting}
public double avgRainForMonth(double rain[][], 
                           int month, int nDays) {
    double total = 0;
    for (int day = 1; day < rain[month].length; day++)
        total = total + rain[month][day];
    return total/nDays;
} // avgRainForMonth()
\end{lstlisting}
\end{jjjlisting}

\noindent Given this definition, we can call this method to
calculate and print the average daily rainfall for March as in 
the following statement:

\begin{jjjlisting}
\begin{lstlisting}
System.out.println("March: " + 
                         avgRainForMonth(rainfall,3,31));
\end{lstlisting}
\end{jjjlisting}

\noindent Note that when passing the entire two-dimensional array to
the method, we just use the name of the array.  We do not have to
follow the name with \mbox{subscripts.}

\subsection{Passing Part of an Array to a Method}
\noindent Instead of passing the entire rainfall array to the
{\tt avgRainForMonth()} method, we could redesign this method so that it
%\begin{marginalnote}\it Method design: What data?\end{marginalnote}
is only passed the particular month that's being averaged.  Remember
that a two-dimensional array is an array of arrays, so if we pass the
month of January, we are passing an array of 32 days.  If we use this
\marginnote{Method design: What data?}
approach, we need only two parameters: the month, which is array of
days, and the number of days in that month:

\begin{jjjlisting}
\begin{lstlisting}
public double avgRainForMonth(double monthRain[], 
                                          int nDays) {
    double total = 0;
    for (int day = 1; day < monthRain.length; day++)
        total = total + monthRain[day];
    return total/nDays;
} // avgRainForMonth()
\end{lstlisting}
\end{jjjlisting}

\noindent Given this definition, we can call it to
calculate and print the average daily rainfall for March as
in the following statement:

\begin{jjjlisting}
\begin{lstlisting}
System.out.println("March: " + 
                      avgRainForMonth(rainfall[3],31));
\end{lstlisting}
\end{jjjlisting}

\noindent In this case, we're passing an array of {\tt double}
to the method, but in order to reference it, we have to pull it out of
the two-dimensional array by giving its {\it row} subscript as well.
Thus, {\tt rainfall[3]} refers to one month of data in the
two-dimensional array, the month of March.  But
{\tt rainfall[3]} is itself a one-dimensional array.
Figure~\ref{fig-twodrow} helps to clarify this point.

\begin{figure}[h!]
%%\figaleft{chptr09/8f15.eps}{Referencing individual elements and
\figa{chptr09/8f15.eps}{Referencing individual elements and
array elements in a two-dimensional array.}
{fig-twodrow}
\end{figure}

It's important to note that deciding whether to use brackets when
%\begin{marginalnote}\it Specifying an argument\end{marginalnote}
\marginnote{Specifying an argument}
passing data to a method is not just a matter of whether you are
passing an array.  It is a matter of what type of data the method
parameter specifies.   So, whenever you call a method that involves a
parameter, you have to look at the method definition to see what kind
of data that parameter specifies.  Then you must supply an argument
that refers to that type of data.

For our two-dimensional {\tt rainfall} array, we can refer to the
entire array as {\tt rainfall}. We can refer to one of its months as
{\tt rainfall[j]}, where {\it j} is any integer between 1 and 12. And
we can refer to any of its elements as {\tt rainfall[j][k]}, where
{\it j} is any integer between 1 and 12, and {\it k} is any integer
between 1 and 31.

\JavaRule{Arguments and Parameters.}{The argument in a
method call must match the data type in the method definition.  This
applies to all parameters, including array parameters.}


\noindent The {\tt Rainfall} class (Figs.~9.21 and \ref{fig-rainfallclass})
\begin{figure}[htb]
\figa{chptr09/p440f1.eps}{The {\tt Rainfall} class.}
{fig-p440f1}
\end{figure}
\begin{figure}[p!]
\jjjprogstart
\begin{jjjlisting}[32pc]
\begin{lstlisting}
public class Rainfall {
  /**
   * Initializes the rainfall array
   * @param rain is a 2D-array of rainfalls
   * Pre:  rain is non null
   * Post: rain[x][y] == 0 for all x,y in the array
   * Note that the loops use unit indexing.
   */
  public void initRain(double rain[][]) {
    for (int month = 1; month < rain.length; month++)
      for (int day = 1; day < rain[month].length; day++)
        rain[month][day] = Math.random() * 2.0;    // Random rainfall
  } // initRain()
  /**
   * Computes average daily rainfall for a year of rainfall data
   * @param rain is a 2D-array of rainfalls
   * @return The sum of rain[x][y] / 356
   * Pre:  rain is non null
   * Post: The sum of rain / 365 is calculated
   * Note that the loops are unit indexed
   */
  public double avgDailyRain(double rain[][]) {
    double total = 0;
    for (int month = 1; month < rain.length; month++)
      for (int day = 1; day < rain[month].length; day++)
        total += rain[month][day];
    return total/365;
  } // avgDailyRain()
  /**
   * Computes average daily rainfall for a given month containing nDays
   * @param monthRain is a 1D-array of rainfalls
   * @param nDays is the number of days in monthRain
   * @return The sum of monthRain / nDays
   * Pre:  1 <= nDays <= 31
   * Post: The sum of monthRain / nDays is calculated
   */
  public double avgRainForMonth(double monthRain[], int nDays) {
    double total = 0;
    for (int day = 1; day < monthRain.length; day++)
      total = total + monthRain[day];
    return total/nDays;
  } // avgRainForMonth()
  public static void main(String args[]) {
    double rainfall[][] = new double[13][32];
    Rainfall data = new Rainfall();
    data.initRain(rainfall);
    System.out.println("The average daily rainfall = " 
                      +  data.avgDailyRain(rainfall));
    System.out.println("The average daily rainfall for March = " 
                      +  data.avgRainForMonth(rainfall[3],31));
  } // main()
}//Rainfall
\end{lstlisting}
\end{jjjlisting}
\jjjprogstop{Definition of the {\tt Rainfall} class.}
{fig-rainfallclass}
\end{figure}
shows how we can test our array algorithms.  It creates the {\tt
rainfall} array in the {\tt main()} method.  It then initializes the
array and prints out average daily rainfall and average daily rainfall
for the month of March.  However, note that we have made a slight
modification to the {\tt initRain()} method.  Instead of just
assigning 0 to each element, we assign a random value between 0 and
2.0:


\begin{jjjlisting}
\begin{lstlisting}
rain[month][day] = Math.random() * 2.0;
\end{lstlisting}
\end{jjjlisting}

\noindent Using the {\tt Math.random()} method in this way enables
us to generate some realistic test data.  In this case, we
%\begin{marginalnote}\it Generating test data\end{marginalnote}
\marginnote{Generating test data}
have scaled the data so that the daily rainfall is between 0 and 2
inches.  (Rainfall like this would probably be appropriate for an
Amazonian rain forest!)  Testing our algorithms with these data
provides some indication that our methods are in fact working
properly.



\JavaTIP{EFFECTIVE DESIGN}{Generating Test Data.}{The {\tt Math.random()}
method can be used to generate numeric test data, when large amounts
of data are required.  The data can be scaled to fit within the
range that the actual data are expected to have.}

\secEXRHone{Self-Study Exercises}
\begin{SSTUDY}

\item  Suppose you're going to keep track of the daily newspaper sales at the
local kiosk.   Declare a $52\;\times\;7$ two-dimensional array of {\tt int}
and initialize each of its elements to 0.


\item  Write a method to calculate the average number of newspapers sold per week,
using the array you declared in the previous \mbox{exercise.}


\item  Write a method to calculate the average number of newspapers sold on Sundays,
using the array you declared in the previous exercise.  Assume that
Sunday is the last day of the week.%

\end{SSTUDY}

\markright{{\color{cyan}SECTION\,\,\arabic{section}\,\,$\bullet$}\,\, Multidimensional Arrays}

\section{Multidimensional Arrays (Optional)}
\label{multidimensional-arrays-optional}

\markright{{\color{cyan}SECTION\,\,\arabic{section}\,\,$\bullet$}\,\, Multidimensional Arrays}

\noindent Java doesn't limit arrays to just two dimensions.  For example, suppose
we decide to extend our rainfall survey to cover a ten-year
period.  For each year we now need a two-dimensional array.  This
results in a three-dimensional array consisting of an array of years,
each of which contains an array of months, each of which contains an
array of days:


\begin{jjjlisting}
\begin{lstlisting}
final int NYEARS = 10;
final int NMONTHS = 13;
final int NDAYS = 32;
double rainfall[][] = new double[NYEARS][NMONTHS][NDAYS];
\end{lstlisting}
\end{jjjlisting}

\noindent Following the design convention of not using the
0 month and 0 days, we end up with a $10 \;\times\; 13 \;\times\; 32$ array.
Note the use of {\tt final} variables to represent the size of each
dimension of the array.  This helps to make the program more readable.

In Figure~\ref{fig-threedarray}, each year of the rainfall data is
represented as a separate page. On each page, there is a
two-dimensional table that consists of 12 rows (1 per month) and 31
columns (1 per day).


\begin{figure}[htb]
\figaright{chptr09/8f17.eps}{Three-dimensional data might be viewed as a
collection of pages, each of which contains a two-dimensional table.
} {fig-threedarray}
\end{figure}

You might imagine that our study could be extended to
cover rainfall data from a number of different cities.  That would
result in a four-dimensional array, with the first dimension now being
the city.   Of course, for this to work, cities would have to be
represented by integers, because array subscripts must be integers.


As you might expect, algorithms for processing each element
in a three-dimensional table would require a three-level
nested loop.  For example, the following algorithm would
be used to initialize all elements of our three-dimensional
rainfall array:

\begin{jjjlistingleft}[28pc]{-2pc}
\begin{lstlisting}
for (int year = 0; year < rainfall.length; year++)
 for (int month = 0; month < rainfall[year].length; month++)
  for (int day = 0; day < rainfall[year][month].length; day++)
   rainfall[year][month][day] = 0.0;
\end{lstlisting}
\end{jjjlistingleft}


\noindent Note again the proper use of the {\tt length}\index{array length}
attribute for each of the three dimensions of the array.   In the outer
loop, {\tt rainfall.length}, we're referring to the number of
years.  In the middle loop, {\tt rainfall[year].length}, we're
referring to number of months within a given year.  In the inner loop,
{\tt rainfall[year][month].length}, we're referring to the number of
days within a month.

If we added a fourth dimension to our array and wanted to extend this
algorithm to initialize it, we would simply embed the three-level loop
within another for loop that would iterate over each city.

\subsection{Array Initializers}
\noindent It is possible to use an initializer\index{initializer, array} with a
{\bf multidimensional array}\index{multidimensional array}.  For instance, the
following examples create several small arrays and initialize their
elements:

\begin{jjjlisting}
\begin{lstlisting}
int a[][] = {{1,2,3}, {4,5,6}};
char c[][] = {{'a','b'}, {'c','d'}};
double d[][][] = 
    {{1.0,2.0,3.0}, {4.0,5.0}, {6.0,7.0,8.0,9.0}};
\end{lstlisting}
\end{jjjlisting}

\noindent The first of these declarations creates a $2\; \times \;3$ array of
integers.   The second example creates a $2\; \times\; 2$ array of
characters, and the third example creates an array of {\tt double}
consisting of three rows, each of which has a different number of
elements.   The first row contains three elements, the second contains two
elements, and the last row contains four elements.  As this last example
shows, the rows in a multidimensional array don't all have to have the
same length.  

Using initializers, as in these examples, is feasible only for
relatively small arrays.  To see why, just imagine what the initializer
expression would be for our three-dimensional {\tt rainfall} array.  It
would require $4,160 \;=\; 10 \; \times \; 13 \; \times \; 32$ zeroes,
separated by commas!

\JavaTIP{PROGRAMMING TIP}{Array Initializers.}{Initializer (assignment)
expressions can be used to assign initial values to relatively small
arrays.  For larger arrays, an initializer method should be designed.}

\markright{{\color{cyan}SECTION\,\,\arabic{section}\,\,$\bullet$}\,\, OOD: Polymorphic Sorting}

\section{OBJECT-ORIENTED DESIGN: \\[2pt]Polymorphic Sorting (Optional)}
\label{objectorienteddesign2pt-polymorphic-sorting-optional}

\markright{{\color{cyan}SECTION\,\,\arabic{section}\,\,$\bullet$}\,\, OOD: Polymorphic Sorting}

\noindent One limitation of the sort routines developed so far is that they only
work on one particular type of data. If you've written an insertion
sort to sort {\tt int}s, you can't use it to sort {\tt double}s. What
would be far more desirable is a {\bf polymorphic sort method}---that
is, one method that could sort any kind of data. This is easily done
by making use of Java wrapper classes, such as {\tt Integer} and {\tt
Double}, together with the {\tt java.lang.Comparable} interface, which
is specially designed for this purpose.

The {\tt java.lang.Comparable} interface consists of the
\mbox{\tt compareTo()} method:

\begin{jjjlisting}
\begin{lstlisting}
public abstract interface Comparable {
  public int compareTo(Object o);// Abstract method
}
\end{lstlisting}
\end{jjjlisting}

\noindent By implementing {\tt compareTo()}, a class
\marginfig{chptr09/p443f1.eps}{Java wrapper classes, such as
{\tt Integer} and {\tt Double}, implement the {\tt Comparable} interface.}
{fig-p443f1}
can impose an order on its objects. The {\tt Comparable}
interface is implemented by all of Java's wrapper classes---that is,
by {\tt Integer}, {\tt Double}, {\tt Float}, {\tt Long}, and so on
(Fig.~9.24).

As we saw in Chapter~8, Java interfaces allow us to create a form of
{\em multiple inheritance}. For example, as Figure~9.24 shows, an
{\tt Integer} is both an \mbox{\tt Object} and a {\tt Comparable}. One
implication of this is that an {\tt Integer} can be used in any method
that takes either an {\tt Object} parameter or a {\tt Comparable}
parameter.

The {\tt compareTo()} method takes an {\tt Object} parameter and
returns an {\tt int}. It is meant to be invoked as {\tt
o1.compareTo(o2)}, where {\em o1} and {\em o2} are objects of the same
type. Classes that implement {\tt compareTo()} must abide by the
following rules for its return value:

\begin{jjjlisting}
\begin{lstlisting}
if (o1 < o2)       then o1.compareTo(o2) < 0
if (o1.equals(o2)) then o1.compareTo(o2) == 0
if (o1 > o2)       then o1.compareTo(o2) > 0
\end{lstlisting}
\end{jjjlisting}

\noindent In other words, if {\tt o1 < o2}, then {\tt o1.compareTo(o2)} 
will return a negative integer. If {\tt o1 > o2}, then {\tt
o1.compareTo(o2)} will return a positive integer. And if {\tt o1} and
{\tt o2} are equal, then {\tt o1.compareTo(o2)} will return~0.

For a class that implements {\tt Comparable}, we can use the \mbox{\tt
compareTo()} method to help sort its elements. For example, the
following revised version of {\tt insertionSort()} method can be used
to sort any array of {\tt Comparable} objects---that is, any array of
objects whose class implements {\tt Comparable}:

\begin{jjjlistingleft}[27pc]{-1pc}
\begin{lstlisting}
public void sort(Comparable[] arr) {
  Comparable temp;   // Temporary variable for insertion
  for (int k = 1; k < arr.length; k++)  {
    temp = arr[k];    // Remove it from array
    int i;
    for (i = k-1; i >= 0 && arr[i].compareTo(temp) > 0; i--) 
      arr[i+1] = arr[i];   // Move it right by one
    arr[i+1] = temp;        // Insert the element
  }
} // sort()
\end{lstlisting}
\end{jjjlistingleft}

\noindent In this version, the parameter is an array of
{\tt Comparable}. Thus, we can pass it any array whose elements
implement {\tt Comparable}, including an array of {\tt Integer} or
{\tt Float}, and so on.  Then, to compare elements of a {\tt
Comparable} array, we use the {\tt compareTo()} method:

\begin{jjjlisting}
\begin{lstlisting}
for (i = k-1; i >= 0 && arr[i].compareTo(temp) > 0; i--) 
\end{lstlisting}
\end{jjjlisting}

\noindent Note that our algorithm no longer refers to {\tt int}s, as
in the original insertion sort. Indeed, it doesn't mention the
specific type---{\tt Integer}, {\tt Float}, or whatever---of the
objects that it is sorting. It refers only to {\tt Comparable}s.
Therefore, we can use this method to sort any type of object, as long
as the object's class implements the {\tt Comparable} interface. Thus,
by using {\tt Comparable}, we have a more general {\tt
insertionSort()} method, one that can sort any one-dimensional array
of {\tt Comparable}s. 

The {\tt TestSort} class (Figs.~9.25 and ~\ref{fig-testsort}) provides
an example of how to use the polymorphic {\tt sort()} method.
\marginfig{chptr09/p445f1.eps}{The {\tt TestSort()} class.}
{fig-p445f1}

It contains three methods: The {\tt sort()} method that we just
described; a polymorphic {\tt print()} method, which can be used to
print the values of any array of {\tt Comparable}; and a {\tt main()}
method. The {\tt main()} method creates arrays of {\tt Integer} and
{\tt Float} and then uses the polymorphic {\tt sort()} method to sort
them.  Note how the {\tt print()} method uses the polymorphic {\tt
toString()} method to print the elements of a {\tt Comparable} array.

\begin{figure}[bth]
\jjjprogstart
\begin{jjjlistingleft}[28pc]{-2pc}
\begin{lstlisting}
public class TestSort {
  public static int MAXSIZE = 25;

  public void sort(Comparable[] arr) {
    Comparable temp;      // Temporary variable for insertion
    for (int k = 1; k < arr.length; k++) { // For each element
      temp = arr[k];                     // Remove it 
      int i;
      for (i = k-1; i >= 0 && arr[i].compareTo(temp) > 0; i--) 
        arr[i+1] = arr[i]; // Move larger to the right
      arr[i+1] = temp;       // Insert the element
    }
  } // sort()
  public void print(Comparable arr[]) {
    for (int k = 0; k < arr.length; k++) {
      if (k % 5 == 0)  System.out.println();  // New row  
        System.out.print(arr[k].toString() + "\t"); 
    }
    System.out.println();
  }
  // Sorts two different types of array with the same method.
  public static void main(String args[]) {
    Integer iArr[] = new Integer[TestSort.MAXSIZE]; 
    Float fArr[] = new Float[TestSort.MAXSIZE];   
    for (int k = 0; k < TestSort.MAXSIZE; k++) { // Create data
      iArr[k] = new Integer((int) (Math.random() * 10000));
      fArr[k] = new Float(Math.random() * 10000);
    }
    TestSort test = new TestSort();
    test.sort(iArr);     // Sort and print Integers
    test.print(iArr);
    test.sort(fArr);     // Sort and print Floats
    test.print(fArr);                      
  } // main()
}
\end{lstlisting}
\end{jjjlistingleft}
\jjjprogstop{{\tt TestSort} uses the polymorphic {\tt sort()} method
to sort either {\tt Integer}s or {\tt Float}s. }
{fig-testsort}
\end{figure}


This example of polymorphic sorting illustrates once again the great
power of inheritance and polymorphism in object-oriented programming.
The {\tt Integer} and {\tt Float} classes use class inheritance to
inherit features from the {\tt Object} class, and they use interface
implementation to inherit the {\tt compareTo()} method from the {\tt
Comparable} class.  By implementing versions of the {\tt toString()}
and {\tt compareTo()} methods that are appropriate for these wrapper
classes, Java makes it easier to use {\tt Integer} and {\tt Float}
objects in a variety of contexts.  Taken together, inheritance and
polymorphism enable us to design very general and extensible
algorithms.  In this example, we defined a {\tt sort()} method that
can sort an array containing any kind of object as long as the object
implements the {\tt Comparable} interface. 

\subsection{The {\tt java.util.Arrays.sort()} Method}

\noindent While sorting algorithms provide a good way to introduce the
concepts of array processing, real-world programmers never write their
own sort algorithms. Instead they use library methods, which have been
written and optimized by programming experts. Moreover, library sort
routines use sort algorithms that are much more efficient than the
ones we've discussed.

The {\tt java.util.Arrays} class contains a polymorphic sort
method that is very simple to use. For example, here's how we
would use it to sort the two arrays declared in the {\tt TestSort}
program:

\begin{jjjlisting}
\begin{lstlisting}
java.util.Arrays.sort(iArr);
java.util.Arrays.sort(fArr);
\end{lstlisting}
\end{jjjlisting}

\noindent That's all there is to it! Obviously, learning how to use
Java's class and method library, saves real-word programmers lots of
effort.

\marginfig{chptr09/p446f1.eps}{The {\tt java.util.Vector} class.}
{fig-p446f1}


\secEXRHone{Self-Study Exercises}
\begin{SSTUDY}

\item  Add a definition of a {\tt compareTo()} method to the
{\tt LetterFreq} class so that it implements the {\tt Comparable}
interface.  The method should define one object to be less than
another object if its {\tt freq} instance variable is less.



\item  Add a definition of a {\tt sort()} method that  can
be added to the definition of the {\tt AnalyzeFreq} class. Make it so
the array in the class can be sorted into ascending order using the
ordering of {\tt LetterFreq} defined in the previous exercise.
Use the {\tt java.util.Arrays.sort()} method.

\item  Rewrite the {\tt main()} of the {\tt AnalyzeFreq} class to make
use of the {\tt sort()} method of the previous exercise.
\end{SSTUDY}


\section{From the Java Library: java.util.Vector}
\label{sec-vectors}
\WWWjava

The {\tt java.util.Vector} class implements an array of objects that
can grow in size as needed.  One limitation of regular arrays is that
their lengths remain fixed.  Once the array is full---once every
element is used---you can't allocate additional elements.

The {\tt Vector} class contains methods for storing and retrieving
objects, and for accessing objects by their index position within the
{\tt Vector} (Fig.~9.27).

One use for a {\tt Vector} would be when a program needs to store
input from the user or a file without knowing in advance how
many items there are.  Using a {\tt Vector} is less efficient than an
array in terms of processing speed, but it gives you the flexibility
of growing the data structure to meet the storage
requirements.

As an illustration of this idea, the program in
Figure~\ref{fig-vectordemo} creates a random number of integers and
then stores them in a {\tt Vector}. The {\tt Vector}, which is
declared and instantiated in {\tt main()}, is initially
empty.  Integers from 0 to the random {\tt bound} are then inserted
into the {\tt Vector}. In this case, insertions are done with the
{\tt addElement()} method, which causes the {\tt Vector} object to insert
the element at the next available location, increasing its size, if
necessary.


\begin{figure}[hb]
\jjjprogstart
\begin{jjjlisting}[27pc]
\begin{lstlisting}
import java.util.Vector;

public class VectorDemo {

  public static void printVector(Vector v) {
    for (int k=0; k < v.size(); k++)
      System.out.println(v.elementAt(k).toString());
  } // printVector()

  public static void main(String args[]) {
    Vector vector = new Vector();  // An empty vector

    int bound = (int)(Math.random() * 20);
    for (int k = 0; k < bound; k++ )   // Insert a random
      vector.addElement(new Integer(k));// number of Integers
    printVector(vector);            // Print the elements
  } // main()
}// VectorDemo
\end{lstlisting}
\end{jjjlisting}
\jjjprogstop{Demonstration of the {\tt Vector} class.  }
{fig-vectordemo}
\end{figure}

Once all the integers have been inserted, the {\tt printVector()}
method is called.  Note that it uses the {\tt size()} method to
determine how many elements the {\tt Vector} contains.  This is similar
to using the {\tt length()} method to determine the number of
characters in a {\tt String}.

Finally, note that a {\tt Vector} stores objects.  It cannot be used to
%\begin{marginalnote}\it Vectors store objects\end{marginalnote}
store primitive data values.  You cannot store an {\tt int} in a
{\tt Vector}. Therefore, we need to use the {\tt Integer} wrapper class to
convert {\tt int}s into {\tt Integer}s before they can be inserted
into the {\tt Vector}. Because you can't just print an {\tt Integer},
or any other {\tt Object}, the {\tt toString()} method is used to
print the string representation of the object.

By defining {\tt Vector} to store {\tt Object}s, Java's designers have
\marginnote{Vectors store objects}
made it as general as possible and, therefore, as widely useful as 
possible.  


\JavaTIP[false]{EFFECTIVE DESIGN}{Generality.}{Defining a data collection, such
as an array or a {\tt Vector}, in terms of the {\tt Object} class
makes it capable of storing and processing any type of value,
including values of primitive data types.  This is because the
{\tt Object} class is the root of~the~Java class hierarchy.}

\section{Case Study: An N-Player Computer Game}
\label{case-study-an-n-player-computer-game}

\noindent In this section we will make use of arrays to extend our
game-playing library by developing a design that can support games
that involve more than two players. We will use an array to store a
variable number of players. Following the object-oriented design
principles described in Chapter~8, we will make use of inheritance and
polymorphism to develop a design that is flexible and extensible, 
one that can be used to implement a wide variety of computer games.
As in our {\tt TwoPlayer} game example from Chapter~8, our design will
allow both humans and computers to play the games.  To help simplify
the example, we will modify the {\tt WordGuess} game that we developed
in the Chapter~8. As you will see, it requires relatively few
modifications to convert it from a subclass of {\tt TwoPlayerGame} to
a subclass of {\tt ComputerGame}, the superclass for our N-Player game
hierarchy.

\subsection{The {\tt ComputerGame} Hierarchy}

\begin{figure}[bt]
\figaright{chptr09/nplayerwordguess.eps}{Overview of the {\tt
ComputerGame} class hierarchy.
} {fig-game}

\end{figure}

Figure~\ref{fig-game} provides a summary overview of the {\tt
ComputerGame} hierarchy.  This figure shows the relationships
among the many classes and interfaces involved.  The two classes whose
symbols are bold, {\tt WordGuess} and {\tt WordGuesser}, are the
classes that define the specific game we will be playing.  The
rest of the classes and interfaces are designed to be used with
any N-player game. 

At the root of this hierarchy is the abstract {\tt ComputerGame}
class.  Note that it uses from 1 to {\em N} {\tt Player}s. These
objects will be stored in a one-dimensional array in {\tt
ComputerGame}.  Recall from Chapter~8 that an {\em IPlayer} was any
class that implements the {\tt makeAMove()} method.  In this design,
we have put the abstract {\tt makeAMove()} method into the {\tt
Player} class, a class that defines a generic player of computer
games.  For the {\tt WordGuess} game, the {\tt WordGuesser} class
extends {\tt Player}.  In order to play Word Guess, we will create a
{\tt WordGuess} instance, plus one or more instances of {\tt
WordGuesser}s. This is similar to the {\tt OneRowNim} example from the
previous chapter,

Note where the {\tt TwoPlayerGame} and {\tt OneRowNim} classes occur
in the hierarchy.  {\tt TwoPlayerGame} will now be an extension of
{\tt ComputerGame}. This is in keeping with the fact that a two-player
game is a special kind of N-player computer game.  As we will see when
we look at the details of these classes, {\tt TwoPlayerGame} will
override some of the methods inherited from {\tt ComputerGame}.

Because it contains the abstract {\tt makeAMove()} method, the {\tt
Player} class is an abstract class.  Its purpose is to define and
store certain data and methods that can be used by any computer
games. For example, one important piece of information defined in {\tt
Player} is whether the player is a computer or a person.  {\tt
Player}'s data and methods will be inherited by {\tt WordGuesser} and
by other classes that extend {\tt Player}.  Given its position in the
hierarchy, we will be able to define polymorphic methods for {\tt
WordGuesser}s that treat them as {\tt Player}s. As we will see, this
will give our design great flexibility and extensibility.

\subsection{The {\tt ComputerGame} Class}

Figure~9.30 shows the design details of the {\tt ComputerGame}
class. One of the key tasks of the {\tt ComputerGame} class is to
manage the one or more computer game players. Because this is a task
that is common to all computer games, it makes sense to manage it here
in the superclass. Toward this end, {\tt ComputerGame} declares four
instance variables and several methods. Three {\tt int} variables
define the total number of players ({\tt nPlayers}), the number of
players that have been added to the game ({\tt addedPlayers}), and the
player whose turn it is ({\tt whoseTurn}).  An array named {\tt
player} stores the {\tt Player}s.  In keeping with the zero indexing
convention of arrays, we number the players from 0 to {\em
nPlayers-1}. These variables are all declared {\tt protected}, so that
they can be referenced directly by {\tt ComputerGame} subclasses, but
as {\tt protected} variables, they remain hidden from all other
classes.

\marginfig{chptr09/computergame.eps}{The {\tt ComputerGame} class.}
{fig-computergame}
The {\tt ComputerGame(int)} constructor allows the number of players
to be set when the game is constructed.  The default constructor sets
the number of players to one. The constructors create an array of
length {\tt nPlayers}:

\begin{jjjlisting}
\begin{lstlisting}
public ComputerGame(int n) {
  nPlayers = n;
  player = new Player[n]; // Create the array
}
\end{lstlisting}
\end{jjjlisting}

\noindent The {\tt setPlayer()} and {\tt getPlayer()} methods are the
mutator and accessor methods for the {\tt whoseTurn} variable. This variable
allows a user to determine and set whose turn it is, a useful feature
for initializing a game.  The {\tt changePlayer()} method uses the default
expression, 

\begin{jjjlisting}
\begin{lstlisting}
 whoseTurn = (whoseTurn + 1) % nPlayers 
\end{lstlisting}
\end{jjjlisting}

\noindent for changing whose turn it is.  Assuming that players are
numbered from 0 to {\em nPlayers-1}, this code gives the turn to the
next player, wrapping around to player 0, if necessary. Of course, a
subclass of {\tt ComputerGame} can override this method if the game
requires some other order of play.

The {\tt addPlayer(Player)} method is used to add a new {\tt Player}
to the game, including any subclass of {\tt Player}.  The
method assumes that {\tt addedPlayers} is initialized to 0. It
increments this variable by 1 each time a new player is added to the
array. For the game {\tt WordGuess}, we would be adding {\tt Player}s
of type {\tt WordGuesser} to the game.

The complete source code for {\tt ComputerGame} is shown in
Figure~\ref{fig-gamesource}.

\begin{figure}[h!]
\scriptsize
\jjjprogstart
\begin{jjjlisting}
\begin{lstlisting}
public abstract class ComputerGame {   
  protected int nPlayers;
  protected int addedPlayers = 0;
  protected int whoseTurn;
  protected Player player[];  // An array of players

  public ComputerGame() {
    nPlayers = 1;       // Default: 1 player game
    player = new Player[1];
  }
  public ComputerGame(int n) {
    nPlayers = n;
    player = new Player[n]; // N-Player game
  }
  public void setPlayer(int starter) { 
    whoseTurn = starter; 
  }
  public int getPlayer() { 
    return whoseTurn;  
  }
  public void addPlayer(Player p) {
    player[addedPlayers] = p;
    ++addedPlayers;
  }
  public void changePlayer() { 
    whoseTurn = (whoseTurn + 1) % nPlayers;
  }
  public String getRules() {
    return "The rules of this game are: ";
  }
  public String listPlayers() {
    StringBuffer result = 
          new StringBuffer("\nThe players are:\n");
    for (int k = 0; k < nPlayers; k++)
      result.append("Player" + k + " " + 
                      player[k].toString() + "\n");
      result.append("\n");
      return result.toString();
  }
  public abstract boolean gameOver(); // Abstract
  public abstract String getWinner(); //  methods
} //ComputerGame
\end{lstlisting}
\end{jjjlisting}
\jjjprogstop{Implementation of the {\tt ComputerGame} class.}
{fig-gamesource}
\end{figure}

\noindent There are several points worth noting about this
implementation. First, note that just as in the case of the abstract
{\tt TwoPlayerGame} class from Chapter~8, the methods {\tt gameOver()}
and {\tt getWinner()} are defined as {\tt abstract} and the {\tt
getRules()} method is given a generic implementation. The intent here
is that the subclass will override {\tt getRules()} and will provide
game-specific implementations for the abstract methods.

Second, note how the {\tt addPlayer()} method is coded. It uses the
{\tt addedPlayers} variable as the index into the {\tt player} array,
which always has length {\tt nPlayers}.  An attempt to call this
method when the array is already full will lead to the following
exception being thrown by Java:

\begin{jjjlisting}
\begin{lstlisting}
Exception in thread ``main'' 
    java.lang.ArrayIndexOutOfBoundsException: 2
    at ComputerGame.addPlayer(ComputerGame.java:22)
    at TwentyOne.main(TwentyOne.java:121)
\end{lstlisting}
\end{jjjlisting}

\noindent In other words, it is an error to try to add more players
than will fit in the {\tt player} array. In Chapter~11, we will learn
how to design our code to guard against such problems. 

Finally, note the implementation of the {\tt listPlayers()} method
(Fig.~\ref{fig-gamesource}).
%%%RAM\begin{jjjlisting}
%%%RAM\begin{lstlisting}
%%%RAMpublic String listPlayers() {
%%%RAM  StringBuffer result =  
%%%RAM             new StringBuffer("\nThe players are:\n");
%%%RAM  for (int k = 0; k < nPlayers; k++)
%%%RAM    result.append("Player" + k + " " + 
%%%RAM                                player[k].toString() + "\n");
%%%RAM  result.append("\n");
%%%RAM  return result.toString();
%%%RAM}
%%%RAM\end{lstlisting}
%%%RAM\end{jjjlisting}
\noindent Here is a good example of polymorphism at work. The
elements of the {\tt player} array have a declared type of {\tt
Player}. Their dynamic type is {\tt WordGuesser}. So when the
expression {\tt player[k].toString()} is invoked, dynamic binding is
used to bind this method call to the implementation of {\tt
toString()} defined in the {\tt WordGuesser} class.  Thus, by
allowing {\tt toString()} to be bound at run time, we are able
to define a method here that doesn't know the exact types of the
objects it will be listing.

The power of polymorphism is the flexibility and extensibility it
\marginnote{polymorphism}
lends to our class hierarchy.  Without this feature, we would not be
able to define {\tt listPlayers()} here in the superclass, and would
instead have to define it in each subclass.  

\JavaTIP[false]{EFFECTIVE DESIGN}{Extensibility.}{Polymorphic
methods allow us to implement methods that can be applied to
yet-to-be-defined subclasses.}

\marginfigscaled{chptr09/wordguess.eps}{0.9}{The {\tt WordGuess} class.}
{fig-wordguess}

\subsection{The {\tt WordGuess} and {\tt WordGuesser} Classes}

We will assume here that you are familiar with the {\tt WordGuess}
example from Chapter~8. If not, you will need to review that section
before proceeding.  Word Guess is a game in which players take turns
trying to guess a secret word by guessing its letters. Players keep
guessing as long as they correctly guess a letter in the word. If they
guess wrong, it becomes the next player's turn. The winner of the game
is the person who guesses the last letter secret letter, thereby
completely identifying the word. 

Figure~9.32 provides an overview of the {\tt WordGuess} class.  If
you compare it with the design we used in Chapter~8, the only change
in the instance methods and instance variables is the addition of a
new constructor, {\tt WordGuess(int)}, and an {\tt init()} method.  This
constructor takes an integer parameter representing the number of
players. The default constructor assumes that there is one player.  Of
course, this version of {\tt WordGuess} extends the {\tt ComputerGame}
class, rather than the {\tt TwoPlayerGame} class.  Both constructors
call the {\tt init()} method to initialize the game:

\begin{jjjlisting}
\begin{lstlisting}
 public WordGuess() { super(1); init(); }
 public WordGuess(int m) { super(m); init(); } 
 public void init() {
     secretWord = getSecretWord();
     currentWord = new StringBuffer(secretWord);
     previousGuesses = new StringBuffer();
     for (int k = 0; k < secretWord.length(); k++)
        currentWord.setCharAt(k,'?');
     unguessedLetters = secretWord.length();
 }
\end{lstlisting}
\end{jjjlisting}

The only other change required to convert {\tt WordGuess} to an
N-player game, is to rewrite its {\tt play()} method.  Because the new
{\tt play()} method makes use of functionality inherited from the {\tt
ComputerGame()} class, it is actually much simpler than the {\tt play()}
method in the Chapter~8 version:

\begin{jjjlisting}[28pc]
\begin{lstlisting}
public void play(UserInterface ui) {
  ui.report(getRules());
  ui.report(listPlayers());
  ui.report(reportGameState());

  while(!gameOver()) {  
    WordGuesser p = (WordGuesser)player[whoseTurn];
    if (p.isComputer())
      ui.report(submitUserMove(p.makeAMove(getGamePrompt())));
    else {
      ui.prompt(getGamePrompt());
      ui.report(submitUserMove(ui.getUserInput()));
    }
    ui.report(reportGameState());
  } // while
}
\end{lstlisting}
\end{jjjlisting}

\noindent The method begins by displaying the game's rules and listing
its players. The {\tt listPlayers()} method is inherited from the {\tt
ComputerGame} class.  After displaying the game's current state, the
method enters the play loop. On each iteration of the loop, a
player is selected from the array:

\begin{jjjlisting}
\begin{lstlisting}
 WordGuesser p = (WordGuesser)player[whoseTurn];
\end{lstlisting}
\end{jjjlisting}

\noindent The use of the {\tt WordGuesser} variable, {\tt p},  
just makes the code somewhat more readable. Note that we have to use
a cast operator, {\tt (WordGuesser)}, to convert the array element, a
{\tt Player}, into a {\tt WordGuesser}.  Because {\tt p} is a {\tt
WordGuesser}, we can refer directly to its {\tt isComputer()} method. 

If the player is a computer, we prompt it to make a move, submit
the move to the {\tt submitUserMove()} method, and then report the
result.  This is all done in a single statement:

\begin{jjjlisting}
\begin{lstlisting}
 ui.report(submitUserMove(p.makeAMove(getGamePrompt())));
\end{lstlisting}
\end{jjjlisting}

\noindent If the player is a human, we prompt the player and use the {\tt
KeyboardReader}'s {\tt getUserInput()} method to read the user's move.
We then submit the move to the {\tt submitUserMove()} method and
report the result.  At the end of the loop, we report the game's
updated state. The following code segment illustrates a small portion
of the interaction generated by this {\tt play()} method:

\begin{jjjlisting}
\begin{lstlisting}
 Current word ???????? Previous guesses GLE
 Player 0 guesses next.Sorry, Y is NOT a new letter 
                       in the secret word
 Current word ???????? Previous guesses GLEY
 Player 1 guesses next.Sorry, H is NOT a new letter 
                       in the secret word
 Current word ???????? Previous guesses GLEYH
 Player 2 guesses next.
 Guess a letter that you think is in the secret word: a
 Yes, the letter A is in the secret word
\end{lstlisting}
\end{jjjlisting}

\noindent In this example, players 0 and 1 are computers and player 2 is
a human. 

\marginfig{chptr09/wordguesser.eps}{The {\tt WordGuesser} class
extends {\tt Player}.}
{fig-wordguesser}

In our new design, the {\tt WordGuesser} class is a subclass of {\tt
Player} (Figure~9.33).  The {\tt WordGuesser} class itself requires
no changes other than its declaration:

\begin{jjjlisting}
\begin{lstlisting}
 public class WordGuesser extends Player
\end{lstlisting}
\end{jjjlisting}

\noindent As we saw when we were discussing the {\tt WordGuess} class, the
{\tt play()} method invokes {\tt WordGuesser}'s {\tt isComputer()}
method. But this method is inherited from the {\tt Player} class.  The
only other method used by {\tt play()} is the {\tt makeAMove()}
method. This method is coded exactly the same as it was in the
previous version of {\tt WordGuesser}.

Figure~\ref{fig-player} shows the implementation of the {\tt Player}
class. Most of its code is very simple. Note that the default value
for the {\tt kind} variable is {\tt HUMAN} and the default {\tt id} is
-1, indicating the lack of an assigned identification.

\begin{figure}[h!]
\jjjprogstart
\begin{jjjlisting}
\begin{lstlisting}
public abstract class Player  {   
  public static final int COMPUTER=0;
  public static final int HUMAN=1;
  protected int id = -1;  // Id between 0 and nPlayers-1
  protected int kind = HUMAN; // Default is HUMAN

  public Player() { }
  public Player(int id, int kind) {
    this.id = id;
    this.kind = kind;
  }
  public void setID(int k) { id = k; }
  public int getID() { return id; }
  public void setKind(int k) { kind = k; }
  public int getKind() { return kind; }
  public boolean isComputer() { return kind == COMPUTER; }
  public abstract String makeAMove(String prompt); 
} // Player
\end{lstlisting}
\end{jjjlisting}
\jjjprogstop{Implementation of the {\tt Player} class.}
{fig-player}
\end{figure}

What gives {\tt Player} its utility is the fact that it encapsulates
those attributes and actions that are common to all computer game
players. Defining these elements, in the superclass, allows them to
be used throughout the {\tt Player} hierarchy.  It also makes it
possible to establish an association between a {\tt Player} and a {\tt
ComputerGame}.

Given the {\tt ComputerGame} and {\tt Player} hierarchies and
the many interfaces they contain, the task of designing and
implementing a new N-player game is made much simpler. This too is due
to the power of object-oriented programming. By learning to use a
library of classes, such as these, even inexperienced programmers can
create relatively sophisticated and complex computer games.

\JavaTIP{EFFECTIVE DESIGN}{Code Reuse.}{Re-using library code by
extending classes and implementing interfaces makes it much
simpler to create sophisticated applications.}

Finally, the following {\tt main()} method instantiates and runs an instance
of the {\tt WordGuess} game in a command-line user interface (CLUI):

\begin{jjjlisting}
\begin{lstlisting}
public static void main(String args[]) 
{ KeyboardReader kb = new KeyboardReader();
  ComputerGame game = new WordGuess(3);
  game.addPlayer(new WordGuesser((WordGuess)game, 0, 
                                           Player.HUMAN));
  game.addPlayer(new WordGuesser((WordGuess)game, 1, 
                                         Player.COMPUTER);
  game.addPlayer(new WordGuesser((WordGuess)game, 2, 
                                         Player.COMPUTER);
  ((CLUIPlayableGame)game).play(kb);
} //main()
\end{lstlisting}
\end{jjjlisting}

\noindent In this example, we create a three player game in which two
of the players are computers.  Note how we create a new {\tt
WordGuesser}, passing it a reference to the game itself, as well as
its individual identification number, and its type (HUMAN or
COMPUTER).  To run the game, we simply invoke its {\tt play()} method.
You know enough now about object-oriented design principles to
recognize that the use of {\tt play()} in this context is an example
of polymorphism.

\markright{{\color{cyan}SECTION\,\,\arabic{section}\,\,$\bullet$}\,\, A GUI-Based Game}

\section{A GUI-Based Game (Optional Graphics)}
\label{agui-based-game-optional-graphics}

\markright{{\color{cyan}SECTION\,\,\arabic{section}\,\,$\bullet$}\,\, A GUI-Based Game}

Most modern computer games do not use a command-line interface.  This
section addresses this shortcoming by expanding our {\tt ComputerGame}
hierarchy so that it works with Graphical User Interfaces (GUIs) as
well as Command-Line User Interfaces (CLUIs).

The Sliding Tile Puzzle is a puzzle game. It is played by one player,
a human. The puzzle consists of six tiles arranged on a board
containing seven spaces. Three of the tiles are labeled {\em L} and
three are labeled {\em R}.  Initially the tiles are arranged as {\em
RRR\_LLL}.  In other words, the {\em R} tiles are arranged to the left
of the {\em L} tiles, with the blank space in the middle. The object
of the puzzle is to rearrange the tiles into {\em LLL\_RRR}. The rules
are that tiles labeled {\em R} can only move right. Tiles labeled {\em
L} can only move left. Tiles may move directly into the blank space or
they can jump over one tile into the blank space.

Our purpose in this section is to develop a GUI that plays this
game. An appropriate GUI is shown Figure~\ref{fig-capture}.  Here the
tiles and the blank space are represented by an array of buttons.  To
make a move the user clicks on the 'tile' he or she wishes to
move. The GUI will assume that the user wants to move that tile into
the blank space. If the proposed move is legal, the GUI will carry out
the move. Otherwise, it will just ignore it. For example, if the user
were to click on the third {\em R} button from the left, a legal move,
the GUI would rearrange the labels on the buttons so that their new
configuration would be {\em RR\_RLLL}. On the other hand, if the user
were to click on the rightmost {\em L} button, the GUI would ignore
that move, because it is illegal.

\begin{figure}[h!]
\figascaled{chptr09/captureGUI.eps}{0.8}{The Sliding Tile Puzzle.}
{fig-capture}
\end{figure}

\subsection{The {\tt GUIPlayableGame} Interface}

How should we extend our game-playing hierarchy to accommodate
GUI-based games? As we learned in Chapter~4, one difference between
GUI-based applications and CLUI-based applications, is the locus of
control. In a CLUI-based application, control resides in the
computational object which, for games, is the game object.  That's why
the {\tt play()} method in our CLUI-based games contains the game's
control loop.  By contrast, control resides in the GUI's event loop in
GUI-based applications.  That's why, we learned how to
manage Java's event hierarchy in Chapter~4. Thus, in the GUI shown in
Figure~\ref{fig-capture}, the GUI will {\em listen} and take {\em
action} when the user clicks one of its buttons.

However, given that control will reside in the GUI, there is still a
need for communication between the GUI and the game
object. In the CLUI-based games, we have used the {\tt
CLUIPlayableGame} interface to manage the communication between the
game and the user interface.  We will follow the same design strategy
in this case.  Thus, we will design a {\tt GUIPlayableGame} interface that
can be implemented by any game that wishes to use a GUI (Fig.~9.36).

What method(s) should this interface contain?  One way to answer this
question is to think about the type of interaction that must take
place when the user clicks  one of the tiles.  If the user clicks
the third {\em R} button, the GUI should pass this information to the
game. The game should then decide whether or not that is a legal move
and communicate this back to the GUI. Assuming it is a legal move, the
game should also update its representation of the game's state to
reflect that the tile array has changed. And it should somehow make
communicate the game's state to the GUI.
\marginfig{chptr09/GUIplayable.eps}{The {\tt GUIPlayableGame} interface
extends the {\tt IGame} interface.}
{fig-GUIplayable}


Because it is impossible to know in advance just what form of data a
game's moves might take, we will use Java {\tt String}s to communicate
between the user interface and the game object. Thus, a method with
the following signature will enable us to submit a {\tt
String} representing the user's move to the game  and receive in return a {\tt
String} representing the game object's response to the move:

\begin{jjjlisting}
\begin{lstlisting}
 submitUserMove(String move): String;
\end{lstlisting}
\end{jjjlisting}

\noindent In addition to this method, a GUI interface could use the
{\tt report\-Game\-State()} and {\tt get\-Game\-Prompt()} methods that are
part of the {\tt IGame} interface.  Th design shown in Figure~9.36
leads to the following definition for the {\tt GUIPlayableGame}
interface:

\begin{jjjlisting}
\begin{lstlisting}
public interface GUIPlayableGame extends IGame {
    public String submitUserMove(String theMove);
}
\end{lstlisting}
\end{jjjlisting}

\noindent Because it extends {\tt IGame}, this interface inherits the
{\tt getGamePrompt()} and {\tt report\-Game\-State()} from the {\tt IGame}
interface. The GUI should be able to communicate with any game that
implements this interface.

\subsection{The {\tt SlidingTilePuzzle}}

Let's now discuss the design and details of the 
{\tt SlidingTilePuzzle} itself. Its design is summarized in
Figure~9.37.  Most of the methods should be familiar
to you, because the design closely follows the design we employed in
the {\tt WordGuess} example. It has implementations of inherited
methods from the {\tt ComputerGame} class and the {\tt
GUIPlayableGame} interface.  

We will represent the sliding tile puzzle in a one-dimensional array
of {\tt char}. We will store the puzzle's solution in a Java {\tt
String} and we will use an {\tt int} variable to keep track of where
the blank space is in the array.  This leads to the following
class-level declarations:

\begin{jjjlisting}
\begin{lstlisting}
 private char puzzle[] = {'R','R','R',' ','L','L','L'};
 private String solution = "LLL RRR";
 private int blankAt = 3;
\end{lstlisting}
\end{jjjlisting}

\marginfig{chptr09/slidingtilepuzzle.eps}{The {\tt SlidingTilePuzzle} is
a {\tt ComputerGame} that implements the {\tt GUIPlayableGame} interface.}
{fig-slidingpuzzle}
\noindent Note how we initialize the {\tt puzzle} array with the
initial configuration of seven characters. Taken together, these
statements initialize the puzzle's state.

Because a puzzle is a one-person game and our sliding tile puzzle will be
played by a human, this leads to a very simple constructor 
definition:

\begin{jjjlisting}
\begin{lstlisting}
 public SlidingTilePuzzle() {
     super(1);
 }
\end{lstlisting}
\end{jjjlisting}

\noindent We call the {\tt super()} constructor ({\tt ComputerGame()})
to create a one-person game.

The puzzle's state needs to be communicated to the GUI as a {\tt
String}. This is the purpose of the {\tt reportGameState()} method:

\begin{jjjlisting}
\begin{lstlisting}
 public String reportGameState() {   
     StringBuffer sb = new StringBuffer();
     sb.append(puzzle);
     return sb.toString();
 }
\end{lstlisting}
\end{jjjlisting}

\noindent We use a {\tt StringBuffer()} to convert the puzzle, a {\tt
char} array, into a {\tt String}.

The most important method for communicating with the GUI is the {\tt
submitUserMove()} method:

\begin{jjjlisting}
\begin{lstlisting}
public String submitUserMove(String usermove) {   
  int tile = Integer.parseInt(usermove);
  char ch = puzzle[tile];
  if (ch == 'L' && 
               (blankAt == tile-1 || blankAt == tile-2))
    swapTiles(tile,blankAt);
  else if (ch == 'R' && 
               (blankAt == tile+1 || blankAt == tile+2))
    swapTiles(tile,blankAt);
  else 
    return "That's an illegal move.\n";
  return "That move is legal.\n";
}
\end{lstlisting}
\end{jjjlisting}

\noindent This is the method that processes the user's move, which is
communicated through the GUI. As we saw, the puzzle's 'tiles' are
represented by an array of buttons in the GUI.  The buttons are
indexed 0 through 6 in the array. When the user clicks a button,
the GUI should pass its index, represented as a {\tt String} to the
{\tt submitUserMove()} method.  Given the index number of the tile
that was selected by the user, this method determines if the move is
legal.

The {\tt Integer.parseInt()} method is used to extract the tile's
index from the method's parameter. This index is used to get a 'tile'
from the {\tt puzzle} array. The logic in this method reflects the
rules of the game. If the tile is an {\em L}, then it can only move
into a blank space that is either 1 or 2 spaces to its left.
Similarly, an {\em R} tile can only move into a blank space that is 1
or 2 spaces to its right. All other moves are illegal. For legal
moves, we simply swap the tile and the blank space in the array, a
task handled by the {\tt swap()} method.  In either case, the method
returns a string reporting whether the move was legal or illegal.

Figure~\ref{fig-puzzlecode} shows the full implementation for the {\tt
SlidingTilePuzzle}, the remaining details of which are straight forward.

\begin{figure}[h!]
\jjjprogstart
\begin{jjjlisting}
\begin{lstlisting}
public class SlidingTilePuzzle extends ComputerGame 
                            implements GUIPlayableGame {
  private char puzzle[] = {'R','R','R',' ','L','L','L'};
  private String solution = "LLL RRR";
  private int blankAt = 3;
  public SlidingTilePuzzle() { super(1); }

  public boolean gameOver() { // True if puzzle solved
    StringBuffer sb = new StringBuffer();
    sb.append(puzzle);
    return sb.toString().equals(solution);    
  } 
  public String getWinner() {   
    if (gameOver())
      return "\nYou did it! Very Nice!\n";
    else return "\nGood try. Try again!\n";
  }
  public String reportGameState() {   
    StringBuffer sb = new StringBuffer();
    sb.append(puzzle);
    return sb.toString();
  } 
  public String getGamePrompt() {    
    return "To move a tile, click on it.";
  } //prompt()

  public String submitUserMove(String usermove) {   
    int tile = Integer.parseInt(usermove);
    char ch = puzzle[tile];
    if (ch=='L' && (blankAt==tile-1 || blankAt==tile-2))
      swapTiles(tile,blankAt);
    else if (ch=='R' && 
                   (blankAt==tile+1 || blankAt==tile+2))
      swapTiles(tile,blankAt);
    else 
      return "That's an illegal move.\n";
    return "That move is legal.\n";
  }
  private void swapTiles(int ti, int bl) {
    char ch = puzzle[ti];
    puzzle[ti] = puzzle[bl];
    puzzle[bl] = ch;
    blankAt = ti;   // Reset the blank
  }
} //SlidingTilePuzzle
\end{lstlisting}
\end{jjjlisting}
\jjjprogstop{Implementation of the {\tt SlidingTilePuzzle} class.}
{fig-puzzlecode}
\end{figure}

\pagebreak
\subsection{The {\tt SlidingGUI} Class}

Let's now implement a GUI that can be used to play the sliding tile
puzzle. We will model the GUI itself after those we designed in
Chapter~4. 

Figure~9.39 provides a summary of the design. As an implementor of the
\marginfig{chptr09/slidingGUI.eps}{The {\tt SlidingGUI} class.}
{fig-slidingGUI}
{\tt ActionListener} interface, {\tt SlidingGUI} implements the {\tt
actionPerformed()} method, which is where the code that controls the
puzzle is located.  The main data structure is an array of seven {\tt
JButton}s, representing the seven tiles in the puzzles. The buttons'
labels will reflect the state of the puzzle.  They will be rearranged
after every legal move by the user. The {\tt reset} button is used to
reinitialize the game. This allows users to play again or to start
over if they get stuck.

The {\tt puzzleState} is a {\tt String} variable that stores the
puzzle's current state, which is updated repeatedly from the {\tt
SlidingTilePuzzle} by calling its {\tt reportGameState()} method.  The
private {\tt labelButtons()} method will read the {\tt puzzleState}
and use its letters to set the labels of the GUI's buttons.

The implementation of {\tt SlidingGUI} is shown in
Figure~\ref{fig-GUIsource}.  Its constructor and {\tt buildGUI()}
methods are responsible for setting up the GUI. We use of a {\tt for}
loop in {\tt buildGUI()} to create the {\tt JButton}s, associate an
{\tt ActionListener} with them, and add them to the GUI. Except for
the fact that we have an array of buttons, this is very similar to the
GUI created in Chapter~4. Recall that associating an {\tt
ActionListener} with the buttons allows the program to respond to
button clicks in its {\tt actionPerformed()} method.

Note how an instance of the {\tt SlidingTilePuzzle} is created in
the constructor, and how its state is retrieved and stored in
the {\tt puzzleState} variable:

\begin{jjjlisting}
\begin{lstlisting}
 puzzleState = sliding.reportGameState();
\end{lstlisting}
\end{jjjlisting}

\noindent The {\tt labelButtons()} method transfers the
letters in {\tt puzzleState} onto the buttons.

\begin{figure}[p]
\scriptsize
\jjjprogstart
\begin{jjjlisting}
\begin{lstlisting}[basicstyle=\scriptsize]
import javax.swing.*;
import java.awt.*;
import java.awt.event.*;
public class SlidingGUI extends JFrame implements ActionListener {
  private JButton tile[] = new JButton[7];
  private JButton reset = new JButton("Reset");
  private SlidingTilePuzzle sliding;
  private String puzzleState;
  private Label label;
  private String prompt = "Goal: [LLL RRR]. " +
    " Click on the tile you want to move." +
    " Illegal moves are ignored.";
  public SlidingGUI(String title) {
    sliding = new SlidingTilePuzzle();
    buildGUI();
    setTitle(title);
    pack();
    setVisible(true);
  } // SlidingGUI()
  private void buildGUI() {
    Container contentPane = getContentPane();
    contentPane.setLayout(new BorderLayout());
    JPanel buttons = new JPanel();
    puzzleState = sliding.reportGameState();
    for (int k = 0; k < tile.length; k++) {
      tile[k] = new JButton(""+puzzleState.charAt(k));
      tile[k].addActionListener(this);
      buttons.add(tile[k]);
    }
    reset.addActionListener(this);
    label = new Label(prompt);
    buttons.add(reset);
    contentPane.add("Center", buttons);
    contentPane.add("South", label);
  } // buildGUI()
  private void labelButtons(String s) {
    for (int k = 0; k < tile.length; k++)
      tile[k].setText(""+ s.charAt(k));
  } // labelButtons()
  public void actionPerformed(ActionEvent e) {
    String result = "";
    if (e.getSource() == reset) { // Reset clicked?
      sliding = new SlidingTilePuzzle();
      label.setText(prompt);
    }
    for (int k = 0; k < tile.length; k++) // Tile clicked?
      if (e.getSource() == tile[k])
        result = ((GUIPlayableGame)sliding).submitUserMove(""+ k);
    if (result.indexOf("illegal") != -1)
      java.awt.Toolkit.getDefaultToolkit().beep();
    puzzleState = sliding.reportGameState();
    labelButtons(puzzleState);
    if (sliding.gameOver()) 
      label.setText("You did it! Very nice!");
  } // actionPerformed()
  public static void main(String args[]) {
    new SlidingGUI("Sliding Tile Puzzle");
  } // main()
} // SlidingGUI
\end{lstlisting}
\end{jjjlisting}
\jjjprogstop{Implementation of the {\tt SlidingGUI} class.}
{fig-GUIsource}
\end{figure}

The most important method in the GUI is the {\tt actionPerformed()}
method. This method controls the GUI's actions and is called
automatically whenever one of the GUI's buttons is clicked. First, we
check whether the {\tt reset} button has been clicked. If so, we reset
the puzzle by creating a new instance of {\tt SlidingTilePuzzle} and
re-initializing the prompt label.

Next we use a for loop to check whether one of the tile buttons has
been clicked.  If so, we use the loop index, {\em k}, as the tile's
identification and submit this to the puzzle as the user's move:

\begin{jjjlisting}[26.5pc]
\begin{lstlisting}
if (e.getSource() == tile[k])
  result = ((GUIPlayableGame)sliding).submitUserMove(""+ k);
\end{lstlisting}
\end{jjjlisting}

\noindent The cast operation is necessary here because we declared
{\tt sliding} as a {\tt SlidingTilePuzzle} rather than as a {\tt
GUIPlayableGame}.  Note also that we have to convert {\em k} to a {\tt
String} when passing it to {\tt submitUserMove()}. 

As a result of this method call, the puzzle returns a {\tt result},
which is checked to see if the user's move was illegal. If
{\tt result} contains the word ``illegal'', the computer beeps to signal an
error:

\begin{jjjlisting}
\begin{lstlisting}
 if (result.indexOf("illegal") != -1)
    java.awt.Toolkit.getDefaultToolkit().beep();
\end{lstlisting}
\end{jjjlisting}

\noindent The {\tt java.awt.Toolkit} is a class that contains lots of
useful methods, including the {\tt beep()} method. Note that no matter
what action is performed, a reset or a tile click, we update {\tt
puzzleState} by calling {\tt reportGameState()} and use it to relabel
the tile buttons.  The last task in the {\tt actionPerformed()} method
is to invoke the puzzle's {\tt gameOver()} method to check if the user
has successfully completed the puzzle. If so, we display a
congratulatory message in the GUI's window.

Finally, the {\tt main()} for a GUI is very simple, consisting of a
single line of code:

\begin{jjjlisting}
\begin{lstlisting}
   new SlidingGUI("Sliding Tile Puzzle");
\end{lstlisting}
\end{jjjlisting}

\noindent Once a {\tt SlidingGUI} is created, with the
title of ``Sliding Tile Puzzle,'' it will open a window
and manage the control of the puzzle.

\secSMH{Chapter Summary}

%\addcontentsline{toc}{section}{Chapter Summary}
\secKTH{Technical Terms}

\begin{KT}
array

array \break initializer

array length

binary search

data structure 

element

element type

insertion sort

multidimensional array

one-dimensional \break array

polymorphic sort method

selection sort

sequential search

sorting

subscript

two-dimensional array

\end{KT}


\secSMHtwo{Summary of Important Points}
\begin{SMBL}

\item An {\bf array}\index{array} is a named collection of contiguous
storage locations, each of which stores a data item of the same data
type.  Each element of an array is referred to by a {\it
subscript\index{subscript}}---that is, by its position in the array.
If the array contains {\it N} elements, then its length is {\it N} and
its indexes are {\tt 0, 1, \dots N-1}.  

\item Array elements are referred to using the following subscript
notation {\em arrayname[subscript]}, where {\it arrayname} is any
valid identifier, and {\it subscript} is an integer value in the range
0 to {\tt arrayname.length - 1}. The array's {\tt length} instance
variable can be used as a bound for loops that process the array.

\item An {\it array declaration} provides the name and type of the
array.  An array instantiation uses the keyword {\tt new} and causes
the compiler to allocate memory for the array's elements:

\begin{jjjlisting}
\begin{lstlisting}
int arr[]; // Declare a one-dimensional array variable
arr = new int[15];// Allocate 15 int locations for it
\end{lstlisting}
\end{jjjlisting}

\item Multidimensional arrays have arrays as their components:

\begin{jjjlisting}
\begin{lstlisting}
int twoDarr[][]; // Declare a two-dimensional array variable
twoDarr = new int[10][15]; // Allocate 150 int locations
\end{lstlisting}
\end{jjjlisting}

\item  An array's values must be initialized by assigning
values to each array location.  An {\it initializer\index{array
initializer}} expression may be included as part of the array
declaration.

\item Insertion sort and selection sort are examples of array sorting
algorithms.  Both algorithms require several passes over the array.

\item When an array is passed as a argument to a method, a reference
to the array is passed rather than the entire array itself.

\item  Swapping two elements of an array, or any
two locations in memory, requires the use of a temporary variable.

\item Sequential search and binary search are examples of array
searching algorithms.  Binary search requires that the array be
sorted.

\item For multidimensional arrays, each dimension of the array has its
own {\tt length} variable.

\item Inheritance and polymorphism are useful design features for
developing a hierarchy of computer games. 

\end{SMBL}


\secANSHleft

%\addcontentsline{toc}{section}{Answers to Self-Study Exercises}
%\begin{itemize}

\begin{ANS}
\item  
Space (in bytes) allocated for each of the following?

\begin{enumerate}
\scriptsize
\item[a.]  \verb|int a[] = new int[5];       |{\color{cyan}\verb|   // 5 * 4 = 20 bytes|}
\item[b.]  \verb|double b[] = new double[10];|{\color{cyan}\verb|   // 10 * 8 = 80 bytes|}
\item[c.]  \verb|char c[] = new char[30];    |{\color{cyan}\verb|   // 30 * 2  = 60 bytes|}
\item[d.]  \verb|String s[] = new String[10];|{\color{cyan}\verb|   // 10 * 4 (reference) = 40 bytes|}
\item[e.]  \verb|Student s[] = new Student[5]; |{\color{cyan}\verb| // 5 * 4 (reference) = 20 bytes|} 
\normalsize
\end{enumerate}


\item  An array containing 10 {\tt float}s, 1.0 to 10.0.

\begin{jjjlisting}
\begin{lstlisting}
float farr[] = {1.0,2.0,3.0,4.0,5.0,6.0,7.0,8.0,9.0,10.0};
\end{lstlisting}
\end{jjjlisting}

\item  Prints the first element of {\tt farr}.

\begin{jjjlisting}
\begin{lstlisting}
System.out.println(farr[0]);
\end{lstlisting}
\end{jjjlisting}

\item  Assigns 100.0 to the last element in {\tt farr}.

\begin{jjjlisting}
\begin{lstlisting}
farr[farr.length-1] = 100.0;
\end{lstlisting}
\end{jjjlisting}

\item  A loop to print all of the elements of {\tt farr}.

\begin{jjjlisting}
\begin{lstlisting}
for (int j = 0; j < farr.length; j++)
    System.out.println(farr[j]);
\end{lstlisting}
\end{jjjlisting}

\item  An array containing the first 100 square roots.

\begin{jjjlisting}
\begin{lstlisting}
double doubarr[] = new double[100];
for (int k = 0; k < doubarr.length; k++)
    doubarr[k] = Math.sqrt(k+1);
\end{lstlisting}
\end{jjjlisting}

\item Analyzing the letter frequencies in a file.

\begin{jjjlisting}
\begin{lstlisting}
import java.io.*;
import java.util.Scanner;
public static void main(String[] argv) {
  Scanner fileScan;  // To read lines of text from the file
  String str;        // To store the line of text
  AnalyzeFreq af = new AnalyzeFreq();
  try {         // Create a Scanner
    File theFile = new File("freqtest.txt");
    fileScan = Scanner.create(theFile);
    fileScan = fileScan.useDelimiter("\r\n"); //For Windows
    while (fileScan.hasNext()) { // Read and count
      str = fileScan.next();
      af.countLetters(str);
    } //while
    af.printArray(); // Print frequencies
  } catch (Exception e) {
     e.printStackTrace();
  } //catch()
} //main()
\end{lstlisting}
\end{jjjlisting}

\item  Sort 24 18 90 1 0 85 34 18 with insertion sort.

\begin{jjjlisting}
\begin{lstlisting}
  24 18 90 1  0  85 34 18 // Initial
  18 24 90 1  0  85 34 18 // Pass 1
  18 24 90 1  0  85 34 18 // Pass 2
  1  18 24 90 0  85 34 18 // Pass 3
  0  1  18 24 90 85 34 18 // Pass 4
  0  1  18 24 85 90 34 18 // Pass 5
  0  1  18 24 34 85 90 18 // Pass 6
  0  1  18 18 24 34 85 90 // Pass 7
\end{lstlisting}
\end{jjjlisting}

\item  Sort 24 18 90 1 0 85 34 18 with selection sort.

\begin{jjjlisting}
\begin{lstlisting}
24 18 90 1   0   85 34 18 // Initial
0  18 90 1   24  85 34 18 // Pass 1
0  1  90 18  24  85 34 18 // Pass 2
0  1  18 90  24  85 34 18 // Pass 3
0  1  18 18  24  85 34 90 // Pass 4
0  1  18 18  24  85 34 90 // Pass 5
0  1  18 18  24  34 85 90 // Pass 6
0  1  18 18  24  34 85 90 // Pass 7
\end{lstlisting}
\end{jjjlisting}


\item  Code to swap two {\tt Student}s.

\begin{jjjlisting}
\begin{lstlisting}
Student tempStud = student1;
student1 = student2;
student2 = tempStud;
\end{lstlisting}
\end{jjjlisting}

\item Implementation of the {\tt selectionSort()}.

\begin{jjjlisting}
\begin{lstlisting}
 public void selectionSort(int arr[]) {
    int smallest;      // Location of smallest element
    for (int k = 0; k < arr.length-1; k++) {  
        smallest = k;
        for (int j = k+1; j < arr.length; j++) 
            if (arr[j] < arr[smallest]) 
                smallest = j;
        if (smallest != k) {   // Swap smallest and kth
            int temp = arr[smallest];      
            arr[smallest] = arr[k];	
            arr[k] = temp;
        } // if
    } // outer for
 } // selectionSort()

\end{lstlisting}
\end{jjjlisting}

\item  After {\tt mystery(myArr,k)}, {\tt myArr} will store 1,2,3,5,5 and {\it k} will store 3.


\item  Binary search trace for 21 in  2, 4, 6, 8, 10, 12, 14, 16, 18, 20, 22, 24, 26, 28:

\begin{jjjlisting}
\begin{lstlisting}
key  iteration  low   high    mid
--------------------------------------
21   0          0     13      6
21   1          7     13      10
21   2          7     9       8
21   3          9     9       9
21   4          10    9       failure
\end{lstlisting}
\end{jjjlisting}

\item  A two-dimensional array with 5 rows of 10 integers.

\begin{jjjlisting}
\begin{lstlisting}
int int2d[][] = new int[5][10];
\end{lstlisting}
\end{jjjlisting}


\item  Prints the last integer in the third row {\tt int2d} and
assigns 100 to its last element.

\begin{jjjlisting}
\begin{lstlisting}
System.out.println(int2d[2][9]);
int2d[4][9] = 100;
\end{lstlisting}
\end{jjjlisting}

\item  Prints all of the elements of {\tt int2d}.

\begin{jjjlisting}
\begin{lstlisting}
for (int k = 0; k < int2d.length; k++) {
     for (int j = 0; j < int2d[k].length; j++)
         System.out.print( int2d[k][j] + " ");
     System.out.println();    // new line
}
\end{lstlisting}
\end{jjjlisting}

\item  A $52\; \times \;7$ two-dimensional array of {\tt int}.

\begin{jjjlisting}
\begin{lstlisting}
int sales[][] = new int[52][7];
for (int k = 0; k < sales.length; k++)
    for (int j= 0; j < sales[k].length; j++)
        sales[k][j] = 0;
\end{lstlisting}
\end{jjjlisting}

\item  A method to calculate average number of newspapers per week.

\begin{jjjlisting}
\begin{lstlisting}
double avgWeeklySales(int arr[][]) {
    double total = 0;
    for (int k = 0; k < arr.length; k++)
        for (int j= 0; j < arr[k].length; j++)
            total += arr[k][j];
    return total/52;
}
\end{lstlisting}
\end{jjjlisting}

\item  A  method to calculate average Sunday newspapers.

\begin{jjjlisting}
\begin{lstlisting}
double avgSundaySales(int arr[][]) {
    double total = 0;
    for (int k = 0; k < arr.length; k++)
        total += arr[k][6];
    return total/52;
}
\end{lstlisting}
\end{jjjlisting}

\item A {\tt compareTo()} for {\tt LetterFreq}.

\begin{jjjlisting}
\begin{lstlisting}
public int compareTo(Object lf) {
     LetterFreq  letFreq = (LetterFreq)lf;
     if (freq < letFreq.getFreq())
         return -1;
     else if (freq > letFreq.getFreq())
         return +1;
     else return 0; //The frequencies must be equal.
} //compareTo()
\end{lstlisting}
\end{jjjlisting}

\item A {\tt sort()} for {\tt AnalyzeFreq}.

\begin{jjjlisting}
\begin{lstlisting}
public void sort() {
     java.util.Arrays.sort(freqArr);
} //sort()
\end{lstlisting}
\end{jjjlisting}

\item A new {\tt AnalyzeFreq.main()} that uses {\tt sort()}.

\begin{jjjlisting}
\begin{lstlisting}
public static void main(String[] argv) {
  AnalyzeFreq af = new AnalyzeFreq();
  af.countLetters("Now is the time for all good students" +
     " to study computer-related topics.");
  af.sort();
  af.printArray();
} //main()
\end{lstlisting}
\end{jjjlisting}

\end{ANS}

\secEXRHtwoleft{Exercises}

\marginnote{\vspace{9pt}\raggedright{\bf Note:} For programming exercises, {\bf first} draw 
a UML class diagram describing all classes and
their inheritance relationships and/or associations.}
\begin{EXRtwo}
%\addcontentsline{toc}{section}{Exercises}

\item  Explain the difference between the following
pairs of terms:

\begin{EXRtwoLL}
\item  An {\it element} and an element {\it type}.
\item  A {\it subscript} and an {\it array element}.
\item  A {\it one-dimensional} array and {\it two-dimensional} array.
\item  An {\it array} and a {\it vector}.
\item  A {\it insertion sort} and a {\it selection sort}.
\item  A {\it binary search} and a {\it sequential search}.
\end{EXRtwoLL}

\item  Fill in the blanks.
\begin{EXRtwoLL}\baselineskip=14pt
\item  The process of arranging an array's elements into a
particular order is known as  \rule{40pt}{0.5pt}\,.
\item  One of the preconditions of the binary search method is that
the array has to be  \rule{40pt}{0.5pt}\,.
\item  An \rule{40pt}{0.5pt} is an object that can store a collection
of elements of the same type.
\item  An \rule{40pt}{0.5pt} is like an array except that it can grow.
\item  For an array, its  \rule{40pt}{0.5pt} is represented by an instance
variable.
\item  An expression that can be used during array instantiation to assign
values to the array is known as an  \rule{40pt}{0.5pt}\,.
\item  A \rule{40pt}{0.5pt} is an array of arrays.
\item  A sort method that can be used to sort different types of data
is known as a \rule{40pt}{0.5pt} method.
\item  To instantiate an array you have to use the \rule{40pt}{0.5pt} operator.
\item  An array of objects stores \rule{40pt}{0.5pt} to the objects.
\end{EXRtwoLL}

\baselineskip=11pt\item  Make each of the following array declarations:
\begin{EXRtwoLL}
\item  A $4 \times 4$ array of {\tt double}s.
\item  A $20 \times 5$ array of {\tt String}s.
\item  A $3 \times 4$ array of {\tt char} initialized to `*';
\item  A $2 \times 3 \times 2$ array of {\tt boolean} initialized to true.
\item  A $3 \times 3$  array of {\tt Student}s.
\item  A $2 \times 3$  array of {\tt String}s initialized to ``one,'' ``two,''
and so on.
\end{EXRtwoLL}

\item Identify and correct the syntax error in
each of the following expressions:
\begin{EXRtwoLL}
\item  \verb|int arr = new int[15];|
\item  \verb|int arr[] = new int(15);|
\item  \verb|float arr[] = new [3];|
\item  \verb|float arr[] = new float {1.0,2.0,3.0};|
\item  \verb|int arr[] = {1.1,2.2,3.3};|
\item  \verb|int arr[][] = new double[5][4];|
\item  \verb|int arr[][] = { {1.1,2.2}, {3.3, 1} };|
\end{EXRtwoLL}


\item  Evaluate each of the following expressions, some
of which may be erroneous:
\begin{jjjlisting}
\begin{lstlisting}
int arr[] = { 2,4,6,8,10 };
\end{lstlisting}
\end{jjjlisting}
\begin{EXRtwoLL}
\begin{multicols}{2}
\item  \verb|arr[4]|
\item  \verb|arr[ arr.length ]|
\item  \verb|arr[ arr[0] ]|
\item  \verb|arr[ arr.length / 2 ]|
\item  \verb|arr[ arr[1] ]|
\item  \verb|arr[ 5 % 2 ]|
\item  \verb|arr[ arr[ arr[0] ] ]|
\item  \verb|arr[ 5 / 2.0 ]|
\item  \verb|arr[ 1 + (int) Math.random() ]|
\item  \verb|arr[ arr[3] / 2 ]|
\end{multicols}
\end{EXRtwoLL}

\item  What would be printed by the following code segment?

\begin{jjjlisting}
\begin{lstlisting}
int arr[] = { 24, 0, 19, 21, 6, -5, 10, 16};
for (int k = 0; k < arr.length; k += 2)
  System.out.println( arr[k] );
\end{lstlisting}
\end{jjjlisting}

\item  What would be printed by the following code segment?

\begin{jjjlisting}
\begin{lstlisting}
int arr[][] = { {24, 0, 19}, {21, 6, -5}, {10, 16, 3}, 
                                           {1, -1, 0} };
for (int j = 0; j < arr.length; j++)
  for (int k = 0; k < arr[j].length; k++)
    System.out.println( arr[j][k] );
\end{lstlisting}
\end{jjjlisting}


\item  What would be printed by the following code segment?

\begin{jjjlisting}
\begin{lstlisting}
int arr[][] = { {24, 0, 19}, {21, 6, -5}, {10, 16, 3}, 
                                           {1, -1, 0} };
for (int j = 0; j < arr[0].length; j++)
    for (int k = 0; k < arr.length; k++)
        System.out.println(arr[k][j]);
\end{lstlisting}
\end{jjjlisting}


\item  What's wrong with the following code segment,
which is supposed to swap the values of the {\tt int}
variables, {\it n1} and {\it n2}?

\begin{jjjlisting}
\begin{lstlisting}
int temp = n1;
n2 = n1;
n1 = temp;
\end{lstlisting}
\end{jjjlisting}

\item  Explain why the following method does
not successfully swap the values of its two parameters?  Hint: Think
about the difference between value and reference parameters.

\begin{jjjlisting}
\begin{lstlisting}
public void swapEm(int n1, int n2) {
    int temp = n1;
    n1 = n2;
    n2 = temp;
}
\end{lstlisting}
\end{jjjlisting}

\item  Declare and initialize an array to store the
following two-dimensional table of values:

\begin{jjjlisting}
\begin{lstlisting}
1   2   3   4
5   6   7   8
9  10  11  12
\end{lstlisting}
\end{jjjlisting}

\item  For the two-dimensional array you created in
the previous exercise, write a nested for loop to print the values in
the following order: 1 5 9 2 6 10 3 7 11 4 8 12. That is, print the
values going down the columns instead of going across the rows.

\item  Define an array that would be suitable for
storing the following values:

\begin{EXRtwoLL}
\item  The GPAs of 2,000 students.
\item  The lengths and widths of 100 rectangles.
\item  A week's worth of hourly temperature measurements,
  stored so that it is easy to calculate the average daily temperature.
\item  A board for a tic-tac-toe game.
\item  The names and capitals of the 50 states.
\end{EXRtwoLL}

\item  Write a code segment that will compute the sum
of all the elements of an array of {\tt int}.

\item  Write a code segment that will compute the
sum of the elements in a two-dimensional array of {\tt int}.

\item  Write a method that will compute the average
of all the elements of a two-dimensional array of {\tt float}.

\item  Write a method that takes two parameters, an
{\tt int} array and an integer, and returns the
location of the last element of the array
that is greater than or equal to the second parameter.

\item  Write a program that tests whether
a $3\; \times \;3$ array, input by the user, is a {\it magic
square}. A magic square\index{square} is an $N\; \times\; N$ matrix of
numbers in which every number from 1 to $N^2$ must appear just once,
and every row, column, and diagonal must add up to the same total---for example,

\begin{jjjlisting}
\begin{lstlisting}
6 7 2
1 5 9
8 3 4
\end{lstlisting}
\end{jjjlisting}

\item  Revise the program in the previous exercise so that
it allows the user to input the dimensions of the array,
up to $4\; \times\; 4$.

\item Modify the {\tt AnalyzeFreq} program so that it can display the
relative frequencies of the 10 most frequent and 10 least frequent
letters.

\item  The {\it merge sort} algorithm takes two collections
of data that have been sorted and merges them together.  Write a
program that takes two 25-element {\tt int} arrays, sorts them, and
then merges them, in order, into one 50-element array.

\item  {\bf Challenge: } Design and implement a {\tt BigInteger}
class that can add and subtract integers with up to 25 digits.   Your
class should also include methods for input and output of the
numbers.  If you're really ambitious, include methods for
multiplication and division.

\item  {\bf Challenge: } Design a data structure
for this problem: As manager of Computer Warehouse, you want to
track the dollar amount of purchases made by those clients that
have regular accounts.   The accounts are numbered from {\tt 0, 1, \dots ,
N}. The problem is that you don't know in advance how many purchases
each account will have.  Some may have one or two purchases.  Others
may have 50 purchases.

\item  An {\it anagram} is a word made by rearranging the
letters of another word.  For example, {\it act} is an
anagram of {\it cat}, and {\it aegllry} is an anagram of
{\it allergy}. Write a Java program that accepts two
words as input and determines if they are anagrams.

\item  {\bf Challenge:} An {\it anagram dictionary} is a dictionary that
organizes words together with their anagrams.  Write a program that lets
the user enter up to 100 words (in a {\tt TextField}, say). After
each word is entered, the program should display (in a {\tt TextArea}
perhaps) the complete anagram dictionary for the words entered.
Use the following sample format for the dictionary.  Here the words
entered by the user were:  {\it felt, left, cat, act, opt, pot, top.}

\begin{jjjlisting}
\begin{lstlisting}
act:  act cat
eflt:  felt left
opt:   opt pot top
\end{lstlisting}
\end{jjjlisting}

\item  Acme Trucking Company has hired you
to write software to help dispatch its trucks.  One important element
of this software is knowing the distance between any two cities that
it services.   Design and implement a {\tt Distance} class that stores
the distances between cities in a two-dimensional array.   This class
will need some way to map a city name, {\it Boise}, into an integer that
can be used as an array subscript.  The class should also contain
methods that would make it useful for looking up the distance between
two cities.  Another useful method would tell the user the closest city
to a given city.

\item Rewrite the {\tt main()} method for the {\tt WordGuess} example so
that it allows the user to input the number of players and whether each 
players is a computer or a human. Use a {\tt KeyboardReader}.

\item Write a smarter version of the {\tt WordGuesser} class that
``knows'' which letters of the English language are most
frequent. HINT: Rather than using random guesses, store the player's
guesses in a string in order of decreasing frequency:
"ETAONRISHLGCMFBDGPUKJVWQXYZ".

\item Write a CLUI version of the {\tt SlidingTilePuzzle}. You will
need to make modifications to the {\tt SlidingTilePuzzle} class. 


\end{EXRtwo}
%
