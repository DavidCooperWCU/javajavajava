%%%  Chapter 14: Threads and Concurrent Programming
%%%  3rd Edition

\setcounter{SSTUDYcount}{1}
\setcounter{chapter}{13}
\chapter{Threads and Concurrent Programming}
\label{chapter-threads}



\CObegin
\secCOBH{Objectives}
\label{objectives}
\noindent After studying this chapter, you will

\begin{COBL}
\item  Understand the concept of a thread.
\item  Know how to design and write multithreaded programs.
\item  Be able to use the {\tt Thread} class and the {\tt Runnable} interface.
\item  Understand the life cycle of a thread.
\item  Know how to synchronize threads.
\end{COBL}

\secCOLH{Outline}
\label{outline}
\begin{COL}
\item {Introduction}
\item {What Is a Thread?}
\item {From the Java Library: {\tt java.lang.Thread}}
\item {Thread States and Life Cycle}
\item {Using Threads to Improve Interface Responsiveness}
\item {Case Study: Cooperating Threads}
\item {Case Study: The Game of Pong}
\par\small\item[] {Chapter Summary}
\par\small\item[] {Solutions to Self-Study Exercises}
\par\small\item[] {Exercises}\
\end{COL}
\COend

\section{Introduction}
\label{introduction}
\noindent This chapter is about doing more than one thing at a time.  Doing
more than one thing at once is commonplace in our everyday lives.
For example, let's say your breakfast today consists of cereal,
toast, and a cup of java.  You have to do three things at once
to have breakfast: eat cereal, eat toast, and drink coffee.

Actually, you do these things ``at the same time'' by alternating
among them: You take a spoonful of cereal, then a bite of toast, and
then sip some coffee.  Then you have another bite of toast, or another
spoonful of cereal, more coffee, and so on, until breakfast is
finished.  If the phone rings while you're having breakfast, you will
probably answer it---and continue to have breakfast, or at least to
sip the coffee.  This means you're doing even more ``at the same
time.''  Everyday life is full of examples where we do more than one
task at the same time.

The computer programs we have written so far have performed one task
at a time.  But there are plenty of applications where a program needs
to do several things at once, or {\bf
concurrently}\index{concurrently}.  For example, if you wrote an
Internet chat program, it would let several users take part in a
discussion group.  The program would have to read messages from
several users at the same time and broadcast them to the other
participants in the group.  The reading and broadcasting tasks would
have to take place concurrently. In Java, concurrent programming is
handled by {\it threads}, the topic of this chapter.

\section{What Is a Thread?}
\label{what-isa-thread}
\noindent A {\bf thread}\index{thread} (or a {\it thread of execution} or a {\it thread of
control}) is a single sequence of executable statements within a
program.  For Java applications, the flow of control begins at the
first statement in {\tt main()} and continues sequentially through the
program statements.  For Java applets, the flow of control
begins with the first statement in {\tt init()}.  Loops within a
program cause a certain block of statements to be repeated.  If-else
structures cause certain statements to be selected and others to be
skipped.  Method calls cause the flow of execution to jump to another
part of the program, from which it returns after the method's
statements are executed.   Thus, within a single thread, you can trace
the sequential flow of execution from one statement to the next.

One way to visualize a thread is to imagine that you could make a
%\begin{marginalnote}\it Visualizing a thread\end{marginalnote}
\marginnote{Visualizing a thread}
list of the program's statements as they are executed by the
computer's central processing unit (CPU).  Thus, for a particular
execution of a program with loops, method calls, and selection
statements, you could list each instruction that was executed,
beginning at the first, and continuing until the program stopped, as a
single sequence of executed statements.  That's a thread!


Now imagine that we break a program up into two or more independent
threads.  Each thread will have its own sequence of instructions.
Within a single thread, the statements are executed one after the
other, as usual.  However, by alternately executing the statements
from one thread and another, the computer can run several threads {\it
concurrently}.  Even though the CPU executes one instruction at at
time, it can run multiple threads concurrently by rapidly alternating
among them. The main advantage of concurrency is that it allows the
computer to do more than one task at a time.  For example, the CPU
could alternate between downloading an image from the Internet and
running a spreadsheet calculation. This is the same way you ate toast
and cereal and drank coffee in our earlier breakfast example.  From
our perspective, it might look as if the computer had several CPUs
working in parallel, but that's just the illusion created by an
effectively scheduling threads.


\BOXtwo{JVM Threads.}{The Java Virtual Machine (JVM)
is itself an example of a multithreaded program.  JVM threads perform
tasks that are essential to the successful execution of Java programs.}

\BOXtwo{Garbage Collector Thread.}{One of the JVM
threads, the {\it garbage collector thread}, automatically reclaims
memory taken up by objects that are not used in your programs.   This
happens at the same time that the JVM is interpreting your program.}

\subsection{Concurrent Execution of Threads}
\noindent The technique of concurrently executing several tasks within a program
%\begin{marginalnote}\it Multitasking\end{marginalnote}
\marginnote{Multitasking}
is known as {\bf multitasking}\index{multitasking}. A {\bf task} in this sense is a
computer operation of some sort, such as reading or saving a file,
compiling a program, or displaying an image on the screen.  Multitasking
requires the use of a separate thread for each of the tasks.  The
methods available in the Java {\tt Thread} class make it possible (and
quite simple) to implement {\bf multithreaded}\index{multithreaded} programs.

Most computers, including personal computers, are {\it sequential}
machines that consist of a single CPU, which is capable of executing
one machine instruction at a time.  In contrast, {\it parallel
computers}, used primarily for large scale scientific and engineering
applications, are made up of multiple CPUs working in tandem.

Today's personal computers, running at clock speeds over 1
gigahertz---1 {\it gigahertz} equals 1 billion cycles per second---are capable of executing 
millions of machine instructions per second.
Despite its great speed, however, a single CPU can process only one
instruction at a time.

Each CPU uses a {\bf fetch-execute cycle}\index{fetch-execute cycle}
to retrieve the next instruction from memory and execute it.  Since
CPUs can execute only one instruction at a time, multithreaded
programs are made possible by dividing the CPU's time and sharing it
among the threads.  The CPU's schedule is managed by a {\bf scheduling
algorithm}, which is an algorithm that schedules threads for execution
on the CPU. The choice of a scheduling algorithm depends on the
platform on which the program is running. Thus, thread scheduling might
be handled differently on Unix, Windows, and Macintosh systems.


%\begin{figure}
\begin{figure}[tb]
\figaleft{chptr14/13f1.eps}{Each thread gets a slice of the CPU's time.
} {fig-timeslicing}

\end{figure}
%\end{figure}

One common scheduling technique is known as {\bf time slicing}\index{time slicing},
%\begin{marginalnote}\it CPUs are sequential\end{marginalnote}
\marginnote{CPUs are sequential}
in which each thread alternatively gets a slice of the CPU's time.
For example, suppose we have a program that consists of two threads.
Using this technique, the system would give each thread a small {\bf
quantum}\index{quantum} of CPU time---say, one thousandth of a second (one {\it
millisecond})---to execute its instructions.  When its quantum
expires, the thread would be {\it preempted} and the other thread
%\begin{marginalnote}\it Time slicing\end{marginalnote}
would be given a chance to run.  The algorithm would then alternate in this
{\bf round-robin}\index{round-robin} fashion between one thread and the other
(Fig.~\ref{fig-timeslicing}). During each millisecond on a 300-megahertz CPU, 
\marginnote{Time slicing}
a thread can execute 300,000 machine
instructions.  One {\bf megahertz} equals 1 million
cycles per second. Thus, within each second of real time, each thread will
receive 500 time slices and will be able to execute something like
150 million machine instructions.


Under {\bf priority scheduling}\index{priority scheduling}, threads of higher priority are
%\begin{marginalnote}\it Priority scheduling\end{marginalnote}
\marginnote{Priority scheduling}
allowed to run to completion before lower-priority threads are given a
chance.  An example of a high-priority thread would be one that
is processing keyboard input or any other kind of interactive input
from the user.  If such tasks were given low priority, users would
experience noticeable delays in their interaction, which would be
quite unacceptable.

The only way a high-priority thread can be preempted is if a thread of
still higher priority becomes available to run.  In many cases,
higher-priority threads are those that can complete their task within
a few milliseconds, so they can be allowed to run to completion
without starving the lower-priority threads. An example would be
processing a user's keystroke, a task that can begin as soon as the
key is struck and can be completed very quickly. Starvation occurs
when one thread is repeatedly preempted by other threads.

\BOXtwoleft{Thread Support.}{Depending on the hardware
platform, Java threads can be supported by assigning different threads
to different processors, by time slicing a single processor, or by
time slicing many hardware processors.}

\subsection{Multithreaded Numbers}
\noindent Let's consider a simple example of a threaded program.  Suppose we give
every individual thread a unique ID number, and each time it runs, it
prints its ID ten times. ~For example, when the thread with ID 1
runs the output produced would just be a sequence of ten 1's: 1111111111.



As shown in Figure~14.2, the {\tt NumberThread} class is defined
\marginfig{chptr14/p706f1.eps}{The {\tt NumberThread} class overrides the inherited {\tt run()}
method.}
{fig-p706f1}

%\begin{marginalnote}\it A {\tt Thread} subclass\end{marginalnote}
as a subclass of {\tt Thread} and overrides the {\tt run()} method.
To set the thread's ID number, the constructor takes a single
parameter that is used to set the thread's ID number.  In the {\tt
run()} method, the thread simply executes a loop that prints its own
number ten times:
%\begin{figure}
%\begin{graphic}
%\end{graphic}
%\end{figure}

\begin{jjjlisting}
\begin{lstlisting}
public class NumberThread extends Thread {
    int num;

    public NumberThread(int n) {
        num = n;
    }
    public void run() {
        for (int k=0; k < 10; k++) {
            System.out.print(num);
        } //for
    } // run()
} // NumberThread
\end{lstlisting}
\end{jjjlisting}
\marginnote{{\tt Thread} subclass}

%\begin{figure}
\begin{figure}[tb]
\figa{chptr14/p707f1.eps}{The {\tt Numbers} object creates several instances of
{\tt NumberThread} and tells each one to {\tt start()}.}
{fig-p707f1}
\end{figure}
%\end{figure}

\noindent Now let's define another class whose task will be to 
create many \mbox{\tt NumberThread}s and get them all running at the
same time (Fig.~\ref{fig-p707f1}).  For each {\tt NumberThread}, we
want to call its constructor and then {\tt start()} it:

\begin{jjjlisting}[27pc]
\begin{lstlisting}
public class Numbers {
  public static void main(String args[]) {
                                  // 5 threads
    NumberThread number1, number2, number3, number4, number5; 

     // Create and start each thread
    number1 = new NumberThread(1); number1.start(); 
    number2 = new NumberThread(2); number2.start();
    number3 = new NumberThread(3); number3.start();
    number4 = new NumberThread(4); number4.start();
    number5 = new NumberThread(5); number5.start();
  } // main()
} // Numbers
\end{lstlisting}
\end{jjjlisting}

\noindent When a thread is started by calling its {\tt start()}
method, it automatically calls its
\marginnote{Starting a thread}
{\tt run()} method.  The output generated by this version of the
{\tt Numbers} application is as follows:

\begin{jjjlisting}
\begin{lstlisting}
11111111112222222222333333333344444444445555555555
\end{lstlisting}
\end{jjjlisting}

\noindent From this output, it appears that the individual threads
were run in the order in which they were created.  In this case, each
thread was able to run to completion before the next thread started
running.

What if we increase the number of iterations that each thread performs?
Will each thread still run to completion? The following output was
generated for 200 iterations per thread:

\begin{jjjlisting}
\begin{lstlisting}[basicstyle=\scriptsize]
111111111111111111111111111111111111111111111111111111111111111111111
111111111111111111111111111111111111111111111111111111111111111111111
111111111111111111111111111111111111111111111111111111111111112222222
222222222222222222222222222222222222222222222222222222222222222222222
222222222222222222222222222222222222222222222222222222222222222222222
222222222222222222222222222222222222222222223333333333333333333333333
333333333333333333333333333333333333333333333333333333333333333333333
333333333333333333333333333444444444444444444444444444444444444444444
444444444444444444444444444444444444444444444444444444444444444444444
444444444455555555555555555555555555555555555555555555555555555555555
555555555555555555555555555555555555555555555555555555555555552222222
222233333333333333333333333333333333333333333333333333333333333333333
333333333333334444444444444444444444444444445555555555555555555555555
555555555555555555555555555555555555555555555555555555444444444444444
4444444444444444444444444444444444
\end{lstlisting}
\end{jjjlisting}

\noindent In this case, only thread 1 managed to run
to completion.  Threads 2, 3, 4, and 5 did not.  As this example
illustrates, the order and timing of a thread's execution are highly
unpredictable.  This example also serves to illustrate one way of
creating a multithreaded program:

%\begin{figure}
%\begin{graphic}
\marginfig{chptr14/p708f1.eps}{The {\tt java.lang.Thread} class. NOTE: NEEDS
REVISION TO ADD PRIORITY, YIELD() and SLEEP().}
{fig-p708f1}

%\end{graphic}
%\end{figure}

\begin{BL}
\item Create a subclass of the {\tt Thread} class.
\item Within the subclass, implement a method with the signature 
 {\tt void run()} that contains the statements to be executed by that
 thread.
\item Create several instances of the subclass and start each
 thread by invoking the {\tt start()} method on each instance.
\end{BL}

\BOXtwoleft{Thread Creation.}{One way to create a thread
in Java is to define a subclass of {\tt Thread} and override the
default {\tt run()} method.}

\section{From the Java Library: {\tt java.lang.Thread}}
\label{fromthe-java-library}
{The} {\tt java.lang.Thread} class contains the public methods shown
in Figure~14.4 (the figure contains only a partial list).  Note that
{\tt Thread} implements the {\tt Runnable} interface, which consists
simply of the {\tt run()} method.  As we will now see, another way to
create a thread is to instantiate a {\tt Thread} object and pass it a
{\tt Runnable} object that will become its body. This approach allows
you to turn an existing class into a separate thread.

A {\tt Runnable} object is any object that implements the {\tt
Runnable} interface---that is, any object that implements the {\tt
run()} method (Fig.~14.5).  The following example provides an
alternative way to implement the {\tt NumberThread} program:
\marginfig{chptr14/p709f1.eps}{Any object that implements the {\tt Runnable} 
interface can be run as a separate thread.}
{fig-p709f1}


\begin{jjjlisting}
\begin{lstlisting}
public class NumberPrinter implements Runnable {
    int num;

    public NumberPrinter(int n) {
        num = n;
    }

    public void run() {
        for (int k=0; k < 10; k++)
            System.out.print(num);
    } // run()
} // NumberPrinter
\end{lstlisting}
\end{jjjlisting}

\noindent Given this definition, we would then pass instances of this class to
the individual threads as we create them:

\begin{jjjlisting}[29pc]
\begin{lstlisting}
public class Numbers {
  public static void main(String args[]) {

    Thread number1, number2, number3, number4, number5;
     // Create and start each thread
    number1 = new Thread(new NumberPrinter(1)); number1.start(); 
    number2 = new Thread(new NumberPrinter(2)); number2.start();
    number3 = new Thread(new NumberPrinter(3)); number3.start();
    number4 = new Thread(new NumberPrinter(4)); number4.start();
    number5 = new Thread(new NumberPrinter(5)); number5.start();
  } // main()
} // Numbers
\end{lstlisting}
\end{jjjlisting}

\noindent The {\tt NumberPrinter} class implements
{\tt Runnable} by defining exactly the same {\tt run()} that was used
previously in the {\tt NumberThread} class.  We then pass instances of
{\tt NumberPrinter} when we create the individual threads.  Doing
things this way gives exactly the same output as earlier.  This example
serves to illustrate another way of creating a multithreaded program:

\begin{BL}
\item Implement the {\tt Runnable} interface for an existing class
 by implementing the {\tt void run()} method, which contains the
 statements to be executed by that thread.

\item Create several {\tt Thread} instances by first creating
 instances of the {\tt Runnable} class and passing each instance as
 an argument to the {\tt Thread()} constructor.

\item For each thread instance, start it by invoking the {\tt start()} 
 method on it.
\end{BL}


\BOXtwo{Thread Creation.}{A thread can be created by
passing a {\tt Runnable} object to a new {\tt Thread} instance.  The
object's {\tt run()} method will be invoked automatically as soon as
the thread's {\tt start()} method is called.}

\pagebreak
\JavaTIP{EFFECTIVE DESIGN}{Converting a Class to a Thread.}{Using the
{\tt Runnable} interface to create threads enables you to turn an existing
class into a thread.  For most applications, using the {\tt Runnable}
interface is preferable to redefining the class as a {\tt Thread}
subclass.}


\secEXRHone{Self-Study Exercise}
\label{self-study-exercise}
\begin{SSTUDY}
\item  Use the {\tt Runnable} interface to convert the
following class into a thread.  You want the
thread to print all the odd numbers up to its bound:

\begin{jjjlisting}
\begin{lstlisting}
public class PrintOdds {
    private int bound;
    public PrintOdds(int b) {
        bound = b;
    }

    public void print() {
        if (int k = 1; k < bound; k+=2)
            System.out.println(k);
    }
} // PrintOdds
\end{lstlisting}
\end{jjjlisting}
\end{SSTUDY}

\subsection{Thread Control}
\noindent The various methods in the {\tt Thread} class (Fig.~14.4)
can be used to exert
%\begin{marginalnote}\it Controlling threads\end{marginalnote}
\marginnote{Controlling threads}
some control over a thread's execution.  The {\tt start()} and
{\tt stop()} methods play the obvious roles of starting and stopping a
thread.  These methods will sometimes be called automatically.
For example, an applet is treated as a thread by the browser,
or appletviewer, which is responsible for starting and stopping it.

As we saw in the {\tt NumberThread} example, the {\tt run()} method
encapsulates the thread's basic algorithm.   It is usually not called
directly.  Instead, it is called by the thread's {\tt start()} method,
which handles any system-dependent initialization tasks before calling
{\tt run()}.

\subsection{Thread Priority}
\noindent The {\tt setPriority(int)} method lets you set a thread's priority to
an integer value between {\tt Thread.MIN\_PRIORITY} and
{\tt Thread.MAX\_PRIOR\-ITY}, the bounds defined as constants in the
{\tt Thread} class.  Using {\tt set\-Prior\-ity()} gives you some control over a
thread's execution.  In general, higher-priority threads get to run
before, and longer than, lower-priority threads.
%%%\begin{textblock}{1}(-0.75,4)
%%%\noindent\epsfig{file=e2kl-D1:PH-e2kl:Morelli:0333700MOREL:CommonArt:BX7_Texture/cyan.eps,%
%%%   height=51pc,width=3.25pc,clip=}
%%%\end{textblock}

\JavaRule[false]{Preemption.}{A higher-priority thread that
wants to run will {\it preempt} any threads of lower priority.}

\noindent To see how {\tt setPriority()} works, suppose we change
{\tt NumberThread}'s constructor to the following:

\begin{jjjlisting}
\begin{lstlisting}
public NumberThread(int n) {
    num = n;
    setPriority(n);
}
\end{lstlisting}
\end{jjjlisting}

\noindent In this case, each thread sets its priority to its ID number.
So, thread five will have priority five, a higher priority than
%\begin{marginalnote}\it Thread priority\end{marginalnote}
\marginnote{Thread priority}
all the other threads.   Suppose we now run 2 million iterations of
each of these threads.  Because 2 million iterations will take a long
time if we print the thread's ID on each iteration, let's modify the
{\tt run()} method, so that the ID is printed every 1 million
iterations:

\begin{jjjlisting}
\begin{lstlisting}
for (int k = 0; k < 10; k++)
    if (k % 1000000 == 0)
        System.out.print(num);
\end{lstlisting}
\end{jjjlisting}

\noindent Given this modification, we get the following output when we
run \break{\tt Numbers}:

\begin{jjjlisting}
\begin{lstlisting}
5544332211
\end{lstlisting}
\end{jjjlisting}

\noindent It appears from this output that the threads ran to completion
in priority order.  Thus, thread five completed 2 million iterations before
thread four started to run, and so on.  This shows that, on my system at
least, the Java Virtual Machine (JVM) supports priority scheduling.

\JavaTIP{PROGRAMMING TIP}{Platform Dependence.}{Thread implementation in
Java is platform dependent.  Adequate testing is necessary to ensure
that a program will perform correctly on a given platform.}

\JavaTIP{EFFECTIVE DESIGN}{Thread Coordination.}{One way to coordinate the
behavior of two threads is to give one thread higher priority than
another.}

\JavaTIP{DEBUGGING TIP}{Starvation.}{A high-priority thread that never
gives up the CPU can starve lower-priority threads by preventing them from
accessing the CPU.}

\subsection{Forcing Threads to Sleep}
\noindent The {\tt Thread.sleep()} and {\tt Thread.yield()} methods also
provide some control over a thread's behavior.  When executed by
a thread, the {\tt yield()} method causes the thread to
%\begin{marginalnote}\it Sleep vs.  yield\end{marginalnote}
yield the CPU, allowing the thread scheduler to choose another
\marginnote{Sleep versus yield}
thread.  The {\tt sleep()} method causes the thread to yield
and not to be scheduled until a certain amount of real time
has passed.
%%%\begin{textblock}{1}(45.7,4)
%%%\noindent\epsfig{file=e2kl-D1:PH-e2kl:Morelli:0333700MOREL:CommonArt:BX7_Texture/cyan.eps,%
%%%   height=51pc,width=3.25pc,clip=}
%%%\end{textblock}

\JavaRule{Sleep Versus Yield.}{Both the {\tt yield()}
and {\tt sleep()} methods yield the CPU, but the {\tt sleep()} method
keeps the thread from being rescheduled for a fixed amount of real
time.}

\noindent The {\tt sleep()} method can halt a running thread for a given number
of milliseconds, allowing other waiting threads to run.  The {\tt
sleep()} method throws an {\tt InterruptedException}, which is a
checked exception.  This means that the {\tt sleep()} call must be
embedded within a {\tt try/catch} block or the method it's in must
throw an {\tt InterruptedException}.  Try/catch blocks were covered in
Chapter~10.

\begin{jjjlisting}
\begin{lstlisting}
try {
    sleep(100);
} catch (InterruptedException e) {
    System.out.println(e.getMessage());
}
\end{lstlisting}
\end{jjjlisting}

\noindent For example, consider the following version of the
{\tt NumberPrinter.run()}:

\begin{jjjlisting}
\begin{lstlisting}
public void run() {
    for (int k=0; k < 10; k++) {
        try {
            Thread.sleep((long)(Math.random() * 1000));
        } catch (InterruptedException e) {
            System.out.println(e.getMessage());
        }
        System.out.print(num);
    } // for
} // run()
\end{lstlisting}
\end{jjjlisting}

\noindent In this example, each thread is forced to sleep for a random
number of milliseconds between 0 and 1,000. When a thread sleeps, it
gives up the CPU, which allows one of the other waiting threads to
run.  As you would expect, the output we get from this example will
reflect the randomness in the amount of time that each thread sleeps:

\begin{jjjlisting}
\begin{lstlisting}
14522314532143154232152423541243235415523113435451
\end{lstlisting}
\end{jjjlisting}

\noindent As we will see, the {\tt sleep()} method provides a
rudimentary form of thread synchronization, in which one thread
yields control to another.
%%%\begin{textblock}{1}(-0.75,4)
%%%\noindent\epsfig{file=e2kl-D1:PH-e2kl:Morelli:0333700MOREL:CommonArt:BX7_Texture/cyan.eps,%
%%%   height=51pc,width=3.25pc,clip=}
%%%\end{textblock}

\pagebreak
\secEXRHone{Self-Study Exercises}
\label{self-study-exercises}
\begin{SSTUDY}

\item  What happens if you run five {\tt NumberThread}s of equal priority
through 2 million iterations each? Run this experiment and note the
output.  Don't print after every iteration! What sort of scheduling
algorithm (round-robin, priority scheduling, or something else) was used
to schedule threads of equal priority on your system?

\item  Try the following experiment and note the output.   Let each thread
sleep for 50 milliseconds (rather than a random number of
milliseconds). How does this affect the scheduling of the threads? To
make things easier to see, print each thread's ID after every 100,000
iterations.

\item  The purpose of the Java garbage collector is to recapture
memory that was used by objects that are no longer being
used by your program.  Should its thread have higher or
lower priority than your program?
\end{SSTUDY}


\subsection{The Asynchronous Nature of Threaded Programs}
\noindent Threads are {\bf asynchronous}\index{asynchronous}. This means that the order of execution
and the timing of a set of threads are unpredictable, at
least from the programmer's point of view.  Threads are executed under
the control of the scheduling algorithm used by the operating system
and the Java Virtual Machine.  In general, unless threads are
explicitly synchronized, it is impossible for the programmer to
predict when and for how long an individual thread will
%\begin{marginalnote}\it Thread preemptions are unpredictable\end{marginalnote}
\marginnote{Thread preemptions are unpredictable}
run.  In some systems, under some circumstances, a thread might run to
completion before any other thread can run.  In other systems, or
under different circumstances, a thread might run for a short time and
then be suspended while another thread runs. Of course, when a thread
is preempted by the system, its state is saved so that its execution
can be resumed without losing any information.

One implication of a thread's asynchronicity is that it is not
generally possible to determine where in its source code an individual
thread might be preempted.  You can't even assume that a thread will
be able to complete a simple Java arithmetic operation once it has started
it.   For example, suppose a thread had to execute the following
operation:

\begin{jjjlisting}
\begin{lstlisting}
int N = 5 + 3;
\end{lstlisting}
\end{jjjlisting}

\noindent This operation computes the sum of 5 and 3 and assigns the result to
{\tt N}.  It would be tempting to think that once the thread started
%\begin{marginalnote}\it An arithmetic operation can be interrupted\end{marginalnote}
\marginnote{\raggedright An arithmetic operation can be interrupted}
this operation, it would be able to complete it, but that is not
necessarily so.  You have to remember that Java code is compiled into
a rudimentary bytecode, which is translated still further into
the computer's machine language.   In machine language, this operation
would break down into something like the following three steps:

\begin{jjjlisting}
\begin{lstlisting}
Fetch 5 from memory and store it in register A.
Add 3 to register A.
Assign the value in register A to N.
\end{lstlisting}
\end{jjjlisting}
%%%\begin{textblock}{1}(45.7,4)
%%%\noindent\epsfig{file=e2kl-D1:PH-e2kl:Morelli:0333700MOREL:CommonArt:BX7_Texture/cyan_02.eps,%
%%%   height=29pc,width=3.25pc,clip=}
%%%\end{textblock}

\noindent Although none of the individual machine instructions can
be preempted, the thread could be interrupted between any two machine
instructions.  The point here is that not even a single Java language
instruction can be assumed to be indivisible or unpreemptible.
Therefore, it is impossible to make any assumptions about when a
particular thread will run and when it will give up the CPU.~This
suggests the following important principle of multithreaded programs:
%\begin{marginalnote}\it Threads are asynchronous\end{marginalnote}
\marginnote{Threads are asynchronous}

\JavaRule{Asynchronous Thread Principle.}{Unless they are
explicitly prioritized or synchronized, threads behave in a completely
{\it asynchronous} fashion.}

\JavaTIP{PROGRAMMING TIP}{Thread Timing.}{Unless they are explicitly
synchronized, you cannot make any assumptions about when, or
in what order, individual threads will execute, or where a thread
might be interrupted or preempted during its execution.}

\noindent As we will see, this principle plays a large role
in the design of multithreaded programs.
%%%\end{javalibrary}

\begin{figure}[b]
\figaleft{chptr14/13f9.eps}{A depiction of a thread's life cycle.
} {fig-lifecycle}

\end{figure}

\section{Thread States and Life Cycle}
\label{thread-statesand-life-cycle}
\noindent Each thread has a {\bf life cycle}\index{life cycle} that
consists of several different states, which are summarized in
Figure~\ref{fig-lifecycle} and Table~14.1.  Thread states are
represented by labeled ovals, and the transitions between states are
represented by labeled arrows. Much
%\begin{marginalnote}\it Ready, running, and sleeping\end{marginalnote}
\marginnote{Ready, running, and sleeping}
of a thread's life cycle is under the control of the operating system
and the Java Virtual \mbox{Machine.}  Those transitions represented by method names---such as
%\begin{marginalnote}\it Controlling a thread\end{marginalnote}
\marginnote{Controlling a thread}
{\tt start()}, {\tt stop()}, {\tt wait()}, {\tt sleep()}, {\tt notify()}---can be controlled
by the program.  Of these methods, the {\tt stop()} method has been
deprecated in JDK 1.2 because it is inherently unsafe to stop a thread
in the middle of its execution.  Other transitions---such as {\it dispatch, I/O
request, I/O done, time expired, done sleeping}---are under the
control of the CPU scheduler. When first created a thread is in the
ready state, which means that it is ready to run.   In the ready state,
a thread is waiting, perhaps with other threads, in the 
{\bf ready queue}\index{ready queue}, 
for its turn on the CPU.  A {\bf queue}\index{queue} is like a waiting
line.  When the CPU becomes available, the first thread in the ready
queue will be {\bf dispatched}\index{dispatched}---that is, it will be given the
CPU.~It will then be in the running state.
\marginnote{The ready queue}

\begin{table}[t]
%\hphantom{\caption{A summary of the different thread states.}}
\TBT{2.5pc}{A summary of the different thread states.}
\hspace*{2pc}\begin{tabular}{ll}
\multicolumn{2}{l}{\color{cyan}\rule{22pc}{1pt}}\\[2pt]
%%%\TBCH{State} & \TBCH{Description}
{State} & {Description}
\\[-4pt]\multicolumn{2}{l}{\color{cyan}\rule{22pc}{0.5pt}}\\[2pt]
Ready      & The thread is ready to run and waiting for the CPU.\\[2pt]
Running    & The thread is executing on the CPU.\\[2pt]
Waiting    & The thread is waiting for some event to happen.\\[2pt]
Sleeping   & The thread has been told to sleep for a time.\\[2pt]
Blocked    & The thread is waiting for I/O to finish.\\[2pt]
Dead       & The thread is terminated.
\\[-4pt]\multicolumn{2}{l}{\color{cyan}\rule{22pc}{1pt}}
\end{tabular}
\endTB
%%
\end{table}

Transitions between the ready and running states happen under the
control of the CPU scheduler, a fundamental part of the Java runtime
%\begin{marginalnote}\it CPU Scheduler\end{marginalnote}
\marginnote{CPU scheduler}
system.  The job of scheduling many threads in a fair and
efficient manner is a little like sharing a single bicycle among
several children.   Children who are ready to ride the bike wait in line
for their turn.  The grown up (scheduler) lets the first child (thread)
ride for a period of time before the bike is taken away and given to
the next child in line.  In round-robin scheduling, each child (thread)
gets an equal amount of time on the bike (CPU).

When a thread calls the {\tt sleep()} method, it voluntarily gives up
the CPU, and when the sleep period is over, it goes back into the
ready queue.   This would be like one of the children deciding to rest
for a moment during his or her turn.   When the rest was over, the
child would get back in line.

When a thread calls the {\tt wait()} method, it voluntarily gives up
the CPU, but this time it won't be ready to run again until it is
%\begin{marginalnote}\it Threads can give up the CPU\end{marginalnote}
\marginnote{Threads can give up the CPU}
notified by some other thread.

This would be like one child giving his
or her turn to another child.  When the second child's turn is up, it
would notify the first child, who would then get back in line.

The system also manages transitions between the {\bf blocked}\index{blocked} and
ready states.  A thread is put into a blocked state when it does some
kind of I/O operation.  I/O devices, such as disk drives, modems, and
%\begin{marginalnote}\it Threads block on I/O operations\end{marginalnote}
\marginnote{Threads block on I/O operations}
keyboards, are very slow compared to the CPU.~Therefore, I/O
operations are handled by separate processors known as {\it
controllers}. For example, when a thread wants to read data from a
disk drive, the system will give this task to the disk controller,
telling it where to place the data.  Because the thread can't do anything
until the data are read, it is blocked, and another thread is allowed
to run.   When the disk controller completes the I/O operation, the
blocked thread is unblocked and placed back in the ready queue.

In terms of the bicycle analogy, blocking a thread would be like
giving the bicycle to another child when the rider has to stop to tie
his or her shoe.  Instead of letting the bicycle just sit there, we let
another child ride it.  When the shoe is tied, the child is ready to
ride again and goes back into the ready line.  Letting other threads
run while one thread is waiting for an I/O operation to complete
improves the overall utilization of the CPU.

\secEXRHone{Self-Study Exercise}
\label{self-study-exercise}
\begin{SSTUDY}

\item  Round-robin scheduling isn't always the best idea.  Sometimes
{\it priority scheduling} leads to a better system.  Can you
think of ways that priority scheduling---higher-priority
threads go to the head of the line---can be used to improve
the responsiveness of an interactive \mbox{program?}
\end{SSTUDY}

\markright{{\color{cyan}SECTION\,\,\arabic{section}\,\,$\bullet$}\,\, Threads and Interface Responsiveness}

\RHrightsec{13.4}{Using Threads to Improve Interface Responsiveness}

\markright{{\color{cyan}SECTION\,\,\arabic{section}\,\,$\bullet$}\,\, Threads and Interface Responsiveness}

\section{Using Threads to Improve \\\hspace*{2.8pc}Interface Responsiveness}
\label{using-threadsto-improvehspace-interface-responsiveness}
\RHrightsec{13.4}{Using Threads to Improve Interface Responsiveness}
\noindent One good use for a {\it multithreaded} program is to help
make a more responsive user interface.  In a single-threaded program,
a program that is executing statements in a long (perhaps even
infinite) loop remains unresponsive to the user's actions until the
loop is exited.  Thus, the user will experience a noticeable and
sometimes frustrating delay between the time an action is initiated
and the time it is actually handled by the program.

\subsection{Single-Threaded Design}
\noindent It's always a good idea that the interface be responsive to
user input, but sometimes it is crucial to an application.  For
example, suppose a psychology experiment is trying to measure how
quickly a user responds to a certain stimulus presented by a program.
Obviously, for this kind of application, the program should take
action as soon as the user clicks a button to indicate a response to
the stimulus. Let's work through an appropriate program design for
the experiment. First, we will formally state the situation and
describe what the program should do.  Then, we will examine the
components that would make up an effective program.
%\begin{figure}
%\begin{graphic}
\marginfig{chptr14/13f2.eps}{Random dots are drawn until the user
clicks the Clear button.}
{fig-dotsscreen}

%\end{graphic}
%\end{figure}

\subsection*{Problem Statement}
\noindent A psychologist is conducting a psychometric experiment to measure user
response to a visual cue and asks you to create the following
program.  The program should have two buttons.  When the Draw button
is clicked, the program begins drawing thousands of black dots at
%\begin{marginalnote}\it Problem specification\end{marginalnote}
\marginnote{Problem specification}
random locations within a rectangular region of the screen
(Fig.~14.7).  After a random time interval, the
program begins drawing red dots.  This change corresponds to the
presentation of the stimulus.  As soon as the stimulus is presented
the user is supposed to click on a Clear button, which clears the
drawing area.  To provide a measure of the user's reaction time, the
program should report how many red dots were drawn before the user
clicked the Clear button.

\pagebreak
Figure~\ref{fig-dotsgui} shows a design for this program's GUI.
%\begin{marginalnote}\it GUI design\end{marginalnote}
It contains a control {\tt JPanel} that contains the two
{\tt JButton}s.  The dots are drawn on a {\tt JPanel}, which is
positioned in the center of a {\tt BorderLayout} design. 
\marginnote{GUI design}

%\begin{figure}
\begin{figure}[tb]
\figa{chptr14/13f3.eps}{GUI design for the dot-drawing program.}
{fig-dotsgui}

\end{figure}
%\end{figure}

\subsection*{Problem Decomposition}

%\begin{marginalnote}\it Interface class and drawing class\end{marginalnote}
\noindent This program should be decomposed into two classes, a GUI to handle
\marginnote{Interface class and drawing class}
the user interface and a drawing class to manage the drawing.
The main features of its classes are as follows:

\begin{BL}
\item  {\tt RandomDotGUI} Class: This class manages the user interface,
responding to user actions by calling methods of the {\tt Dotty} class
(Fig.~14.9).

\item  {\tt Dotty} Class:  This class contains {\tt draw()} and {\tt clear()}
methods for drawing on the GUI's drawing panel (Fig.~14.10).
\end{BL}



\subsection*{The {\tt RandomDotGUI} Class}
\marginfig{chptr14/p716f1.eps}{The {\tt RandomDotGUI}.}
{fig-p716f1}

\noindent The implementation of {\tt RandomDotGUI} is shown in
Figure~\ref{fig-randomdotgui}.  The GUI arranges the control and
drawing panels in a {\tt BorderLayout} and listens for action events
on its {\tt JButton}s.  When the user clicks the Draw button, the
GUI's {\tt action\-Per\-formed()} method will create a new {\tt Dotty}
instance and call its {\tt draw()} method:

\begin{jjjlisting}
\begin{lstlisting}
dotty = new Dotty(canvas, NDOTS);
dotty.draw();
\end{lstlisting}
\end{jjjlisting}

\noindent Note that {\tt Dotty} is passed a reference to the drawing
{\tt canvas} as well as the number of dots to be drawn.   When the user
clicks the Clear button, the GUI should call the {\tt dotty.clear()}
method.  Of course, the important question is, how responsive will
the GUI be to the user's action?
\marginfig{chptr14/p716f2.eps}{The {\tt Dotty} class manages the drawing actions.}
{fig-p716f2}



\begin{figure}[h!]
\jjjprogstart
\begin{jjjlisting}
\begin{lstlisting}
import java.awt.*;
import javax.swing.*;
import java.awt.event.*;

public class RandomDotGUI extends JFrame 
                            implements ActionListener  {
  public final int NDOTS = 10000;
  private Dotty dotty;        // The drawing class
  private JPanel controls = new JPanel();
  private JPanel canvas = new JPanel();
  private JButton draw = new JButton("Draw");
  private JButton clear = new JButton("Clear");
 	
  public RandomDotGUI() {
    getContentPane().setLayout(new BorderLayout());
    draw.addActionListener(this);
    clear.addActionListener(this);
    controls.add(draw);
    controls.add(clear);
    canvas.setBorder(
      BorderFactory.createTitledBorder("Drawing Canvas"));
    getContentPane().add("North", controls);
    getContentPane().add("Center", canvas);
    getContentPane().setSize(400, 400);
  } 
  public void actionPerformed(ActionEvent e) {
    if (e.getSource() == draw) {
      dotty = new Dotty(canvas, NDOTS);
      dotty.draw();
    } else {
      dotty.clear();
    }
  } // actionPerformed()
  public static void main(String args[]){
    RandomDotGUI gui = new RandomDotGUI();
    gui.setSize(400,400);
    gui.setVisible(true);        
  }
} // RandomDotGUI
\end{lstlisting}
\end{jjjlisting}
\jjjprogstop{The {\tt RandomDotGUI} class.}
{fig-randomdotgui}
\end{figure}

\subsection*{The {\tt Dotty} Class}

\noindent The purpose of the {\tt Dotty} class will be to draw the
dots and to report how many red dots were drawn before the canvas was
cleared.  Because it will be passed a reference to the drawing panel
and the number of dots to draw, the {\tt Dotty} class will need
instance variables to store these two values.  It will also need a
variable to keep track of how many
%\begin{marginalnote}\it What data do we need?\end{marginalnote}
%%\marginnote{What data do we need?}
dots were drawn.  Finally, since it will be drawing within a fixed
rectangle on the panel, the reference coordinates and dimensions
of the drawing area are declared as class constants. 

The {\tt Dotty()} constructor method will be passed a reference to a
drawing panel as well as the number of dots to be drawn and will
merely assign these parameters to its instance variables.   In addition
to its constructor method, the {\tt Dotty} class will have public
{\tt draw()} and {\tt clear()} methods, which will be called from the
GUI.   The {\tt draw()} method will use a loop to draw random
dots.  The {\tt clear()} will clear the canvas and report the
number of dots drawn.

\begin{figure}[h!]
\jjjprogstart
\begin{jjjlisting}[30pc]
\begin{lstlisting}
import java.awt.*;
import javax.swing.*;   // Import Swing classes

public class Dotty {
                            // Coordinates
  private static final int HREF = 20, VREF = 20, LEN = 200; 
  private JPanel canvas;
  private int nDots;        // Number of dots to draw
  private int nDrawn;       // Number of dots drawn
  private int firstRed = 0; // Number of the first red dot

  public Dotty(JPanel canv, int dots) {
    canvas = canv;
    nDots = dots;
  }
  public void draw() {
    Graphics g = canvas.getGraphics();
    for (nDrawn = 0; nDrawn < nDots; nDrawn++) {
      int x = HREF + (int)(Math.random() * LEN);
      int y = VREF + (int)(Math.random() * LEN);
      g.fillOval(x, y, 3, 3);           // Draw a dot

      if ((Math.random() < 0.001) && (firstRed == 0)) {
        g.setColor(Color.red); // Change color to red
        firstRed = nDrawn;
      }
    } //for
  } // draw()
  public void clear() {  // Clear screen and report result
    Graphics g = canvas.getGraphics();
    g.setColor(canvas.getBackground());
    g.fillRect(HREF, VREF, LEN + 3, LEN + 3);
    System.out.println(
     "Number of dots drawn since first red = " + (nDrawn-firstRed));
  } // clear()
} // Dotty
\end{lstlisting}
\end{jjjlisting}
\jjjprogstop{The {\tt Dotty} class, single-threaded version.}
{fig-singledotty}
\end{figure}

The complete implementation of {\tt Dotty} is shown in
Figure~\ref{fig-singledotty}. Note how its {\tt draw()} method is
designed.  The drawing loop is bounded by the number of dots to be
drawn.  On each iteration, the {\tt draw()} method picks a
%\begin{marginalnote}\it Drawing algorithm\end{marginalnote}
random location within the rectangle defined by the coordinates
(HREF,VREF) and (HREF+LEN, VREF+LEN), and draws a dot there.   On each
iteration it also generates a random number.  If the random number is
less than 0.001, it changes the drawing color to red and keeps track
of the number of dots drawn up to that point.

The problem with this design is that as long as the {\tt draw()}
method is executing, the program will be unable to respond to
the GUI's Clear button.  In a single-threaded design, both the
%\begin{marginalnote}\it Single-threaded design: Waiting for the loop to end \end{marginalnote}
%%\marginnote{Single-threaded design: Waiting for the loop to end }
GUI and {\tt dotty} are combined into a single thread of execution
(Fig.~\ref{fig-dottytrace}).
\begin{figure}[h!]
\figaleft{chptr14/13f6.eps}{A single-threaded execution of random
dot drawing.
} {fig-dottytrace}

\end{figure}
When the user clicks the Draw button, the GUI's {\tt
actionPerformed()} method is invoked.  It then invokes {\tt Dotty}'s
{\tt draw()} method, which must run to completion before anything else
can be done.  If the user clicks the Clear button while the dots are
being drawn, the GUI won't be able to get to this until all the dots
are drawn.

If you run this program with {\tt nDots} set to 10,000, the program
will not clear the drawing panel until all 10,000 dots are drawn, no
matter when the Clear button is pressed.  Therefore, the values
reported for the user's reaction time will be wrong.   Obviously, since
it is so unresponsive to user input, this design completely fails to
satisfy the program's specifications.

\JavaRule{Single-Threaded Loop.}{In a single-threaded
design, a loop that requires lots of iterations will completely
dominate the CPU during its execution, which forces other tasks, including
user I/O tasks, to wait.}

\secEXRHone{Self-Study Exercise}
\label{self-study-exercise}
\begin{SSTUDY}

\item  Suppose the Java Virtual Machine (JVM) was single threaded and your
program got stuck in an infinite loop.  Would you be able to break out
of the loop by typing some special command (such as Control-C) from
the keyboard?
\end{SSTUDY}

\marginfig{chptr14/p719f1.eps}{In a multithreaded design, the {\tt Dotty} class implements {\tt Runnable}.}
{fig-p719f1}

\subsection{Multithreaded Drawing: The {\tt Dotty} Thread}
\noindent One way to remedy this problem is to create a second thread (in
addition to the GUI itself) to do the drawing.  The drawing thread
will be responsible just for drawing, while the GUI thread will be
responsible for handling user actions in the interface.   The trick to
making the user interface more responsive will be to interrupt the
drawing thread periodically so that the GUI thread has a chance to
\marginnote{Multithreaded design: Interrupt the drawing loop}
handle any events that have occurred.

As Figure~14.14 illustrates, the easiest way to convert
{\tt Dotty} into a thread is to have it implement the {\tt Runnable}
interface:

\begin{jjjlisting}
\begin{lstlisting}
public class Dotty implements Runnable {

    // Everything else remains the same

    public void run() {
        draw();
    }
}
\end{lstlisting}
\end{jjjlisting}

\noindent This version of {\tt Dotty} will perform the same task as before
except that it will now run as a separate thread of execution.  Note
that its {\tt run()} method just calls the {\tt draw()} method that we
defined in the previous version.   When the {\tt Dotty} thread is
started by the {\tt RandomDotGUI}, we will have a multithreaded
program.

However, just because this program has two threads doesn't
necessarily mean that it will be any more responsive to the user.
There's no guarantee that the drawing thread will stop as soon as the
Clear button is clicked.  On most systems, if both threads have
%\begin{marginalnote}\it Thread control\end{marginalnote}
\marginnote{Thread control}
equal priority, the GUI thread won't run until the drawing thread
finishes drawing all {\it N} dots.


\JavaTIP{DEBUGGING TIP}{Thread Control.}{Just breaking a program into two
separate threads won't necessarily give you the desired
performance.  It might be necessary to {\it coordinate} the threads.}

\noindent Therefore, we have to modify our design in order to guarantee that the
GUI thread will get a chance to handle the user's actions.   One
good way to do this is to have {\tt Dotty} sleep for a short
instance after it draws each dot.  When a thread sleeps, any other
%\begin{marginalnote}\it Using {\tt sleep()} to interrupt the drawing\end{marginalnote}
\marginnote{\raggedright Using {\tt sleep()} to interrupt the drawing}
threads that are waiting their turn will get a chance to run.  If the
GUI thread is waiting to handle the user's click on Clear, it
will now be able to call {\tt Dotty}'s {\tt clear()} method.

The new version of {\tt draw()} is shown in
Figure~\ref{fig-threadeddotty}.  In this version of {\tt draw()}, the
thread sleeps for 1 millisecond on each iteration of the loop.  This
will make it possible for the GUI to run on every iteration, so it
will handle user actions immediately.

\begin{figure}[p]
\jjjprogstart
\begin{jjjlistingleft}[31pc]{-5pc}
\begin{lstlisting}
import java.awt.*;
import javax.swing.*;     // Import Swing classes

public class Dotty implements Runnable {
                                     // Coordinates
  private static final int HREF = 20, VREF = 20, LEN = 200; 
  private JPanel canvas;
  private int nDots;           // Number of dots to draw
  private int nDrawn;          // Number of dots drawn
  private int firstRed = 0;    // Number of the first red dot
  private boolean isCleared = false; // Panel is cleared

  public void run() {
    draw();
  }
  public Dotty(JPanel canv, int dots) {
    canvas = canv;
    nDots = dots;
  }
  public void draw() {
    Graphics g = canvas.getGraphics();
    for (nDrawn = 0; !isCleared && nDrawn < nDots; nDrawn++) {
      int x = HREF + (int)(Math.random() * LEN);
      int y = VREF + (int)(Math.random() * LEN);
      g.fillOval(x, y, 3, 3);          // Draw a dot

      if (Math.random() < 0.001 && firstRed == 0) {
        g.setColor(Color.red);  // Change color to red
        firstRed = nDrawn;
      }
      try {
        Thread.sleep(1);        // Sleep for an instant
      } catch (InterruptedException e) {
        System.out.println(e.getMessage());
      }
    } //for
  } // draw()
  public void clear() {
    isCleared = true;
    Graphics g = canvas.getGraphics();
    g.setColor( canvas.getBackground() );
    g.fillRect(HREF,VREF,LEN+3,LEN+3);
    System.out.println("Number of dots drawn since first red = "
                                             + (nDrawn-firstRed));
  } // clear()
} // Dotty
\end{lstlisting}
\end{jjjlistingleft}
\jjjprogstop{By implementing the {\tt Runnable} interface, this version of {\tt Dotty} 
can run as a separate thread.}
{fig-threadeddotty}
\end{figure}

Another necessary change is that once the {\tt clear()} method is
called, the {\tt Dotty} thread should stop running (drawing). The
correct way to stop a thread is to use some variable whose value will
cause the run loop (or in this case the drawing loop) to exit, so the
new version of {\tt Dotty} uses the {\tt boolean} variable
{\tt isCleared} to control when drawing is stopped.  Note that the variable
is initialized to {\tt false} and then set to {\tt true} in the
{\tt clear()} method.  The for loop in {\tt draw()} will exit when
{\tt isCleared} becomes {\tt true}. This causes the {\tt draw()} method
to return, which causes the {\tt run()} method to return, which
causes the thread to stop in an orderly fashion.

\JavaTIP{EFFECTIVE DESIGN}{Threaded GUIs.}{Designing a multithreaded
GUI involves creating a secondary thread that will run concurrently
with the main GUI thread.  The GUI thread handles the user interface,
while the secondary thread performs CPU-intensive calculations.}

\JavaTIP{PROGRAMMING TIP}{Threading an GUI.}{Creating a
second thread within a GUI requires three steps: (1) Define the
secondary thread to implement the {\tt Runnable} interface, (2)
override its {\tt run()} method, and (3) incorporate some mechanism,
such as a {\tt sleep()} state, into the thread's run algorithm so that
the GUI thread will have a chance to run periodically.}

\subsection*{Modifications to {\tt RandomDotGUI}}

\noindent We don't need to make many changes in {\tt RandomDotGUI} to
get it to work with the new version of {\tt Dotty}. The primary change
comes in the {\tt actionPerformed()} method.  Each time the Draw
button was clicked in the original version of this method, we created
a {\tt dotty} instance and then called its {\tt draw()} method.  In
the revised version we must create a new {\tt Thread} and pass it
%\begin{marginalnote}\it Starting the drawing thread\end{marginalnote}
\marginnote{Starting the drawing thread}
an instance of {\tt Dotty}, which will then run as a separate
thread:

\begin{jjjlisting}
\begin{lstlisting}
public void actionPerformed(ActionEvent e) { 
    if (e.getSource() == draw) { 
        dotty = new Dotty(canvas, NDOTS); 
        dottyThread = new Thread(dotty); 
        dottyThread.start(); 
    } else { 
        dotty.clear(); 
    } 
} // actionPerformed()
\end{lstlisting}
\end{jjjlisting}

\noindent Note that in addition to a reference to {\tt dotty}
we also have a reference to a {\tt Thread} named {\tt dottyThread}.
This additional variable must be declared within the GUI.

Remember that when you call the {\tt start()} method, it automatically
calls the thread's {\tt run()} method.   When {\tt dottyThread} starts
to run, it will immediately call the {\tt draw()} method and start
drawing dots.  After each dot is drawn, {\tt dottyThread} will sleep
for an instant.

Notice how the GUI stops the drawing thread.  In the new version, {\tt
Dotty.clear()} will set the {\tt isCleared} variable, which will cause
the drawing loop to terminate.  Once again, this is the proper way to
stop a thread.  Thus, as soon as the user clicks the Clear button, the
{\tt Dotty} thread will stop drawing and report its result.

\JavaTIP{DEBUGGING TIP}{Stopping a Thread.}{The best way to stop a thread is to use
a {\tt boolean} control variable whose value can be set to true or false to
exit the {\tt run()} loop.}

\subsection{Advantages of Multithreaded Design}
\noindent By creating a separate thread for {\tt Dotty}, we have turned
a single-threaded program into a multithreaded program.   One thread,
the GUI, handles the user interface.  The second thread handles the
drawing task.  By forcing the drawing to sleep on each iteration, we
guarantee that the GUI thread will remain responsive to the user's
actions.  Figure~\ref{fig-threadedtrace} illustrates the difference
\marginnote{Divide and conquer!}
between the single- and multithreaded designs.  Note that the GUI
thread starts and stops the drawing thread, and the GUI thread
executes {\tt dotty.clear()}.  The drawing thread simply executes its
{\tt draw()} method.  In the single-threaded version, all of these
actions are done by one thread.

\begin{figure}[h!]
\figaleft{chptr14/13f8.eps}{Two independent threads: one for drawing,
the other for the GUI.
} {fig-threadedtrace}

\end{figure}

The trade-off involved in this design is that it will take
longer to draw {\it N} random dots, since {\tt dottyThread.draw()} will
%\begin{marginalnote}\it Trade-off: speed vs.  responsiveness\end{marginalnote}
\marginnote{Trade-off: speed vs.  responsiveness}
sleep for an instant on each iteration.  However, the extra time is hardly
noticeable.  By breaking the program into two separate threads of
control, one to handle the drawing task and one to handle the user
interface, the result is a much more responsive program.

\JavaTIP{EFFECTIVE DESIGN}{Responsive Interfaces.}{In order to give a
program a more responsive user interface, divide it into separate
threads of control.  Let one thread handle interactive tasks, such as
user input, and let the second thread handle CPU-intensive
computations.}

\secEXRHone{Self-Study Exercises}
\label{self-study-exercises}
\begin{SSTUDY}

\item  Someone might argue that because the Java Virtual Machine uses a
round-robin scheduling algorithm, it's redundant to use the
{\tt sleep()} method, since the GUI thread will get its chance to
run.  What's wrong with this argument for interface
responsiveness?

\item  Instead of sleeping on each iteration, another way to make the
interface more responsive would be to set the threaded {\tt Dotty}'s
priority to a low number, such as 1. Make this change, and experiment
with its effect on the program's responsiveness.   Is it more or less
responsive than sleeping on each iteration?  Why?
\end{SSTUDY}

\section{CASE STUDY: Cooperating Threads}
\label{casestudy-cooperating-threads}

\noindent For some applications it is necessary to synchronize and coordinate
the behavior of threads to enable them to carry out a cooperative
task.   Many cooperative applications are based on the {\bf
producer/consumer model}\index{producer/consumer model}. According to this model, two threads
cooperate at producing and consuming a particular resource or piece of
data.  The producer thread creates some message or result, and the
consumer thread reads or uses the result.  The consumer has
to wait for a result to be produced, and the producer has to take care
not to overwrite a result that hasn't yet been consumed.  Many types
of coordination problems fit the producer/consumer model.

One example of an application for this model would be to control the display of data
that is read by your browser.  As information arrives from the
%\begin{marginalnote}\it Producer and consumer threads\end{marginalnote}
\marginnote{Producer and consumer threads}
Internet, it is written to a buffer by the producer thread.   A
separate consumer thread reads information from the buffer and
displays it in your browser window.  Obviously, the two threads must be
carefully synchronized.  


\subsection{Problem Statement}
\noindent To illustrate how to address the sorts of problems that can arise when
you try to synchronize threads, let's consider a simple application in
which several threads use a shared resource.   You're familiar with
%\begin{marginalnote}\it Simulating a waiting line\end{marginalnote}
\marginnote{Simulating a waiting line}
those take-a-number devices that are used in bakeries to manage a
waiting line.  Customers take a number when they arrive, and the clerk
announces who's next by looking at the device.  As customers are
called, the clerk increments the ``next customer'' counter by one.

There are some obvious potential coordination problems here.   The device
must keep proper count and can't skip customers.  Nor can it give the
same number to two different customers.  Nor can it allow the clerk to
serve nonexistent customers.

Our task is to build a multithreaded simulation that uses a
model of a take-a-number device to coordinate the behavior of
customers and a (single) clerk in a bakery waiting line.  To help
illustrate the various issues involved in trying to coordinate
threads, we will develop more than one version of the program.

\subsection*{Problem Decomposition}
\noindent This simulation will use four classes of objects.
\marginnote{What classes do we need?}  Figure~\ref{fig-p726f1}
provides a UML representation of the interactions among the
objects. 
\begin{figure}[h!]
\figa{chptr14/p726f1.eps}{The {\tt Bakery} creates the {\tt Customer}
and {\tt Clerk} threads and the {\tt TakeANumber} gadget.
Then {\tt Customer}s request and receive waiting numbers
and the {\tt Clerk} requests and receives the number of 
the next customer to serve.}
{fig-p726f1}
\end{figure}
The {\tt TakeANumber} object will serve as a model of a
take-a-number device.  This is the resource that will be shared by the
threads, but it is not a thread itself.  The {\tt Customer} class, a
subclass of {\tt Thread}, will model the behavior of a customer who
arrives on line and takes a number from the {\tt TakeANumber} device.
There will be several {\tt Customer} threads created that then compete
for a space in line.  The {\tt Clerk} thread, which simulates the
behavior of the store clerk, should use the {\tt TakeANumber} device
to determine who the next customer is and should serve that customer.
Finally, there will be a main program that will have the task of
creating and starting the various threads.  Let's call this the {\tt
Bakery} class, which gives us the following list of classes:

\begin{BL}
\item  {\tt Bakery}---creates the threads and starts
the \mbox{simulation.}
\item  {\tt TakeANumber}---represents the gadget that keeps track of the
next customer to be served.
\item  {\tt Clerk}---uses the {\tt TakeANumber} to determine the
next customer and will serve the customer.
\item  {\tt Customer}---represents the customers who will use
the {\tt TakeANumber} to take their place in line.
\end{BL}


\subsection{Design: The {\tt TakeANumber} Class}
\noindent The {\tt TakeANumber} class must track two things: Which
customer will be served next, and which waiting number the next
customer will be given.  This suggests that it should have at least two
%\begin{marginalnote}\it What methods do we need?\end{marginalnote}
%\marginnote{What methods do we need?}
public methods: {\tt nextNumber()}, which will be used by customers to
get their waiting numbers, and {\tt nextCustomer()}, which will be used
by the clerk to determine who should be served (Fig.~14.18).
Each of these methods will simply retrieve the values of the instance
\marginfig{chptr14/p726f2.eps}{The {\tt TakeANumber} object keeps 
track of numbers and customers.}
{fig-p726f2}

variables, {\tt next} and {\tt serving}, which keep track of these two
values.  As part of the object's state, these variables should be {\tt
private}.


How should we make this {\tt TakeANumber} object accessible to all of
the other objects---that is, to all of the customers and to the
clerk?  The easiest way to do that is to have the main program pass a
reference to the {\tt TakeANumber} when it constructs the
%\begin{marginalnote}\it Passing a reference to a shared object\end{marginalnote}
{\tt Customer}s and the {\tt Clerk}.  They can each store the reference as
an instance variable.   In this way, all the objects in the simulation
can share a {\tt TakeANumber} object as a common resource.
Our design considerations lead to the definition of
\marginnote{Passing a reference to a shared object}
the {\tt TakeANumber} class shown in Figure~\ref{fig-takeanumber1}.

\begin{figure}[h]
\jjjprogstart
\begin{jjjlisting}
\begin{lstlisting}
class TakeANumber {
    private int next = 0;     // Next place in line
    private int serving = 0;  // Next customer to serve

    public synchronized int nextNumber() {
        next = next + 1;
        return next;
    } // nextNumber()

    public int nextCustomer() {
        ++serving;
        return serving;
    } // nextCustomer()

} // TakeANumber
\end{lstlisting}
\end{jjjlisting}
\jjjprogstop{Definition of the {\tt TakeANumber} class, Version 1.}
{fig-takeanumber1}
\end{figure}

Note that the {\tt nextNumber()} method is declared
{\tt synchronized}.  As we will discuss in more detail, this ensures that
only one customer at a time can take a number.   Once a thread begins
executing a {\tt synchronized} method, no other thread can execute that
method until the first thread finishes.
%\begin{marginalnote}\it Synchronized methods\end{marginalnote}
\marginnote{Synchronized methods} This is important because,
otherwise, several {\tt Customer}s could call the {\tt nextNumber}
method at the same time.  It's important that the customer threads
have access only one at a time, also called {\em mutually exclusive
access} to the {\tt TakeANumber} object.  This form of {\bf mutual
exclusion}\index{mutual exclusion} is important for the correctness of
the simulation.

\secEXRHone{Self-Study Exercise}
\label{self-study-exercise}
\begin{SSTUDY}

\item What is the analogue to mutual
exclusion in the real-world bakery situation?
\end{SSTUDY}

\subsection{Java Monitors and Mutual Exclusion}
\noindent An object that contains {\tt synchronized} methods has a {\bf monitor}\index{monitor}
associated with it.  A monitor is a widely used synchronization
%\begin{marginalnote}\it The monitor concept\end{marginalnote}
\marginnote{The monitor concept}
mechanism that ensures that only one thread at a time can execute a
{\tt synchronized} method.   When a {\tt synchronized} method is
called, a {\bf lock}\index{lock} is acquired on that object.  For example, if one
of the {\tt Customer} threads calls {\tt nextNumber()}, a lock will be
placed on that {\tt TakeANumber} object.   While an object is {\it
locked}, no other {\tt synchronized} method can run in that
object.  Other threads must wait for the lock to be released before
they can execute a {\tt synchronized} method.

While one {\tt Customer} is executing {\tt nextNumber()}, all other
\marginnote{Mutually exclusive access to a shared object}
{\tt Customer}s will be forced to wait until the first {\tt Customer}
is finished.  When the {\tt synchronized} method is exited, the lock
on the object is released, allowing other {\tt Customer} threads to
access their {\tt synchronized} methods.  In effect, a {\tt
synchronized} method can be used to guarantee mutually exclusive
access to the {\tt TakeANumber} object among the competing customers.

\JavaRule{{\tt synchronized}.}{Once a thread begins to
execute a {\tt synchronized} method in an object, the object is {\it
locked} so that no other thread can gain access to that object's
synchronized methods.}

\JavaTIP{EFFECTIVE DESIGN}{Synchronization.}{In order to restrict access
of a method or set of methods to one object at a time (mutual
exclusion), declare the methods {\tt synchronized}.}

\noindent One cautionary note here is that although a synchronized method blocks
access to other synchronized methods, it does not block access to
nonsynchronized methods.   This could cause problems.   We
will return to this issue in the next part of our case study when we discuss the
testing of our program.

\subsection{The {\tt Customer} Class}
\noindent A {\tt Customer} thread should model the behavior of taking a number
from the {\tt TakeANumber} gadget.   For the sake of this simulation,
let's suppose that after taking a number, the {\tt Customer} object just prints it
%\begin{marginalnote}\it Simulating the customer\end{marginalnote}
%\marginnote{Simulating the customer}
out.  This will serve as a simple model of ``waiting on line.'' What
about the {\tt Customer}'s state?  To help distinguish one customer
\marginfig{chptr14/p728f1.eps}{The {\tt Customer} thread.}
{fig-p728f1}

from another, let's give each customer a unique ID number starting at
10001, which will be set in the constructor method.  Also, as we noted
earlier, each {\tt Customer} needs a reference to the {\tt
TakeANumber} object, which is passed as a constructor parameter
(Fig.~14.20).  This leads to the definition of {\tt
Customer} shown in Figure~\ref{fig-customer1}.  Note that before
taking a number the customer sleeps for a random interval of up to
1,000 milliseconds.  This will introduce a bit of randomness into the
simulation.

\begin{figure}[h!]
\jjjprogstart
\begin{jjjlisting}
\begin{lstlisting}
public class Customer extends Thread {

  private static int number = 10000; // Initial ID number
  private int id;
  private TakeANumber takeANumber;

  public Customer( TakeANumber gadget ) {
    id = ++number;
    takeANumber = gadget;
  }

  public void run() {
    try {
      sleep( (int)(Math.random() * 1000 ) );
      System.out.println("Customer " + id + 
       " takes ticket " + takeANumber.nextNumber());
    } catch (InterruptedException e) {
      System.out.println("Exception " + e.getMessage());
    }
  } // run()
} // Customer
\end{lstlisting}
\end{jjjlisting}
\jjjprogstop{Definition of the {\tt Customer} class, Version 1.}
{fig-customer1}
\end{figure}

Another important feature of this definition is the use of the
{\tt static} variable {\tt number} to assign each customer a unique ID
number.   Remember that a {\tt static} variable belongs to the class
%\begin{marginalnote}\it Static (class) variables\end{marginalnote}
\marginnote{Static (class) variables}
itself, not to its instances.  Therefore, each {\tt Customer} that is
created can share this variable.  By incrementing it and assigning its
new value as the {\tt Customer}'s ID, we guarantee that each customer
has a unique ID number.

\JavaRule[false]{Static (Class) Variables.}{Static variables
are associated with the class itself and not with its instances.}

\JavaTIP[false]{EFFECTIVE DESIGN}{Unique IDs.}{Static variables are often used to
assign a unique ID number or a unique initial value to each instance
of a class.}



\subsection{The {\tt Clerk} Class}

\noindent The {\tt Clerk} thread should simulate the behavior of serving the
next customer in line, so the {\tt Clerk} thread will repeatedly
%\begin{marginalnote}\it Simulating the clerk\end{marginalnote}
%\marginnote{Simulating the clerk}
access {\tt TakeANumber.\-nextCustomer()} and then serve that customer.
For the sake of this simulation, we'll just print a message to
indicate which customer is being served.  Because there's only one
clerk in this simulation, the only variable in its internal state will
be a reference to the {\tt TakeANumber} object
(Fig.~14.22).  In addition to the constructor, all we
really need to define for this class is the {\tt run()} method.  This
%\begin{figure}
%\begin{graphic}
\marginfig{chptr14/p729f1.eps}{The {\tt Clerk} thread.}
{fig-p729f1}

%\end{graphic}
%\end{figure}
leads to the definition of {\tt Clerk} shown in
Figure~\ref{fig-clerk1}.  In this case, the {\tt sleep()} method is
necessary to allow the {\tt Customer} threads to run.  The {\tt Clerk}
will sit in an infinite loop serving the next customer on each
iteration.

\begin{figure}[h]
\jjjprogstart
\begin{jjjlisting}[27pc]
\begin{lstlisting}
public class Clerk extends Thread {
  private TakeANumber takeANumber;

  public Clerk(TakeANumber gadget) {
    takeANumber = gadget;
  }

  public void run() {
    while (true) {
      try {
        sleep( (int)(Math.random() * 50));
        System.out.println("Clerk serving ticket " + 
                        takeANumber.nextCustomer());
      } catch (InterruptedException e) {
        System.out.println("Exception " + e.getMessage() );
      }
    } //while
  } //run()
} // Clerk
\end{lstlisting}
\end{jjjlisting}
\jjjprogstop{Definition of {\tt Clerk}, Version 1.}
{fig-clerk1}
\end{figure}


\subsection{The {\tt Bakery} Class}
\noindent Finally, {\tt Bakery} is the simplest class to design.   It contains
the {\tt main()} method, which gets the whole simulation started.   As
we said, its role will be to create one {\tt Clerk} thread and
%\begin{marginalnote}\it The main program\end{marginalnote}
\marginnote{The main program}
several {\tt Customer} threads, and get them all started
(Fig.~\ref{fig-bakery}). Notice that the {\tt Customer}s and the
{\tt Clerk} are each passed a reference to the shared
{\tt TakeANumber} gadget.

\begin{figure}[h!]
\jjjprogstart
\begin{jjjlistingleft}[28pc]{-2pc}
\begin{lstlisting}
public class Bakery {
  public static void main(String args[]) {
    System.out.println("Starting clerk and customer threads");
    TakeANumber numberGadget = new TakeANumber();
    Clerk clerk = new Clerk(numberGadget);
    clerk.start();
    for (int k = 0; k < 5; k++) {
      Customer customer = new Customer(numberGadget);
      customer.start();
    }
  } // main()
} // Bakery
\end{lstlisting}
\end{jjjlistingleft}
\jjjprogstop{Definition of the {\tt Bakery} class.}
{fig-bakery}
\end{figure}

\subsection*{Problem: Nonexistent Customers}
\noindent Now that we have designed and implemented the classes, let's run
several experiments to test that everything works as intended.  Except
for the {\tt synchronized} {\tt nextNumber()} method, we've made
little attempt to make sure that the {\tt Customer} and {\tt Clerk}
threads will work together cooperatively, without violating the
%\begin{marginalnote}\it Testing and debugging\end{marginalnote}
real-world constraints that should be satisfied by the simulation.   If
we run the simulation as it is presently coded, it will generate five
customers and the clerk will serve all of them.  But we get something
\marginnote{Testing and debugging}
like the following output:

%%%%% output from bakery1

\begin{jjjlisting}
\begin{lstlisting}
Starting clerk and customer threads
  Clerk serving ticket 1
  Clerk serving ticket 2
  Clerk serving ticket 3
  Clerk serving ticket 4
  Clerk serving ticket 5
Customer 10004 takes ticket 1
Customer 10002 takes ticket 2
  Clerk serving ticket 6
Customer 10005 takes ticket 3
  Clerk serving ticket 7
  Clerk serving ticket 8
  Clerk serving ticket 9
  Clerk serving ticket 10
Customer 10001 takes ticket 4
Customer 10003 takes ticket 5
\end{lstlisting}
\end{jjjlisting}

\noindent Our current solution violates an important
real-world constraint: You can't serve customers before they enter the
%\begin{marginalnote}\it Problem: The clerk thread doesn't wait for customer threads\end{marginalnote}
\marginnote{Problem: The clerk thread doesn't wait for customer threads}
line! How can we ensure that the clerk doesn't serve a customer unless
there's actually a customer waiting?

The wrong way to address this issue would be to increase the amount of
sleeping that the {\tt Clerk} does between serving customers.  Indeed,
this would allow more customer threads to run, so it might appear to
have the desired effect, but it doesn't truly address the main
problem: A clerk cannot serve a customer if no customer is waiting.

The correct way to solve this problem is to have the clerk check that
there are customers waiting before taking the next customer.   One way
to model this would be to add a {\tt customerWaiting()} method to our
{\tt TakeANumber} object.   This method would return {\tt true}
whenever {\tt next} is greater than {\tt serving}. That will
correspond to the real-world situation in which the clerk can see
%\begin{marginalnote}\it The clerk checks the line\end{marginalnote}
\marginnote{The clerk checks the line}
customers waiting in line.  We can make the following
modification to {\tt Clerk.run()}:

\begin{jjjlisting}
\begin{lstlisting}
public void run() {
  while (true) {
    try {
      sleep((int)(Math.random() * 50));
      if (takeANumber.customerWaiting())    
        System.out.println("Clerk serving ticket " 
                    + takeANumber.nextCustomer());
    } catch (InterruptedException e) {
      System.out.println("Exception " + e.getMessage());
    }
  } // while
} // run()
\end{lstlisting}
\end{jjjlisting}

\noindent And we add the following method to {\tt TakeANumber} 
(Fig.~14.25):

\begin{jjjlisting}
\begin{lstlisting}
public boolean customerWaiting() {
    return next > serving;
}
\end{lstlisting}
\end{jjjlisting}
\marginfig{chptr14/p731f1.eps}{The revised {\tt TakeANumber} class.}
{fig-p731f1}


\noindent In other words, the {\tt Clerk} won't serve a customer
unless there are customers waiting---that is, unless {\tt next} is
greater than {\tt serving}.  Given these changes, we get the following
type of output when we run the simulation:

%%% output from bakery2
\begin{jjjlisting}
\begin{lstlisting}
Starting clerk and customer threads
Customer 10003 takes ticket 1
  Clerk serving ticket 1
Customer 10005 takes ticket 2
  Clerk serving ticket 2
Customer 10001 takes ticket 3
  Clerk serving ticket 3
Customer 10004 takes ticket 4
  Clerk serving ticket 4
Customer 10002 takes ticket 5
  Clerk serving ticket 5
\end{lstlisting}
\end{jjjlisting}

\noindent This example illustrates that when application design
involves cooperating threads, the algorithm used must ensure the
proper cooperation and coordination among the threads.

\JavaTIP{EFFECTIVE DESIGN}{Thread Coordination.}{When two or more threads
must behave cooperatively, their interaction must be carefully
coordinated by the algorithm.}

\subsection{Problem: Critical Sections}
%<sec_asynchronicity:\thisBheadno><pg_sec_asynchronicity:\thispageno>
\noindent It is easy to forget that thread behavior is asynchronous.  You can't
predict when a thread might be interrupted or might have to give up
\marginnote{Thread interruptions are \hfill\break unpredictable}
the CPU to another thread.   In designing applications that involve
cooperating threads, it's important that the design incorporates
features to guard against problems caused by
asynchronicity.   To illustrate this problem, consider the following
statement from the {\tt Customer.run()} method:

\begin{jjjlisting}
\begin{lstlisting}
System.out.println("Customer " + id + 
       " takes ticket " + takeANumber.nextNumber());
\end{lstlisting}
\end{jjjlisting}

\noindent Even though this is a single Java statement, it breaks
up into several Java bytecode statements.  A {\tt Customer} thread
could certainly be interrupted between getting the next number back
from {\tt TakeANumber} and printing it out.  We can simulate this by
breaking the {\tt println()} into two statements and putting a
{\tt sleep()} in their midst:

\begin{jjjlisting}
\begin{lstlisting}
public void run() {
  try {
    int myturn = takeANumber.nextNumber();
    sleep( (int)(Math.random() * 1000 ) );
    System.out.println("Customer " + id + 
                   " takes ticket " + myturn);
  } catch (InterruptedException e) {
    System.out.println("Exception " + e.getMessage());
  }
} // run()
\end{lstlisting}
\end{jjjlisting}

\noindent If this change is made in the simulation, you might get
the following output:

%%%Output from bakery3

\begin{jjjlisting}
\begin{lstlisting}
Starting clerk and customer threads
  Clerk serving ticket 1
  Clerk serving ticket 2
  Clerk serving ticket 3
Customer 10004 takes ticket 4
  Clerk serving ticket 4
  Clerk serving ticket 5
Customer 10001 takes ticket 1
Customer 10002 takes ticket 2
Customer 10003 takes ticket 3
Customer 10005 takes ticket 5
\end{lstlisting}
\end{jjjlisting}

\noindent Because the {\tt Customer} threads are now interrupted
in between taking a number and reporting their number, it looks as if
they are being served in the wrong order.  Actually, they
are being served in the correct order.  It's their reporting of
their numbers that is wrong!

The problem here is that the {\tt Customer.run()} method is being
interrupted in such a way that it invalidates the simulation's output.
%\begin{marginalnote}\it Problem: An interrupt in a critical section\end{marginalnote}
\marginnote{Problem: An interrupt in a critical section}
A method that displays the simulation's state should be designed so
that once a thread begins reporting its state, that thread will be
allowed to finish reporting before another thread can start reporting
its state.   Accurate \mbox{reporting} of a thread's state is a critical
element of the simulation's overall integrity.

A {\bf critical section}\index{critical section} is any section of a thread that should not be
interrupted during its execution.  In the bakery simulation, all of the
statements that report the simulation's progress are critical
sections.  Even though the chances are small that a thread will be
interrupted in the midst of a {\tt println()} statement, the faithful
reporting of the simulation's state should not be left to
chance.  Therefore, we must design an algorithm that prevents the
interruption of critical sections.

\subsection*{Creating a Critical Section}
\noindent The correct way to address this problem is to treat the reporting of
the customer's state as a critical section.  As we saw earlier when we
discussed the concept of a monitor, a {\tt synchronized} method within
a shared object ensures that once a thread starts the method, it will
be allowed to finish it before any other thread can start
%\begin{marginalnote}\it Making a critical section uninterruptible\end{marginalnote}
\marginnote{\raggedright Making a critical section uninterruptible}
it.  Therefore, one way out of this dilemma is to redesign the {\tt
nextNumber()} and {\tt nextCustomer()} methods in the {\tt
TakeANumber} class so that they report which customer receives a
ticket and which customer is being served
(Fig.~\ref{fig-takeanumber2}).  In this version all of the methods are
{\tt synchronized}, so all the actions of the {\tt TakeANumber} object
are treated as critical sections.

\begin{figure}[h!]
\jjjprogstart
\begin{jjjlisting}
\begin{lstlisting}
public class TakeANumber {
  private int next = 0;     // Next place in line
  private int serving = 0;  // Next customer to serve

  public synchronized int nextNumber(int custId) {
    next = next + 1;
    System.out.println( "Customer " + custId + "
                                  takes ticket " + next );
    return next;
  } // nextNumber()
  public synchronized int nextCustomer() {
    ++serving;
    System.out.println("  Clerk serving ticket " 
                                              + serving );
    return serving;
  } // nextCustomer()
  public synchronized boolean customerWaiting() {
    return next > serving;
  } // customerWaiting()
} // TakeANumber
\end{lstlisting}
\end{jjjlisting}
\jjjprogstop{Definition of the {\tt TakeANumber} class, Version 2.}
{fig-takeanumber2}
\end{figure}


Note that the reporting of both the
next number and the next customer to be served are now handled by
{\tt TakeANumber} in Figure~\ref{fig-takeanumber2} . Because the methods that handle these actions are
{\tt synchronized}, they cannot be interrupted by any threads involved in
the simulation.  This guarantees that the simulation's output will
faithfully report the simulation's state.

\pagebreak
Given these changes to {\tt TakeANumber}, we must remove the {\tt
println()} statements from the {\tt run()} methods in {\tt Customer}:

\begin{jjjlisting}
\begin{lstlisting}
public void run() {
  try {
    sleep((int)(Math.random() * 2000));
    takeANumber.nextNumber(id);
  } catch (InterruptedException e) {
    System.out.println("Exception: "+ e.getMessage());
  }
} // run()
\end{lstlisting}
\end{jjjlisting}

\noindent and from the {\tt run()} method in {\tt Clerk}:

\begin{jjjlisting}
\begin{lstlisting}
public void run() {
  while (true) {
    try {
      sleep( (int)(Math.random() * 1000));
      if (takeANumber.customerWaiting())
        takeANumber.nextCustomer();    
    } catch (InterruptedException e) {
      System.out.println("Exception: "+e.getMessage());
    }
  } // while
} // run()
\end{lstlisting}
\end{jjjlisting}

\noindent Rather than printing their numbers, these
methods now just call the appropriate methods in {\tt TakeANumber}.
Given these design changes, our simulation now produces the
following correct output:

%% OUTPUT PRODUCED BY bakery4

\begin{jjjlisting}
\begin{lstlisting}
Starting clerk and customer threads
Customer 10001 takes ticket 1
  Clerk serving ticket 1
Customer 10003 takes ticket 2
Customer 10002 takes ticket 3
  Clerk serving ticket 2
Customer 10005 takes ticket 4
Customer 10004 takes ticket 5
  Clerk serving ticket 3
  Clerk serving ticket 4
  Clerk serving ticket 5
\end{lstlisting}
\end{jjjlisting}

\noindent The lesson to be learned from this is that in
designing multithreaded programs, it is important to assume that if a
thread can be interrupted at a certain point, it will be interrupted at
%\begin{marginalnote}\it Preventing undesirable interrupts\end{marginalnote}
\marginnote{Preventing undesirable interrupts}
that point.  The fact that an interrupt is unlikely to occur is no
substitute for the use of a critical section.  This is something like
``Murphy's Law of Thread Coordination.''


\JavaTIP[false]{EFFECTIVE DESIGN}{The Thread Coordination Principle.}{Use
critical sections to coordinate the behavior of cooperating threads.
By designating certain methods as synchronized, you can ensure their
mutually exclusive access.  Once a thread starts a synchronized method,
no other thread will be able to execute the method until the first
thread is \mbox{finished.}}

\noindent In a multithreaded application, the classes and
methods should be designed so that undesirable interrupts will
not affect the correctness of the algorithm.

\JavaTIP[false]{PROGRAMMING TIP}{Critical Sections.}{Java's monitor mechanism
will ensure that while one thread is executing a synchronized method,
no other thread can gain access to it.  Even if the first thread is
interrupted, when it resumes execution again it will be allowed to
finish the synchronized method before other threads can access
synchronized methods in that object.}

\secEXRHone{Self-Study Exercise}
\label{self-study-exercise}
\begin{SSTUDY}

\item  Given the changes we've described, the bakery simulation should now
run correctly regardless of how slow or fast the {\tt Customer} and
{\tt Clerk} threads run.  Verify this by placing different-sized sleep
intervals in their {\tt run()} methods.  ({\it Note}: You don't want to
put a {\tt sleep()} in the synchronized methods because that would
undermine the whole purpose of making them {\tt synchronized} in the
first place.)
\end{SSTUDY}


\subsection{Using {\tt wait/notify} to Coordinate Threads}
\noindent The examples in the previous sections were designed to illustrate the
issue of thread asynchronicity and the principles of mutual exclusion
and critical sections.  Through the careful design of the algorithm and
the appropriate use of the {\tt synchronized} qualifier, we have
managed to design a program that correctly coordinates the behavior of
the {\tt Customer}s and {\tt Clerk} in this bakery simulation.

\subsection*{The Busy-Waiting Problem}
%<sec_busywait:\thisCheadno><pg_sec_busywait:\thispageno>
\noindent One problem with our current design of the Bakery algorithm is that it
uses {\em busy waiting}\index{busy waiting} 
%\begin{marginalnote}\it Busy waiting\end{marginalnote}
\marginnote{Busy waiting} on the part of the {\tt Clerk} thread. {\bf
Busy waiting} occurs when a thread, while waiting for some condition
to change, executes a loop instead of giving up the CPU.  Because busy
waiting is wasteful of CPU time, we should modify the algorithm.

\pagebreak
As it is presently designed, the {\tt Clerk} thread sits in a loop
that repeatedly checks whether there's a customer to serve:


\begin{jjjlisting}
\begin{lstlisting}
public void run() {
  while (true) {
    try {
      sleep( (int)(Math.random() * 1000));
      if (takeANumber.customerWaiting())
        takeANumber.nextCustomer();
    } catch (InterruptedException e) {
      System.out.println("Exception: " + e.getMessage());
    }
  } // while
} // run()
\end{lstlisting}
\end{jjjlisting}

A far better solution would be to force the {\tt Clerk} thread to wait
until a customer arrives without using the CPU.  Under such a design,
the {\tt Clerk} thread can be notified and enabled to run as
soon as a {\tt Customer} becomes available.  Note that
this description views the customer/clerk relationship as one-half of
the producer/consumer
%\begin{marginalnote}\it Producer/consumer\end{marginalnote}
\marginnote{Producer/consumer} relationship.  When a customer takes a
number, it {\it produces} a customer in line that must be served (that
is, {\it consumed}) by the clerk.

This is only half the producer/consumer relationship because we
haven't placed any constraint on the size of the waiting line.  There's
no real limit to how many customers  can be produced.   If we
did limit the line size, customers might be forced to wait before
taking a number if, say, the tickets ran out, or the bakery filled up.
In that case, customers would have to wait until the line resource
became available and we would have a full-fledged producer/consumer
relationship.

\subsection*{The {\tt wait/notify} Mechanism}
\noindent So, let's use Java's {\tt wait/notify} mechanism to
eliminate busy waiting from our simulation.  As noted in
Figure~\ref{fig-lifecycle}, the {\tt wait()} method puts a thread into
a waiting state, and {\tt notify()} takes a thread out of waiting and
places it back in the ready queue.  To use these methods in this
program we need to modify the {\tt nextNumber()} and {\tt
nextCustomer()} methods.  If there is no customer in line when the
{\tt Clerk} calls the {\tt nextCustomer()} method, the {\tt Clerk}
should be made to {\tt wait()}:

\begin{jjjlisting}[26.5pc]
\begin{lstlisting}
public synchronized int nextCustomer() {
  try {
    while (next <= serving)
      wait();
  } catch(InterruptedException e) {
    System.out.println("Exception: " + e.getMessage());
  } finally {
    ++serving;
    System.out.println(" Clerk serving ticket " + serving);
    return serving;
  }
}
\end{lstlisting}
\end{jjjlisting}

\noindent Note that the {\tt Clerk} still checks whether
there are customers waiting.  If there are none, the {\tt Clerk} calls
the {\tt wait()} method.  This removes the {\tt Clerk} from the CPU
%\begin{marginalnote}\it A waiting thread gives up the CPU\end{marginalnote}
\marginnote{A waiting thread gives up the CPU}
until some other thread notifies it, at which point it will be ready
to run again.  When it runs again, it should check that there is in
fact a customer waiting before proceeding.  That's why we use a
while loop here.  In effect, the {\tt Clerk} will wait until there's a
customer to serve.   This is not busy waiting because the
{\tt Clerk} thread loses the CPU and must be notified each time a customer
becomes available.

When and how will the {\tt Clerk} be notified? Clearly, the {\tt
Clerk} should be notified as soon as a customer takes a number.
Therefore, we put a {\tt notify()} in the {\tt nextNumber()} method,
which is the method called by each {\tt Customer} as it gets in line:

\begin{jjjlisting}
\begin{lstlisting}
public synchronized int nextNumber( int custId) {
  next = next + 1;
  System.out.println("Customer " + custId + 
                     " takes ticket " + next);
  notify();
  return next;
}
\end{lstlisting}
\end{jjjlisting}

\noindent Thus, as soon as a {\tt Customer} thread executes the
{\tt nextNumber()} method, the {\tt Clerk} will be notified and allowed
to proceed. 

What happens if more than one {\tt Customer} has executed a {\tt
wait()}?  In that case, the JVM will maintain a queue of waiting {\tt
Customer} threads.  Then, each time a {\tt notify()} is executed, the
JVM will take the first {\tt Customer} out of the queue and allow it
to proceed.

If we use this model of thread coordination, we no longer need to test
{\tt customerWaiting()} in the {\tt Clerk.run()} method.  It is
to be tested in the {\tt TakeANumber.nextCustomer()}. Thus,
the {\tt Clerk.run()} can be simplified to

\begin{jjjlisting}
\begin{lstlisting}
public void run() {
  while (true) {
    try {
      sleep((int)(Math.random() * 1000));
      takeANumber.nextCustomer();
    } catch (InterruptedException e) {
      System.out.println("Exception: "+ e.getMessage());
    }
  } // while
} // run()
\end{lstlisting}
\end{jjjlisting}

\noindent The {\tt Clerk} thread may be forced to wait
when it calls the {\tt nextCustomer} method.


Because we no longer need the {\tt customerWaiting()} method, we end
up with the new definition of {\tt TakeANumber} shown in
\marginfig{chptr14/p737f1.eps}{In the final design of {\tt TakeANumber}, 
its methods are synchronized.}
{fig-p737f1}

Figures~14.27 and~\ref{fig-takeanumber3}.  

\begin{figure}[h]
\jjjprogstart
\begin{jjjlisting}
\begin{lstlisting}
public class TakeANumber {
  private int next = 0;
  private int serving = 0;

  public synchronized int nextNumber(int custId) {
    next = next + 1;
    System.out.println( "Customer " + custId + 
                        " takes ticket " + next);
    notify();
    return next;
  } // nextNumber()

  public synchronized int nextCustomer() {
    try {
      while (next <= serving)  {
        System.out.println("  Clerk waiting ");
        wait();
      }
    } catch(InterruptedException e) {
       System.out.println("Exception " + e.getMessage());
    } finally {
       ++serving;
       System.out.println("  Clerk serving ticket " 
                                          + serving);
       return serving;
    }
  } // nextCustomer()
} // TakeANumber
\end{lstlisting}
\end{jjjlisting}
\jjjprogstop{The {\tt TakeANumber} class, Version 3.}
{fig-takeanumber3}
\end{figure}

\noindent Given this version of the program, the following kind of
output will be generated:

%%  Output from bakery5

\begin{jjjlisting}
\begin{lstlisting}
Starting clerk and customer threads
Customer 10004 takes ticket 1
Customer 10002 takes ticket 2
  Clerk serving ticket 1
  Clerk serving ticket 2
Customer 10005 takes ticket 3
Customer 10003 takes ticket 4
  Clerk serving ticket 3
Customer 10001 takes ticket 5
  Clerk serving ticket 4
  Clerk serving ticket 5
  Clerk waiting
\end{lstlisting}
\end{jjjlisting}



\JavaTIP[false]{PROGRAMMING TIP}{Busy Waiting.}{Java's {\tt wait/notify}
mechanism can be used effectively to eliminate busy waiting from a
multithreaded application.}

\JavaTIP[false]{EFFECTIVE DESIGN}{\hspace*{2.5pt}P\hspace*{0.5pt}r\hspace*{0.5pt}o\hspace*{0.5pt}d\hspace*{0.5pt}u%
\hspace*{0.5pt}c\hspace*{0.5pt}e\hspace*{0.5pt}r\hspace*{1pt}/\hspace*{1pt}C\hspace*{0.5pt}o%
\hspace*{0.5pt}n\hspace*{0.5pt}s\hspace*{0.5pt}u\hspace*{0.5pt}m\hspace*{0.5pt}e\hspace*{0.5pt}r%
\hspace*{0.5pt}.\hspace*{4.5pt}}
{The producer/consumer model is a useful design for coordinating the wait/notify interaction.}

\secEXRHone{Self-Study Exercise}
\label{self-study-exercise}
\begin{SSTUDY}

\item  An interesting experiment to try is to make the {\tt Clerk} a little
slower by making it sleep for up to 2,000 milliseconds.  Take a
guess at what would happen if you ran this experiment.  Then
run the experiment and observe the results.
\end{SSTUDY}

\subsection*{The {\tt wait/notify} Mechanism}
\noindent There are a number of important restrictions that must be observed
%\begin{marginalnote}\it Wait/notify go into synchronized methods\end{marginalnote}
\marginnote{\raggedright Wait/notify go into synchronized methods}
when using the {\tt wait/notify} mechanism:

\begin{BL}
\item  Both {\tt wait()} and {\tt notify()} are methods of the
{\tt Object} class, not the {\tt Thread} class.  This enables
them to lock objects, which is the essential feature of
Java's monitor mechanism.

\item  A {\tt wait()} method can be used within a {\tt synchronized}
method.  The method doesn't have to be part of a {\tt Thread}.


\item  You can only use {\tt wait()} and {\tt notify()}  within
{\tt synchronized} methods.  If you use them in other methods,
you will cause an {\tt Illegal\-Moni\-torStateException} with
the message ``current thread not owner.''

\item  When a {\tt wait()}---or a {\tt sleep()}---is used within a
{\tt synchronized} method, the lock on that object is released
so that other methods can access the object's {\tt synchronized}
methods.
\end{BL}

\JavaTIP[false]{DEBUGGING TIP}{Wait/Notify.}{It's easy to forget that the
{\tt wait()} and \mbox{\tt notify()} methods can only be used within
{\tt synchronized} methods.}

\section{CASE STUDY: The Game of Pong}
\label{casestudy-the-gameof-pong}

The game of Pong was one of the first computer video games and was all
the rage in the 1970s. The game consists of a ball that moves
horizontally and vertically within a rectangular region, and a single
paddle, which is located at the right edge of the region that can
be moved up and down by the user. When the ball hits the top, left,
or bottom walls or the paddle, it bounces off in the opposite
direction. If the ball misses the paddle, it passes through the right
wall and re-emerges at the left wall. Each time the ball bounces off
a wall or paddle, it emits a pong sound.

\subsection{A Multithreaded Design}

Let's develop a multithreaded GUI to play the game of Pong.
\marginfig{chptr14/pongscreensmall.eps}{The UI for Pong.}
{fig-pongscreen}

Figure~14.29 shows how the game's GUI should appear.  There
are three objects involved in this program: the frame, which serves
as the GUI, the ball, which is represented as a blue circle in the
program, and the paddle, which is represented by a red rectangle along
the right edge of the frame. What cannot be seen in this figure is 
that the ball moves autonomously, bouncing off the walls and paddle.
The paddle's motion is controlled by the user by pressing the up- and
down-arrow keys on the keyboard. 

We will develop class definitions for the ball, paddle, and the
frame. Following the example of our dot-drawing program earlier in
the Chapter, we will employ two independent threads, one for the GUI
and one for the ball.  Because the user will control the movements of
the paddle, the frame will employ a listener object to listen for and
respond to the user's key presses. 

Figure~\ref{fig-pongdesign} provides an overview of the
object-oriented design of the Pong program. The {\tt PongFrame} class
is the main class. It uses instances of the {\tt Ball} and {\tt
Paddle} classes.  {\tt PongFrame} is a subclass of {\tt JFrame} and
\begin{figure}[h!]
\figaleft{chptr14/pongdesign-sm.png}{Design of the Pong program.
} {fig-pongdesign}

\end{figure}
implements the {\tt KeyListener} interface. This is another of the
several event handlers provided in the {\tt java.awt} library. This
one handles {\tt KeyEvent}s and the {\tt KeyListener} interface
consists of three abstract methods: {\tt keyPressed()}, {\tt
keyTyped()}, and {\tt keyReleased()}, all of which are associated with
the act of pressing a key on the keyboard. All three of these
methods are implemented in the {\tt PongFrame} class.  A key-typed
event occurs when a key is pressed down. A key-release event occurs
when a key that has been pressed down is released.  A key-press event
is a combination of both of these events.

The {\tt Ball} class is a {\tt Thread} subclass. Its data and methods
are designed mainly to keep track of its motion within the program's
drawing panel. The design strategy employed here leaves the drawing of
the ball up to the frame. The {\tt Ball} thread itself just 
handles the movement within the program's drawing panel.  Note that the {\tt
Ball()} constructor takes a reference to the {\tt PongFrame}. As we
will see, the {\tt Ball} uses this reference to set the dimensions of
the frame's drawing panel. Also, as the {\tt Ball} moves, it will
repeatedly call the frame's {\tt repaint()} method to draw the ball.

The {\tt Paddle} class is responsible for moving the paddle up and
down along the drawing panel's right edge. Its public methods, {\tt
moveUP()} and {\tt moveDown()},  will be called by the frame in
response to the user pressing the up and down arrows on the
keyboard.  Because the frame needs to know where to draw the paddle,
the paddle class contains several public methods, {\tt getX()}, {\tt
getY()}, and {\tt resetLocation()}, whose tasks are to report the
paddle's location or to adjust its location in case the frame
is resized.

The {\tt PongFrame} controls the overall activity of the program.
Note in particular its {\tt ballHitsPaddle()} method. This method has
the task of determining when the ball and paddle come in contact as
the ball continuously moves around in the frame's drawing panel.  As
in the {\tt ThreadedDotty} example earlier in the chapter, it is
necessary for the {\tt Ball} and the the frame to be implemented as
separated threads so that the frame can be responsive to the user's
key presses.

\subsection{Implementation of the Pong Program}

We begin our discussion of the program's implementation with the {\tt Paddle} class
implementation (Fig.~\ref{fig-paddle}).
\begin{figure}[htb]
\jjjprogstart
\begin{jjjlisting}[27pc]
\begin{lstlisting}
public class Paddle {
    public static final int HEIGHT = 50;  // Paddle size
    public static final int WIDTH = 10;       
    private static final int DELTA = HEIGHT/2; // Move size
    private static final int BORDER = 0;
    private int gameAreaHeight;
    private int locationX, locationY;
    private PongFrame frame;

    public Paddle (PongFrame f) {
       frame = f;
       gameAreaHeight  = f.getHeight();
       locationX = f.getWidth()-WIDTH;
       locationY = gameAreaHeight/2;
    } // Paddle()
    public void resetLocation() {
       gameAreaHeight  = frame.getHeight();
       locationX = frame.getWidth()-WIDTH;
    }
    public int getX() {
         return locationX;
    }
    public int getY() {
         return locationY;
    }
    public void moveUp () {
       if (locationY > BORDER )
           locationY -= DELTA;
    } // moveUp()
    public void moveDown() {
       if (locationY + HEIGHT < gameAreaHeight - BORDER)
           locationY += DELTA;
    } // moveDown()
} // Paddle
\end{lstlisting}
\end{jjjlisting}
\jjjprogstop{Definition of the {\tt Paddle} class.}
{fig-paddle}
\end{figure}

Class constants, {\tt HEIGHT} and {\tt WIDTH} are used to
define the size of the {\tt Paddle}, which is represented on
the frame as a simple rectangle. The frame will use the
{\tt Graphics.fillRect()} method  to draw the paddle:

\begin{jjjlisting}[28pc]
\begin{lstlisting}
g.fillRect(pad.getX(),pad.getY(),Paddle.WIDTH,Paddle.HEIGHT);
\end{lstlisting}
\end{jjjlisting}

\noindent Note how the frame uses the paddle's {\tt getX()}
and {\tt getY()} methods to get the paddle's current location. 


The class constants {\tt DELTA} and {\tt BORDER} are used to control
the paddle's movement. {\tt DELTA} represents the number of pixels
that the paddle moves on each move up or down, and {\tt BORDER} is
used with {\tt gameAreaHeight} to keep the paddle within the
drawing area.  The {\tt moveUp()} and {\tt moveDown()} methods are
called by the frame each time the user presses an up- or down-arrow key.
They simply change the paddle's location by {\tt DELTA} pixels up or
down.

\pagebreak
The {\tt Ball} class (Fig.~\ref{fig-ball}) uses the class constant
{\tt SIZE} to determine the size of the oval that represents the ball,
drawn by the frame as follows:

\begin{jjjlisting}
\begin{lstlisting}
g.fillOval(ball.getX(),ball.getY(),ball.SIZE,ball.SIZE);
\end{lstlisting}
\end{jjjlisting}

\noindent As with the paddle, the frame uses the
ball's {\tt getX()} and {\tt getY()} method to determine the ball's
current location.

\begin{figure}[p]
\jjjprogstart
\begin{jjjlisting}[34.5pc]
\begin{lstlisting}
import javax.swing.*;
import java.awt.Toolkit;

public class Ball extends Thread {
  public static final int SIZE = 10;   // Diameter of the ball
  private PongFrame frame;           // Reference to the frame
  private int topWall, bottomWall, leftWall, rightWall;  // Boundaries
  private int locationX, locationY;     // Current location of the ball
  private int directionX = 1, directionY = 1; //x- and y-direction (1 or -1)
  private Toolkit kit = Toolkit.getDefaultToolkit(); //For beep() method

  public Ball(PongFrame f) {
    frame = f;
    locationX = leftWall + 1;                // Set initial location
    locationY = bottomWall/2;
  } // Ball()
  public int getX() {
    return locationX;
  } // getX()
  public int getY() {
    return locationY;
  } // getY()
  public void move() {
    rightWall = frame.getWidth() - SIZE;   // Define bouncing region
    leftWall = topWall = 0;                 // And location of walls
    bottomWall = frame.getHeight() - SIZE;
    locationX = locationX + directionX; // Calculate a new location
    locationY = locationY + directionY;

    if (frame.ballHitsPaddle()){
      directionX = -1;   //  move toward left wall
      kit.beep();
    } //if ball hits paddle
    if (locationX <= leftWall){
      directionX = + 1;   //  move toward right wall
      kit.beep();
    } //if ball hits left wall
    if (locationY + SIZE >= bottomWall || locationY <= topWall){
      directionY = -directionY;      //  reverse direction
      kit.beep();
    } //if ball hits top or bottom walls
    if (locationX  >= rightWall + SIZE) {
      locationX = leftWall + 1;   // jump back to left wall
    } //if ball goes through right wall
  } // move()
  public void run() {
    while (true) {
      move();                // Move
      frame.repaint();
      try {  sleep(15);
      } catch (InterruptedException e) {}
    } // while
  } // run()
} // Ball
\end{lstlisting}
\end{jjjlisting}
\jjjprogstop{Definition of the {\tt Ball} class.}
{fig-ball}
\end{figure}

Unlike the paddle, however, the ball moves autonomously.  Its {\tt
run()} method, which is inherited from its {\tt Thread} superclass,
repeatedly moves the ball, draws the ball, and then sleeps for a brief
interval (to slow down the speed of the ball's apparent motion).  The
{\tt run()} method itself is quite simple because it consists of a
short loop.  We will deal with the details of how the ball is painted
on the frame when we discuss the frame itself.

The most complex method in the {\tt Ball} class is the {\tt move()}
method. This is the method that controls the ball's movement within
the boundaries of the frame's drawing area. This method begins
by moving the ball by one pixel left, right, up, or down by adjusting
the values of its {\tt locationX} and {\tt locationY} coordinates:

\begin{jjjlisting}
\begin{lstlisting}
locationX = locationX + directionX; // Calculate location
locationY = locationY + directionY;
\end{lstlisting}
\end{jjjlisting}

\noindent The {\tt directionX} and {\tt directionY} variables are set
to either $+1$ or $-1$, depending on whether the ball is moving left
or right, up or down. After the ball is moved, the method uses a
sequence of {\tt if} statements to check whether the ball is touching
one of the walls or the paddle.  If the ball is in contact with the
top, left, or bottom walls or the paddle, its direction is changed by
reversing the value of the {\tt directionX} or {\tt directionY}
variable. The direction changes depend on whether the ball has touched
a horizontal or vertical wall. When the ball touches the right wall,
having missed the paddle, it passes through the right wall and
re-emerges from the left wall going in the same direction.

Note how the frame method, {\tt ballHitsPaddle()} is used to
determine whether the ball has hit the paddle. This is necessary
because only the frame knows the locations of both the ball and the
paddle.

\subsection{The {\tt KeyListener} Interface}

The implementation of the {\tt PongFrame} class is shown in
figure~\ref{fig-pongapplet}. The frame's main task is to manage the
drawing of the ball and paddle and to handle the user's key presses.
Handling keyboard events is a simple matter of implementing the {\tt
KeyListener} interface. This works in much the same way as the {\tt
ActionListener} interface, which is used to handle button clicks and
other {\tt ActionEvent}s.  Whenever a key is pressed, it generates
{\tt KeyEvent}s, which are passed to the appropriate methods of the
{\tt KeyListener} interface.

\begin{figure}[p]
\jjjprogstart
\begin{jjjlistingleft}[32pc]{-6pc}
\begin{lstlisting}
import javax.swing.*;
import java.awt.*;
import java.awt.event.*;

public class PongFrame extends JFrame implements KeyListener {
     private Ball ball;
     private Paddle pad;

     public PongFrame() {
         setBackground(Color.white);
         addKeyListener(this);
         pad = new Paddle(this);  // Create the paddle
         ball = new Ball(this);      // Create the ball
         ball.start();
     } // PongFrame()

     public void paint (Graphics g ) {
          g.setColor(getBackground());   // Erase the drawing area
          g.fillRect(0,0,getWidth(),getHeight());

          g.setColor(Color.blue);       // Paint the ball
          g.fillOval(ball.getX(),ball.getY(),ball.SIZE,ball.SIZE);

          pad.resetLocation();          // Paint the paddle
          g.setColor(Color.red);
          g.fillRect(pad.getX(),pad.getY(),Paddle.WIDTH,Paddle.HEIGHT);
     } // paint()

      public boolean ballHitsPaddle() {
          return ball.getX() + Ball.SIZE >= pad.getX()
             && ball.getY() >= pad.getY()
             && ball.getY() <= pad.getY() + Paddle.HEIGHT;
     } // ballHitsPaddle()

     public void keyPressed( KeyEvent e) { // Check for arrow keys
         int keyCode = e.getKeyCode();
         if (keyCode == e.VK_UP)            // Up arrow
             pad.moveUp();
         else if (keyCode == e.VK_DOWN)     // Down arrow
             pad.moveDown();
     } // keyReleased()
     public void keyTyped(KeyEvent e) {}     // Unused
     public void keyReleased( KeyEvent e) {} // Unused
     public static void main(String[] args) {
       PongFrame f = new PongFrame();
       f.setSize(400,400);
       f.setVisible(true);
} // PongFrame
\end{lstlisting}
\end{jjjlistingleft}
\jjjprogstop{Definition of the {\tt PongFrame} class.}
{fig-pongapplet}
\end{figure}

There's a bit of redundancy in the {\tt KeyListener} interface in the
sense that a single key press and release generates three {\tt
KeyEvent}s: A key-typed event, when the key is pressed, a key-released
event, when the key is released, and a key-pressed event, when the key
is pressed and released.  While it is important for some programs to
be able to distinguish between a key-typed and key-released event, for
this program, we will take action whenever one of the arrow keys is
pressed (typed and released). Therefore, we implement the {\tt
keyPressed()} method as follows:

\begin{jjjlisting}
\begin{lstlisting}
public void keyPressed( KeyEvent e) { // Check arrow keys
   int keyCode = e.getKeyCode();
   if (keyCode == e.VK_UP)            // Up arrow
      pad.moveUp();
   else if (keyCode == e.VK_DOWN)     // Down arrow
      pad.moveDown();
} // keyReleased()
\end{lstlisting}
\end{jjjlisting}

\noindent Each key on the keyboard has a unique code that 
identifies the key. The key's code is gotten from the {\tt
KeyEvent} object by means of the {\tt getKeyCode()} method. Then it is
compared with the codes for the up-arrow and down-arrow keys, which
are implemented as class constants, {\tt VK\_UP} and {\tt VK\_DOWN},
in the {\tt KeyEvent} class.  If either of those keys were typed, the
appropriate paddle method, {\tt moveUP()} or {\tt moveDown()}, is
called.  

Note that even though we are not using the {\tt keyPressed()} and {\tt
keyReleased()} methods in this program, it is still necessary to
provide implementations for these methods in the frame.  In order to
implement an interface, such as the {\tt KeyListener} interface, you
must implement {\em all} the abstract methods in the interface.  That
is why we provide trivial implementations of both the {\tt
keyPressed()} and {\tt keyReleased()} methods.

\subsection{Animating the Bouncing Ball}

Computer animation is accomplished by repeatedly drawing, erasing, and
re-drawing an object at different locations on the drawing panel.  The
frame's {\tt paint()} method is used for drawing the ball and the
paddle at their current locations.  The {\tt paint()} method is never
called directly. Rather, it is called automatically after the
constructor 
method {\tt PongFrame()}, when the program is started. It is then invoked indirectly by
the program by calling the {\tt repaint()} method, which is called in
the {\tt run()} method of the {\tt Ball} class.  The reason that {\tt
paint()} is called indirectly is because Java needs to pass it the
frame's current {\tt Graphics} object. Recall that in Java all
drawing is done using a {\tt Graphics} object.

In order to animate the bouncing ball, we first erase the current
image of the ball, then we draw the ball in its new location. We also
draw the paddle in its current location. These steps are carried out
in the frame's {\tt paint()} method. First, the drawing area is
cleared by painting its rectangle in the background color.  Then the
ball and paddle are painted at their current locations. Note that
before painting the paddle, we first call its {\tt resetLocation()}
method. This causes the paddle to be relocated in case the user has
resized the frame's drawing area.  There is no need to do this for
the ball because the ball's drawing area is updated within the {\tt
Ball.move()} method every time the ball is moved.

One problem with computer animations of this sort is that the repeated
\marginnote{Double buffering}
drawing and erasing of the drawing area can cause the screen to
flicker. In some drawing environments a technique known as {\bf double
buffering} is used to reduce the flicker. In double buffering, an
invisible, off-screen, buffer is used for the actual drawing
operations and it is then used to replace the visible image all at
once when the drawing is done.  Fortunately, Java's Swing components,
including {\tt JApplet} and {\tt JFrame}, perform an automatic form of
double buffering, so we needn't worry about it.  Some graphics
environments, including Java's AWT environment, do not perform double
buffering automatically, in which case the program itself must carry
it out.

Like the other examples in this chapter, the game of Pong provides a
simple illustration of how threads are used to coordinate concurrent
actions in a computer program. As most computer game fans will
realize, most modern interactive computer games utilize a
multithreaded design. The use of threads allows our interactive
programs to achieve a responsiveness and sophistication that is not
possible in single-threaded programs.  One of the great advantages of
Java is that it simplifies the use of threads, thereby making thread
programming accessible to programmers.  However, one of the lessons
that should be drawn from this chapter is that multithreaded programs
must be carefully designed in order to work effectively.

\secEXRHone{Self-Study Exercise}
\label{self-study-exercise}
\begin{SSTUDY}

\item Modify the {\tt PongFrame} program so that it contains a second
ball that starts at a different location from the first ball.

\end{SSTUDY}



\secSMHleft{Chapter Summary}
\label{chapter-summary}

\secKTH{Technical Terms}
\label{technical-terms}
%\parindent=0pt

\begin{KT}
asynchronous

blocked

busy waiting

concurrent

critical section

dispatched

fetch-execute cycle

lock

monitor

multitasking

multithreaded

mutual exclusion

priority scheduling

producer/consumer model

quantum

queue

ready queue

round-robin scheduling

scheduling algorithm 

task

thread

thread life cycle

time slicing

\end{KT}

\secSMHtwo{Summary of Important Points}
\label{summaryof-important-points}
\begin{SMBL}\baselineskip=11.9pt
\item  {\it Multitasking} is the technique of executing several
tasks at the same time within a single program.  In Java
we give each task a separate {\it thread of execution},
thus resulting in a {\it multithreaded} program.

\item  A {\it sequential} computer with a single {\it central
processing unit (CPU)} can execute only one machine instruction
at a time.  A {\it parallel} computer uses multiple CPUs operating
simultaneously to execute more than one instruction at a time.

\item  Each CPU uses a {\it fetch-execute cycle} to retrieve the
next machine instruction from memory and execute it.  The cycle is
under the control of the CPU's internal clock, which typically runs at
several hundred {\it megahertz}---where 1 megahertz (MHz) is 1
million cycles per second.

\item  {\it Time slicing} is the technique whereby several threads
can share a single CPU over a given time period.  Each thread is given
a small slice of the CPU's time under the control of some kind of
scheduling algorithm.  

\item In {\it round-robin scheduling}, each thread is
given an equal slice of time, in a first-come--first-served order.  In
{\it priority scheduling}, higher-priority threads are allowed to
run before lower-priority threads are run.

\item  There are generally two ways of creating threads in a program.
One is to create a subclass of {\tt Thread} and implement a
{\tt run()} method.  The other is to create a {\tt Thread} instance
and pass it a {\tt Runnable} object---that is, an object that
implements {\tt run()}.

\item  The {\tt sleep()} method removes a thread from the CPU
for a determinate length of time, giving other threads a chance
to run.

\item  The {\tt setPriority()} method sets a thread's priority.
Higher-priority threads have more and longer access to the CPU.

\item  Threads are {\it asynchronous}. Their timing and duration
on the CPU are highly sporadic and unpredictable.  In designing
threaded programs, you must be careful not to base your
algorithm on any assumptions about the threads' timing.

\item  To improve the responsiveness of interactive programs,
you could give compute-intensive tasks, such as drawing
lots of dots, to a lower-priority thread or to a thread
that sleeps periodically.

\item  A thread's life cycle consists of ready, running, waiting,
sleeping, and blocked states.  Threads start in the ready
state and are dispatched to the CPU by the scheduler,
an operating system program.  If a thread performs an
I/O operation, it blocks until the I/O is completed.  If
it voluntarily sleeps, it gives up the CPU.

\item  According to the {\it producer/consumer} model, two
threads share a resource, one serving to produce the
resource and the other to consume the resource.  Their
cooperation must be carefully synchronized.

\item  An object that contains {\tt synchronized} methods
is known as a {\it monitor}. Such objects ensure that only one thread
at a time can execute a synchronized method.   The object is {\it
locked} until the thread completes the method or voluntarily
sleeps.  This is one way to ensure mutually exclusive access to
a resource by a collection of cooperating threads.

\item  The {\tt synchronized} qualifier can also be used to
designate a method as a {\it critical section}, whose execution
should not be preempted by one of the other cooperating
threads.

\item  In designing multithreaded programs, it is useful to
assume that if a thread {\it can} be interrupted at a
certain point, it {\it will} be interrupted there.  Thread
coordination should never be left to chance.

\item  One way of coordinating two or more cooperating
threads is to use the {\tt wait/notify} combination.
One thread waits for a resource to be available, and the
other thread notifies when a resource becomes available.

\end{SMBL}

\pagebreak
\secANSHleft
%\addcontentsline{toc}{section}{\S~~~ Answers to Self-Study Exercises}

%%%\begin{ANSstr}
\begin{ANS}
\item \mbox{ }

\begin{jjjlisting}
\begin{lstlisting}
public class PrintOdds implements Runnable {
  private int bound;
  public PrintOdds(int b) {
   bound = b;
  }

  public void print() {
    if (int k = 1; k < bound; k+=2)
      System.out.println(k);
  }

  public void run() {
    print();
  }
}
\end{lstlisting}
\end{jjjlisting}

\item  On my system, the experiment yielded the following output, if
each thread printed its number after every 100,000 iterations:

\begin{jjjlisting}
\begin{lstlisting}
1111112222222211111111333333322222221111113333333
222224444444433333344444445555555544444555555555555
\end{lstlisting}
\end{jjjlisting}

\noindent This suggests that round-robin scheduling
is being used.


\item  If each thread is given 50 milliseconds of sleep on each
iteration, they tend to run in the order in which they
were created:

\begin{jjjlisting}
\begin{lstlisting}
123451234512345...
\end{lstlisting}
\end{jjjlisting}

\item  The garbage collector runs whenever the available memory drops
below a certain threshold.  It must have higher priority than
the application, since the application won't be able to run
if it runs out of memory.


%\item{EXERCISE 13.8}
\item  To improve the responsiveness of an interactive program, the system
could give a high priority to the threads that interact with the user
and a low priority to those that perform noninteractive computations,
such as number crunching.

\item  If the JVM were single threaded, it wouldn't be possible to break out
of an infinite loop, because the program's loop would completely
consume the CPU's attention.


%\item{EXERCISE 13.6}
\item  If round-robin scheduling is used, each thread will be get a portion
of the CPU's time, so the GUI thread will eventually get its turn.
But you don't know how long it will be before the GUI gets its
turn, so there might still be an unacceptably long wait before the
user's actions are handled.   Thus, to {\it guarantee} responsiveness,
it is better to have the drawing thread sleep on every iteration.

%\item{EXERCISE 13.7}
\item  If {\tt Dotty}'s priority is set to 1, a low value, this
does improve the responsiveness of the interface, but it
is significantly less responsive than using a {\tt sleep()}
on each iteration.

\item  In a real bakery only one customer at a time can take a number.  The
take-a-number gadget ``enforces'' mutual exclusion by virtue of its
design: There's room for only one hand to grab the ticket and there's
only one ticket per number.  If two customers got ``bakery rage'' and
managed to grab the same ticket, it would rip in half and neither
would benefit.

\item  One experiment to run would be to make the clerk's performance very
slow by using large sleep intervals.   If the algorithm is correct,
this should not affect the order in which customers are
served.  Another experiment would be to force the clerk to work fast
but the customers to work slowly.  This should still not affect the order
in which the customers are served.

\item  You should observe that the waiting line builds up as
customers enter the bakery, but the clerk should
still serve the customers in the correct order.
%%%\end{ANSstr}

\item A two-ball version of Pong would require the following
changes to the original version:
\begin{enumerate}
\item A new {\tt Ball()} constructor that has parameters to
set the initial location and direction of the ball.
\item The {\tt PongFrame} should create a new {\tt Ball}
instance, start it, and draw it.
\end{enumerate}

\end{ANS}

\secEXRHtwo{Exercises}
\label{exercises}

\begin{EXRtwo}
\item  Explain the difference between the following
pairs of terms:

\begin{EXRtwoLL}
\begin{multicols}{2}
\item  {\it Blocked} and {\it ready}.
\item  {\it Priority} and {\it round-robin} scheduling.
\item  {\it Producer} and {\it consumer}.
\item  {\it Monitor} and {\it lock}.
\item  {\it Concurrent} and {\it time slicing}.
\item  {\it Mutual exclusion} and {\it critical section}.
\item  {\it Busy} and {\it nonbusy} waiting.
\end{multicols}
\end{EXRtwoLL}

\marginnote{\raggedright{\bf Note:} For programming exercises, {\bf first} draw 
a UML class diagram describing all classes and
their inheritance relationships and/or associations.}

\item  Fill in the blanks.
\begin{EXRtwoLL}\baselineskip=12pt
\item  \rule{40pt}{0.5pt} happens when a CPU's time is divided among several
different threads.
\item  A method that should not be interrupted during its
execution is known as a  \rule{40pt}{0.5pt}\,.
\item  The scheduling algorithm in which each thread gets
an equal portion of the CPU's time is known as  \rule{40pt}{0.5pt}\,.
\item  The scheduling algorithm in which some threads can
preempt other threads is known as \rule{40pt}{0.5pt}.
\item  A \rule{40pt}{0.5pt} is a mechanism that enforces mutually exclusive
access to a synchronized method.
\item  A thread that performs an I/O operation may be forced
into the \rule{40pt}{0.5pt}  state until the operation is completed.
\end{EXRtwoLL}\baselineskip=11pt

\item  Describe the concept of {\it time slicing} as
it applies to CPU \mbox{scheduling.}

\item  What's the difference in the way concurrent
threads would be implemented on a computer with several processors and
on a computer with a single processor?

\item  Why are threads put into the {\it blocked} state
when they perform an I/O operation?

\item  What's the difference between a thread in the
sleep state and a thread in the ready state?

\item  {\bf Deadlock} is a situation that occurs when one thread is
holding a resource that another thread is waiting for, while 
the other thread is holding a resource that the first
thread is waiting for. Describe how deadlock can occur at a 
four-way intersection with cars entering from each
branch. How can it be avoided?

\item  {\bf Starvation} can occur if one thread is repeatedly 
preempted by other threads. Describe how starvation
can occur at a four-way intersection and how it can be avoided.

\item  Use the {\tt Runnable} interface to define
a thread that repeatedly generates random numbers
in the interval 2 through 12.

\item  Create a version of the {\tt Bakery} program that
uses two clerks to serve customers.

\item  Modify the {\tt Numbers} program so that the user can
interactively create {\tt NumberThread}s and assign them a priority.
Modify the {\tt NumberThread}s so that they print their numbers
indefinitely (rather than for a fixed number of iterations). Then
experiment with the system by observing the effect of introducing
threads with the same, lower, or higher priority.  How do the threads
behave when they all have the same priority? What happens when you
introduce a higher-priority thread into the mix? What happens when you
introduce a lower-priority thread into the mix?

\item  Create a bouncing ball simulation in which
a single ball (thread) bounces up and down in a vertical line.  The
ball should bounce off the bottom and top of the enclosing frame.

\item  Modify the simulation in the previous exercise so
that more than one ball can be introduced.   Allow the user to
introduce new balls into the simulation by pressing the space bar or
clicking the mouse.

\item  Modify your solution to the previous problem by having
the balls bounce off the wall at a random angle.


\item  {\bf Challenge:} One type of producer/consumer problem
is the {\it reader/writer} problem.  Create a subclass of
{\tt JTextField} that can be shared by threads, one of which writes a
random number to the text field, and the other of which reads the
value in the text field.  Coordinate the two threads so that the
overall effect of the program will be to print the values from 0
to 100 in the proper order.  In other words, the reader thread
shouldn't read a value from the text field until there's a value to
be read.   The writer thread shouldn't write a value to the text field
until the reader has read the previous value.

\item  {\bf Challenge:} Create a streaming banner thread that
moves a simple message across a panel.  The message should repeatedly
enter at the left edge of the panel and exit from the right
edge.  Design the banner as a subclass of {\tt JPanel} and have it
implement the {\tt Runnable} interface.  That way it can be added to
any user interface.  One of its constructors should take a {\tt String}
argument that lets the user set the banner's message.

\item {\bf Challenge: } Create a slide show program, which repeatedly
cycles through an array of images.  The action of displaying the
images should be a separate thread.  The frame thread should handle
the user interface.  Give the user some controls that let it pause,
stop, start, speed up, and slow down the images.

\item  {\bf Challenge:} Create a horse race simulation, using separate
threads for each of the horses.  The horses should race horizontally
across the screen, with each horse having a different vertical
coordinate.  If you don't have good horse images to use, just
make each horse a colored polygon or some other shape.  Have the
horses implement the {\tt Drawable} interface, which we introduced
in Chapter~{chapter-inheritance}.

\item  {\bf Challenge:} Create a multithreaded digital clock application.
One thread should keep time in an endless while loop. The other
thread should be responsible for updating the screen each second.

%%%\end{EXRtwostr}
\end{EXRtwo}
%% LocalWords:  multithreaded java nother
