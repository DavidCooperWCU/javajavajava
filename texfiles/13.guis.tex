%%%  Chapter 13: Graphical User Interfaces
%%%  3rd Edition

\setcounter{SSTUDYcount}{1}
\setcounter{chapter}{12}
\chapter{Graphical User Interfaces}
\label{chapter-guis}


\CObegin
\secCOBH{Objectives}
\label{objectives}
\noindent After studying this chapter, you will

\begin{COBL}
\item  Gain more experience with the Swing component set.
\item  Understand the relationship between the AWT and Swing.
\item  Learn more about Java's event model.
\item  Be able to design and build useful Graphical User Interfaces (GUIs).
\item  Appreciate how object-oriented design principles were used to
 extend Java's GUI capabilities.
\end{COBL}

\secCOLH{Outline}
\label{outline}
\begin{COL}
\item {Introduction}
\item {Java GUIs: From AWT to Swing}
\item {The Swing Component Set}
\item {Object-Oriented Design: Model-View-Controller Architecture}
\item {The Java Event Model}
\item {Case Study: Designing a Basic GUI}
\item {Containers and Layout Managers}
\item {Checkboxes, Radio Buttons, and Borders}
\item {Menus and Scroll Panes}
\item[] {{\color{cyan}Special Topic:} Are Computers Intelligent?}
\par\small\item[] {Chapter Summary}
\par\small\item[] {Solutions to Self-Study Exercises}
\par\small\item[] {Exercises}
\end{COL}
\COend


\section{Introduction}
\label{introduction}
\noindent As we have seen, a {\it Graphical User Interface (GUI)}
creates a certain way of interacting with a program.  It is what gives
a program its {\it look and feel}. In preceding chapters, we have
already used the basic components from which GUIs are created,
including buttons, text fields, labels, and text areas.  Throughout
this chapter, we will focus on designing and building GUIs that are
easy for users to navigate.  However, Java's GUI libraries are so
large that we will concentrate on only a handful of additional
components, including containers, check boxes, radio buttons, and
menus.

We will try to identify design principles that can be applied to the
design of more advanced interfaces.  Also, because Java's GUI classes
provide an excellent example of object-oriented design, we will
highlight some of the important design decisions and principles that
have influenced the development of Java's GUI classes in both the AWT
and Swing.  Let's begin with a brief history Java's GUI libraries.

\section{Java GUIs: From AWT to Swing}
\label{java-gu-is-from-aw-tto-swing}
\WWWjava
%\addcontentsline{toc}{section}{\S~~~ From the Java Library: AWT to Swing}
\tBOXseven{Ever since the release} of version 1.2 of the Java
Development Kit (JDK) in 2000, Java has contained two distinct
libraries of GUI components.  The {\it Abstract Windowing Toolkit
(AWT)} has been part of Java since the original 1.0 version of the JDK
1.0.  The more advanced {\em Swing component set} was first introduced
in JDK 1.1 and was extensively revised in JDK 1.2.

Although the original version of the AWT was suitable for developing
Java applets, it wasn't powerful enough to support full-fledged
applications. Commonly used programs, such as word processors and
spreadsheets, have GUI requirements that were just too much for the
original AWT.  The main problem was that the AWT was dependent on the
underlying operating system. That meant that Java GUI programs were
forced to rely on GUI elements that were part of the underlying
operating system. A Java GUI program running on a Windows platform had
to depend on Windows code for implementations of its buttons and text
fields.  A Java program running on Unix depended upon underlying Unix
code for its GUI components. Such dependence on the underlying
operating system made the AWT less portable and less efficient.

In contrast, the Swing GUI components are part of the {\it Java
Foundation Classes (JFC)}, a collection of classes that do not depend
as much on the underlying platform. The Swing library makes it
possible to write GUI programs entirely in Java. Because they are
rendered entirely by Java code, Swing components make it possible to
design GUIs that are truly platform independent.  Such programs are
much more portable than those which rely on AWT components and the
underlying platform.  A program that uses Swing components will have
the same look and feel on a Mac, Windows, or Unix platform.

\subsection{Heavyweight Versus Lightweight Components}
\noindent AWT components are based on the {\bf peer model}, a design
in which every AWT component has a corresponding class (a {\em peer})
written in the underlying \marginnote{The AWT peer model} system's
code.  For example, the {\tt java.awt.Button} class has a peer named
{\tt java.awt.peer.Button}. The peer class serves as the interface
between the Java code and the computer's underlying  windowing system.
The methods in the peer class are written in so-called {\em native}
code--that is, the non-Java code of the underlying operating
system. Therefore, AWT components are inherently platform dependent.

AWT components are called {\it heavyweight} because they depend on the
native (peer) system for their drawing and rendering.  Since every
AWT component has an associated peer component, a Java AWT component
would look just like the peer component. This is why an AWT button on
a Windows platform looks just like a Windows button.  In effect, the
AWT button, via its peer, creates and uses a Windows button.  When you
change the Java button's label, it must call a method in the peer
class that changes the label of the peer button.  This interaction
between Java and the native windowing system requires a good deal of
overhead, thereby affecting the overall efficiency of the system.

By contrast, a {\bf lightweight component} is one that is written entirely 
\marginnote{Lightweight components} 
in Java. Instead of depending on a native component for its rendering, a
lightweight component is drawn and rendered entirely by Java code.
Because they do not depend on underlying system code, Swing components are 
more efficient and more portable than corresponding AWT components.

%\begin{figure}
\begin{figure}[tb]
\figa{chptr13/p472f1.eps}{Swing classes, part 1: Relationship between
the AWT and the top-level Swing windows. REVISION: {\tt JWindow}
extends {\tt Window}.}
{fig-swing1-guis}

\end{figure}
%\end{figure}

Figures~\ref{fig-swing1-guis} and \ref{fig-swing2-guis} show the relationship
between AWT and Swing classes. The top-level Swing classes---the {\tt
JApplet}, {\tt JDialog}, {\tt JFrame}, and {\tt JWindow}---are direct
subclasses of their corresponding AWT counterparts. These are the
top-level GUI windows. The remaining Swing components
(Fig.~\ref{fig-swing2-guis}) are subclasses of {\tt java.awt.Component}
and {\tt java.awt.Container}.  As you can see, the names of Swing and
AWT components are very similar. Swing components that have
corresponding AWT components have names that begin with ``J.''

One might think that because Swing components are superior
to their AWT counterparts, the AWT package will eventually be dropped.
However, this is not likely. Even if a Java program uses Swing
components exclusively, that will still not break the dependence on
the AWT.

%\begin{figure}
\begin{figure}[tb]
\figaleft{chptr13/p473f1.eps}{Swing classes, part 2: Swing GUI components are
derived from the {\tt JComponent} class.
} {fig-swing2-guis}

\end{figure}
%\end{figure}

There are several reasons for this dependence. First, Swing's
top-level window classes---{\tt JApplet}, {\tt JDialog}, {\tt JFrame},
and {\tt JWindow}---are defined as extensions to their AWT
counterparts.  This means that Swing-based GUIs are still dependent on
the AWT. Java programs need to have some way to map their windows to
the windowing system used on the native (Windows, Unix, or Macintosh)
platform.  The AWT's top-level windows---{\tt Window}, {\tt Frame},
{\tt Dialog}, and {\tt Panel}---provide that mapping.

Second, the {\tt JComponent} class, which is the basis for all Swing
components, is derived from {\tt java.awt.Container}. There are many
more such dependencies.  Fundamentally, Swing components are based on
the AWT.

Finally, all GUI applications and applets use layout managers ({\tt
java.\-awt.FlowLayout}), fonts ({\tt java.awt.Font}), colors ( {\tt
java.awt.Color}), and other non-component classes that are defined in
the AWT. There is just no way to design a GUI without using AWT classes.
Therefore, the programs presented in this and subsequent chapters will
use Swing components instead of corresponding AWT components, but
they also will use layouts and other elements from the AWT.

\JavaTIP{PROGRAMMING TIP}{Swing Documentation.}{Complete documentation of
the Swing classes is available for downloading or browsing on Sun's
Web site at
{\tt http://java.sun.com/reference/api/index.html}}


\section{The Swing Component Set}
\label{the-swing-component-set}

\noindent Java's Swing components are defined in a
collection of packages named {\tt javax.swing.*}, which is imported by
the code shown in this and subsequent chapters.  Swing packages
include the following:

\begin{jjjlisting}
\begin{lstlisting}
javax.swing.event.*
javax.swing.text.*
javax.swing.plaf.*
\end{lstlisting}
\end{jjjlisting}

\noindent The {\tt javax.swing.event} package defines the
various Swing events and their listeners, such as the {\tt MenuEvent}
and the {\tt MenuListener}.  (In the AWT, the AWT events and
listeners were defined in {\tt java.awt.event}.)

The {\tt javax.swing.text} package contains the classes for {\tt
JTextField} and {\tt JTextComponent}.  The Swing text components are
more complex than their AWT counterparts.  For example, one of their
important features is the ability to undo changes made to the text
they contain.  This feature is crucial for building sophisticated
word-processing applications.

The {\tt javax.swing.plaf} package contains Swing's look-and-feel
\marginnote{Look and feel}
classes.  The term {\it plaf} is an acronym for {\bf pluggable
look and feel}. It refers to the fact that changing an application's
look and feel is a simple matter of ``plugging in'' a different plaf
model.  Changing how a program looks does not change what it does.

Swing's platform-independent look and feel is achieved by placing all
the code responsible for drawing a component in a class that is
separate from the component itself.  For example, in addition to {\tt
JButton}, the class that defines the button control, there will be a
separate class responsible for drawing the button on the screen.  The
drawing class will control the button's color, shape, and other
characteristics of its appearance.

There are several look-and-feel packages built into Swing.  For
example, the {\tt javax.swing.plaf.motif} package contains the classes
that implement the Motif interface, a common Unix-based interface.
The {\tt javax.swing.plaf.windows} packages contains classes that
support a Windows look and feel, and the {\tt javax.swing.plaf.metal}
package provides classes that support the Metal interface, a Java look
and feel. These classes know how to draw each component and how to
react to mouse, keyboard, and other events associated with these
components.  


\section{OBJECT-ORIENTED DESIGN:  \\Model-View-Controller Architecture}
\label{objectorienteddesign-model-view-controller-architecture}
%\addcontentsline{toc}{section}{\S~~~ Object-Oriented Design: Model-View-Controller Architecture}

\noindent Java's Swing components have been implemented using an
object-oriented design known as the {\bf model-view-controller (MVC)}
model.  Any Swing component can be considered in terms of three
independent aspects: what state it's in (its model), how it looks (its
view), and what it does (its controller).

For example, a button's role is to appear on the interface waiting to
be clicked.   When it is clicked, the button's appearance changes.  It
looks pushed in or it changes color briefly, and then it changes back
to its original (unclicked) appearance.   In the MVC model, this aspect
of the button is its {\bf view}\index{view}. If you were designing an interface
\marginnote{View}
for a button, you would need visual representations for both the
clicked and the unclicked button (as well as other possible states).

When you click a button, its internal state changes from pressed to
unpressed.  You've also probably seen buttons that were
disabled---that is, in a state where they just ignore your clicks.
Whether a button is enabled or disabled and whether it is pressed or
\marginnote{Model}
not are properties of its internal state.  Taken together, such properties
constitute the button's {\bf model}\index{model}.  Of
course, a button's view---how it looks---depends on its model.  When a
button is pressed, it has one appearance, and when it is disabled, it
has another.

Because a button's state will change when it is clicked or when it is
enabled by the program, some object needs to keep track of these
changes.  That part of the component is its {\bf
controller}\index{controller}.  
\marginnote{Controller}

Figure~\ref{fig-mvc} shows how the button's model, view, and
controller interact with each other.  Suppose the user clicks the
button.  This action is detected by the controller.  Whenever the mouse
button is pressed, the controller tells the model to change into the
pressed state.  The model, in turn, generates an event that is passed
to the view.  The event tells the view that the button needs to be
redrawn to reflect its change in state.

%\begin{figure}
\begin{figure}[t]
\figa{chptr13/9f3.eps}{The model-view- controller architecture.}
{fig-mvc}

\end{figure}
%\end{figure}

When the mouse button is released, a similar sequence of events occurs.
The model is told to change to the unpressed state.  It in turn
generates an event, handled by the view, which changes the button's
appearance.

A change in the button's appearance does not necessarily depend on
direct action by the user.  For example, the program itself could call a
method that disables the button.  In this case, the program issues a
\mbox{command} \mbox{directly} to the model, which in turn generates an event that
causes the view to change the object's appearance.

For some Swing components, such as the text components, this
three-part model is implemented almost exactly as we just
described.  For others, such as {\tt JButton}, one class is used to
implement both the view and the controller.  The {\tt JButton} model is
defined in the {\tt DefaultButtonModel} class, and its view and
controller are defined in the {\tt BasicButtonUI} class (The {\tt UI}
acronym stands for User Interface).  The point is that for some
components, Swing has organized the view and control---the look and
the feel---into a single class.

\subsection{Pluggable Look and Feel}
%\addcontentsline{toc}{subsection}{~~~\S~~~ Pluggable Look and Feel}
\noindent The MVC model uses a clear division of labor to implement a GUI
component.   The main advantage of this design is the independence
between the model, the view, and the controller.  If you want to give a
button a different look and feel, you can redefine its view and its
controller.

By combining the view and controller into a single class, Swing makes
it even easier to change a component's look and feel.  For example, to
design your own look and feel for a {\tt JButton}, you would define a
class that implemented all of the methods in the
{\tt BasicButtonUI}. Of course, this is a job for an experienced software
developer.

However, if you just want to set your program to use one of the
pre-defined look and feel models, you can simply use the {\tt
UIManager.set\-LookAndFeel()} method:

\begin{jjjlisting}
\begin{lstlisting}
public static void main (String args[]){
  try{
     UIManager.setLookAndFeel(
        "javax.swing.plaf.metal.MetalLookAndFeel");
  }catch (Exception e) {
     System.out.err("Exception: " + e.getMessage());
  }
}//main()
\end{lstlisting}
\end{jjjlisting}

Java's default, the {\it Metal} look and feel, has been designed
specifically for Java applications.  For a Windows look, you can use
the following argument:
\verb|com.sun.java.swing.plaf.windows.WindowsLookAndFeel|.\break

\noindent Figure~13.4
%\begin{figure}
%\begin{figure}[tb]
\marginfig{chptr13/9f4.eps}{The same Java application using the
Motif, Windows, and Metal look and feel.
} {fig-lafs}
%\end{figure}
%\end{figure}
shows how the simple application would appear under the
three different look-and-feel styles.

%\end{objectorienteddesign}


%%
\secEXRHone{Self-Study Exercise}
\label{self-study-exercise}
\begin{SSTUDY}

\item  The MVC architecture is a model of object-oriented design.  But if a
{\tt JButton} is really composed of three separate parts, how can we
still call it a component? Isn't it really three things?

\end{SSTUDY}

\section{The Java Event Model}
\label{the-java-event-model}

\noindent As we saw in Chapter 4, whatever happens while the computer
is running is classified as an event.  Every keystroke and mouse
click, every time a disk is inserted into a disk drive, an event is
generated.  The handling of events are an important element of GUI
programming. Therefore, before we begin discussing how to design GUIs,
it will be useful to review the main concepts of Java's {\bf event
model}\index{event model}.

When a Java program is running, events generated by the hardware are
passed up through the operating system (and through the browser, for
applets) to the program.  Those events that belong to the program must
be handled by the program (refer to Fig.~4.18 in
Chapter~4). For example, if you click your browser's menu bar, that
event will be handled by the browser itself.  If you click a button
contained in the Java program, that event should be handled by the
program.

In Java, whenever something happens within a GUI component, an event
object is generated and passed to the {\it event listener} that has
been registered to handle that component's events.  You've seen
numerous examples of this process in earlier chapters, but we've
included a simple example to serve as a reminder.

Suppose you create a {\tt JButton} in a GUI as follows:

\begin{jjjlisting}
\begin{lstlisting}
private JButton clickme = new JButton("ClickMe");
\end{lstlisting}
\end{jjjlisting}

\noindent Whenever the user clicks the {\tt JButton}, an {\tt
ActionEvent} is generated.  In order to handle these events, the GUI
must register the {\tt JButton} with a listener object that listens
for action events.  This can be done in an applet's {\tt init()}
method or in an application's constructor method, as in this example:

\begin{jjjlisting}
\begin{lstlisting}
public MyGUI() {
  // Add clickme to the GUI and assign it a listener
  add(clickme);  
  clickme.addActionListener(this); 
}
\end{lstlisting}
\end{jjjlisting}

\noindent In this case, we have designated the GUI itself ({\tt this})
as an {\tt ActionListener} for {\tt clickme} (Fig.~13.5).
%\begin{figure}
%\begin{graphic}
\marginfig{chptr13/p478f1.eps}{The GUI listens for action events on the {\tt JButton}.}
{fig-p478f1}
%\end{graphic}
%\end{figure}
A {\bf listener}\index{listener} is any object that implements a {\it listener
interface}, which is one of the interfaces derived from {\tt
java.util.Event\-Listener}.  An {\tt ActionListener} is an object that
listens for and receives {\tt ActionEvent}s.

\pagebreak
In order to complete the event-handling code, the GUI must implement
the {\tt ActionListener} interface.  As Figure~\ref{fig-clickme}
shows, implementing an interface is a matter of declaring the
interface in the class heading and implementing the methods contained
in the interface, in this case the {\tt actionPerformed()} method.

\begin{figure}[h]
\jjjprogstart
\begin{jjjlisting}
\begin{lstlisting}
import javax.swing.*;
import java.awt.event.*;

public class MyGUI extends JFrame 
                              implements ActionListener {
    private JButton clickme = new JButton("ClickMe");

    public MyGUI() {
     // Add clickme to the GUI and assign it a listener
        getContentPane().add(clickme);   
        clickme.addActionListener(this);
        setSize(200,200);
        setVisible(true);
    } // init()
    public void actionPerformed(ActionEvent e) {
        if (e.getSource() == clickme) {
	    clickme.setText(clickme.getText()+"*");
        }
    } // actionPerformed()
    public static void main(String args[]) {
        MyGUI gui = new MyGUI();
    }
} // MyGUI
\end{lstlisting}
\end{jjjlisting}
\jjjprogstop{A simple GUI application that handles action events
on a {\tt JButton}.}
{fig-clickme}
\end{figure}

%\begin{figure}
\begin{figure}[tb]
\figaleft{chptr13/p479f1.eps}{A UML depiction of the sequence of actions
and events that take place when a button is clicked. The vertical
lines represent time lines, with time running from top to bottom. The
arrows between lines represent messages passing between objects.
} {fig-p479f1}

\end{figure}
%\end{figure}

Now that we have implemented the code in Figure~\ref{fig-clickme},
whenever the user clicks {\tt clickme}, that action is encapsulated
within an {\tt ActionEvent} object and passed to the {\tt
actionPerformed()} method.  This method contains Java code that will
handle the user's action in an appropriate way.  For this example, it
modifies the button's label by appending an asterisk to it each time
it is clicked.  Figure~\ref{fig-p479f1} depicts the
sequence of actions and events that occur when the the user 
clicks a button.

\pagebreak
The methods used to handle the {\tt ActionEvent} are derived from the
\marginfig{chptr13/p479f2.eps}{An {\tt EventObject}. The {\tt getSource()} method is used to
get the object that caused the event.}
{fig-p479f2}
{\tt java.util.EventObject} class, the root class for all events
(Fig.~13.8).  Our example (Fig.~\ref{fig-clickme}) uses
the {\tt getSource()} method to get a reference to the object that
generated the event.  To see what information is contained in an
event object, we can use the {\tt toString()} method to print
a string representation of the event that was generated.  Here's what
it displays:

\begin{jjjlisting}
\begin{lstlisting}
java.awt.event.ActionEvent[ACTION_PERFORMED,cmd=ClickMe]
  on javax.swing.JButton[,58,5,83x27,
  layout=javax.swing.OverlayLayout]
\end{lstlisting}
\end{jjjlisting}

\noindent As you can see, the event generated was an {\tt ACTION\_PERFORMED}
event, in response to the {\tt ClickMe} command.  The source of the event
was the {\tt JButton}.

\subsection{Event Classes}

\noindent Although the event model is the same for both AWT and Swing
classes, the Swing package introduces many additional events.
Table~13.1 lists the events that are generated by both AWT and Swing
components.  You already have worked with some of these. We have
written GUIs that handled {\tt ActionEvent}s for {\tt JButton}s and
{\tt JTextField}s in preceding chapters.

\begin{table}[hbt]
\TBTvar{0pc}{Java's {\tt AWTEvent}s\index{AWTEvent} for each {\tt Component} type
%%%\TBTvar{-10pc}{Java's {\tt AWTEvent}s\index{AWTEvent} for each {\tt Component} type
(Original source: David Flanagan, \\
%%%\hspace*{-5pc}{\it Java in a Nutshell}, 2d ed., O'Reilly Associates, 1997. Modified for Swing components.)}
\hspace*{3pc}{\it Java in a Nutshell}, 2d ed., O'Reilly Associates, 1997. Modified for Swing components.)}
\hspace*{0pc}\begin{tabular}{lll}
%%%\hspace*{-10pc}\begin{tabular}{lll}
\multicolumn{3}{l}{\color{cyan}\rule{40.5pc}{1pt}}\\[2pt]
%%%\TBCH{Components}&\TBCH{Events}&\TBCH{Description}
{Components}&{Events}&{Description}
\\[-4pt]\multicolumn{3}{l}{\color{cyan}\rule{40.5pc}{0.5pt}}\\[2pt]
{\tt Button, JButton}                      &  {\tt ActionEvent}    & User clicked button\cr
{\tt CheckBox, JCheckBox}                  &  {\tt ItemEvent}      & User toggled a checkbox\cr
{\tt CheckboxMenuItem, JCheckboxMenuItem}  &  {\tt ItemEvent}      & User toggled a checkbox\cr
{\tt Choice, JPopupMenu}                   &  {\tt ItemEvent}      & User selected a choice\cr
{\tt Component, JComponent}                &  {\tt ComponentEvent} & Component was moved or resized\cr
                                           &  {\tt FocusEvent}     & Component acquired or lost focus\cr
                                           &  {\tt KeyEvent}       & User typed a key\cr
                                           &  {\tt MouseEvent}     & User manipulated the mouse\cr
{\tt Container, JContainer}                &  {\tt ContainerEvent} & Component added/removed from container\cr
{\tt List, JList}                          &  {\tt ActionEvent}    & User double-clicked a list item\cr
                                           &  {\tt ItemEvent}      & User clicked a list item\cr
{\tt Menu, JMenu}                          &  {\tt ActionEvent}    & User selected menu item\cr
{\tt Scrollbar, JScrollbar}                &  {\tt AdjustmentEvent}& User moved scrollbar\cr
{\tt TextComponent, JTextComponent}        &  {\tt TextEvent}      & User edited text\cr
{\tt TextField, JTextField}                &  {\tt ActionEvent}    & User typed Enter key\cr
{\tt Window, JWindow}                      &  {\tt WindowEvent}    & User manipulated window
\\[-4pt]\multicolumn{3}{l}{\color{cyan}\rule{40.5pc}{1pt}}
\end{tabular}
\endTB
\end{table}

In viewing Table~13.1, it's important to remember that the
classes listed there are arranged in a hierarchy.  This will affect the
events that a particular object can generate.  For example, a
{\tt JButton} is a {\tt JComponent} (Fig.~\ref{fig-swing2-guis}), so in addition
to generating {\tt ActionEvent}s when the user clicks on it, it can
also generate {\tt MouseEvent}s when the user moves the mouse over
it.   Similarly, because a {\tt JTextField} is also a {\tt JComponent},
it can generate {\tt KeyEvent}s as well as {\tt ActionEvent}s.

Note that the more generic events, such as those that involve moving,
focusing, or resizing a component, are associated with the more
generic components.  For example, the {\tt JComponent} class contains
methods that are used to manage {\tt ComponentEvent}s.  Because they
are subclasses of {\tt JComponent}, {\tt JButton}s and {\tt
JTextField}s can also use these methods.  Defining the more generic
methods in the {\tt JComponent} superclass is another example of the
effective use of inheritance.


\JavaTIP{EFFECTIVE DESIGN}{Inheritance.}{The higher a method is defined
in the inheritance hierarchy, the broader is its use.}

Table~13.2 lists events that are new with the Swing
classes.  Some of the events apply to new components.  For example,
{\tt JTable} and {\tt JTree} do not have AWT counterparts.  Other events
provide Swing components with capabilities that are not available in
their AWT counterparts.   For example, a {\tt CaretEvent} allows the
programmer to have control over mouse clicks that occur within a
text component.

\begin{table}[h!]
%\hphantom{\caption{Some of the events that are newly defined in the Swing library.}}
%%%\TBT{0pc}{Some of the events that are defined in the Swing library.}
\TBT{-5.5pc}{Some of the events that are defined in the Swing library.}
\hspace*{-5.5pc}
%%%\hspace*{0pc}
\begin{tabular}{lll}
\multicolumn{3}{l}{\color{cyan}\rule{32pc}{1pt}}\\[2pt]
%%\TBCH{Component}&\TBCH{Events}&\TBCH{Description}
{Component}&{Events}&{Description}
\\[-4pt]\multicolumn{3}{l}{\color{cyan}\rule{32pc}{0.5pt}}\\[2pt]
{\tt JPopupMenu}&{\tt PopupMenuEvent}&User selected a choice\cr
{\tt JComponent}&{\tt AncestorEvent}&An event occurred in an ancestor\cr
{\tt JList}&{\tt ListSelectionEvent}&User double-clicked a list item\cr
&{\tt ListDataEvent}&List's contents were changed\cr
{\tt JMenu}&{\tt MenuEvent}&User selected menu item\cr
{\tt JTextComponent}&{\tt CaretEvent}&Mouse clicked in text\cr
&{\tt UndoableEditEvent}&An undoable edit has occurred\cr
{\tt JTable}&{\tt TableModelEvent}&Items added/removed from table\cr
&{\tt TableColumnModelEvent}&A table column was moved\cr
{\tt JTree}&{\tt TreeModelEvent}&Items added/removed from tree\cr
&{\tt TreeSelectionEvent}&User selected a tree node\cr
&{\tt TreeExpansionEvent}&User expanded or collapsed a tree node\cr
{\tt JWindow}&{\tt WindowEvent}&User manipulated window
\\[-4pt]\multicolumn{3}{l}{\color{cyan}\rule{32pc}{1pt}}
\end{tabular}
\endTB
\end{table}


Tables~13.1 and 13.2 provide only a brief summary of these classes and
Swing components. For further details you should consult the JDK
online documentation at

\WWW
\begin{jjjlisting}
\begin{lstlisting}[commentstyle=\color{black}]
http://java.sun.com/j2se/1.5.0/docs/api/
\end{lstlisting}
\end{jjjlisting}

%%\pagebreak
\secEXRHone{Self-Study Exercises}
\label{self-study-exercises}
\begin{SSTUDY}

\item  Is it possible to register a component with more than one
listener?

\item  Is it possible for a component to have two different kinds
of listeners?

\end{SSTUDY}


\section{CASE STUDY: Designing a Basic GUI}
\label{casestudy-designinga-basic-gui}
%\begin{casestudy}
\noindent What elements make up a basic user interface?  If you think about all
of the various interfaces you've encountered---and don't just limit
yourself to computers---they all have the following elements:

\begin{BL}
\item  Some way to provide help/guidance to the user.
\item  Some way to allow input of information.
\item  Some way to allow output of information.
\item  Some way to control the interaction between the user and the device.
\end{BL}

\noindent Think about the interface on a beverage machine.
Printed text on the machine will tell you what choices you have, where
to put your money, and what to do if something goes wrong.   The coin
slot is used to input money.  There's often some kind of display to
tell you how much money you've inserted.  And there's usually a bunch
of buttons and levers that let you control the interaction with the
machine.

These same kinds of elements make up the basic computer interface.
Designing a Graphical User Interface is primarily a process of
choosing components that can effectively perform the tasks of input,
output, control, and guidance.

\JavaTIP{EFFECTIVE DESIGN}{User Interface.}{A user interface must
effectively perform the tasks of input, output, control, and guidance.}

In the programs we designed in the earlier chapters, we used two
different kinds of interfaces.  In the {\it command-line} interface, we
used printed prompts to inform the user, typed commands for data
entry and user control, and printed output to report results.  Our
GUI interfaces used {\tt JLabel}s to guide and prompt the user,
{\tt JTextField}s and {\tt JTextArea}s as basic input and output devices, and
either {\tt JButton}s or {\tt JTextField}s for user control.

Let's begin by building a basic GUI in the form of a Java application.
To keep the example as close as possible to the GUIs we've already
used, we will build it out of the following Swing components: {\tt
JLabel}, {\tt JTextField}, {\tt JTextArea}, and {\tt JButton}.

\subsection{The Metric Converter Application}
\noindent Suppose the coach of the cross-country team asks you to write a Java
application that can be used to convert miles to kilometers.  The
program should let the user input a distance in miles, and the program should
report the equivalent distance in kilometers.
\marginfig{chptr13/p483f1.eps}{The {\tt MetricConverter} class has a single method to convert
miles to kilometers.}
{fig-p483f1}


Before we design the interface for this, let's first define a {\tt
Metric\-Converter} class that can be used to perform the
conversions (Fig.~13.9).  For now at least, this class's
only task will be to convert miles to kilometers, for which it will
use the formula that 1 kilometer equals 0.62 miles:

\begin{jjjlisting}
\begin{lstlisting}
public class MetricConverter {
    public static double milesToKm(double miles) {
        return miles / 0.62;
    }
}
\end{lstlisting}
\end{jjjlisting}

\noindent Note that the method takes a {\tt double} as input
and returns a {\tt double}. Also, by declaring the method {\tt static},
we make it a class method, so it can be invoked simply by

\begin{jjjlisting}
\begin{lstlisting}
MetricConverter.milesToKm(10);
\end{lstlisting}
\end{jjjlisting}

\subsubsection*{Choosing the Components}
\noindent Let's now design a GUI to handle the interaction with the user.
First, let's choose Swing components for each of the four interface
%\begin{marginalnote}\it Which components do we need?\end{marginalnote}
\marginnote{Which components do we need?}
tasks of input, output, control, and guidance.  For each component, it
might be useful to refer back to Figure~\ref{fig-swing2-guis} to note its
location in the Swing hierarchy.

\begin{BL}

\item  A {\tt JLabel} is a display area for a short string of text,  an
image, or both.  Its AWT counterpart, the {\tt Label}, cannot display
images.  A \mbox{\tt JLabel} does not react to input.  Therefore, it is used
primarily to display a graphic or small amounts of static text.  It is
perfectly suited to serve as a prompt, which is what we will use it
for in this interface.

\item  A {\tt JTextField} is a component that allows the user to edit
a single line of text.  It is identical to its AWT counterpart,
the {\tt TextField}. By using its {\tt getText()} and {\tt setText()}
methods, a {\tt JTextField} can be used for either input or output, or
both.  For this problem, we'll use it to perform the interface's input
task.

\item A {\tt JTextArea} is a multiline text area that can be used for
either input or output.  It is almost identical to the AWT {\tt
TextArea} component.  One difference, however, is that a {\tt
JTextArea} does not contain scrollbars by default.  For this program,
we'll use the {\tt JTextArea} for displaying the results of
conversions.  Because it is used solely for output in this program,
we'll make it {\it uneditable} to prevent the user from typing in it.

\item  Let's use a {\tt JButton} as our main
control for this interface.  By implementing the {\tt ActionListener}
interface we will handle the user's action events.

\end{BL}

\subsubsection*{Choosing the Top-Level Window}
\noindent The next issue we must decide is what kind of top-level
%\begin{marginalnote}\it What top-level window should we use?\end{marginalnote}
\marginnote{What top-level window to use?}
window to use for this interface.  For applet interfaces, the top-level
component would be a {\tt JApplet}.  For Java applications, you would
typically use a {\tt JFrame} as the top-level window.  Both of these
classes are subclasses of {\tt Container}, so they are suitable for
holding the components that make up the interface
(Fig.~\ref{fig-swing1-guis}).


Also, as noted earlier, {\tt JApplet}s and {\tt JFrame}s are both
examples of heavyweight components, so they both have windows
associated with them.  To display a {\tt JFrame} we just have to give
it a size and make it visible.  Because a frame runs as a stand-alone
window, not within a browser context, it should also be able to exit
the application when the user closes the frame.

\subsubsection*{Designing a Layout}
\noindent The next step in designing the interface is deciding how to arrange the
%\begin{marginalnote}\it How should the components be arranged?\end{marginalnote}
\marginnote{\raggedright How should the components be arranged?}
components so that they will be visually appealing and comprehensible,
as well as easy to use.
%\begin{figure}
\begin{figure}[b]
\figaleft{chptr13/9f6.eps}{A design and layout for the Metric
Converter GUI.~The {\it containment hierarchy} (also called a {\it widget
hierarchy}) shows the containment relationships among the components.
} {fig-metricgui}

\end{figure}
%\end{figure}

Figure~\ref{fig-metricgui} shows a design for the layout. The largest
component is the output text area, which occupies the center of the
{\tt JFrame}. The prompt, input text field, and control button are
arranged in a row above the text area.  This is a simple and
straightforward layout.

Figure~\ref{fig-metricgui} also provides a {\bf containment
hierarchy}\index{containment hierarchy}, also called a {\bf widget
hierarchy}\index{widget hierarchy}, which shows the containment
relationships among the various components.  Although it might not
seem so for this simple layout, the containment hierarchy plays an
important role in showing how the various components are grouped
in the interface.  For this design, we have a relatively
simple hierarchy, with only one level of containment.  All of the
components are contained directly in the {\tt JFrame}.

\pagebreak
Figure~13.11 shows the design of the {\tt Converter} class, which
extends the {\tt JFrame} class and implements the {\tt ActionListener}
interface.  As a {\tt JFrame} subclass, a {\tt Converter} can contain
GUI components.  As an implementor of the {\tt ActionListener}
interface, it also will be able to handle action events through the {\tt
actionPerformed()} method.

\marginfig{chptr13/p484f1.eps}{The {\tt Converter} class is a subclass
of {\tt JFrame} and implements the {\tt ActionListener} interface.}
{fig-p484f1}


Figure~\ref{fig-converterclass} gives the implementation of the {\tt
Converter} class. Note the three packages that are imported.  The
first contains definitions of the Swing classes, and the other two
contain definitions of AWT events and layout managers that are used in
the program.

\begin{figure}[h!]
\jjjprogstart
\begin{jjjlisting}
\begin{lstlisting}
import javax.swing.*;             // Packages used
import java.awt.*;
import java.awt.event.*;

public class Converter extends JFrame 
                            implements ActionListener{
  private JLabel prompt = 
                  new JLabel("Distance in miles: ");
  private JTextField input = new JTextField(6);
  private JTextArea display = new JTextArea(10,20);
  private JButton convert = new JButton("Convert!");

  public Converter() {
    getContentPane().setLayout(new FlowLayout());
    getContentPane().add(prompt);
    getContentPane().add(input);
    getContentPane().add(convert);
    getContentPane().add(display);
    display.setLineWrap(true);
    display.setEditable(false);
    convert.addActionListener(this);
  } // Converter()

  public void actionPerformed( ActionEvent e ) {
    double miles = 
       Double.valueOf(input.getText()).doubleValue();
    double km = MetricConverter.milesToKm(miles);
    display.append(miles + " miles equals " + 
                     km + " kilometers\n");
  } // actionPerformed()
  public static void main(String args[]) {
    Converter f = new Converter();
    f.setSize(400, 300);
    f.setVisible(true);
    f.addWindowListener(new WindowAdapter() {      
      public void windowClosing(WindowEvent e) {
        System.exit(0); // Quit the application
      }
    });
  } // main()
} // Converter
\end{lstlisting}
\end{jjjlisting}
\jjjprogstop{The {\tt Converter} class implements a simple GUI interface.}
{fig-converterclass}
\end{figure}

We have to do all initializing tasks in the constructor.  First,
we have to set the {\tt JFrame}'s layout to {\tt FlowLayout}.  A {\bf
layout manager}\index{layout manager} is the object that is
responsible for sizing and arranging the components in a container so
that the elements are organized in the best possible manner.  A flow
layout is the simplest arrangement: The components are arranged left
to right in the window, wrapping around to the next ``row'' if
necessary.


Second, note the statements used to set the layout and to add
components directly to the {\tt JFrame}. Instead of adding components
directly to the {\tt JFrame}, we must add them to its content pane:

\begin{jjjlisting}
\begin{lstlisting}
getContentPane().add(input);
\end{lstlisting}
\end{jjjlisting}

\noindent A {\bf content pane}\index{content pane} is a {\tt JPanel} that serves as the
working area of the {\tt JFrame}. It contains all of the frame's
components.  Java will raise an exception if you attempt to add a
component directly to a {\tt JFrame}.

\JavaTIP{DEBUGGING TIP}{Content Pane}{A {\tt JFrame} cannot directly
contain GUI elements.  Instead, they must be added to its content
pane, which can be retrieved using the {\tt getContentPane()} method.}

\noindent The {\tt JFrame} and all the
other top-level Swing windows have an internal structure made up of
several distinct objects that can be manipulated by the
program.  Because of this structure, GUI elements can be organized into
different layers within the window to create many types of
sophisticated layouts.  Also, one layer of the structure makes it
possible to associate a menu with the frame.   

Finally, note how the {\tt Converter} frame is instantiated, made
visible, and eventually exited in the application's {\tt main()}
method:

\begin{jjjlisting}
\begin{lstlisting}
public static void main(String args[]) {
   Converter f = new Converter();
   f.setSize(400, 300);
   f.setVisible(true);
   f.addWindowListener(new WindowAdapter() {  
                         // Quit the application
       public void windowClosing(WindowEvent e) {
           System.exit(0);
       }
   });
} // main()
\end{lstlisting}
\end{jjjlisting}

\noindent It is necessary to set both the size and visibility of the
frame, since these are not set by default.  Because we are using a
{\tt FlowLayout}, it is especially important to give the frame an
appropriate size.   Failure to do so can cause the components to be
arranged in a confusing way and might even cause some components to not
appear in the window.  These are limitations we will fix when
we learn how to use some of the other layout managers.

\subsection{Inner Classes and Adapter Classes}

\noindent In this section we introduce two new language features, {\em
inner classes} and {\em adapter classes}, which are used in the {\tt
main()} method shown above to handle the closing of the
{\tt Converter} application's window when the program is exited: 

\begin{jjjlisting}
\begin{lstlisting}
 f.addWindowListener(new WindowAdapter() { 
       public void windowClosing(WindowEvent e) {
           System.exit(0);
       }
 });
\end{lstlisting}
\end{jjjlisting}

\noindent This code segment provides a listener that listens for
\marginnote{Inner classes}
window closing events.  When such an event occurs, it exits the
application by calling {\tt System.exit()}.

The syntax used here is an example of an {\it anonymous inner class}.
An {\bf inner class} is a class defined within another class.  The
syntax is somewhat ugly, because it places the class definition right
where a reference to a window listener object would go.  In effect
what the code is doing is defining a subclass of {\tt WindowAdapter}
and creating an instance of it to serve as a listener for window
closing events.

Anonymous inner classes provide a useful way of creating classes and
objects on the fly to handle just this kind of listener task.  The
syntax used actually enables us to write one expression that both
defines a class and creates an instance of it to listen for window
closing events.   The new subclass has local scope limited here to the
{\tt main()} method.  It is anonymous, meaning we aren't even
giving it a name, so you can't create other instances of it in the
program.   Note that the body of the class definition is placed right
after the {\tt new} keyword, which takes the place of the argument to
the {\tt addWindowListener()} method.   For more details on the inner
and anonymous classes, see Appendix~F.

\BOXtwo{Inner Class.}{An inner class is
a class defined within another class.  Inner classes are
mostly used to handle a task that supports the work of
the containing class.}

An {\bf adapter class} is a wrapper class that implements trivial
%\begin{marginalnote}\it Adapter class\end{marginalnote}
\marginnote{Adapter class}
versions of the abstract methods that make up a particular
interface. (Remember from Chapter~4 that a wrapper class contains
methods for converting primitive data into objects and for
converting data from one type to another.)

The {\tt Window\-Adapter} class implements the methods of the {\tt
Window\-Listener} interface.  When you implement an interface, such as
{\tt ActionListener}, you must implement all the abstract methods
defined in the interface.  For {\tt ActionListener} there's just one
method, the {\tt actionPerformed()} method, so we can implement it as
part of our applet or frame class.  However, we want to use only one
of the seven methods available in the {\tt WindowListener} interface,
the {\tt windowClosing()} method, which is the method implemented in
the anonymous inner class:

\begin{jjjlisting}
\begin{lstlisting}
public void windowClosing(WindowEvent e) {
    System.exit(0);
}
\end{lstlisting}
\end{jjjlisting}

\noindent The {\tt WindowAdapter} is defined simply as

\begin{jjjlisting}
\begin{lstlisting}
public abstract class WindowAdapter 
                          implements WindowListener {
    public void windowActivated(WindowEvent e) {}
    public void windowClosed(WindowEvent e) {}
    ...
    // Five other window listener methods
}
\end{lstlisting}
\end{jjjlisting}

\noindent Note that each method is given a trivial implementation ({}).
To create a subclass of {\tt WindowAdapter}, you must override at
least one of its trivially implemented methods.

Another way to manage the application's window closing event
is to define a subclass of {\tt WindowAdapter}:

\begin{jjjlisting}
\begin{lstlisting}
import javax.swing.*;
import java.awt.*;
import java.awt.event.*;
public class WindowCloser extends WindowAdapter {
    public void windowClosing(WindowEvent e) {
        System.exit(0);
    }
}
\end{lstlisting}
\end{jjjlisting}

\noindent Given this class, we can then place the following statement
in {\tt Converter}'s {\tt main()} method:

\begin{jjjlisting}
\begin{lstlisting}
f.addWindowListener(new WindowCloser());
\end{lstlisting}
\end{jjjlisting}

\noindent This is somewhat more familiar looking than the inner class
construct. If you prefer this way of handling things, you can use this
method in place of the inner classes here and in other examples.

\JavaTIP{EFFECTIVE DESIGN}{Anonymous Adapter Classes.}{Anonymous adapter
classes provide a useful way of creating an object to handle one
particular kind of event within a program.}

\pagebreak
\subsection{GUI Design Critique}
\noindent Figure~13.13 shows the converter interface.
\marginfig{chptr13/9f8.eps}{The first version of the metric converter GUI.}
{fig-convertscreen}
Although our basic GUI design satisfies the demands of input, output,
control, and guidance, it has a few significant design flaws.

First, it forces the user to manually clear the input field after each
conversion.  Unless it is important that the user's input value remain
displayed until another value is entered, this is just an
inconvenience to the user.  In this case, the user's input value is
displayed along with the result in the {\tt JTextArea}, so there's no
reason not to clear the input text field:

\begin{jjjlisting}
\begin{lstlisting}
input.setText("");  // Clear the input field
\end{lstlisting}
\end{jjjlisting}

\JavaTIP{EFFECTIVE DESIGN}{Reduce the User's Burden.}{A GUI should aim to
minimize the responsibility placed on the user.  In general, the
program should perform any task that it can perform, unless, of course, there is
a compelling reason that the user should do the task.}

A second problem with our design is that it forces the user to switch
between the keyboard (for input) and the mouse (for
control). Experienced users will find this annoying.   An easy way to
fix this problem is to make both the {\tt JTextField} and the
{\tt JButton} serve as controls.  That way, to get the program to do the
conversion, the user can just press the Enter key after typing a
number into the text field.


To give the interface this type of control, we only need to add an
{\tt ActionLis\-tener} to the {\tt JTextField} during the initialization
step:

\begin{jjjlisting}
\begin{lstlisting}
input.addActionListener(this);
\end{lstlisting}
\end{jjjlisting}

\noindent A {\tt JTextField} generates an {\tt
ActionEvent} whenever the Enter key is pressed.  We don't even need to
modify the {\tt actionPerformed()} method, since both controls will
generate the same action event.  This will allow users who prefer the
keyboard to use just the keyboard.

\JavaTIP{EFFECTIVE DESIGN}{User Interface.}{A GUI should aim to minimize the
number of different input devices (mouse, keyboard) that the user has to
manipulate to perform a particular task.}

\noindent Given that the user can now interact with the
interface with just the keyboard, a question arises over whether we
should keep the button at all.  In this case, it seems justifiable to
keep both the button and the text field controls.  Some users dislike
typing and prefer to use the mouse.  Also, having two independent sets
of controls is a desirable form of redundancy.  You see it frequently
in menu-based systems that allow menu items to be selected either by
mouse or by special control keys.

\JavaTIP[false]{EFFECTIVE DESIGN}{Desirable Redundancy.}{Certain forms of
redundancy in an interface, such as two sets of independent controls
(mouse and keyboard), make it a more flexible or more widely usable program.}

\secEXRHone{Self-Study Exercises}
\label{self-study-exercises}
\begin{SSTUDY}

\item  Another deficiency in the converter interface is that it doesn't round
off its result, leading sometimes to numbers with 20 or so
digits.  Develop Java code to fix this problem.


\item  Give an example of desirable redundancy in automobile design.

\end{SSTUDY}

\subsection{Extending the Basic GUI: Button Array}
\label{sec-keypad}
\noindent Suppose the coach likes our program but complains that some of the
folks in the office are terrible typists and would prefer not to have
to use the keyboard at all.  Is there some way we could modify the
interface to accommodate these users?


This gets back to the point we were just making about incorporating
\marginnote{What components do we need?}  
redundancy into the interface.  One way to satisfy this requirement
would be to implement a numeric keypad for input, similar to a
calculator keypad.  Regular {\tt JButton}s can be used as the keypad's
keys.  As a user clicks keypad buttons, their face values---0 through
9---are inserted into the text field.  The keypad will also need a
button to clear the text field and one to serve as a decimal point.

This new feature will add 12 new {\tt JButton} components to our
\marginnote{\raggedright How should the components be organized?}
interface.  Instead of inserting them into the {\tt JFrame}
individually, it will be better to organize them into a separate
panel and to insert the entire panel into the frame as a single
unit.  This will help reduce the complexity of the display, especially
if the keypad buttons can be grouped together visually.   Instead of
having to deal with 16 separate components, the user will see the
keypad as a single unit with a unified function.  This is an example of
the abstraction principle, similar to the way we break long
strings of numbers (1-888-889-1999) into subgroups to make them easier
to remember.

\JavaTIP{EFFECTIVE DESIGN}{Reducing Complexity.}{Organizing elements
into distinct groups by function helps to reduce the GUI's complexity.}

\noindent Figure~\ref{fig-metricgui2} shows the revised  
converter interface design.  The containment hierarchy shows that the 12
keypad {\tt JButton}s are contained within a {\tt JPanel}. In the
frame's layout, the entire panel is inserted just after the text area.


%\begin{figure}
\begin{figure}[tb]
\figaright{chptr13/9f9.eps}{A widget hierarchy showing the
containment relationships among the components.
} {fig-metricgui2}

\end{figure}
%\end{figure}

Incorporating the keypad into the interface requires several changes
%\begin{marginalnote}\it The {\tt JPanel}\end{marginalnote}
%\marginnote{The {\tt JPanel}}
in the program's design. Because the keypad has such a clearly defined
role, let's make it into a separate object by defining a {\tt KeyPad}
class (Fig.~13.15). The {\tt KeyPad} will be a subclass of {\tt
JPanel} and will handle its own {\tt ActionEvent}s.  As we saw in
Chapter~4, a {\tt JPanel} is a generic container.  It is a subclass of
{\tt Container} via the {\tt JComponent} class
(Fig.~\ref{fig-swing2-guis}).  Its main purpose is to contain and organize
\marginfig{chptr13/p490f1.eps}{A {\tt KeyPad} is a {\tt JPanel} of {\tt JButton}s that handles its
own action events.}
{fig-p490f1}
components that appear together on an interface.

In this case, we will use a {\tt JPanel} to hold the keypad buttons.
As you might recall from Chapter~4, to add elements to a {\tt JPanel},
you use the {\tt add()} method, which is inherited from {\tt
Container}.  (A {\tt JApplet} is also a subclass of {\tt
Container} via the {\tt Panel} class.)
%\begin{figure}
%\begin{graphic}
%\end{graphic}
%\end{figure}

As a subclass of {\tt JPanel}, the {\tt KeyPad} will take care of
%\marginnote{\em What data do we need?}
holding and organizing the {\tt JButton}s in the visual
display. We also need some way to organize and manage the 12 keypad
buttons within the program's memory. Clearly, this is a good job for an
array. Actually, two arrays would be even better, one for the buttons
and one for their labels:

\begin{jjjlisting}
\begin{lstlisting}
private JButton buttons[];
private String labels[] = // An array of button labels
            { "1","2","3",
              "4","5","6",
              "7","8","9",
              "C","0","." };
\end{lstlisting}
\end{jjjlisting}

\noindent The {\tt label} array stores the strings that we will
use as the buttons' labels.  The main advantage of the array is that
we can use a loop to instantiate the buttons:

\begin{jjjlisting}
\begin{lstlisting}
buttons = new JButton[NBUTTONS]; // Create the array
                          // For each labeled button
for(int k = 0; k < buttons.length; k++) { 
 buttons[k] = new JButton(labels[k]); // Create button
 buttons[k].addActionListener(this); // and a listener
 add(buttons[k]);           // and add it to the panel
} // for
\end{lstlisting}
\end{jjjlisting}

\noindent This code should be placed in the {\tt KeyPad()}
\marginnote{Algorithm design}
constructor. It begins by instantiating the array itself. It then uses
a for loop, bounded by the size of the array, to instantiate each
individual button and insert it into the array. Note how the loop
variable here, {\em k}, plays a dual role. It serves as the index into
both the button array ({\tt buttons}) and the array of strings that
serves as the buttons' labels ({\tt labels}). In that way the labels
are assigned to the appropriate buttons. Note also how each button is
assigned an {\tt ActionListener} and added to the panel:

\begin{jjjlisting}
\begin{lstlisting}
buttons[k].addActionListener(this); // Add listener
add(buttons[k]);            //  Add button to panel
\end{lstlisting}
\end{jjjlisting}

\noindent An important design issue for our {\tt KeyPad} object concerns how it will
interact with the {\tt Converter} that contains it.  When the user
clicks a keypad button, the key's label has to be displayed in the
{\tt Converter}'s text area. But because the text area is private to the
converter, the {\tt KeyPad} does not have direct access to it. To
address this problem, we will use a Java interface to implement a 
\marginnote{Callback design}
{\bf callback design}\index{callback design}. In this design, whenever a {\tt KeyPad}
button is pressed, the {\tt KeyPad} object calls a method in the
{\tt Converter} that displays the key's label in the text area.


Figure~\ref{fig-p493f1} provides a summary of the callback design.
Note that the association between the {\tt Converter} and the {\tt
KeyPad} is bi-directional. This means that each object has a reference
to the other and can invoke the other's public methods. This will be
effected by having the {\tt Converter} pass a reference to itself when
it constructs the {\tt KeyPad}:

\begin{figure}[b!]
\figaleft{chptr13/p493f1.eps}{In a callback design, the {\tt Converter} implements
the {\tt KeyPadClient} interface. It passes a reference to
itself when it creates the {\tt KeyPad} object. The {\tt KeyPad}
object can then invoke the {\tt keypressCallback()} method whenever
a keypad button is pressed, and the {\tt Converter} can display the
result of the keypress.
} {fig-p493f1}

\end{figure}

\begin{jjjlisting}
\begin{lstlisting}
private KeyPad keypad = new KeyPad(this);
\end{lstlisting}
\end{jjjlisting}

\noindent Another important design issue is that the {\tt KeyPad} needs 
to know the name of the callback method and the {\tt Converter} needs
to have an implementation of that method. This is a perfect job for an
abstract interface:

\begin{jjjlisting}
\begin{lstlisting}
public abstract interface KeyPadClient {
    public void keypressCallback(String s);
}
\end{lstlisting}
\end{jjjlisting}

\noindent The {\tt KeyPad} can interact with any class that implements 
the {\tt Key\-Pad\-Client} interface.  Note that the {\tt KeyPad} has a
reference to the \break{\tt KeyPadClient}, which it will use to invoke the
{\tt keypressCallback()} method.

The implementation of {\tt KeyPad} is shown in Figure~13.17. 
Note that its constructor takes a reference to a {\tt KeyPadClient}
and saves it in an instance variable. Its {\tt actionPerformed()}
method then passes the key's label to the {\tt KeyPadClient}'s 
callback method.

%% proglist ch9/converter2/Converter.java
\begin{figure}[hbt]
\jjjprogstart
\begin{jjjlisting}[29pc]
\begin{lstlisting}
import java.awt.*;
import java.awt.event.*;
import javax.swing.*;

public class KeyPad extends JPanel implements ActionListener{
 
  private final static int NBUTTONS = 12;  
  private KeyPadClient kpc;  // Owner of the KeyPad
  private JButton buttons[];
  private String labels[] =  // An array of button labels
                { "1","2","3",
                  "4","5","6",
                  "7","8","9",
                  "C","0","." };

  public KeyPad(KeyPadClient kpc) {
    this.kpc = kpc;
    buttons = new JButton[NBUTTONS];         // Create the array
    for(int k = 0; k < buttons.length; k++) { // For each button
      buttons[k] = new JButton(labels[k]);   //  Create a button
      buttons[k].addActionListener(this);     //  and a listener
      add(buttons[k]);                    // and add it to panel
    } // for
  } // KeyPad()

  public void actionPerformed(ActionEvent e) {
    String keylabel = ((JButton)e.getSource()).getText();
    kpc.keypressCallback(keylabel);
  } // actionPerformed()
} // KeyPad
\end{lstlisting}
\end{jjjlisting}
\jjjprogstop{The {\tt KeyPad} object implements a 12-key keypad
in a {\tt JPanel}. It has a reference to the {\tt KeyPadClient} that
contains the keypad.}
{fig-keypad}
\end{figure}

Given the {\tt KeyPad} design, we need to revise
our design of the {\tt Converter} class (Fig.~\ref{fig-p493f1}).
The {\tt Converter} will now implement the {\tt KeyPad\-Client}
interface, which means it must provide an implementation of the
{\tt keypressCall\-back()} method:

\begin{jjjlisting}
\begin{lstlisting}
public void keypressCallback(String s) {
    if (s.equals("C"))
        input.setText("");
    else
        input.setText(input.getText() + s);
}
\end{lstlisting}
\end{jjjlisting}

\noindent Recall that whenever the {\tt KeyPad} object calls the 
{\tt keypressCallback()} method, it passes the label of the button that
was pressed. The {\tt Converter} object simply appends the key's label
to the input text field, just as if the user typed the key in the
text field.

The complete implementation of this revised version of the interface
is shown in Figure~13.18 on the next page. The appearance of the
interface itself is shown in Figure~3.19.

%% proglist ch9/converter2/Converter.java
\begin{figure}[p]
\jjjprogstart
\begin{jjjlisting}[27pc]
\begin{lstlisting}
import javax.swing.*;      // Packages used 
import java.awt.*;
import java.awt.event.*;

public class Converter extends JFrame   // Version 2
                 implements ActionListener, KeyPadClient { 

  private JLabel prompt = new JLabel("Distance in miles: "); 
  private JTextField input = new JTextField(6);
  private JTextArea display = new JTextArea(10,20);
  private JButton convert = new JButton("Convert!");
  private KeyPad keypad = new KeyPad(this);
  
  public Converter () {
    getContentPane().setLayout(new FlowLayout()); 
    getContentPane().add(prompt);
    getContentPane().add(input);
    getContentPane().add(convert);  
    getContentPane().add(display);
    getContentPane().add(keypad);
    display.setLineWrap(true);
    display.setEditable(false);
    convert.addActionListener(this);   
    input.addActionListener(this);
  } // Converter()
  
  public void actionPerformed(ActionEvent e) {
    double miles = 
      Double.valueOf(input.getText()).doubleValue();
    double km = MetricConverter.milesToKm(miles);
    display.append(miles + " miles equals " + 
                   km + " kilometers\n");
    input.setText("");
  } // actionPerformed()

  public void keypressCallback(String s) {
    if (s.equals("C"))
        input.setText("");
     else
        input.setText(input.getText() + s);
  }

  public static void main(String args[]) {
    Converter f = new Converter();
    f.setSize(400, 300);  
    f.setVisible(true);
    f.addWindowListener(new WindowAdapter() {  
      public void windowClosing(WindowEvent e) {
        System.exit(0);  // Quit the application
      }
    });
  } // main()
} // Converter
\end{lstlisting}
\end{jjjlisting}
\jjjprogstop{The second version of the {\tt Converter} class, which
implements the GUI shown in Figure 13.19.}
{fig-convertclass2}
\end{figure}

\subsection{GUI Design Critique}
\noindent Figure~3.19 shows that despite our
\marginfig{chptr13/9f12.eps}{The second version of the metric converter
GUI uses a set of keypad buttons for input, but they are not properly
arranged.}
{fig-convertscreen2}
efforts to group the keypad into a rectangular array, it doesn't
appear as a single entity in the interface itself, which indicates a
layout problem.  The default layout for our {\tt KeyPad} (which is a
{\tt JPanel}) is {\tt FlowLayout}, which is not appropriate for a
numeric keypad that needs to be arranged into a two-dimensional grid
pattern, which is the kind of layout our design called for
(Fig.~\ref{fig-metricgui2}).

Fortunately, this flaw can easily be fixed by using an appropriate
layout manager from the AWT. In the next version of the
program, we employ the {\tt java.awt.GridLayout}, which is perfectly suited for
a two-dimensional keypad layout (Section~13.7.2).

The lesson to be learned from this example is that screen layout is an
important element of an effective GUI.~If not done well, it can
undermine the GUI's effort to guide the user toward the appointed
tasks.  If done poorly enough, it can even keep the user from doing the
task at all.

\JavaTIP{EFFECTIVE DESIGN}{Layout Design.}{The appropriate layout and
management of GUI elements is an important part of interface
design.  It contributes to the interface's ability to guide the user's
action toward the interface's goals.}


%%%\begin{figure}[tb]
%%%\figa{chptr13/9f12.eps}{The second version of the metric converter
%%%GUI uses a set of keypad buttons for input, but they are not properly
%%%arranged.
%%%\label{fig-convertscreen2}
%%%\label{pg-fig-convertscreen2}}
%%%\end{figure}

%%%\endcasestudy


\section{Containers and Layout Managers}
\label{containersand-layout-managers}
\noindent A {\tt Container} is a component that can contain other components.
Because containers can contain other containers, it is possible to
create a hierarchical arrangement of components, as we did in the
second version of our {\tt Converter} interface.  In its present form,
the hierarchy for {\tt Converter} consists of a {\tt JFrame} as the
top-level container (Fig.~\ref{fig-metricgui2}). Contained within
the frame is a {\tt KeyPad} (subclass of {\tt JPanel}), which contains 12 {\tt JButton}s.
Most GUIs will have a similar kind of containment hierarchy.


\marginfig{chptr13/p495f1.eps}{A {\tt Container} contains {\tt Component}s.}
{fig-p495f1}


A {\tt Container} is a relatively simple object whose main task is
primarily to hold its components in a particular order.  It has
methods to add and remove components (Fig.~13.20).  As you
can see from these methods, a container keeps track of the order of
its elements, and it is possible to refer to a component by its index
order.

\subsection{Layout Managers}
\noindent The hard work of organizing and managing the elements within
a container is the task of the layout manager.  Among other tasks, the
layout manager determines

\begin{BL}
\item  The overall size of the container.
\item  The size of each element in the container.
\item  The spacing between elements.
\item  The positioning of the elements.
\end{BL}

\noindent Although it is possible to manage your own layouts, it is
not easy to do.  For most applications you are much better off by
learning to use one of the AWT's built-in layouts.  Table~13.3 gives a
brief summary of the available layouts.  We will show examples of {\tt
FlowLayout}, {\tt GridLayout}, and {\tt BorderLayout}. Some of the
widely used Swing containers have a default layout manager assigned to
them (Table~13.4).

\begin{table}[b]
%\hphantom{\caption{Some of Java's AWT and Swing layout managers.}}
\TBT{-8pc}{Some of Java's AWT and Swing layout managers.}
\hspace*{-8.5pc}\begin{tabular}{ll}
\multicolumn{2}{l}{\color{cyan}\rule{35pc}{1pt}}\\[2pt]
%%%\TBCH{Manager}&\TBCH{Description}
{Manager}&{Description}
\\[-4pt]\multicolumn{2}{l}{\color{cyan}\rule{35pc}{0.5pt}}\\[2pt]
{\tt java.awt.BorderLayout} &  Arranges elements along the north, south, east, west,\cr
                            &   and in the center of the container.\cr
{\tt java.swing.BoxLayout}   &  Arranges elements in a single row or single column.\cr
{\tt java.awt.CardLayout}   &  Arranges elements like a stack of cards, with one visible at a time.\cr
{\tt java.awt.FlowLayout}   &  Arranges elements left to right across the container.\cr
{\tt java.awt.GridBagLayout}   & Arranges elements in a grid of variably sized cells (complicated).\cr
{\tt java.awt.GridLayout}   &  Arranges elements into a two-dimensional grid of equally sized cells.\cr
{\tt java.swing.OverlayLayout}  &  Arranges elements on top of each other.
\\[-4pt]\multicolumn{2}{l}{\color{cyan}\rule{35pc}{1pt}}
\end{tabular}
\endTB
\end{table}

To override  the default layout for any of the {\tt JApplet},
{\tt JDialog}, {\tt JFrame}, and {\tt JWindow} containers, you
must remember to use the \mbox{\tt getContentPane()}. The correct
statement is

\begin{jjjlisting}
\begin{lstlisting}
getContentPane().setLayout(new FlowLayout());
\end{lstlisting}
\end{jjjlisting}

\JavaTIP{DEBUGGING TIP}{Content Pane.}{Attempting to add a component
directly to a {\tt JApplet} or a {\tt JFrame} will cause an
exception.  For these top-level containers, components must be added to
their content panes.}

\begin{table}[tb]
%\hphantom{\caption{Default layouts for some of the common Swing containers.}}
\TBT{4pc}{Default layouts for some of the common \\\hspace*{4pc}Swing containers.}
\hspace*{3.5pc}\begin{tabular}{ll}
\multicolumn{2}{l}{\color{cyan}\rule{18pc}{1pt}}\\[2pt]
%%%\TBCH{Container}&\TBCH{Layout Manager}
{Container}&{Layout Manager}
\\[-4pt]\multicolumn{2}{l}{\color{cyan}\rule{18pc}{0.5pt}}\\[2pt]
{\tt JApplet}                 & {\tt BorderLayout} (on its content pane) \cr
{\tt JBox}                    & {\tt BoxLayout}                           \cr
{\tt JDialog}                 & {\tt BorderLayout} (on its content pane) \cr
{\tt JFrame}                  & {\tt BorderLayout} (on its content pane) \cr
{\tt JPanel}                  & {\tt FlowLayout}                         \cr
{\tt JWindow}                 & {\tt BorderLayout} (on its content pane)
\\[-4pt]\multicolumn{2}{l}{\color{cyan}\rule{18pc}{1pt}}
\end{tabular}
\endTB
\end{table}


\subsection{The {\tt GridLayout} Manager}
%<sec_gridlayout:\thisBheadno><pg_sec_gridlayout:\thispageno>
\noindent It is simple to remedy the layout problem that affected the keypad in
the most recent version of the {\tt Converter} program.  The problem
was caused by the fact that as a subclass of {\tt JPanel}, the {\tt
KeyPad} uses a default {\tt FlowLayout}, which causes its buttons to
be arranged in a row.  A more appropriate layout for a numeric keypad
would be a two-dimensional grid, which is exactly the kind of layout
supplied by the {\tt java.awt.GridLayout}.  Therefore, to fix this
problem, we need only set the keypad's layout to a {\tt
GridLayout}. This takes a single statement, which should be added to
the beginning of the {\tt KeyPad()} constructor:

\begin{jjjlisting}
\begin{lstlisting}
setLayout(new GridLayout(4,3,1,1));
\end{lstlisting}
\end{jjjlisting}

\noindent This statement creates a {\tt GridLayout} object and
assigns it as the layout manager for the keypad. It will ensure that
the keypad will have four rows and three columns of buttons
(Fig.~\ref{fig-convertscreen3}). The last two arguments in the
constructor affect the relative spacing between the rows and the
columns.  The higher the number, the larger the spacing.  As
components are added to the keypad, they will automatically be
arranged by the manager into a $4 \times 3$ grid.
\marginfig{chptr13/9f13.eps}{This version of the metric converter GUI
uses a keypad for mouse-based input.  It has an attractive overall layout.}
{fig-convertscreen3}


Note that for a {\tt JPanel}, the {\tt setLayout()} method applies to
the panel itself.  Unlike the top-level containers, such as
{\tt JFrame}, other containers don't have content panes.  The same point
would apply when adding components to a {\tt JPanel}: They are
added directly to the panel, not to a content pane.  Confusion
over this point could be the source of bugs in your programs.

\JavaTIP{DEBUGGING TIP}{Content Pane.}{Top-level containers, such as
{\tt JFrame}, are the only ones that use a content pane.  For other
containers, such as {\tt JPanel}, components are added directly to the
container itself.}

\noindent As its name suggests, the {\tt GridLayout} layout manager arranges
components in a two-dimensional grid.  When components are added to the
container, the layout manager starts inserting elements into the grid
at the first cell in the first row and continues left to right across
row 1, then row 2, and so on.  If there are not enough components to
fill all cells of the grid, the remaining cells are left blank.   If an
attempt is made to add too many components to the grid, the layout
manager will try to extend the grid in some reasonable way in order to
accommodate the components.  However, despite its effort in such cases,
it usually fails to achieve a completely appropriate layout.

\JavaTIP{PROGRAMMING TIP}{Grid Layouts.}{Make sure the number of
components added to a {\tt GridLayout} is equal to the number of rows
times the number of columns.}

\subsection{GUI Design Critique}
\noindent Although the layout in Figure~13.21 is much
improved, there are still some deficiencies.  One problem is
that the {\tt convert} button seems to be misplaced.   It would seem to
make more sense if it were grouped with the keypad rather
than with the input text field.

A more serious problem results from the fact that we are still using a
{\tt FlowLayout} for the program's main window, the
{\tt JFrame}. Among all of Java's layouts, {\tt FlowLayout} gives you the
least amount of control over the arrangement of the components.  Also,
{\tt FlowLayout} is most sensitive to changes in the size and shape of
its container.

\subsection{The {\tt BorderLayout} Manager}
\noindent One way to fix these problems is to use a {\tt BorderLayout} to divide
the frame into five areas: north, south, east, west, and center, as
shown in Figure~\ref{fig-borderlayout-guis}. The {\tt BorderLayout} class contains two constructors:
\marginfig{chptr13/9f14.eps}{Arrangement
of components in  a border
layout.  The relative size of the areas will vary.}
{fig-borderlayout-guis}

\begin{jjjlisting}
\begin{lstlisting}
public BorderLayout();
public BorderLayout(int hgap, int vgap);
\end{lstlisting}
\end{jjjlisting}

\noindent The two parameters in the second version of the constructor
allow you to insert spacing between the areas.

Components are added to a {\tt BorderLayout} by using the {\tt
add(Com\-po\-nent, String)} method found in the {\tt Container} class.
For example, to set the application window to a border layout and to
add the keypad to its east area, we would use the following
statements:

\begin{jjjlisting}
\begin{lstlisting}
getContentPane().setLayout(new BorderLayout(2, 2));
getContentPane().add(keypad,"East");
\end{lstlisting}
\end{jjjlisting}

\noindent In this version of the {\tt add()} method, the second parameter
must be a capitalized {\tt String} with one of the names, ``North,''
``South,'' ``East,'' ``West,'' or ``Center.''  The order in which components
are added does not matter.

One limitation of the {\tt BorderLayout} is that only one component
%\begin{marginalnote}\it Containment hierarchy\end{marginalnote}
\marginnote{Containment hierarchy}
can be added to each area.  That means that if you want to add
several components to an area, you must first enclose them within a
{\tt JPanel} and then add the entire panel to the area.   For example,
let's create a panel to contain the prompt and the text field and
place it at the north edge of the frame:

\begin{jjjlisting}
\begin{lstlisting}
JPanel inputPanel = new JPanel(); // Create panel
inputPanel.add(prompt);              // Add label
inputPanel.add(input);           // Add textfield
                    // Add the panel to the frame
getContentPane().add(inputPanel,"North");
\end{lstlisting}
\end{jjjlisting}

The same point would apply if we want to group the keypad
with the convert button and place them at the east edge.
There are several ways these elements could be grouped.  In this
example, we give the panel a border layout and put the keypad
in the center and the convert button at the south edge:

\begin{jjjlisting}
\begin{lstlisting}
JPanel controlPanel= new JPanel(new BorderLayout(0,0));
controlPanel.add(keypad,"Center");
controlPanel.add(convert, "South");
                    // Add the panel to the frame
getContentPane.add(controlPanel,"East"); 
\end{lstlisting}
\end{jjjlisting}

Given these details about the {\tt BorderLayout}, a more appropriate
design for the converter application is shown in
Figure~\ref{fig-metricgui3}.  Notice that the border layout for the
top-level {\tt JFrame} uses only the center, north, and east areas.
Similarly, the border layout for the control panel uses just the
center and south areas.

\begin{figure}[h!]
\figa{chptr13/9f15.eps}{A border layout design for the metric
converter program. The dotted lines show the panels.}
{fig-metricgui3}
\end{figure}

In a {\tt BorderLayout}, when one (or more) border area is not used,
then one or more of the other areas will be extended to fill the
unused area.  For example, if {\tt West} is not used, then {\tt North},
{\tt South}, and {\tt Center} will extend to the left edge of the
{\tt Container}. If {\tt North} is not used, then {\tt West}, {\tt East,} and
{\tt Center} will extend to the top edge.  This is true for all areas except
{\tt Center}. If {\tt Center} is unused, it is left blank.

\pagebreak
Figure~\ref{fig-convertscreen5} shows the results we get when we incorporate these
\marginfig{chptr13/9f16.eps}{The metric converter, showing its appearance
when a border design is used.}
{fig-convertscreen5}
changes into the program.  The only changes to the program itself
occur in the constructor method, which in its revised form is defined
as follows:

\begin{jjjlisting}
\begin{lstlisting}
public Converter() {
  getContentPane().setLayout(new BorderLayout());
  keypad = new KeyPad(this);

  JPanel inputPanel = new JPanel(); // Input panel
  inputPanel.add(prompt);
  inputPanel.add(input);
  getContentPane().add(inputPanel,"North");

  JPanel controlPanel= new JPanel(new BorderLayout(0,0)); 
                                  // Control panel
  controlPanel.add(keypad, "Center");
  controlPanel.add(convert, "South");
  getContentPane().add(controlPanel, "East");
                                 // Output display
  getContentPane().add(display,"Center");  
  display.setLineWrap(true);
  display.setEditable(false);

  convert.addActionListener(this);
  input.addActionListener(this);
} // Converter()
\end{lstlisting}
\end{jjjlisting}
%\begin{figure}

\noindent This layout divides the interface into three main panels,
an input panel, display panel, and control panel, and gives each panel
its own layout.  In addition, the control panel contains the keypad
panel.  Thus, the containment hierarchy for this design is
much more complex than in our original design.

\secEXRHone{Self-Study Exercises}
\label{self-study-exercises}
\begin{SSTUDY}

\item  The border layout for the top window uses the north, center, and east
regions.  What other combinations of areas might be used for these
three components?

\item  Why wouldn't a flow layout be appropriate for the control panel?

\end{SSTUDY}
%\clearpage

\section{Checkboxes, Radio Buttons, and Borders}
\label{checkboxes-radio-buttonsand-borders}
\noindent Suppose you are the software developer for your own software business
%\begin{marginalnote}\it Problem statement\end{marginalnote}
\marginnote{Problem statement}
specializing in computer games.  You want to develop an applet-based
order form that customers can use to order software over the Web.  At the
moment you have three software titles---a chess game, a checkers
game, and a crossword puzzle game.  The assumption is that the user
will choose one or more of these titles from some kind of menu.  The
user must also indicate a payment option---either E-cash, credit card,
or debit card.  These options are mutually exclusive---the user can
choose one and only one.

Let's design an applet interface for this program.   Unlike the
previous problem where the input was a numeric value, in this problem
the input will be the user's selection from some kind of menu.   The
result will be the creation of an order.  Let's suppose that this part
\marginnote{Interface design}
of the task happens behind the scenes---that is, we don't have to
worry about creating an actual order.  The output the user sees will
simply be an acknowledgment that the order was successfully
submitted.

There are several kinds of controls needed for this interface.   First,
%\begin{marginalnote}\it What components do we need?\end{marginalnote}
\marginnote{What components do we need?}
a conventional way to have users indicate their purchase decisions
is to have them click a Submit button.  They should
also have the option to cancel the transaction at any time.

In addition to these button controls, a couple of menus must be
presented, one for the software titles, and one for the payment
choices.  Swing and AWT libraries provide many options for building
menus.

One key requirement for this interface is the mutually exclusive
payment options.  A conventional way to handle this kind of selection
is with a {\tt JRadioButton}---a button that belongs to a group of
mutually exclusive alternatives.  Only one button from the group may be
selected at one time.   The selection of software titles could be
handled by a collection of checkboxes.  A {\tt JCheckbox} is a button
that can be selected and deselected and that always displays its
current state to the user.  Using a checkbox will make it obvious to
the user exactly what software has been selected.

To complete the design, let's use a {\tt JTextArea} again to serve
as something of a printed order form.  It will confirm the user's
order and display other messages needed during the transaction.


Given these decisions, we arrive at the design shown in
Figure~\ref{fig-acmegui}. In this case, our design uses a {\tt JPanel}
%\begin{figure}
\begin{figure}[b]
\figa{chptr13/9f17.eps}{A design for an online order form
interface.}
{fig-acmegui}
\end{figure}
%\end{figure}
as the main container, instead of using the top window itself.  The
reason for this decision is that we want to use Swing {\tt Borders}
around the various {\tt JPanel}s to enhance the overall visual appeal
of the design.  The borders will have titles that help explain the
purpose of the various panels.

Note that the top-level window in this case is a {\tt JApplet}. By default
%\begin{marginalnote}\it What top-level windows do we use?\end{marginalnote}
\marginnote{What top-level windows do we use?}
it will have a border layout.  For the main {\tt JPanel} we are using a
$3 \times 1$ {\tt GridLayout}. The components in the main panel are
the {\tt JTextArea} and two other {\tt JPanel}s.  The {\tt GridLayout}
will take care of sizing these so they are all of equal size.


The center panel, which uses a flow layout, contains panels for the
checkboxes and
%\begin{marginalnote}\it Component layout\end{marginalnote}
\marginnote{Component layout}
the radio buttons.  These elements are grouped within
their own panels.  Again, we can put a border around them in the final
implementation (Fig.~13.26).  The button panels use a
{\tt BoxLayout}, which we will discuss later.
%\begin{marginalnote}\it Containment hierarchy\end{marginalnote}
\marginnote{Containment hierarchy}
This design leads to the most complex containment hierarchy thus far.

\subsection{Checkbox and Radio Button Arrays}
\noindent Because we will need three checkboxes, one for each title, and three
%\begin{marginalnote}\it What data structures do we need?\end{marginalnote}
\marginnote{What data structures do we need?}
radio buttons, one for each payment option, it will be useful again to
use arrays to store both the buttons and their titles:

\begin{jjjlisting}[26.5pc]
\begin{lstlisting}
private ButtonGroup optGroup = new ButtonGroup();
private JCheckBox titles[] = new JCheckBox[NTITLES];
private JRadioButton options[] = new JRadioButton[NOPTIONS];
private String titleLabels[] =
    {"Chess Master - $59.95", "Checkers Pro - $39.95",
                           "Crossword Maker - $19.95"};
private String optionLabels[] = {"Credit Card", 
             "Debit Card", "E-cash"};
\end{lstlisting}
\end{jjjlisting}

\noindent Again, the advantage of this design is that it simplifies the
instantiation and initialization of the buttons:
\marginfigvspace{-76pt}{chptr13/9f18.eps}{Borders around containers help make
them stand out more.
} {fig-acmescreen}

\begin{jjjlisting}
\begin{lstlisting}
for(int k = 0; k < titles.length; k++) {
    titles[k] = new JCheckBox(titleLabels[k]);
    titles[k].addItemListener(this);
    choicePanel.add(titles[k]);
}
\end{lstlisting}
\end{jjjlisting}

\noindent The only difference between this array of checkboxes and the keypad
array of buttons that we used in the {\tt Converter} program is that
checkboxes generate {\tt ItemEvent}s instead
{\tt ActionEvent}s. Therefore, each checkbox must be registered with an
{\tt ItemListener} (and, of course, the applet itself must implement the
{\tt ItemListener} interface).  We'll show how {\tt ItemEvent}s are
handled later.

The code for instantiating and initializing the radio buttons is
almost the same:

\begin{jjjlisting}
\begin{lstlisting}
for(int k = 0; k < options.length; k++) {
    options[k] = new JRadioButton(optionLabels[k]);
    options[k].addItemListener(this);
    optionPanel.add(options[k]);
    optGroup.add(options[k] );
}
options[0].setSelected(true); // Set first button 'on'
\end{lstlisting}
\end{jjjlisting}
%$
\noindent Radio buttons also generate {\tt ItemEvent}s, so they too
must be registered with an {\tt ItemListener}. Note that the
first button is set on, which represents a default payment
option for the user.

The difference between checkboxes and radio buttons is that radio
buttons must be added to a {\tt ButtonGroup}---here named {\tt
optGroup}---in order to enforce mutual exclusion among them.  A {\tt
ButtonGroup} is an object whose sole task is to enforce mutual
exclusion among its members. Whenever you click one radio button, the
{\tt ButtonGroup} will automatically be notified of this event and
will turn off whatever other button was turned on.  As
Figure~\ref{fig-p504f1} illustrates, radio buttons are monitored by
two different objects, a {\tt ButtonGroup}, which manages the radio
buttons' states, and an {\tt ItemListener}, which listens for clicks
on the buttons and takes appropriate actions.

\begin{figure}[h!]
\figa{chptr13/p504f1.eps}{The {\tt ButtonGroup} object tracks each radio button's
state, ensuring that only one is selected at a time. The {\tt ItemListener}
listens for events on each button.}
{fig-p504f1}

\end{figure}

Note the effective division of labor in the design of
%\begin{marginalnote}\it Divide and conquer\end{marginalnote}
\marginnote{Divide and conquer}
the various objects to which a radio button belongs. The
{\tt optionPanel} is a GUI component (a {\tt JPanel}) that contains the
button within the visual interface. Its role is to help manage the
graphical aspects of the button's behavior.  The {\tt ButtonGroup} is
just an {\tt Object}, not a GUI component. Its task is to monitor the
button's relationship to the other buttons in the group.  Each object
has a clearly delineated task.

This division of labor is a key feature of object-oriented design. It
is clearly preferable to giving one object broad responsibilities.
For example, a less effective design might have given the task of
managing a group of buttons to the {\tt JPanel} that contains them.
However, this would lead to all kinds of problems, not least of which
is the fact that not everything in the container belongs to the same
button group. So a clear division of labor is a much preferable
design.

\JavaTIP{EFFECTIVE DESIGN}{Division of Labor.}{In good object-oriented
design, objects are specialists (experts) for very narrow, clearly
defined tasks. If there's a new task that needs doing, design a new
object to do it.}

\subsection{Swing Borders}
\noindent The Swing {\tt Border} and {\tt BorderFactory} classes can place
borders around virtually any GUI element. Using borders is an
effective way to make the grouping of components more
apparent. Borders can have titles, which enhance the GUI's ability to
guide and inform the user. They can also have a wide range of styles
and colors, thereby helping to improve the GUI's overall appearance.



A border occupies some space around the edge of a
{\tt JComponent}. For the Acme Software Titles interface, we place titled borders around
four of the panels (Fig.~13.26). The border on the
main panel serves to identify the company again. The one around the
button panel serves to group the two control buttons. The borders
around both the checkbox and the radio button menus help to set them
apart from other elements of the display and help identify the
purpose of the buttons.


Attaching a titled border to a component---in this case to a
{\tt JPanel}---is very simple. It takes one statement:

\begin{jjjlisting}
\begin{lstlisting}
choicePanel.setBorder(
       BorderFactory.createTitledBorder("Titles"));
\end{lstlisting}
\end{jjjlisting}

\noindent The {\tt setBorder()} method is defined in {\tt JComponent},
is inherited by all Swing components, and takes a {\tt Border}
argument. In this case, we use the {\tt BorderFactory} class to create
a border and assign it a title.  There are several versions of the static
{\tt createTitledBorder()} method. This version lets us specify the
border's title. It uses default values for type of border (etched),
the title's position (sitting on the top line), justification
(left), and for font's type and color.

As you would expect, the {\tt Border} and {\tt BorderFactory} classes
contain methods that let you exert significant control over the
border's look and feel. You can even design and create your own custom
borders.

\subsection{The {\tt BoxLayout} Manager}
\noindent Another type of layout to use is the {\tt BoxLayout}. This can be
associated with any container, and it comes as the default with the
Swing {\tt Box} container. We use it in this example to arrange the
checkboxes and radio buttons (Fig.~\ref{fig-acmegui}).

A {\tt BoxLayout} is like a one-dimensional grid layout. It allows
multiple components to be arranged either vertically or horizontally
in a row.  The layout will not wrap around, as does the
{\tt FlowLayout}. Unlike the {\tt GridLayout}, the {\tt BoxLayout} does not
force all its components to be the same size. Instead, it tries to use
each component's preferred width (or height) in arranging them
horizontally (or vertically).
(Every Swing component has a preferred
size that is used by the various layout managers in determining the
component's actual size in the interface.)  The {\tt BoxLayout}
manager also tries to align its components' heights (for horizontal
layouts) or widths (for vertical layouts).


Once again, to set the layout manager for a container you use the
{\tt setLayout()} method:

\begin{jjjlisting}
\begin{lstlisting}
choicePanel.setLayout(new 
             BoxLayout(choicePanel,BoxLayout.Y_AXIS));
\end{lstlisting}
\end{jjjlisting}

\noindent The {\tt BoxLayout()} constructor has two parameters. The
first is a reference to the container that's being managed, and the
second is a constant that determines whether horizontal ({\it x}-axis) or
vertical ({\it y}-axis) alignment is used.

One nice feature of the {\tt BoxLayout} is that it can be used in
combinations to imitate the look of the very complicated
{\tt GridBoxLayout}. For example, Figure~\ref{fig-nestedbox} shows an
example with two panels (Panel1 and Panel2) arranged horizontally
within an outer box (Panel0), each containing four components arranged
\marginfig{chptr13/9f19.eps}{Complex layouts can be achieved by
nesting containers that use the {\tt BoxLayout}.}
{fig-nestedbox}
vertically. The three panels all use the {\tt BoxLayout}.


\subsection{The {\tt ItemListener} Interface }
\noindent In this section, we will describe how to handle menu
selections.  Whenever the user makes a menu selection, or clicks a
check box or radio button, an {\tt ItemEvent} is generated.  {\tt
ItemEvent}s are associated with items that make up menus, including
{\tt JPop\-upMenu}s, {\tt JCheckboxes}, {\tt JRadioButton}s, and other
types of menus. Item events are handled by the {\tt ItemListener}
interface, which consists of a single method, the {\tt
itemStateChanged()} method:

\begin{jjjlisting}
\begin{lstlisting}
public void itemStateChanged(ItemEvent e) {
  display.setText("Your order so far (Payment by: ");
  for (int k = 0; k < options.length; k++ )
    if (options[k].isSelected())
      display.append(options[k].getText() + ")\n");
  for (int k = 0; k < titles.length; k++ )
    if (titles[k].isSelected())
      display.append("\t" + titles[k].getText() + "\n");
} // itemStateChanged()
\end{lstlisting}
\end{jjjlisting}

\noindent This version of the method handles item changes for both the
checkbox menu and the radio buttons menu.  The code uses two
consecutive for loops. The first iterates through the {\tt options}
menu (radio buttons) to determine what payment option the user has
selected. Since only one option can be selected, only one title will
be appended to the display. The second loop iterates through the
titles menu (checkboxes) and appends each title the user selected to
the display. This way the complete status of the user's order is
displayed after every selection. The {\tt isSelected()} method is used
to determine if a checkbox or radio button is selected or not.

In this example, we have no real need to identify the item that caused
the event. No matter what item the user selected, we want to display
the entire state of the order. However, like the \mbox{\tt ActionEvent}
class, the \mbox{\tt ItemEvent}  class contains methods that can 
retrieve the item that caused the event:

\begin{jjjlisting}
\begin{lstlisting}
getItem();  // Returns a menu item within a menu
\end{lstlisting}
\end{jjjlisting}

\noindent The {\tt getItem()} method is the {\tt ItemListener}'s analogue
to the {\tt ActionEvent}'s {\tt getSource()} method. It enables you to obtain
the object that generated the event but returns a representation of
the item that was selected or deselected.

\subsection{The {\tt OrderApplet} }
\noindent The design of the {\tt OrderApplet} is summarized in
Figure~13.29 and its complete implementation is given in
Figure~\ref{fig-orderapplet}. There are several important points to
make about this program. First, five {\tt JPanel}s are used to
organize the components into logical and visual groupings.  This
conforms to the design shown in Figure~\ref{fig-acmegui}.
%\begin{figure}
%\begin{graphic}
\marginfig{chptr13/p507f1.eps}{The {\tt OrderApplet} makes extensive use of GUI components.}
{fig-p507f1}
%\end{graphic}
%\end{figure}

\begin{figure}[p]
\jjjprogstart
\begin{jjjlisting}[28pc]
\begin{lstlisting}
import javax.swing.*;
import javax.swing.border.*;
import java.awt.*;
import java.awt.event.*;

public class OrderApplet extends JApplet 
                     implements ItemListener, ActionListener {
  private final int NTITLES = 3, NOPTIONS = 3;
  private JPanel mainPanel = new JPanel(),
                 centerPanel = new JPanel(),
                 choicePanel = new JPanel(),
                 optionPanel = new JPanel(),
                 buttonPanel = new JPanel();
  private ButtonGroup optGroup = new ButtonGroup();
  private JCheckBox titles[] = new JCheckBox[NTITLES];
  private JRadioButton options[] = new JRadioButton[NOPTIONS];
  private String titleLabels[] =
    {"Chess Master - $59.95", "Checkers Pro - $39.95",
                                "Crossword Maker - $19.95"};
  private String optionLabels[] = {"Credit Card", 
                                    "Debit Card", "E-cash"};
  private JTextArea display = new JTextArea(7, 25);
  private JButton submit = new JButton("Submit Order"),
                  cancel = new JButton("Cancel");

  public void init() {
    mainPanel.setBorder(BorderFactory.createTitledBorder(
                                   "Acme Software Titles"));
    mainPanel.setLayout(new GridLayout(3, 1, 1, 1));
    cancel.addActionListener(this);
    submit.addActionListener(this);
    initChoices();
    initOptions();
    buttonPanel.setBorder(BorderFactory.createTitledBorder(
                                              "Order Today"));
    buttonPanel.add(cancel);
    buttonPanel.add(submit);
    centerPanel.add(choicePanel);
    centerPanel.add(optionPanel);

    mainPanel.add( display);
    mainPanel.add(centerPanel);
    mainPanel.add( buttonPanel);
    getContentPane().add(mainPanel);
    setSize(400,400);
  } // init()
\end{lstlisting}
\end{jjjlisting}
\jjjprogstop{The {\tt OrderApplet} class, Part I.}
{fig-orderapplet}
\end{figure}
%$
\begin{figure}[p]
\addtocounter{figure}{-1}
\jjjprogstart
\begin{jjjlistingleft}[29pc]{-3pc}
\begin{lstlisting}
  private void initChoices() {
    choicePanel.setBorder(
             BorderFactory.createTitledBorder("Titles"));
    choicePanel.setLayout(
            new BoxLayout(choicePanel, BoxLayout.Y_AXIS));
    for (int k = 0; k < titles.length; k++) {
      titles[k] = new JCheckBox(titleLabels[k]);
      titles[k].addItemListener(this);
      choicePanel.add(titles[k]);
    }
  } // initChoices()

  private void initOptions() {
    optionPanel.setBorder(
             BorderFactory.createTitledBorder("Payment By"));
    optionPanel.setLayout(
             new BoxLayout(optionPanel, BoxLayout.Y_AXIS));
    for (int k = 0; k < options.length; k++) {
      options[k] = new JRadioButton(optionLabels[k]);
      options[k].addItemListener(this);
      optionPanel.add(options[k]);
      optGroup.add(options[k]);
    }
    options[0].setSelected(true);
  } // initOptions()

  public void itemStateChanged(ItemEvent e) {
    display.setText("Your order so far (Payment by: ");
    for (int k = 0; k < options.length; k++ )
      if (options[k].isSelected())
        display.append(options[k].getText() + ")\n");
    for (int k = 0; k < titles.length; k++ )
      if (titles[k].isSelected())
        display.append("\t" + titles[k].getText() + "\n");
  } // itemStateChanged()

  public void actionPerformed(ActionEvent e){
    String label = submit.getText();
    if (e.getSource() == submit) {
      if (label.equals("Submit Order")) {
        display.append(
        "Thank you. Press 'Confirm' to submit your order!\n");
        submit.setText("Confirm Order");
      } else {
        display.append(
         "Thank you. You will receive your order tomorrow!\n");
        submit.setText("Submit Order");
      }
    } else
      display.setText(
        "Thank you. Maybe we can serve you next time!\n");
  } // actionPerformed()
} // OrderApplet
\end{lstlisting}
\end{jjjlistingleft}
\jjjprogstop{{\it (continued)} The {\tt OrderApplet} class, Part II.} {fig:orderappletpart2cont}
\end{figure}

Second, note the use of titled borders around the four internal
panels. These help reinforce that the components within the
border are related by function.

The applet {\tt init()} method is used to initialize the
interface. This involves setting the layouts for the various
containers and filling the containers with their components. Because
their initializations are relatively long, the checkboxes and radio
buttons are initialized in separate methods, the {\tt initChoices()}
and {\tt initOptions()} methods, respectively.

Finally, note how the {\tt actionPerformed()} method creates a mock
order form in the display area. This allows the user to review the
order before it is submitted. Also note that the algorithm used for
submittal requires the user to confirm an order before it is actually
submitted. The first time the user clicks the Submit button, the
button's label is changed to, ``Confirm Order,'' and the user is
prompted in the display area to click the \mbox{Confirm} button to submit the
order. This design allows the interface to catch inadvertent button
clicks.

A user interface should anticipate errors by the user. When
a program involves an action that can't be undone---such as placing
an order---the program should make sure the user really wants to
take the action before carrying it out.

\JavaTIP{EFFECTIVE DESIGN}{Anticipate the User.}{A well-designed interface
should make it difficult for the user to make errors and should
make it easy to recover from mistakes when they do happen.}

\secEXRHone{Self-Study Exercise}
\label{self-study-exercise}
\begin{SSTUDY}

\item What's your favorite interface horror
story?  How would you have remedied the problem? The interface needn't
be a computer interface.

\end{SSTUDY}

\pagebreak
\section{Menus and Scroll Panes}
\label{menusand-scroll-panes}
\noindent Pop-up and pull-down menus allow an application or applet to grow in
complexity and functionality without cluttering its interface. Menus
are hierarchical in nature. A particular menu is divided into a number
of menu items, which can themselves be further subdivided.  
Java makes it simple to implement menus.

A {\tt JMenuBar} is an implementation of a menu bar---a horizontal
list of names that appears at the top of a window
(Fig.~\ref{fig-menuterms}). 

\begin{figure}[h!]
\figa{chptr13/9f20.eps}{An application with a menu bar that is
showing its edit menu. The edit menu contains a cascading drop-down menu that
can show recent cuts.}
{fig-menuterms}
\end{figure}

Almost all applications have a menu bar. To construct a menu, you add
{\tt JMenu} objects to a {\tt JMenuBar}. A {\tt JMenu} is essentially
a clickable area on a menu bar that is associated with a {\tt
JPopupMenu}, a small window that pops up and displays the menu's {\tt
JMenuItem}s.  A menu can also contain {\tt JSeparator}s, which are
dividers that can be placed between menu items to organize them into
logical groupings.

\subsection{Adding a Menu Bar to an Application}
\noindent It is easy to create menus in Swing. The process involves three steps,
although you needn't perform them in this order:

\begin{NL}
\item  Create the individual {\tt JMenuItem}s.
\item  Create a {\tt JMenu} and add the {\tt JMenuItem}s to it.
\item  Create a {\tt JMenuBar} and add the {\tt JMenu}s to it.
\end{NL}

For example, suppose you're building the interface for a text
editor. A text editor typically contains at least two standard
menus. The file menu is used to create new documents, open and close
files, save your document, and so on.  The edit menu is used to cut
and paste selected text from the document. 

Here's how you would create the file menu for this program. First, you
create a menu bar and make it the menu bar for the application's
{\tt JFrame} or for the {\tt JApplet}. This is usually done in the
application's constructor or in the applet's {\tt init()} method:

\begin{jjjlisting}
\begin{lstlisting}
JMenuBar mBar = new JMenuBar();// Create menu bar
this.setMenuBar(mBar);   // Add it to this window
\end{lstlisting}
\end{jjjlisting}

\noindent The next step involves creating and adding menus and menu
items to the menu bar. This is also usually done in the constructor
or the {\tt init()} method. If the menu is large, you should break
this task into subtasks and define a method for each subtask.

\JavaTIP[false]{EFFECTIVE DESIGN}{Method Size.}{A method that gets longer
than 20 to 25 lines is probably trying to do too much and should
be divided into separate methods, each with a clearly defined
task.}

\noindent Here's the definition of the file menu for our
simple text editor:

\begin{jjjlisting}
\begin{lstlisting}
private void initFileMenu() {
    fileMenu = new JMenu("File");   // Create menu
    mBar.add(fileMenu);      // Add it to menu bar

    openItem = new JMenuItem("Open"); // Open item
    openItem.addActionListener( this );
    openItem.setEnabled(false);
    fileMenu.add(openItem);

    saveItem = new JMenuItem("Save"); // Save item
    saveItem.addActionListener(this);
    saveItem.setEnabled(false);
    fileMenu.add(saveItem);
    fileMenu.addSeparator();  // Logical separator

    quitItem = new JMenuItem("Quit"); // Quit item
    quitItem.addActionListener(this);
    fileMenu.add(quitItem);
} // initFileMenu()
\end{lstlisting}
\end{jjjlisting}

\noindent The first two statements in the method create the file
menu and add it to the menu bar.  The rest of the statements create the
individual menu items that make up the file menu.  Note the use of a
{\it separator} item after the save item. This has the effect of
grouping the file-handling items (open and save) into one logical
category and distinguishing them from the quit item. A separator is
represented as a line in the menu (Fig.~\ref{fig-menuterms}).

\JavaTIP{EFFECTIVE DESIGN}{Logical Design.}{In designing interfaces, an
effort should be made to use visual cues, such as menu item separators
and borders, to group items that are logically related. This will help
to orient the user.}

\noindent Note that each menu item is given an {\tt ActionListener}.  As
we'll see shortly, action events for menu items are handled the same
way as action events for buttons. Finally, note how the
{\tt setEnabled()} method is used to disable both the open and save menu
items. Implementation of these actions is left as an exercise.

\subsection{Menu Hierarchies}
\noindent Menus can be added to other menus to create a hierarchy. For
example, the edit menu will include the standard cut, copy, and paste
menu items.  Some edit menus also contain an ``Undo'' item, which can
be used to undo the last editing operation that was performed. In
other words, if you cut a piece of text, you can undo that operation
and get that cut back. Many editors seem to allow just a single
undo. If you cut two pieces of text, the first piece is lost to the
user to undo. This can be an issue, especially if you didn't mean to
do the first cut.

To help remedy this type of situation, let's add a feature to our
editor that will keep track of cuts by storing them in a {\tt
Vector}. This function will be like an ``Unlimited Undo'' operation
for cuts. For this example, we won't place any limit on the size of
the vector. Every cut the user makes will be inserted at the beginning
of the vector.  To go along with this feature we need a menu that can
grow dynamically during the program. Each time the user makes a cut,
the string that was cut will be added to the menu.

%\begin{figure}
%\begin{graphic}
%\includegraphics{./art/9/9f22.eps}
%\caption{A cascading
%drop-down menu to keep track of the user's cuts.
%\label{fig-cutsmenu}
%\label{pg-fig-cutsmenu}}
%\end{graphic}
%\end{figure}

This kind of menu should occur within the edit menu, but it will have
its own items. This is a menu within a menu
(Fig.~\ref{fig-menuterms}), an example of a {\it cascading}
drop-down menu. The edit menu itself drops down from the menu bar,
and the recent cuts menu drops down and to the right of where its
arrow points. The following method was used to create the edit menu:

\begin{jjjlisting}
\begin{lstlisting}
private void initEditMenu() {
  editMenu = new JMenu("Edit"); // Create edit menu
  mBar.add(editMenu);            // Add to menu bar

  cutItem = new JMenuItem ("Cut");      // Cut item
  cutItem.addActionListener(this);
  editMenu.add(cutItem);
  copyItem = new JMenuItem("Copy");    // Copy item
  copyItem.addActionListener(this);
  editMenu.add(copyItem);
  pasteItem = new JMenuItem("Paste"); // Paste item
  pasteItem.addActionListener(this);
  editMenu.add(pasteItem);
  editMenu.addSeparator();
  selectItem = new JMenuItem("Select All");// Select
  selectItem.addActionListener(this);
  editMenu.add(selectItem);
  editMenu.addSeparator();
  cutsMenu = new JMenu("Recent Cuts");//Cuts submenu
  editMenu.add(cutsMenu);
} // initEditMenu()
\end{lstlisting}
\end{jjjlisting}

\noindent The main difference between this method and the
one used to create the file menu is that here we insert an entire
submenu as one of the items in the edit menu. The {\tt cutsMenu} will
be used to hold the strings that are cut from the document.  Initially,
it will be empty.


\subsection{Handling Menu Actions}
\noindent Handling {\tt JMenuItem} actions is no different from handling
{\tt JButton} actions. Whenever a user makes a menu selection, an
{\tt ActionEvent} is generated. Programs that use menus must implement the
{\tt action\-Performed()} method of the {\tt ActionListener} interface.
In the text editor example, there are a total of six enabled menu
items, including the recent cuts menu. This translates into a large
if-else structure, with each clause handling a single menu item. 

The following {\tt actionPerformed()} method is used to handle the
menu selections for the text editor:

\begin{jjjlistingleft}[33pc]{-7pc}
\begin{lstlisting}
public void actionPerformed(ActionEvent e) {
  JMenuItem m  = (JMenuItem)e.getSource(); // Get selected menu item
  if ( m == quitItem ) {                   // Quit
    dispose();}
  } else if (m == cutItem) {                // Cut the selected text
    scratchPad = display.getSelectedText(); // Copy text to scratchpad
    display.replaceRange("",                //  and delete
        display.getSelectionStart(),   //  from the start of selection
        display.getSelectionEnd());    //  to the end
        addRecentCut(scratchPad);      // Add text to the cuts menu
  } else if (m == copyItem)            // Copy text to scratchpad
    scratchPad = display.getSelectedText();
  } else if (m == pasteItem) { // Paste scratchpad to document at caret
    display.insert(scratchPad, display.getCaretPosition()); // position
  } else if ( m == selectItem ) {
    display.selectAll();              // Select the entire document
  } else {
    JMenuItem item = (JMenuItem)e.getSource(); // Default is cutsMenu
    scratchPad = item.getActionCommand(); // Put cut back in scratchpad
  }
} // actionPerformed()
\end{lstlisting}
\end{jjjlistingleft}

\noindent The method begins by getting the source of the {\tt ActionEvent}
and casting it into a {\tt JMenuItem}. It then checks each case of the
if-else structure. Because the actions taken by this program are
fairly short, they are mostly coded within the {\tt actionPerformed()}
method itself. However, for most programs it will be necessary to write
a separate method corresponding to each menu item and then call
the methods from {\tt actionPerformed()}.

Our text editor's main task is to implement the cut/copy/paste
functions, which are simple to do in Java. The text that's being
edited is stored in a {\tt JTextArea}, which contains instance methods
that make it very easy to select, insert, and replace text. To copy a
piece of text, the program need only get the text from the
{\tt JTextArea} ({\tt getSelectedText()}) and assign it to the
{\tt scratchpad}, which is represented as a {\tt String}. To paste a piece
of text, the program inserts the contents of the {\tt scratchpad} into
the {\tt JTextArea} at the location marked by the {\it caret}, a
cursor-like character in the document that marks the next insertion
point.


The structure of this if-else statement is significant. Note how the
default case of the if-else is designed. We are using the last else
clause as a ``catch all'' condition to catch and handle selections
from the {\tt cutsMenu}.  All of the other
%\begin{marginalnote}\it Default logic\end{marginalnote}
menu items can be referred to by name. However, the menu items in the
{\tt cutsMenu} are just snippets of a string that the user has
\marginnote{Default logic}
previously cut from the text, so they can't be referenced by
name. Luckily, we don't really need to. For any {\tt JMenuItem}, the
{\tt getActionCommand()} method returns its text, which in this case
is the previously cut text. So we just assign the cut text from the
menu to the {\tt scratchpad}.


\JavaTIP{PROGRAMMING TIP}{Default Cases.}{Although the order of the
clauses in an if-else structure is usually not important,
the default clause can sometimes be used to handle cases that can't be
referenced by name.}

\subsubsection*{Handling Previously Cut Text}
\noindent The most difficult function in our program is the cut operation. Not
only must the selected text be removed from the document and stored in
the {\tt scratchpad}, but it must also be inserted into the vector that is
storing all the previous cuts. The {\tt addRecentCut()} method takes
care of this last task. The basic idea here is to take the cut string
%\begin{marginalnote}\it Algorithm design\end{marginalnote}
\marginnote{Algorithm design}
and insert it at the beginning of the vector, so that cuts will be
maintained in a last-in--first-out order. Then the {\tt cutsMenu} must
be completely rebuilt by reading its entries out of the vector, from
first to last. That way the most recent cut will appear first in the
menu:

\begin{jjjlisting}
\begin{lstlisting}
private void addRecentCut(String cut) {
  recentCuts.insertElementAt(cut,0);
  cutsMenu.removeAll();
  for (int k = 0; k < recentCuts.size(); k++) {
    JMenuItem item = 
     new JMenuItem((String)recentCuts.elementAt(k));
    cutsMenu.add( item );
    item.addActionListener(this);
  }
} // addRecentCut()
\end{lstlisting}
\end{jjjlisting}

\noindent The {\tt recentCuts Vector} stores the cut strings. Note
the use of the {\tt insert\-ElementAt()} method to insert strings into the
vector and the {\tt elementAt()} method to get strings from the
vector. (You may find it helpful to review the section on vectors in Chapter~9.)

Note also how menu items are removed and inserted in menus.  The
{\tt cutsMenu} is reinitialized, using the {\tt removeAll()} method.  Then
the for loop iterates through the strings stored in the vector, making
new menu items from them, which are then inserted into the
%\begin{figure}
%\begin{graphic}
\marginfig{chptr13/p514f1.eps}{Design of the {\tt SimpleTextEditor}.}
{fig-p514f1}
%\end{graphic}
%\end{figure}
{\tt cutsMenu}.  In this way, the {\tt cutsMenu} is changed dynamically
each time the user cuts a piece of text from the document.

\subsection{Adding Scrollbars to a Text Area}
\noindent The design of the {\tt SimpleTextEditor} class is summarized in
Figure~\ref{fig-p514f1} and its complete implementation is shown in
Figure~\ref{fig-texteditor}. 
\pagebreak
\begin{figure}[h!]
\jjjprogstart
\begin{jjjlistingleft}[37pc]{-11pc}
\begin{lstlisting}
import javax.swing.*;
import java.awt.*;
import java.awt.event.*;
import java.util.Vector;

public class SimpleTextEditor extends JFrame implements ActionListener{
  private JMenuBar mBar = new JMenuBar();        // Create the menu bar
  private JMenu fileMenu, editMenu, cutsMenu;   // Menu references and items
  private JMenuItem cutItem, copyItem, pasteItem, selectItem,recentcutItem; 
  private JMenuItem quitItem, openItem, saveItem;       // File items
  private JTextArea display = new JTextArea(); // Here's where the editing occurs
  private String scratchPad = "";              // Scratch pad for cut/paste
  private Vector recentCuts = new Vector();

  public SimpleTextEditor() {
    super("Simple Text Editor");     // Set the window title
    this.getContentPane().setLayout(new BorderLayout());
    this.getContentPane().add("Center", display);
    this.getContentPane().add(new JScrollPane(display));
    display.setLineWrap(true);
    this.setJMenuBar(mBar);         // Set this program's menu bar
    initFileMenu();                 // Create the menus
    initEditMenu();
  } // SimpleTextEditer()

  private void initEditMenu() {
    editMenu = new JMenu("Edit");     // Create the edit menu
    mBar.add(editMenu);               //  and add it to menu bar
    cutItem = new JMenuItem ("Cut");          // Cut item
    cutItem.addActionListener(this);
    editMenu.add(cutItem);
    copyItem = new JMenuItem("Copy");         // Copy item
    copyItem.addActionListener(this);
    editMenu.add(copyItem);
    pasteItem = new JMenuItem("Paste");       // Paste item
    pasteItem.addActionListener(this);
    editMenu.add(pasteItem);
    editMenu.addSeparator();
    selectItem = new JMenuItem("Select All"); // Select item
    selectItem.addActionListener(this);
    editMenu.add(selectItem);
    editMenu.addSeparator();
    cutsMenu = new JMenu("Recent Cuts");      // Recent cuts submenu
    editMenu.add(cutsMenu);
  } // initEditMenu()

  private void initFileMenu() {
    fileMenu = new JMenu("File");       // Create the file menu
    mBar.add(fileMenu);                 //  and add it to the menu bar
    openItem = new JMenuItem("Open");   // Open item
    openItem.addActionListener( this );
    openItem.setEnabled(false);
    fileMenu.add(openItem);
    saveItem = new JMenuItem("Save");   // Save item
\end{lstlisting}
\end{jjjlistingleft}
\jjjprogstop{A menu-based {\tt SimpleTextEditor} application, Part I.}
{fig-texteditor}
\end{figure}

\clearpage
\begin{figure}[h!]
\addtocounter{figure}{-1}
\jjjprogstart
\begin{jjjlisting}[37pc]
\begin{lstlisting}
    saveItem.addActionListener(this);
    saveItem.setEnabled(false);
    fileMenu.add(saveItem);
    fileMenu.addSeparator();    // Logical separator
    quitItem = new JMenuItem("Quit");   // Quit item
    quitItem.addActionListener(this);
    fileMenu.add(quitItem);
  } // initFileMenu()
  public void actionPerformed(ActionEvent e) {
    JMenuItem m  = (JMenuItem)e.getSource();// Get selected menu item
    if ( m == quitItem ) {                  // Quit
      dispose();
    } else if (m == cutItem) {   // Cut the selected text
      scratchPad = display.getSelectedText(); // Copy text to scratchpad
      display.replaceRange("",         //  and delete
          display.getSelectionStart(), //  from the start of the selection
          display.getSelectionEnd());  //  to the end
      addRecentCut(scratchPad);   // Add the cut text to the cuts menu
    } else if (m == copyItem) {  // Copy the selected text to the scratchpad
      scratchPad = display.getSelectedText();
    } else if (m == pasteItem) { // Paste the scratchpad to the document at caret
      display.insert(scratchPad, display.getCaretPosition()); // position
    } else if ( m == selectItem ) {
      display.selectAll();       // Select the entire document
    } else {
      JMenuItem item = (JMenuItem)e.getSource(); // Default is cutsMenu
      scratchPad = item.getActionCommand(); // Put cut back in the scratchpad
    }
  } // actionPerformed()
  private void addRecentCut(String cut) {
    recentCuts.insertElementAt(cut,0);
    cutsMenu.removeAll();
    for (int k = 0; k < recentCuts.size(); k++) {
      JMenuItem item = 
         new JMenuItem((String)recentCuts.elementAt(k));
      cutsMenu.add( item );
      item.addActionListener(this);
    }
  } // addRecentCut()

  public static void main(String args[]) {
    SimpleTextEditor f = new SimpleTextEditor();
    f.setSize(300, 200);
    f.setVisible(true);
    f.addWindowListener(new WindowAdapter() {     
      public void windowClosing(WindowEvent e) {
        System.exit(0);   // Quit the application
      }
    });
  } // main()
} // SimpleTextEditor
\end{lstlisting}
\end{jjjlisting}
\jjjprogstop{{\it (continued)} The {\tt SimpleTextEditor}, Part II.}{fig:simpletexteditorpart2cont}
\end{figure}

\clearpage
\noindent It uses a {\tt BorderLayout}, with the {\tt JTextArea}
placed at the center.  Note how simple it is to add
%\begin{marginalnote}\it Scrollbars\end{marginalnote}
\marginnote{Scrollbars}
scrollbars to the text area:

\begin{jjjlisting}
\begin{lstlisting}
this.getContentPane().add(new JScrollPane(display));
\end{lstlisting}
\end{jjjlisting}

\noindent This statement creates a {\tt JScrollPane} and
adds it to the application's container.  A {\tt JScrollPane} is one of
Swing's scrollbar classes. Its function is to manage the viewing and
scrolling of a scrollable component, such as a {\tt JTextArea}.  A
{\tt JScrollPane} is actually a container, which is why it takes the
{\tt display} as an argument. The {\tt display} is being added to the
{\tt JScrollPane}.

Just about any {\tt Component} can be added to a {\tt JScrollPane}.
Once a component is added, the scroll pane will manage the
scrolling functions for the component. The default constructor used in
this example takes a single {\tt Component} parameter. This refers to
the scrollable component, in this case to the {\tt JTextArea}.
Another constructor that you might use takes the following form:

\begin{jjjlisting}
\begin{lstlisting}
public JScrollPane(Component comp, int vsbPolicy, 
                                     int hsbPolicy);
\end{lstlisting}
\end{jjjlisting}

\noindent The two integers refer to the vertical and horizontal
scrolling policies. These cover properties such as whether the
scrollbars are always present or just as needed. The default policy is
to attach scrollbars to the component only when needed. Thus, to see
the scrollbars in the {\tt SimpleText} {\tt Editor}, you would have to shrink
the window to the point where all of the text cannot be viewed
(Fig.~\ref{fig-scrollexample}).  Because the text area in this example
is wrapping the text, the horizontal scrollbar will never be needed.
%\begin{figure}
%\begin{figure}[tb]
\marginfig{chptr13/9f25.eps}{The scrollbars appear on the text area
only when they are needed. In this case, only a vertical scrollbar
is necessary.}
{fig-scrollexample}
%\end{figure}
%\end{figure}


\secEXRHone{Self-Study Exercises}
\label{self-study-exercises}
\begin{SSTUDY}

\item  Modify the {\tt addRecentCut()} method so it limits the cuts stored
in the vector to the last ten cuts.


\item  Modify the {\tt addRecentCut()} method so that it doesn't duplicate
cuts already stored in the vector. ({\it Hint}: Use the 
{\tt indexOf(String)} method in the {\tt Vector} class.)

\end{SSTUDY}

\section*{{\color{cyan}Special Topic:} Are Computers Intelligent?}

Contemporary computer interfaces are largely visual and graphical, and
many things we use a computer for, such as word processing, still
require us to type. Will there come a day when instead of typing a
letter or e-mail message, we'll be able to dictate it to our computer?
Will computers eventually have the same kind of interface we have---that is, will 
we someday be able to carry on conversations with our
computers?  Clearly, a ``conversational interface'' would require
substantial intelligence on the part of the computer. Do computers
have any chance of acquiring such intelligence?\\
\hspace*{1pc}The question of machine intelligence or {\it artificial intelligence
(AI)} has been the subject of controversy since the very first
computers were developed. In 1950, in an article in the journal {\it
Mind}, Alan Turing proposed the following test to settle the question
of whether computers could be intelligent. Suppose you put a person
and a computer in another room, and you let a human interrogate both
with any kind of question whatsoever. The interrogator could ask them
to parse a Shakespearian sonnet, or solve an arithmetic problem, or
tell a joke. The computer's task would be to try to fool the
interrogator into thinking that it was the human. And the (hidden)
human's task would be to try to help the interrogator see that he or
she was the human.\\
\hspace*{1pc}Turing argued that someday computers would be able to play this game
so well that interrogators would have no better than a 50/50 chance of
telling which was which. When that day came, he argued, we would have
to conclude that computers were intelligent.\\
\hspace*{1pc}This so-called {\it Turing test} has been the subject of controversy ever
since.  Many of the founders of AI and many of its current
practitioners believe that computation and human thinking are
basically the same kind of process and that eventually computers will
develop enough capability that we'll have to call them intelligent.
Skeptics argue that even if computers could mimic our intelligence,
there's no way they will be self-conscious and, therefore, they can
never be truly intelligent. According to the skeptics, merely
executing programs, no matter how clever the programs are, will never
add up to intelligence.\\
\hspace*{1pc}Computers have made some dramatic strides lately. In 1997, an IBM
computer named Deep Blue beat world chess champion
Gary Kasparov in a seven-game chess match. In 1998, a computer
at Los Alamos National Laboratory proved a mathematical theorem
that some of the best mathematicians were unable to prove
for the past 40 years.\\
\hspace*{1pc}However, despite these achievements, most observers would agree that
computers are not yet capable of passing the Turing test. One area
where computers fall short is in natural language
understanding. Although computers are good at understanding Java and
other computer languages, human languages are still too complex and
require too much common sense knowledge for computers to understand
them perfectly. Another area where computers still fall somewhat short
is in speech recognition. However, an American company recently
demonstrated a telephone that could translate between English and
German (as well as some other languages) in real time.  The device's
only limitation was that its discourse was limited to the travel
domain.  As computer processing speeds improve, this limitation is
expected to be only temporary.  Thus, we may be closer than we think
to having our ``conversational user interface.''\\
\hspace*{1pc}Natural language understanding, speech recognition, learning,
perception, chess playing, and problem solving are the kinds of
problems addressed in AI, one of the major applied areas of computer
science. Almost every major research group in AI has a Web site that
describes its work. To find some of these, just do a search for
``artificial intelligence'' and then browse through the links that
are returned.

\pagebreak
\secSMHleft{Chapter Summary}
\label{chapter-summary}
%\addcontentsline{toc}{section}{\S~~~ Chapter Summary}

\nopagebreak\secKTH{Technical Terms}
\label{technical-terms}

\begin{KT}
adapter class

callback design

content pane

containment hierarchy

controller

event model

inner class

layout manager

lightweight  component

listener

model

model-view-controller (MVC)

peer model

pluggable look and feel

view

widget hierarchy

\end{KT}


\secSMHtwo{Summary of Important Points}
\label{summaryof-important-points}
\begin{SMBL}
\item  Java provides two sets of Graphical User Interface
(GUI) components, the Abstract Windowing Toolkit (AWT), which was part
of Java 1.0 and the Swing component set, the
GUI part of the Java Foundation Classes (JFC), introduced in JDK 1.1.

\item  Unlike their AWT counterparts, Swing components are written
entirely in Java. This allows programs written in Swing to have
a platform-independent look and feel. There are three built-in
look-and-feel packages in Swing: a Windows  style, a Unix-like
Motif style, and a purely Java Metal style.

\item  Swing components are based on the {\it model-view-controller (MVC)}
architecture, in which the component is divided into three separate
objects: how it looks ({\it view}), what state it's in ({\it model}),
and what it does ({\it controller}). The view and controller parts are
sometimes combined into a single {\it user interface} class, which can
be changed to create a customized look and feel.

\item  AWT components are based on the {\it peer model}, in which
every AWT component has a peer in the native windowing system. This
model is less efficient and more platform dependent than the MVC model.

\item  Java's {\it event model} is based on {\it event listeners}. When
a GUI component is created, it is registered with an appropriate event
listener, which takes responsibility for handling the component's events.

\item  A user interface combines four functions: guidance/information for the user,
input, output, and control.

\item  Components in a GUI are organized into a {\it containment
hierarchy} that is rooted at the top-level window. {\tt JPanel}s and other
{\tt Container}s can be used to organize the components into a hierarchy
according to function or some other criterion.

\item  The top-level Swing classes---{\tt JApplet}, {\tt JDialog},
{\tt JFrame}, and {\tt JWindow}---use a {\it content pane} as their
component container.

\item  A GUI should minimize the number of input devices the user
needs to manipulate, as well as the complexity the user needs
to deal with. Certain forms of redundancy---such as two independent
but complete sets of controls---are desirable because they make
the interface more flexible and more widely applicable.

\item  A {\it layout manager} is an object that manages the size and
arrangement of the components in a container. The AWT and Swing
provide a number of built-in layouts, including flow, border,
grid, and box layouts.

\item  A {\it radio button} is a toggle button that belongs to a group
in which only one button from the group may be selected at the same
time. A {\it checkbox} is a toggle button that always displays its
state.

\item  A well-designed interface should reduce the chance of
user error and should make it as easy as possible to recover from
errors when they do occur.

\end{SMBL}

%%\secANSHleft
\secANSH
%%\begin{ANSstr}
\begin{ANS}
%\addcontentsline{toc}{section}{\S~~~ Answers to Self-Study Exercises}

%\begin{figure}
\begin{figure}[tb]
\figa{chptr13/p528f1.eps}{A {\tt JButton} has internal model-view-controller
components that interact with each other to produce the button's
overall behavior.}
{fig-p528f1}

\end{figure}
%\end{figure}

\item  How can a button still be considered a component under the MVC model?
This is a good question. The {\tt JButton} class acts as a wrapper
class and hides the model-view-controller details
(Fig.~\ref{fig-p528f1}). When you instantiate a {\tt JButton}, you
still get a single instance.  Think of it this way: Your body consists
of several systems that interact (internally) among themselves, but
it's still one body that other bodies interact with as a single
object. 

\item  A component can indeed be registered with more than one listener.  For
example, the {\tt ToggleButton} that we defined in
Chapter~\ref{chapter-io} has two listeners. The first is the button
itself, which takes care of toggling the button's label. The second is the
frame in which the button is used, which takes care of handling
whatever action the button is associated with.

\item  Some components can have two different kinds of listeners.  For
example, imagine a ``sticky button'' that works like this. When you
click and release the button, it causes some action to take place,
just like a normal button. When you click and hold the mouse button
down, the button ``sticks'' to the cursor and you can then move it to
a new location. This button would need listeners for
{\tt ActionEvent}s, {\tt MouseEvent}s, and {\tt MouseMotionEvent}s.

\item  To round a double you could use the {\tt Math.round()} method.  For
example, suppose the number you want to round is {\tt d}. Then the
expression {\tt Math.round(100 * d)/100.0} will round to two
decimal places. Alternatively, you could use the
{\tt java.text.Num\-ber\-Format} class. Both of these approaches were covered
in Chapter~5.

\item  Many cars today have cruise control as a alternative way to control the
accelerator. Push buttons, usually located on the steering wheel, are
used to speed up and slow down, so you can drive with your foot
or your hand.

\item As an alternative, a north-west-center border layout for the
top-level window in the {\tt Converter} might work. So might
center-south-east and center-south-west. What makes these possible is
the fact that the layout manager will use up space in any edge area
that is not assigned a component.

\item  A flow layout would not be appropriate for the control panel because
you would have little control of where the {\tt convert} button would be
placed relative to the keypad.

\item  Interface design disaster: My car uses the same kind of on/off switch
for the headlights and the windshield wipers. One is a stem on the
left side of the steering wheel, and the other is on a stem on the
right side of the steering wheel. On more than one occasion, I've
managed to turn off the headlights when I intended to turn on the
wipers.

\item  Modify the {\tt addRecentCut()} method so it limits the cuts
stored in the vector to the last ten cuts. Solution: Check
the size of the vector after inserting the cut. If it
exceeds ten, remove the last element in the vector.

\begin{jjjlisting}
\begin{lstlisting}
private void addRecentCut(String cut) {
  recentCuts.insertElementAt(cut, 0);
  if (recentCuts.size() > 10) { // If more than 10 cuts
    recentCuts.removeElementAt(10);  // remove oldest cut
  }
  cutsMenu.removeAll();
  for (int k = 0; k < recentCuts.size(); k++) {
    JMenuItem item = 
      new JMenuItem((String) recentCuts.elementAt(k));
    cutsMenu.add(item);
    item.addActionListener(this);
  }
} // addRecentCut()
\end{lstlisting}
\end{jjjlisting}

\item  Modify the {\tt addRecentCut()} method so that it doesn't duplicate
cuts stored in the vector. Solution: Use the {\tt indexOf()} method
to search for the cut in the vector. If it's already there,
don't insert the cut.

\begin{jjjlisting}
\begin{lstlisting}
private void addRecentCut(String cut) {
  if (recentCuts.indexOf(cut) == -1) {// If not already cut
    recentCuts.insertElementAt(cut,0);
    if (recentCuts.size() > 10) { // If more than 10 cuts
        recentCuts.removeElementAt(10); // remove oldest
    }
    cutsMenu.removeAll();
    for (int k = 0; k < recentCuts.size(); k++) {
        JMenuItem item = 
          new JMenuItem((String) recentCuts.elementAt(k));
        cutsMenu.add(item);
        item.addActionListener(this);
    }
  } // if not already cut
} // addRecentCut()
\end{lstlisting}
\end{jjjlisting}

%%%\end{ANSstr}
\end{ANS}


%\section*{}{Exercises}

\pagebreak
\secEXRHtwo{Exercises}
\label{exercises}

%%\begin{EXRtwostr}
\begin{EXRtwo}

\item  Explain the difference between the following pairs of terms:

\begin{EXRtwoLL}
\item  A {\it model} and a {\it view}.
\item  A {\it view} and a {\it controller}.
\item  A {\it lightweight} and {\it heavyweight} component.
\item  A {\tt JButton} and a {\tt Button}.
\item  A {\it layout manager} and a {\it container}.
\item  A {\it containment hierarchy} and an {\it inheritance hierarchy}.
\item  A {\it content pane} and a {\tt JFrame}.
\end{EXRtwoLL}

\marginnote{\raggedright{\bf Note:} For programming exercises, {\bf first} draw 
a UML class diagram describing all classes and
their inheritance relationships and/or associations.}

\item  Fill in the blanks.

\begin{EXRtwoLL}\baselineskip=14pt
\item  A GUI component that is written entirely in Java is
known as a  \rule{40pt}{0.5pt} component.
\item   The AWT is not platform independent because it
uses the \rule{40pt}{0.5pt} model to implement its GUI components.
\item  The visual elements of a GUI are arranged in a \rule{40pt}{0.5pt}\,.
\item  A \rule{40pt}{0.5pt} is an object that takes responsibility
for arranging the components in a container.
\item  The default layout manager for a {\tt JPanel} is \rule{40pt}{0.5pt}\,.
\item  The default layout manager for a {\tt JApplet} is \rule{40pt}{0.5pt}\,.
\end{EXRtwoLL}\baselineskip=11pt

\item  Describe in general terms what you would have to do
to change the standard look and feel of a Swing {\tt JButton}.

\item  Explain the differences between the model-view-controller
design of a {\tt JButton} and the design of an AWT {\tt Button}.
Why is MVC superior?

\item  Suppose you have a GUI that contains a \mbox{\tt JButton}
and a {\tt JLabel}. Each time the button is clicked, the GUI
rearranges the letters in the label. Using Java's event model as a
basis, explain the sequence of events that happens in order for this
action to take place.

\item Draw a containment hierarchy for the most recent GUI version of the
{\tt OneRowNim} program.

\item  Create a GUI design, similar to the one shown in
Figure~\ref{fig-acmegui}, for a program that would be
used to buy tickets online for a rock concert.

\item  Create a GUI design, similar to the one shown in
Figure~\ref{fig-acmegui}, for an online program that
would be used to play musical recordings.

\item  Design and implement a GUI for the {\tt CDInterest}
program (Fig.~5.18). This program should
let the user input the interest rate, principal, and period
and should accumulate the value of the investment.

\item  Design and implement a GUI for the {\tt Temperature}
class (Fig.~5.5). One challenge of
this design is to find a good way for the user to indicate
whether a Fahrenheit or Celsius value is being input. This
should also determine the order of the conversion: F to C
or C to F.

\item  Design an interface for a 16-button integer calculator that supports
addition, subtraction, multiplication, and division. Implement
the interface so that the label of the button is displayed in the
calculator's display---that is, it doesn't actually do the math.

\item  {\bf Challenge: } Design and implement a {\tt Calculator}
class to go along with the interface you developed in the previous
exercise. It should function the same way as a hand calculator except
it only handles integers.

\item  Modify the {\tt Converter} application so that it can convert
in either direction: from miles to kilometers or from kilometers to
miles. Use radio buttons in your design to let the user select one or
the other alternative.

\item  Here's a design problem for you. A biologist needs an
interactive program that calculates the average of some field data
represented as real numbers. Any real number could be a data value, so
you can't use a sentinel value, such as 9999, to indicate the end of
the input. Design and implement a suitable interface for this problem.

\item  {\bf Challenge:} A dialog box is a window associated
with an application that appears only when needed. Dialog boxes have
many uses. An error dialog is used to report an error message.  A file
dialog is used to help the user search for and open a file. Creating a
basic error dialog is very simple in Swing. The {\tt JOptionPane}
class has class methods that can be used to create the kind of dialog
shown in Figure~\ref{fig-errdialog}. Such a dialog box can be created
with a single statement:

\marginfiglarge{chptr13/9f28.eps}{A basic {\tt J\-Option\-Pane} error dialog.}
{fig-errdialog}

\begin{jjjlisting}
\begin{lstlisting}
JOptionPane.showMessageDialog(this, 
         "Sorry, your number is out of range.");
\end{lstlisting}
\end{jjjlisting}

\noindent Convert the {\tt Validate} program (Fig.~6.12 from
Chapter~6) to a GUI interface and use the {\tt JOptionPane} dialog to
report errors.


\item  {\bf Challenge:} Design and implement a version of the
game {\it Memory}.  In this game you are given a two-dimensional grid
of boxes that contains pairs of matching images or strings. The object
is to find the matching pairs. When you click a box, its contents
are revealed. You then click another box. If its contents match the
first one, their contents are left visible.  If not, the boxes are
closed up again. The user should be able to play multiple games
without getting the same arrangement every time.

\item {\bf Challenge:} Extend the {\tt SimpleTextEditor} program by
adding methods to handle the opening, closing and saving of text
files.

%%\end{EXRtwostr}
\end{EXRtwo}
