%%%  Chapter 16: Data Structures: Lists, Stacks, Queues
%%%  3rd Edition


\setcounter{SSTUDYcount}{1}
\setcounter{chapter}{15}

\chapter{Data Structures: Lists, Stacks, and Queues}
\label{chapter-datastructures}


%%\markboth{{\color{cyan}CHAPTER\,\arabic{chapter}\,\,$\bullet$}\,\, \chaptername} %%Data Structures\,\,}
%%{{\color{cyan}SECTION\,\arabic{section}\,\,$\bullet$}\,\, \sectionname\,\,}

\CObegin
\secCOBH{Objectives}
\noindent After studying this chapter, you will

\begin{COBL}
\item  Understand the concepts of a dynamic data structure and an 
Abstract Data Type (ADT).
\item  Be able to create and use dynamic data structures
such as linked lists and binary search trees.
\item  Understand the stack, queue, set, and map ADTs.
\item  Be able to use inheritance to define extensible data structures.
\item  Know how to use the TreeSet, TreeMap, HashSet, and HashMap 
library classes.
\item  Be able to use the Java generic type construct.
\end{COBL}

\secCOLH{Outline}
\begin{COL}
\item  {Introduction}
\item  {The Linked List Data Structure}
\item  {Object-Oriented Design: The List Abstract Data Type (ADT)}
\item  {The Stack ADT}
\item  {The Queue ADT}
\item[] {{\color{cyan}Special Topic:}  The LISP Language}
\item {From the Java Library: The Java Collections Framework and Generic Types}
\item {Using the {\tt Set} and {\tt Map} interfaces}
\item {The Binary Search Tree Data Structure}
\par\small\item[] {Chapter Summary}
\par\small\item[] {Solutions to Self-Study Exercises}
\par\small\item[] {Exercises}
\end{COL}
\COend


\section{Introduction}
\label{introduction}
\noindent A {\bf data structure}\index{data structure} is used to
organize information that a computer can access and process easily and
efficiently.  You are already familiar with one type of data
structure---arrays, which we discussed in Chapter~9.  If you remember,
an array is an example of a data structure in which all of the data
are of the same type or class and in which individual elements are
accessed by their position (index or subscript). An array is an
example of a {\bf static structure}\index{static structure}, because
its size is fixed for the duration of the program's execution.  (This
is a different meaning of {\it static} than the Java keyword {\tt
static}.)

The {\tt Vector}\index{vector} class from Chapter~9 is another example
of a data structure.  Like an array, individual vector elements are
accessed by their position.  However, unlike arrays, a vector is an
example of a {\bf dynamic structure}\index{dynamic structure}---that
is, one that can grow and shrink during a program's execution.

These are only two of the many data structures developed by computer
scientists.   For more advanced problems, it is often necessary to
develop specialized structures to store and manipulate
information.  Some of these structures---linked lists, stacks,
queues, binary trees, hash tables---have become classic objects
of study in computer science.

This chapter describes how to implement a linked list and how to use
inheritance to extend the list to implement the stack and queue
structures. Then the Java Collections Framework implementation of
numerous data structures in the {\tt java.util} package will be
described.  The data structure classes in this library make use of a
new Java construct, called {\em generic types}.  Finally, the binary
tree data structure that is used in the Java Collections Framework
will be studied briefly.

\section{The Linked List Data Structure}
\label{the-linked-list-data-structure}
\noindent As we said, a {\it static} data structure is one whose size is fixed
\marginnote{Static vs.  dynamic}
during a program's execution---a static structure's memory is allocated at compile
time.  By contrast, a {\it dynamic} structure is one that can grow and
shrink as needed.  In this section, we will develop a dynamic {\bf
list}\index{list}, which is a data structure whose elements are
arranged in a linear sequence.  There is a first element in the list,
a second element, and so on.  Lists are quite general and, as we will
discuss later, lists have a broad range of applications.  Depending on
how elements are inserted and removed from a list, they can be used
for a range of specialized purposes.

\subsection{Using References to Link Objects}
\noindent As you know from earlier chapters, when you create an 
object using the
\marginnote{Referring to objects}
{\tt new} operator you get back a {\em reference} to the object
that you then can assign to a reference variable.  In the following
example, {\it b} is a reference to a {\tt JButton}:

\begin{jjjlisting}
\begin{lstlisting}
JButton b = new JButton();
\end{lstlisting}
\end{jjjlisting}

\noindent We have defined many classes that contained
references to other objects:

\begin{jjjlisting}
\begin{lstlisting}
public class Student {
     private String name;
}
\end{lstlisting}
\end{jjjlisting}

\noindent In this example, {\tt name} is a reference to a {\tt String}
object.


%\begin{figure}
\begin{figure}[tb]
\figa{chptr16/linkedlist.eps}{A linked list of {\tt Node}s terminated
by a {\tt null} link.}
{fig-linkedlist}
\end{figure}
%\end{figure}

A {\bf linked list}\index{linked list} is a list in which a 
collection of nodes are
linked together by references from one node to the next.  To make a
linked list, we will define a class of self-referential
\marginnote{Self-referential objects}
objects.  A {\bf self-referential object}\index{self-referential
object} is an object that contains a reference to an object of the
same class.  The convention is to name these objects {\tt Node}s:

\begin{jjjlisting}
\begin{lstlisting}
public class Node {
     private String name;
     private Node next;
}
\end{lstlisting}
\end{jjjlisting}

\marginfigvspace{10pt}{chptr16/nodeuml.eps}{The {\tt Node} class.
} {fig-nodeuml}
\noindent In addition to the reference to a {\tt String} object, each
{\tt Node} object contains a reference to another {\tt Node} object.
The {\tt next} variable is often called a {\bf link}\index{link}
because it is used to link together two {\tt Node} objects.  For
example, Figure~\ref{fig-linkedlist} provides an illustration of a
linked list of {\tt Node}s.

By assigning references to the {\tt next} variables in each {\tt
Node}, we can chain together arbitrarily long lists of objects.
Therefore, we will want to add methods to our {\tt Node} class that
enable us to manipulate a {\tt Node}'s next variable
(Fig.~16--2).  By assigning it a reference to another {\tt
Node}, we can link two {\tt Node}s together.  By retrieving the
link's value, we can find the next {\tt Node} in the list.

\JavaRule{Self-Referential Object.}{A {\it
self-referential object} is one that contains an instance variable
that refers to an object of the same class.}

\noindent In addition to the link variable, each {\tt Node} stores
some data.  In this example, the data is a single {\tt String}. But
there's no real limit to the amount and type of data that can be
\marginnote{Linking objects together}
stored in a linked list.  Therefore, in addition to methods that
manipulate a {\tt Node}'s link, we will also want methods to
manipulate its data.  These points suggest the following basic design
for a {\tt Node}:


\begin{jjjlisting}
\begin{lstlisting}
public class Node {
     private Object data;
     private Node next;

     public Node(Object obj);          // Constructor

     public void setData(Object obj);  // Data access
     public Object getData();

     public void setNext(Node link);   // Link access
     public Node getNext();
} // Node
\end{lstlisting}
\end{jjjlisting}


\noindent Note that we have defined the {\tt Node}'s data in the
most general possible way: As a reference to an {\tt Object}.  Because
the {\tt Object} class is the root of Java's entire class hierarchy,
an {\tt Object} can encompass any kind of data.  By using Java's
wrapper classes, such as {\tt Integer} and {\tt Double}, a
{\tt Node}'s data can even include primitive data.

The important point is that regardless of its type of data, a
{\tt Node} will have data access methods and link access methods.  The data
%\begin{marginalnote}\it Divide and conquer\end{marginalnote}
\marginnote{Divide and conquer}
access methods differ, depending on the type of data, but the link
access methods will generally be the same.

\JavaTIP{EFFECTIVE DESIGN}{Link Versus Data.}{Making a clear distinction
between an object's data and those elements used to manipulate the
object is an example of the divide-and-conquer principle.}

\secEXRHone{Self-Study Exercises}
\begin{SSTUDY}
\item  Write a statement to create a new {\tt Node} whose data
consist of the {\tt String} ``Hello.''

\item  Write a statement to create a new {\tt Node} whose data
consist of the {\tt Student} named ``William.''  Assume that the {\tt Student}
class has a constructor with a {\tt String} parameter for the student's name.
\end{SSTUDY}

\subsection{Example: The Dynamic Phone List}


\noindent Let's define a {\tt PhoneListNode} class that can be used to
implement a phone list (Fig.~\ref{fig-phoneuml}).  This definition
will be a straightforward specialization of the generic {\tt Node}
list defined in the previous section.  Each element of the phone list
will consist of a person's name and phone number.  These will be the
node's data and can be stored in two {\tt String} variables.  To
access these data, we will provide a
%\begin{marginalnote}\it Accessing a list's data\end{marginalnote}
\marginnote{Accessing a list's data}
constructor and a basic set of access methods.  Thus, we have the
definition shown in Figure~\ref{fig-phonelistnode}.

\begin{figure}[t]
\figaright{chptr16/phoneuml.eps}{Design of the {\tt PhoneListNode} class.
} {fig-phoneuml}
\end{figure}

\begin{figure}[h!]
\jjjprogstart
\begin{jjjlisting}
\begin{lstlisting}
public class PhoneListNode {
     private String name;
     private String phone;
     private PhoneListNode next;

     public PhoneListNode(String s1, String s2) {
         name = s1;
         phone = s2;
         next = null;
     } // PhoneListNode()

     public void setData(String s1, String s2) {
         name = s1;
         phone = s2;
     } // setData()

     public String getName() {
         return name;
     } // getName()

     public String getData() {
         return name + " " + phone;
     } // getData()

     public String toString() {
         return name + " " + phone;
     } // toString()

     public void setNext(PhoneListNode nextPtr) {
         next = nextPtr;
     } // setNext()

     public PhoneListNode getNext() {
         return next;
     } // getNext()
} // PhoneListNode
\end{lstlisting}
\end{jjjlisting}
\jjjprogstop{The {\tt PhoneListNode} class.
}{fig-phonelistnode}
\end{figure}

\clearpage
The constructor and data access methods should be familiar to
you.  Note that the constructor sets the initial value of {\tt next} to
{\tt null}, which means that it refers to no object.

\JavaTIP{DEBUGGING TIP}{Null Reference.}{A common programming error
is the attempt to use a {\tt null} reference to refer to an object.
This usually means the reference has not been successfully initialized.}

\noindent Let's discuss the details of the link access methods---the
{\tt setNext()} and {\tt getNext()} methods---which are also simple
to implement.   Because this is a {\tt PhoneListNode}, these methods
%\begin{marginalnote}\it Manipulating a list's nodes\end{marginalnote}
\marginnote{Manipulating a list's nodes}
take {\tt PhoneListNode} as a parameter and return type, respectively.
Given a reference to a {\tt PhoneListNode}, the {\tt setNext()} method
assigns it to {\tt next}.  The {\tt getNext()} method simply
returns the value of its {\tt next} link.

Let's now see how we would use these methods to construct a list.
The following statements create three nodes:

\begin{jjjlisting}
\begin{lstlisting}
PhoneListNode node1 = 
       new PhoneListNode("Roger M","090-997-2918");
PhoneListNode node2 = 
       new PhoneListNode("Jane M","090-997-1987");
PhoneListNode node3 = 
       new PhoneListNode("Stacy K","090-997-9188");
\end{lstlisting}
\end{jjjlisting}

\noindent The next two statements chain the nodes together into
the list shown in Figure~16.5:

\begin{jjjlisting}
\begin{lstlisting}
node1.setNext(node2);
node2.setNext(node3);
\end{lstlisting}
\end{jjjlisting}

\noindent If we wanted to add a fourth node to the end of this
\marginfig{chptr16/phonelistpic.eps}{The phone list: a linked list of nodes,
each of which contains a person's name and phone number.}
{fig-phonelist}
list, we could use the following statements:

\begin{jjjlisting}
\begin{lstlisting}
PhoneListNode node4 =
     new PhoneListNode("gary g","201-119-8765");
node3.setNext(node4);
\end{lstlisting}
\end{jjjlisting}

\noindent Although this example illustrates the basic technique
for inserting nodes at the end of the list, it depends too much on our
knowledge of the list.  In order to be truly useful we will have to
develop a more general set of methods to create and manipulate a list
of nodes. As we will see, a better design would be able to find
the end of the list without knowing anything about the list's data.


\JavaTIP{EFFECTIVE DESIGN}{Generality.}{In a well-designed list data
structure, you should be able to manipulate its elements without
knowing anything about its data.}


\pagebreak
\secEXRHone{Self-Study Exercise}
\begin{SSTUDY}

\item Suppose you know that {\tt nodeptr} is
a reference to the last element of a linked list of {\tt
PhoneList\-Node}s.  Create a new element for ``Bill C'' with phone
number ``111-202-3331'' and link it into the end of the list.
\end{SSTUDY}

\subsection{Manipulating the Phone List}

\noindent In addition to the {\tt Node}s that make a list, we must define a
class containing methods to manipulate the list.  This class will
include the insert, access, and remove methods.  It must also contain
a reference to the list itself.  This leads to the basic design shown
in Figure~16.6. Because this is a list of {\tt
PhoneListNode}s, we need a {\tt PhoneListNode} reference to point to
%\begin{figure}
%\begin{graphic}
\marginfigvspace{-120pt}{chptr16/phonelistclass.eps}{The {\tt PhoneList} class
has a reference to the first node of the list ({\tt head}) and methods to
insert, remove, and look up information.}
{fig-phonelistclass}
%\end{graphic}
%\end{figure}
the list, which is the purpose of the {\tt head} variable.


A preliminary coding of the {\tt PhoneList} class is shown in
Figure~\ref{fig-phonelistprelim}.  As you can see there, when a new
{\tt PhoneList} instance is constructed, {\tt head} is initialized to
{\tt null}, meaning the list is initially empty.
%\begin{marginalnote}\it An empty list\end{marginalnote}
\marginnote{An empty list}
Since we will frequently want to test whether the list is empty, we
define the {\tt boolean} {\tt isEmpty()} method for that purpose.  As
you can see, its definition says that a list is empty when the
reference to the {\tt head} of this list is {\tt null}.

\JavaTIP{PROGRAMMING TIP}{The {\tt null} Reference.}{A {\tt null}
reference is useful for defining limit cases, such as an empty list or
an uninstantiated object.}

\begin{figure}[hbt]
\jjjprogstart
\begin{jjjlisting}
\begin{lstlisting}
public class PhoneList {
    private PhoneListNode head;

    public PhoneList() {
        head = null;         // Start with empty list
    }
    public boolean isEmpty() {  // Defines an empty list
        return head == null;
    }
    public void insert(PhoneListNode node) { }
    public String getPhone(String name) { }
    public String remove(String name) { }
    public void print() { }
} // PhoneList
\end{lstlisting}
\end{jjjlisting}
\jjjprogstop{A preliminary version of the {\tt PhoneList} class.
}{fig-phonelistprelim}
\end{figure}

\subsection*{Inserting Nodes into a List}

\noindent The {\tt insert()} method will have the task of inserting new
{\tt Phone\-List\-Node}s into the list.  There are a number of ways to do
this.  The node could be inserted at the beginning or at
the end of the list, or in alphabetical order, or possibly in other
ways.  As we'll see, it is easiest to insert a new node at the
head of the list.   But for this example, let's develop a method that
inserts the node at the end  of the list.

%\begin{figure}
\begin{figure}[tb]
\figaleft{chptr16/insertions.eps}{Two cases.  (a) The list is empty before
the insertion, which takes place at {\tt head}. (b) The list is not empty,
so the insertion takes place at the end of the list.
} {fig-insertions}
\end{figure}
%\end{figure}

There are two cases we need to worry about for this algorithm.   First,
\marginnote{Insertion algorithm} 
if the list is empty, we can insert the node by simply setting {\tt
head} to point to the node [Figure~\ref{fig-insertions}(a)].  Second,
if the list is not empty, we must move through, or {\em
traverse}\index{traverse}, the links of the list until we find the
last node and insert the new node after it
[Figure~\ref{fig-insertions}(b)]. In this case, we want to set the
{\tt next} variable of the last node to point to the new node.  This
gives us the following algorithm:

\begin{jjjlisting}
\begin{lstlisting}
public void insert(PhoneListNode newNode) {
  if (isEmpty())
    head = newNode;        // Insert at head of list
  else {
    PhoneListNode current = head; // Start traversal at head
    while (current.getNext() != null)// While not last node
      current = current.getNext(); //   go to next node
      current.setNext( newNode );  // Do the insertion
    }
} // insert()
\end{lstlisting}
\end{jjjlisting}

\noindent Recall that when nodes are linked, their {\tt next}
variables are non-{\tt null}. So when a node's {\tt next} variable is
{\tt null}, that indicates the end of the list---there's no next
node.   Thus, our algorithm begins by checking if the list is empty.  If
so, we assign {\tt head} the reference to {\tt newNode}, the
{\tt PhoneListNode} that's being inserted.

If the list is not empty, then we need to find the last node.  In order
%\begin{marginalnote}\it Traversing a list\end{marginalnote}
\marginnote{Traversing a list}
to traverse the list, we will need a temporary variable,
{\tt current}, which will always point to the current node.   It's important
to understand the while loop used here:

\begin{jjjlisting}
\begin{lstlisting}
PhoneListNode current = head;       // Initializer
while (current.getNext() != null)   // Entry condition
     current = current.getNext();    // Updater
\end{lstlisting}
\end{jjjlisting}



\noindent The loop variable, {\tt current}, is initialized by
setting it to point to the head of the list.  The entry condition
tests whether the {\tt next} link, leading out of {\tt current}, is
{\tt null} (Fig.~\ref{fig-inserting}). That is, when the link coming
out of a node is {\tt null}, then that node is the last node in the
list [Figure~\ref{fig-inserting}(c)]. Inside the while loop, the
update expression simply assigns the next node to {\tt current}. In
that way, {\tt current} will point to each successive node until the
last node is found.  It's very important that the loop exits when {\tt
current.getNext()} is {\tt null}---that is, when the next pointer of
the current node is null.  That way {\tt current} is pointing to
%\begin{marginalnote}\it Loop exit condition\end{marginalnote}
\marginnote{Loop-exit condition}
the last node and can be used to set its {\tt next} variable to the
node being inserted [Figure~\ref{fig-inserting}(d)].  Thus, after the
loop is exited, {\tt current} still points to the last node.  At that
point, the {\tt setNext()} method is used to link {\tt newNode} into
the list as the new last node.

\begin{figure}[tbh]
\figa{chptr16/inserting.eps}{The temporary variable {\tt current} is
used to traverse the list to find its end.}
{fig-inserting}
\end{figure}

\JavaTIP{DEBUGGING TIP}{List Traversal.}{A common error in designing
list-traversal algorithms is an erroneous loop-entry or loop-exit
condition.  One way to avoid this error is to hand trace your
algorithm to make sure your code is correct.}

\subsection*{Printing the Nodes of a List}

\noindent The {\tt print()} method also uses a traversal strategy to print the
%\begin{marginalnote}\it List traversal\end{marginalnote}
\marginnote{List traversal}
data from each node of the list.  Here again it is necessary to test
whether the list is empty.  If so, we must print an error message.
(This would be a good place to throw a programmer-defined exception,
such as an {\tt EmptyList\-Exception}.)  If the list is not empty, then
we use a temporary variable to traverse the list, printing
each node's data along the way:

\begin{jjjlisting}
\begin{lstlisting}
public void print() {
  if (isEmpty())
     System.out.println("Phone list is empty");
  PhoneListNode current = head; // Start traversal at head
  while (current != null) {      // While not end of list
    System.out.println( current.toString() ); // print data
    current = current.getNext();       //  go to next node
  }
} // print()
\end{lstlisting}
\end{jjjlisting}

\noindent Note the differences between this while loop and
the one used in the {\tt insert()} method.  In this case, we exit the
loop when {\tt current} becomes {\tt null}; there's no action to be
taken after the loop is exited.  The printing takes place within the
loop.  Thus, in this case, the entry condition, {\tt (current != null)},
signifies that the task has been completed.

\JavaTIP{PROGRAMMING TIP}{Terminating a Traversal.}{In designing
list-traversal algorithms where the reference, {\it p}, points to the
nodes in the list, if you need to refer to the last node in the list
after the traversal loop exits, then your exit condition should be
{\tt p.getNext() == null}.  If you have finished processing the nodes
when the loop exits, your exit condition should be {\tt p == null}.}

\subsection*{Looking up a Node in a List}

\noindent Because the record associated with a person can be located
anywhere in the list, the traversal strategy must also be used to look
\marginnote{List traversal}
up someone's phone number in the {\tt PhoneList}. Here again we start
at the {\tt head} of the list and traverse through the {\tt next} links
until we find the node containing the desired phone number.  This
method takes the name of the person as a parameter.  There are three
cases to worry about: (1) The list is empty; (2) the normal case where
the person named is found in the list; and (3) the person named is not
in the list.  Because the method returns a {\tt String}, we can return
error messages in the first and third cases:

\begin{jjjlisting}
\begin{lstlisting}
public String getPhone(String name) {
  if (isEmpty())                    // Case 1: Empty list
    return "Phone list is empty";
  else {
    PhoneListNode current = head;
    while ((current.getNext() != null) &&
                 (!current.getName().equals(name)))
      current = current.getNext();
      if (current.getName().equals(name))
        return current.getData(); // Case 2: Found name
      else                        // Case 3: No such person
        return ("Sorry.  No entry for " + name);
   }
} // getPhone()
\end{lstlisting}
\end{jjjlisting}

\noindent Note the while loop in this case.   As in the {\tt insert()}
method, when the loop exits, we need a reference to the {\tt current}
node so that we can print its phone number [{\tt current.getData()}].  But
here there are three ways to exit the loop: (1) We reach the end of
the list without finding the named person; (2) we find the named
person in the interior of the list; or (3) we find the named person
%\begin{marginalnote}\it Compound exit condition\end{marginalnote}
\marginnote{Compound exit condition}
in the last node of the list.  In any case, it is necessary to test
whether the name was found or not after the loop is exited.  Then
appropriate action can be taken.

\secEXRHone{Self-Study Exercise}
\begin{SSTUDY}

\item  What if the exit condition for the while loop
in {\tt getPhone()} were stated as
\begin{jjjlisting}
\begin{lstlisting}
((current.getNext() != null) ||
     (!current.getName().equals(name)))
\end{lstlisting}
\end{jjjlisting}
\end{SSTUDY}

\subsection*{Removing a Node from a List}

\noindent By far the most difficult task is that of removing a node from a
%\begin{marginalnote}\it Node removal algorithm\end{marginalnote}
\marginnote{Node-removal algorithm}
list.  In the {\tt PhoneList} we use the person's name to identify the
node, and we return a {\tt String} that can be used to report either
success or failure.   There are four cases to worry about in designing
this algorithm: (1) The list is empty, (2) the first node is being
removed, (3) some other node is being removed, and (4) the named person is
not in the list.   The same traversal strategy we used in
{\tt getPhone()} is used here, with the same basic while loop for cases 3
and 4.


As Figure~\ref{fig-remove} shows, the first two cases are
easily handled.  If the list is empty, we just return an error message.  We
\begin{figure}[h!]
\jjjprogstart
\begin{jjjlistingleft}[32pc]{-6pc}
\begin{lstlisting}
public String remove(String name) { // Remove an entry by name
  if (isEmpty())                         // Case 1: empty list
    return "Phone list is empty";
  PhoneListNode current = head;
  PhoneListNode previous = null;
  if (current.getName().equals(name)) { // Case 2: remove first node
    head = current.getNext();
    return "Removed " + current.toString() ;
  }
  while ((current.getNext() != null) && 
                              (!current.getName().equals(name)))  {
    previous = current;
    current = current.getNext();
  }
  if (current.getName().equals(name)) { // Case 3: remove named node
    previous.setNext(current.getNext());
    return "Removed " + current.toString();
  } else
    return ("Sorry.  No entry for " + name); // Case 4: node not found
} // remove()
\end{lstlisting}
\end{jjjlistingleft}
\jjjprogstop{The {\tt remove()} method.
}{fig-remove}
\end{figure}
use {\tt current} as the traversal variable.  If the named node is the
first node, we simply need to set {\tt head} to
{\tt current.getNext()}, which has the effect of making {\tt head} point to
the second node in the list [Figure~\ref{fig-removal}(a)]. Once the node
is cut out from the chain of links, there will be no further reference
to it.  In this case, Java will recapture the memory it uses when it
does garbage collection.

\JavaRule[false]{Garbage Collection.}{Java's garbage collector handles the
disposal of unused objects automatically.  This helps to simplify
linked-list applications.  In languages such as C++, the programmer
would have to {\it dispose} of the memory occupied by the deleted
node.}

\begin{figure}[tb]
\figaright{chptr16/removal.eps}{Removing different nodes from a linked list.
} {fig-removal}
\end{figure}

\noindent In order to remove some other node besides the first, two traversal
variables are needed: {\tt previous} and {\tt current}.  They
proceed together down the list, with {\tt previous} always pointing to
\marginnote{Tandem traversal}
the node just before the {\tt current} node.  The reason, of course, is
that to remove the {\tt current} node, you need to adjust the link
pointing to it contained in the \mbox{\tt previous} node
[Figure~\ref{fig-removal}(b)].  That is, the new value of
{\tt previous.next} will be the current value of {\tt current.next}.
We use the {\tt getNext()} and \mbox{\tt setNext()} methods to
effect this change:

\begin{jjjlisting}
\begin{lstlisting}
previous.setNext(current.getNext());
\end{lstlisting}
\end{jjjlisting}

\pagebreak
\subsection*{Testing the List}

\noindent In developing list-processing programs, it is important to
design good test data.  As we have seen, both the insertion and
removal operations involve several distinct cases.  Proper testing of
\marginnote{Designing test data}
these methods ideally would test every possible case.  
The {\tt main()} program in Figure~\ref{fig-testlist} illustrates the
kinds of tests that should be performed.  This method could be
incorporated directly into the {\tt PhoneList} class, or it could be
made part of a separate class.

\begin{figure}[h!]
\jjjprogstart
\begin{jjjlisting}
\begin{lstlisting}
public static void main(String argv[]) {
                  // Create list and insert nodes
  PhoneList list = new PhoneList();
  list.insert( new PhoneListNode("Roger M", "997-0020"));
  list.insert( new PhoneListNode("Roger W", "997-0086"));
  list.insert( new PhoneListNode("Rich P", "997-0010"));
  list.insert( new PhoneListNode("Jane M", "997-2101"));
  list.insert( new PhoneListNode("Stacy K", "997-2517"));
                  // Test whether insertions worked
  System.out.println( "Phone Directory" );
  list.print();
                  // Test whether lookups work
  System.out.println("Looking up numbers by name");
  System.out.println(list.getPhone("Roger M"));
  System.out.println(list.getPhone("Rich P"));
  System.out.println(list.getPhone("Stacy K"));
  System.out.println(list.getPhone("Mary P"));
  System.out.println(list.remove("Rich P"));

  System.out.println("Phone Directory");
  list.print();
      // Test removals, printing list after each removal
  System.out.println(list.remove("Roger M"));
  System.out.println("Phone Directory");
  list.print();
  System.out.println(list.remove("Stacy K"));
  System.out.println("Phone Directory");
  list.print();
  System.out.println(list.remove("Jane M"));
  System.out.println("Phone Directory");
  list.print();
  System.out.println(list.remove("Jane M"));
  System.out.println("Phone Directory");
  list.print();
  System.out.println(list.remove("Roger W"));
  System.out.println("Phone Directory");
  list.print();
  System.out.println(list.remove("Roger W"));
  System.out.println("Phone Directory");
  list.print();
} // main()
\end{lstlisting}
\end{jjjlisting}
\jjjprogstop{A {\tt main()} method containing a set of tests for the
{\tt PhoneList} class.
}{fig-testlist}
\end{figure}
\noindent Of course, there are often so many combinations of list operations
that exhaustive testing might not be feasible.  At the very least you
should design test data that test each of the different conditions
identified in your algorithms.  For example, in testing removals from
a list, you should test all four cases that we discussed.  In testing
insertions or lookups, you should test all three cases that we
identified.

\JavaTIP{EFFECTIVE DESIGN}{Test Data.}{Test data for validating
list-processing algorithms should (at least) test each of the cases
identified in each of the removal and insertion methods.}

\secEXRHone{Self-Study Exercises}
\begin{SSTUDY}

\item  Trace through the {\tt main()} 
method line by line and predict its
output.

\item  Design a test of {\tt PhoneList} that shows that
new elements can be inserted into a list after
some or all of its previous nodes have been removed.
\end{SSTUDY}

\section{OBJECT-ORIENTED DESIGN:  \\The List Abstract Data Type (ADT)}
\label{objectorienteddesign-the-list-abstract-data-type-adt}
\markright{{\color{cyan}SECTION\,\,\arabic{section}\,\,$\bullet$}\,\,OOD: The List ADT}

\noindent The {\tt PhoneList} example from the previous section
illustrates the basic concepts of the linked list.  Keep in mind that
there are other implementations that could have been described.  For
example, some linked lists use a reference to both the first and last
elements of the list.  Some lists use nodes that have two pointers,
one to the next node and one to the previous node.  This enables
traversals in two directions---front to back and back to front---as
well as making it easier to remove nodes from the list.  The example
we showed was intended mainly to illustrate the basic techniques
involved in list processing.

Also, the {\tt PhoneList} example is limited to a particular type of
%\begin{marginalnote}\it A generic list structure\end{marginalnote}
\marginnote{A generic list structure}
data---namely, a {\tt PhoneListNode}. Let's develop a more general
linked list class and a more general node class that can be used to
store and process lists of any kind of data.

An {\bf Abstract Data Type (ADT)}\index{Abstract Data Type (ADT)} 
involves two components: the data
that are being stored and manipulated and the methods and operations
that can be performed on those data.  For example, an {\tt int} is an
ADT.~The data are the integers ranging from some
{\tt MININT} to some {\tt MAXINT}. The operations are the various integer
operations: addition, subtraction, multiplication, and division.
These operations prescribe the ways that {\tt int}s can be used.  There
are no other ways to manipulate integers.

Moreover, in designing an ADT, it's important to hide the
implementation of the operations from the users of the operations.
%\begin{marginalnote}\it Information hiding\end{marginalnote}
\marginnote{Information hiding}
Thus, our programs have used all of these integer operations on {\tt
int}s, but we have no real idea how they are implemented---that is,
what exact algorithm they use.

Objects can be designed as ADTs, because we can easily
distinguish an object's use from its implementation.  Thus, the
{\tt private} parts of an object---its instance variables and private
methods---are hidden from the user while the object's interface---its 
{\tt public} methods---are available.  As with the integer
operators, the object's public methods prescribe just how the object
can be used.

So let's design a list ADT. We want it to be able to store any kind of
%\begin{marginalnote}\it Design specifications\end{marginalnote}
data, and we want to prescribe the operations that can be performed on
those data---the insert, delete, and so on.  Also, we want to design
\marginnote{Design specifications}
the ADT so that it can be extended to create more specialized
kinds of lists.

\subsection*{The {\tt Node} Class}

\JavaTIP{EFFECTIVE DESIGN}{Generalizing a Type.}{An effective strategy
for designing a list abstract data type is to start with a specific
list and generalize it.  The result should be a more abstract version
of the original list.}

\noindent Our approach will be to generalize the classes we created in the {\tt
Phone\-List} example.  Thus, the {\tt PhoneListNode} will become a
generic {\tt Node} that can store any kind of data
(Fig.~16--13).  Some of the changes are merely name changes.
Thus, wherever we had {\tt PhoneListNode}, we now have just {\tt Node}.
The link access methods have not changed significantly.  What has
%\begin{figure}
%\begin{graphic}
\marginfig{chptr16/listnode.eps}{The {\tt Node} class is a generalization of
the {\tt PhoneListNode} class.}
{fig-listnode}
%\end{graphic}
%\end{figure}
changed is that instead of instance variables for the name, phone
number, and so on, we now have just a single data reference to an {\tt
Object}. This is as general as you can get, because, as we pointed out
earlier, {\tt data} can refer to any object, even to
primitive data.

The implementation of the {\tt Node} class is shown in
Figure~\ref{fig-nodeclass}.  Note that the data access methods, {\tt
\begin{figure}[h!]
\jjjprogstart
\begin{jjjlisting}
\begin{lstlisting}
public class Node {
   private Object data;        // Stores any kind of data
   private Node next;

   public Node(Object obj) {  // Constructor
      data = obj;
      next = null;
   }                               // Data access methods
   public void setData(Object obj) {
      data = obj;
   }
   public Object getData() {
      return data;
   }
   public String toString() {
      return data.toString();
   }                               // Link access methods
   public void setNext( Node nextPtr ) {
      next = nextPtr;
   }
   public Node getNext() {
      return next;
   }
} // Node
\end{lstlisting}
\end{jjjlisting}
\jjjprogstop{The {\tt Node} class is a more abstract version of
the {\tt Phone- ListNode} class.
}{fig-nodeclass}
\end{figure}
getData()} and {\tt setData()}, use references to {\tt Object} for
their parameter and return type.  Note also how we've defined the {\tt
toString()} method.  It just invokes {\tt data.toString()}. Because
{\tt toString()} is defined in {\tt Object}, every type of data will
have this method.  And because {\tt toString()} is frequently
overridden in defining new objects, it is useful here.


\subsection*{The {\tt List} Class}


\noindent Let's now generalize the {\tt PhoneList} class
(Fig.~16--15).  The {\tt List} class will still contain a
reference to the {\tt head} of the list, which will now be a list of
{\tt Node}s.  It will still define its constructor, its
%\begin{figure}
%\begin{graphic}
\marginfigvspace{-90pt}{chptr16/listuml.eps}{The {\tt List} class 
contains a pointer to the
head of the list and public methods to insert and remove
objects from both the front and rear of the list.
} {fig-listuml}
%\end{graphic}
%\end{figure}
{\tt isEmpty()} method, and its {\tt print()} method in the same way as in the
{\tt PhoneList}.

However, in designing a generic {\tt List} class, we want to design
%\begin{marginalnote}\it Generic list methods\end{marginalnote}
%¥¥¥\marginnote{Generic list methods}
some new methods, particularly because we want to use this class as the
basis for more specialized lists.  The {\tt PhoneList.insert()} method
was used to insert nodes at the end of a list.  In addition to this
method, let's design a method that inserts at the head of the
list.  Also, {\tt PhoneList} had a method to remove nodes by
name.  However, now that we have generalized our data, we don't know if
the list's {\tt Object}s have a name field, so we'll scrap this method
in favor of two new methods that remove a node from the beginning or
end of the list, respectively.

\begin{figure}[p!]
\jjjprogstart
\begin{jjjlisting}[32pc]
\begin{lstlisting}
public class List {
  private Node head;
  public List() { head = null; }
  public boolean isEmpty() { return head == null; }
  public void print() {
    if (isEmpty())
      System.out.println("List is empty");
    Node current = head;
    while (current != null) {
      System.out.println(current.toString());
      current = current.getNext();
    }
  } // print()
  public void insertAtFront(Object obj) {
    Node newnode =  new Node(obj);
    newnode.setNext(head);
    head = newnode;
  }
  public void insertAtRear(Object obj) {
    if (isEmpty())
      head = new Node(obj);
    else {
      Node current = head;              // Start at head of list
      while (current.getNext() != null) // Find the end of the list
        current = current.getNext();
        current.setNext(new Node(obj));  // Create and insert newNode
      }
  } // insertAtRear()
  public Object removeFirst() {
    if (isEmpty())              // Empty List
       return null;
    Node first = head;
    head = head.getNext();
    return first.getData();
  } // removeFirst()
  public Object removeLast() {
    if (isEmpty())  // empty list
       return null;
    Node current = head;
    if (current.getNext() == null) {// Singleton list
       head = null;
       return current.getData();
    }
    Node previous = null;           // All other cases
    while (current.getNext() != null) {
      previous = current;
      current = current.getNext();
    }
    previous.setNext(null);
    return current.getData();
  } // removeLast()
} // List
\end{lstlisting}
\end{jjjlisting}
\jjjprogstop{The {\tt List} ADT.
}{fig-listadt}
\end{figure}

We already know the basic strategies for implementing these
new methods, which are shown in the definition in
Figure~\ref{fig-listadt}.
We have renamed the {\tt insertAtRear()} method,
which otherwise is very similar to the {\tt PhoneList.\-insert()}
method.  The key change is that now its parameter must be an
{\tt Object}, because we want to be able to insert any kind of object into
our list.  At the same time, our list consists of {\tt Node}s, so we
have to use the {\tt Object} to create a {\tt Node} in our insert
methods:

\begin{jjjlisting}
\begin{lstlisting}
head = new Node(obj);
\end{lstlisting}
\end{jjjlisting}

\noindent Recall that the {\tt Node} constructor takes an {\tt Object}
argument and simply assigns it to the {\tt data} reference.  So when we
insert an {\tt Object} into the list, we make a new {\tt Node} and set
its {\tt data} variable to point to that {\tt Object}.  Note that
we check whether the list is empty {\it before} traversing to the last node.

The new {\tt insertAtFront()} method (Fig.~\ref{fig-listadt}) is
simple to implement, since no traversal of the list is necessary.  You
just need to create a new {\tt Node} with the {\tt Object} as its
data element and then link the new node into the head of the list:

\begin{jjjlisting}
\begin{lstlisting}
Node newnode = new Node(obj);
newnode.setNext(head);
head = newnode;
\end{lstlisting}
\end{jjjlisting}

\noindent See Figure~\ref{fig-insertions}a for a graphical 
representation of this
type of insertion.

The new {\tt removeFirst()} method is also quite simple to implement.
In this case, you want to return a reference to the {\tt Object} that's
stored in the first node, but you need to adjust {\tt head} so that it
points to whatever the previous {\tt head.next} was pointing to before
the removal.  This requires the use of a temporary variable, as shown
in the method.

The new {\tt removeLast()} method is a bit more complicated.  It
handles three cases: (1) The empty list case, (2) the single node
list, and (3) all other lists.  If the list is empty, it returns {\tt
null}. Obviously, it shouldn't even be called in this case.  In
designing subclasses of {\tt List} we will first invoke {\tt
isEmpty()} before attempting to remove a node.

If the list contains a single node, we treat it as a special case and
set {\tt head} to {\tt null}, thus resulting in an empty list.  In the
typical case, case 3, we traverse the list to find the last node,
again using the strategy of maintaining both a {\tt previous} and a
{\tt current} pointer.  When we find the last node, we must adjust
{\tt previous.next} so that it no longer points to~it.

\subsection*{Testing the List ADT}

\noindent Testing the list ADT follows the same strategy used in the
{\tt PhoneList} example.  However, one of the things we want to test is that
%\begin{marginalnote}\it Heterogeneous lists\end{marginalnote}
\marginnote{Heterogeneous lists}
we can indeed create lists of heterogeneous types---lists that include
{\tt Integer}s mixed with {\tt Float}s, mixed with other types of
objects.  The {\tt main()} method in Figure~\ref{fig-testlistadt}
illustrates this feature.

\begin{figure}[bth]
%%
\jjjprogstart
\begin{jjjlisting}[27pc]
\begin{lstlisting}
public static void main( String argv[] ) {
         // Create list and insert heterogeneous nodes
  List list = new List();
  list.insertAtFront(new PhoneRecord("Roger M", "997-0020"));
  list.insertAtFront(new Integer(8647));
  list.insertAtFront(new String("Hello, World!"));
  list.insertAtRear(new PhoneRecord("Jane M", "997-2101"));
  list.insertAtRear(new PhoneRecord("Stacy K", "997-2517"));

         // Print the list
  System.out.println("Generic List");
  list.print();
         // Remove objects and print resulting list
  Object o;
  o = list.removeLast();
  System.out.println(" Removed " + o.toString());
  System.out.println("Generic List:");
  list.print();
  o = list.removeLast();
  System.out.println(" Removed " + o.toString());
  System.out.println("Generic List:");
  list.print();
  o = list.removeFirst();
  System.out.println(" Removed " +o.toString());
  System.out.println("Generic List:");
  list.print();
} // main()
\end{lstlisting}
\end{jjjlisting}
\jjjprogstop{A series of tests for the {\tt List} ADT.
}{fig-testlistadt}
\end{figure}


\JavaTIP[false]{EFFECTIVE DESIGN}{The {\tt List} ADT.}{One advantage of defining
a {\tt List} ADT is that it lets you avoid having to write the
relatively difficult list-processing algorithms each time you need a
list structure.}

The list we create here involves various types of data.  The {\tt
Phone\-Record} class is a scaled-down version of the {\tt
PhoneListNode} we used in the previous example
(Fig.~\ref{fig-phonerecorduml}).  Its definition is shown in
Figure~\ref{fig-phonerecord}.  Note how we use an {\tt Object}
reference to remove objects from the list in {\tt main()}.  We use the
{\tt Object.toString()} method to display the object that was removed.

\begin{figure}[bht]
\figaright{chptr16/phonerecorduml.eps}{The {\tt PhoneRecord} class
stores data for a phone directory.
} {fig-phonerecorduml}
\end{figure}

\begin{figure}[htb]
\jjjprogstart
\begin{jjjlisting}
\begin{lstlisting}
public class PhoneRecord {
     private String name;
     private String phone;

     public PhoneRecord(String s1, String s2) {
         name = s1;
         phone = s2;
     }
     public String toString() {
         return name + " " + phone;
     }
     public String getName( ) {
         return name;
     }
     public String getPhone( ) {
         return phone;
     }
} // PhoneRecord
\end{lstlisting}
\end{jjjlisting}
\jjjprogstop{A {\tt PhoneRecord} class.
}{fig-phonerecord}
\end{figure}
%\end{objectorienteddesign}

\pagebreak
\secEXRHone{Self-Study Exercises}
\begin{SSTUDY}

\item  Trace through the {\tt main()} 
method line by line and predict its
output.

\item  Design a test of the {\tt List} program that shows that
it is possible to insert new elements into a list after
some or all of its previous nodes have been removed.
\end{SSTUDY}


\section{The Stack ADT}
\label{the-stack-adt}

\noindent A {\bf stack}\index{stack} is a special type of list that 
allows insertions and
removals to be performed only to the front of the list.  Therefore, it enforces {\bf
last-in--first-out (LIFO)}\index{last-in--first-out (LIFO)} behavior 
on the list.  Think of a stack of
dishes at the salad bar.
%\begin{marginalnote}\it Stack operations\end{marginalnote}
%\marginnote{Stack operations}
When you put a dish on the stack, it goes onto the top of the stack.
When you remove a dish from the stack, it comes from the top of the
stack  (Fig.~\ref{fig-stack}).  

\begin{figure}[thb]
\figaright{chptr16/stack.eps}{A stack is a list that permits
insertions and removals only at its top.
} {fig-stack}
\end{figure}

The stack operations are conventionally called {\bf push}\index{push},
for insert, and {\bf pop}\index{pop}, for remove, respectively.  Thus,
the stack ADT stores a list of data and supports the following
operations:

\begin{BL}
\item  Push---inserts an object onto the top of the stack.
\item  Pop---removes the top object from the stack.
\item  Empty---returns {\tt true} if the stack is empty.
\item  Peek---retrieves the top object without removing it.
\end{BL}

Stacks are useful for a number of important computing tasks.  For
\marginnote{Stack applications} example, during program execution,
method call and return happens in a LIFO fashion.  The last method
called is the first method exited.  Therefore, a stack structure known
as the {\em run-time stack} is used to manage method calls during
program execution.  When a method is called, an activation block is
created, which includes the method's parameters, local variables, and
return address.  The activation block is pushed onto the stack.  When
that method call returns, the return address is retrieved from the
activation block and the whole block is popped off the stack.  The
{\tt Exception.printStackTrace()} method uses the run-time stack to
print a trace of the method calls that led to an exception.

\subsection{The {\tt Stack} Class}

\noindent Given our general definition of {\tt List} and {\tt Node}, it is
practically trivial to define the stack ADT as a subclass of {\tt
List} (Fig.~16--21).  As a subclass of {\tt List}, a {\tt
Stack} will inherit all of the public and protected methods defined in
{\tt List}. Therefore, we can simply use the {\tt insertAtFront()} and
{\tt removeFirst()} methods for the push and pop operations,
\marginfig{chptr16/stackuml.eps}{As a subclass of {\tt List}, a {\tt Stack}
inherits all of its public (+) and protected (\#) elements.
Therefore, {\tt push()} can be defined in terms of {\tt insertAtFront()}
and {\tt pop()} can be defined in terms of {\tt removeFirst()}.}
{fig-stackuml}
respectively (Fig.~\ref{fig-stackadt}). Because the {\tt isEmpty()}
method is defined in {\tt List}, there's no need to override it in
\begin{figure}[h]
\jjjprogstart
\begin{jjjlisting}
\begin{lstlisting}
public class Stack extends List {
     public Stack() {
         super();          // Initialize the list
     }
     public void push( Object obj ) {
         insertAtFront( obj );
     }
     public Object pop() {
         return removeFirst();
     }
} // Stack
\end{lstlisting}
\end{jjjlisting}
\jjjprogstop{The {\tt Stack} ADT.
}{fig-stackadt}
\end{figure}
{\tt Stack}.  In effect, the {\tt push()} and {\tt pop()} methods
merely rename the {\tt insertAtFront()} and {\tt removeFirst()}
methods. Note that the {\tt Stack()} constructor calls the superclass
constructor. This is necessary so that the list can be initialized.


Do we have to make any changes to the {\tt List} class in order to use
it this way?  Yes.  We want to change the declaration of {\tt head}
from {\tt private} to {\tt protected}, so it can be accessed in the
{\tt Stack} class.  And we want to declare {\tt List}'s {\tt public}
%\begin{marginalnote}\it Protected methods are inherited\end{marginalnote}
%\marginnote{Protected methods are inherited}
access methods, such as {\tt insertAtFront()} and {\tt removeFirst()},
as {\tt protected}.  That will allow them to be used in {\tt Stack},
and in any classes that extend {\tt List}, but not by other
classes.  This is essential.  Unless we do this we haven't really
restricted the stack operations to push and pop and, therefore, we
haven't really defined a stack ADT.  Remember, an ADT defines the data
and the operations on the data.   A stack ADT must restrict access to
the data to just the push and pop operations.

\JavaRule{Protected Elements.}{An object's {\tt protected} elements
are hidden from all other objects except instances of the same class
or its subclasses.}

\JavaTIP{EFFECTIVE DESIGN}{Information Hiding.}{Use the
{\tt private} and {\tt protected} qualifiers to hide an ADT's
implementation details from other objects.  Use {\tt public} to define
the ADT's interface.}

\secEXRHone{Self-Study Exercise}
\begin{SSTUDY}

\item  Define the {\tt peek()} method for the
{\tt Stack} class.  It should
take no parameters and return an {\tt Object}. It should return the
{\tt Object} on the top of the stack.
\end{SSTUDY}


%%%REW\subsection{Testing the {\ttHtwo Stack} Class}
\subsection{Testing the {\tt Stack} Class}
\noindent Now let's test our {\tt stack} class by 
using a {\tt stack} to reverse the letters in a {\tt String}.  The
algorithm is this: Starting at the front of the {\tt String}, push
%\begin{marginalnote}\it Reversing a string\end{marginalnote}
\marginnote{Reversing a string}
each letter onto the stack until you reach the end of the
{\tt String}. Then pop letters off the stack and concatenate them, left to
right, into another {\tt String}, until the stack is empty
(Fig.~\ref{fig-stackadttest}).

Note that because our {\tt Node}s store {\tt Object}s,
we must convert each {\tt char} into a {\tt Character}, using the
wrapper class.  Note also that we can use the {\tt toString()} method
to convert from {\tt Object} to {\tt String} as we are popping the
stack.

\begin{figure}[h]
\jjjprogstart
\begin{jjjlisting}
\begin{lstlisting}
public static void main( String argv[] ) {
     Stack stack = new Stack();
     String string = "Hello this is a test string";

     System.out.println("String: " + string);
     for (int k = 0; k < string.length(); k++)
         stack.push(new Character( string.charAt(k)));

     Object o = null;
     String reversed = "";
     while (!stack.isEmpty()) {
         o  = stack.pop();
         reversed = reversed + o.toString();
     }
     System.out.println("Reversed String: " + reversed);
} // main()
\end{lstlisting}
\end{jjjlisting}
\jjjprogstop{A method to test the {\tt Stack} ADT, which is used
here to reverse a String of letters.
}{fig-stackadttest}
\end{figure}

\section{The {Queue} ADT}
\label{the-adt}
\noindent A {\bf queue}\index{queue} is a special type of list that
limits insertions to the end of the list and removals to the front of
the list.  Therefore, it enforces {\bf first-in--first-out
(FIFO)}\index{first-in--first-out (FIFO)} behavior on the list.  Think
of the waiting line at the salad bar.  You enter the line at the rear
and you leave the line at the front (Fig.~\ref{fig-queue}).  

\begin{minipage}{27pc}
The queue operations are conventionally called {\bf
enqueue}\index{enqueue}, for insert, and {\bf dequeue}\index{dequeue},
for remove, respectively.  Thus, the queue ADT stores a list of data
and supports the following operations:

\begin{BL}
\item  Enqueue---insert an object onto the rear of the list.
\item  Dequeue---remove the object at the front of the list.
\item  Empty---return true if the queue is empty.
\end{BL}
\end{minipage}

Queues are useful for a number of computing tasks.  For example, the
%\begin{marginalnote}\it Queue operations\end{marginalnote}
%\marginnote{Queue operations}
ready, waiting, and blocked queues used by the CPU scheduler all use a
FIFO protocol.   Queues are also useful in implementing certain kinds
of simulations.  For example, the waiting line at a bank or a bakery
can be modeled using a queue.

\begin{figure}[h!]
\figa{chptr16/queue.eps}{A queue is a list that permits insertions
at the rear and removals at the front only.}
{fig-queue}
\end{figure}

\pagebreak
\subsection{The {\tt Queue} Class}

\noindent The {\tt Queue} class is also trivial to derive
from {\tt List} (Fig.~16.25).  Here we just restrict operations to
the {\tt insertAtRear()} and {\tt removeFirst()} methods
(Fig.~\ref{fig-queueadt}).  To test the methods of this class, we
replace the {\tt push()} and {\tt pop()} operations of 
the last example to {\tt enqueue()} and {\tt dequeue()}, respectively
(Fig.~\ref{fig-queueadttest}).  In this case, the letters of the test
string will come out of the queue in the same order they went
in---FIFO.

\marginfig{chptr16/queueuml.eps}{The {\tt Queue}'s {\tt enqueue()} and
{\tt dequeue()} methods can use the {\tt List}'s
{\tt insertAtRear()} and {\tt removeFirst()} methods,
respectively.}
{fig-queueuml}

\begin{figure}[h!]%
\jjjprogstart
\begin{jjjlisting}
\begin{lstlisting}
public class Queue extends List {
     public Queue() {
         super();          // Initialize the list
     }
     public void enqueue(Object obj) {
         insertAtRear( obj );
     }
     public Object dequeue() {
         return removeFirst();
     }
}// Queue
\end{lstlisting}
\end{jjjlisting}
\jjjprogstop{The {\tt Queue} ADT.
}{fig-queueadt}
\end{figure}%


\begin{figure}[h!]
\jjjprogstart
\begin{jjjlisting}
\begin{lstlisting}
public static void main(String argv[]) {
     Queue queue = new Queue();
     String string = "Hello this is a test string";
     System.out.println("String: " + string);
     for (int k = 0; k < string.length(); k++)
         queue.enqueue( new Character(string.charAt(k)));
     System.out.println("The current queue:");
     queue.print();

     Object o = null;
     System.out.println("Dequeuing:");
     while (!queue.isEmpty()) {
         o  = queue.dequeue();
         System.out.print( o.toString() );
     }
} // main()
\end{lstlisting}
\end{jjjlisting}
\jjjprogstop{A method to test the {\tt Queue} ADT. Letters inserted
in a queue come out in the same order they went in.
}{fig-queueadttest}
\end{figure}

\JavaTIP{EFFECTIVE DESIGN}{ADTs.}{ADTs encapsulate and manage
the difficult tasks involved in manipulating the data structure.
But because of their extensibility, they can be used in
a wide range of applications.}



\pagebreak
\secEXRHone{Self-Study Exercise}
\begin{SSTUDY}

\item  Define a {\tt peekLast()} method for 
the {\tt Queue} class.  It
should take no parameters and return an {\tt Object}. It should 
return a reference
to the {\tt Object} stored in the last {\tt Node} of the list without
removing~it.
\end{SSTUDY}

\section*{{\color{cyan}Special Topic:} The LISP Language}

\noindent One of the very earliest computer languages, and the one
that's most often associated with artificial intelligence (AI), is
LISP, which stands for {\it LIS}t {\it P}rocessor.  LISP has been, and
still is, used to build programs for human learning, natural language
processing, chess playing, human vision processing, and a wide range
of other applications.

The earliest (pure) versions of LISP had no control structures and the
only data structure they contained was the list structure.  Repetition
in the language was done by recursion.  Lists are used for everything
in LISP, including LISP programs themselves.  LISP's unique syntax is
simple.  A LISP program consists of symbols, such as {\it 5} and {\it
x}, and lists of symbols, such as {\it (5)}, \mbox{\it (1 2 3 4 5)},
and {\it ((this 5) (that 10))}, where a list is anything enclosed
within parentheses.  The null list is represented by {\tt ()}.\\

Programs in LISP are like mathematical functions.  For example, here's
a definition of a function that computes the square of two numbers:

\begin{jjjlisting}
\begin{lstlisting}
(define (square x) (* x x) )
\end{lstlisting}
\end{jjjlisting}

\noindent The expression {\it (square x)} is a list giving the name of
the function and its parameter.  The expression {\it (* x x)} gives
the body of the function.  

LISP uses {\it prefix notation}, in which the operator is the first
symbol in the expression, as in {\it (* x x)}. This is equivalent to
{\it (x * x)} in Java's {\it infix notation}, where the operator
occurs between the two operands.  To run this program, you would
simply input an expression like {\it (square 25)} to the LISP
interpreter, and it would evaluate it to 625.

LISP provides three basic list operators.  The expression {\it (car
x)} returns the first element of the (nonempty) list x.  The
expression {\it (cdr x)} returns the tail of the list {\it x}.
Finally, {\it (cons z x)} constructs a list by making {\it z} the head
of the list and {\it x} its tail.  For example, if {\it x} is the list
{\it (1 3 5)}, then {\it (car x)} is 1, {\it (cdr x)} is {\it (3 5)},
and {\it (cons 7 x)} is {\it (7 1 3 5)}.  %%\\

Given these basic list operators, it is practically trivial to define
a stack in LISP:

\begin{jjjlisting}
\begin{lstlisting}
(define (push x stack) (cons x stack))        
(define (pop stack)(setf stack (cdr stack))(car stack))
\end{lstlisting}
\end{jjjlisting}

\noindent The push operation creates a new stack by forming the {\it cons}
of the element {\it x} and the previous version of the
stack.  The pop operation returns the {\it car} of the stack but
first changes the stack (using {\tt setf}) to the tail of the original
stack.

These simple examples show that you can do an awful lot of computation
using just a simple list structure.  The success of LISP, particularly
its success as an AI language, shows the great power and generality
inherent in recursion and lists.

\section{From the Java Library: The Java Collections Framework and Generic Types}
\label{fromthe-java-library-the-java-collections-frameworkand-generic-types}

\markright{{\color{cyan}SECTION\,\,\arabic{section}\,\,$\bullet$}\,\,Java Collections and Generic Types}
\WWWjava

\tBOXseven{The Java class library} contains implementations of some
abstract data types.  The Java utility package, {\tt java.util.*},
contains a good number of classes and interfaces designed to
facilitate storing and manipulating groups of objects.  This family of
related interfaces and classes is called the {\bf Java collections
framework}.  It contains structures that correspond to the ADTs that
we have just discussed, plus other data structures.  Java 5.0 has
reimplemented the Java collections framework using {\bf generic types}
that allow a programmer to specify a type for the objects that are
stored in the structure.

\subsection{Generic types in Java}

Declaring classes that use the generic type construct introduced in Java 5.0
\marginnote{generic types}
involves using new syntax to refer to the class name. Such classes and
interfaces, including those in the collections framework, use angle
brackets containing one or more variables (separated by commas) to
refer to unspecified type names.  For example, you would use {\tt <E>}
or {\tt <K,V>} to refer to unspecified type names. Thus, names of
classes or interfaces implemented with generic types are written with
the syntax {\tt ClassName<E>}.

Let's reconsider the {\tt Vector} class, which was introduced in
Chapter~9.  The {\tt Vector} class, which is part of the Java collections
framework, has a generic type implementation in Java 5.0.
\marginfig{chptr16/libvectoruml.eps}{The {\tt java.util.\-Vector}
class is implemented with a generic type in Java 5.0.}
{fig-libvectoruml}
Figure~16.28 describes the {\tt Vector<E>} class. Notice that the
{\tt E} refers to an unspecified type name, that is, the name of a
class or interface.  This type is specified when a corresponding
variable is declared. The type must also be included after a
constructor's type name when an object is instantiated and assigned to
the variable.  The following code demonstrates how to create a {\tt
Vector<E>} object for storing {\tt String} objects.

\begin{jjjlisting}
\begin{lstlisting}
   Vector<String> strVec = new Vector<String>();
   strVec.addElement("alpha");
   strVec.addElement("beta");
   String str = strVec.elementAt(0);
\end{lstlisting}
\end{jjjlisting}

\noindent In effect, the {\tt <E>} serves as parameter for the
type of objects that will be stored in the {\tt Vector}. 
Java 5.0 still allows the use of the unparameterized {\tt
Vector} class which is equivalent to instantiating a {\tt
Vector<Object>} object.  If you use a {\tt Vector} object, the above
code would be written as follows.

\begin{jjjlisting}
\begin{lstlisting}
   Vector strVec = new Vector();
   strVec.addElement("alpha");
   strVec.addElement("beta");
   String str = (String)strVec.elementAt(0);
\end{lstlisting}
\end{jjjlisting}

One benefit a generic type provides is type checking of method arguments
at compile time.  If {\tt strVec} is a {\tt Vector<String>} object, then the
statement

\begin{jjjlisting}
\begin{lstlisting}
   strVec.addElement(new Integer(57));
\end{lstlisting}
\end{jjjlisting}

\noindent will generate a compile-time error.  By contrast, if {\tt
strVec} was just a plain {\tt Vector} object, no error would be found
at compile time.  Thus, if a programmer wishes to create an array of
{\tt String} objects, using generic types will help guarantee that the
objects being stored are actually of type {\tt String}. In this way,
using generic types helps to reduce the number of programming errors
and thereby makes programs safer and more robust.

A second benefit of using generic types is that the return type of
objects retrieved from the data structure will be of the specified
type rather than of type {\tt Object}.  If you compare the last
statement in each of the two code segments above, you can see that
using a generic type eliminates the need to cast an {\tt Object} to a
{\tt String}.  This is a big convenience for the programmer, because
forgetting to cast objects from one type to another is a common
programming error.

\marginfig{chptr16/libstackuml.eps}{The {\tt java.util.\-Stack<E>}
class is a subclass of {\tt Vector<E>}.}
{fig-libstackuml}

\subsection*{The {\tt java.util.Stack<E>} class}

The Java collections framework includes the {\tt Stack<E>} class,
implemented as a subclass of the {\tt Vector<E>} class.  It contains
the methods shown in Figure~16.29.  For the most part, its methods
provide the same functionality as the methods we developed earlier in
this chapter.  


\noindent Note that the methods provide the functionality of a stack ADT but the
details of its implementation are hidden from the user.  An object of
this class can be declared, instantiated, and used in a manner 
like the {\tt Vector<E>} code.

\begin{jjjlisting}
\begin{lstlisting}
   Stack<String> stk = new Stack<String>();
   stk.push("alpha");
   stk.push("beta");
   String str = stk.pop();
\end{lstlisting}
\end{jjjlisting}

\secEXRHone{Self-Study Exercise}
\begin{SSTUDY}

\item  Write a class with only a {\tt main()} method
that modifies Figure~\ref{fig-stackadttest} so that it uses the parameterized
{\tt java.util.Stack<E>} class instead of the {\tt Stack} class used there.
\end{SSTUDY}

\subsection{The {\tt List<E>} interface \\and the {\tt LinkedList<E>} class}

The {\tt java.util.LinkedList<E>} is an implementation of a linked
list (Fig.~16.30).  Like our implementation earlier in this
chapter, it contains methods that can be used to define the standard
stack and queue methods.

\marginfig{chptr16/liblinlist.eps}{The {\tt LinkedList<E>} class
implements the  {\tt List<E>} interface. Only a partial list of 
methods are shown.}
{fig-liblinlist}

Many of the standard list-processing methods are defined as part of
the {\tt java.util.List<E>} interface.  The advantage of defining list
operations as an interface is that they can be implemented by a number
of data structures.  Code for using the list methods can be written to
work independently of the data structure being used.  

For example, the collections framework contains {\tt LinkedList<E>}
and {\tt ArrayList<E>}, both of which implement the {\tt List<E>}
interface.  In this section, we will demonstrate how to make
appropriate use of the {\tt List<E>} interface and data structures
that implement it.

Suppose that a programmer is developing an application to track
activity of employees working at a small company's phone-in help desk.
The programmer has decided to use the {\tt LinkedList<E>} data
structure to store objects of the {\tt PhoneRecord} class that was
defined earlier in this chapter and will use methods of the {\tt
List<E>} interface to manipulate the data.  A list seems to be an
appropriate structure for this problem since
\begin{itemize}
\item  {An unknown (but relatively small) amount of data will be involved.}
\item  {The company wants the data stored in the order it is generated.}
\item  {The main use of the data will be to print out the list's 
phone records.}
\end{itemize}

\noindent The programmer might write a short method like that in
Figure~\ref{fig-listtest} to demonstrate how the {\tt List<E>} and
{\tt LinkedList<E>} structures will be used.

\begin{figure}[htb]
\jjjprogstart
\begin{jjjlistingleft}[27pc]{-1pc}
\begin{lstlisting}
public static void testList() {
  List<PhoneRecord> theList = new LinkedList<PhoneRecord>();
  // new ArrayList<PhoneRecord>(); could also be used.
  theList.add(new PhoneRecord("Roger M", "090-997-2918"));
  theList.add(new PhoneRecord("Jane M", "090-997-1987"));
  theList.add(new PhoneRecord("Stacy K", "090-997-9188"));
  theList.add(new PhoneRecord("Gary G", "201-119-8765"));
  theList.add(new PhoneRecord("Jane M", "090-997-1987"));
  System.out.println("Testing a LinkedList List");
  for (PhoneRecord pr : theList)
     System.out.println(pr);
} // testList
\end{lstlisting}
\end{jjjlistingleft}
\jjjprogstop{A method that demonstrates the interface {\tt List<E>}
and the class {\tt LinkedList<E>}.
}{fig-listtest}
\end{figure}


\noindent Note that the statement

\begin{jjjlisting}
\begin{lstlisting}
List<PhoneRecord> theList = new LinkedList<PhoneRecord>();
\end{lstlisting}
\end{jjjlisting}

\noindent declares a variable {\tt theList} of interface type {\tt
List<E>} but assigns an object of class type {\tt LinkedList<E>}.
This is appropriate because the class implements the interface and the
code uses only methods from the interface.  The class {\tt
ArrayList<E>} in the collections framework also implements the {\tt
List<E>} interface.  It uses an array rather than a linked list to
store elements and has a constructor with an {\tt int} parameter that
sets the size of the array.  If the programmer knew that {\tt theList}
would contain close to, but always less than, 100 elements, then it
might be better to declare:

\begin{jjjlisting}
\begin{lstlisting}
List<PhoneRecord> theList = new ArrayList<PhoneRecord>(100);
\end{lstlisting}
\end{jjjlisting}

\noindent Also note the unusual looking {\tt for} loop at the end of
the method.  This is a new feature of Java 5.0 which can be used to simplify
the coding of loops that iterate through every object in a collection
of objects. The statement 
\marginnote{The for--each loop}

\begin{jjjlisting}
\begin{lstlisting}
     for (PhoneRecord pr : theList) { *** }
\end{lstlisting}
\end{jjjlisting}

\noindent should be thought of as executing the enclosed statements
{\em for each} {\tt PhoneRecord} object, {\tt pr}, in the {\tt
theList} data structure.  In previous versions of Java, an interface
named {\tt Iterator} had to be used to enumerate all the elements in a
collection.  The {\tt Iterator} approach is more flexible---for
example, it allows you to iterate through just some of the members of
the collection--- and will therefore still have to be used for more
complex loops.

The output of the method will be:
\begin{jjjlisting}
\begin{lstlisting}
          Roger M 090-997-2918
          Jane M 090-997-1987
          Stacy K 090-997-9188
          Gary G 201-119-8765
          Jane M 090-997-1987
\end{lstlisting}
\end{jjjlisting}

In the next section we will examine two other structures in the collections
framework, the {\tt Set} interface and the {\tt Map} interface.

\JavaTIP{EFFECTIVE DESIGN}{Code Reuse.}{Given the relative difficulty
of writing correct and efficient list-processing algorithms, applications
that depend on lists should make use of library classes whenever possible.}

\section{ Using the {\tt Set} and {\tt Map} Interfaces}
\label{usingthe-interfaces}

The {\tt Set} and {\tt Map} interfaces are similar to the {\tt List}
interface in that there are multiple classes in the collections framework
that implement them.

\subsection{Using the {\tt Set} Interface.}

The {\tt Set} interface is modeled after the {\em set theory}
principles taught in mathematics.  In mathematics, sets are groups of
elements with a clearly defined algorithm for deciding if any given
element is in any given set.  Elements can be added to sets and can be
removed from sets. Sets cannot have duplicate elements; if an element
is added to a set that already contains an element equal to it, the
new set still has a single such element.  The elements of a set have
no natural order; two sets that have the same elements listed in
different orders are considered to be the same set.

In computer science and in Java, data structures that model sets are
designed for large collections of data.  Such data structures have a
method that determines if an object is in a given set with an
efficient algorithm. For large data sets, using such a method is much
faster than iterating through a list. Sometimes, you may or may not be
able to list the elements of a set data structure in some natural
\marginfig{chptr16/libset.eps}{A partial list of methods of the {\tt 
Set<E>} interface.}
{fig-libset}
\noindent order, depending on how the data structure is implemented. An
incomplete listing of the methods of the {\tt Set<E>} interface is
given in the UML diagram in Figure~16.32.

{\tt TreeSet<E>} and {\tt HashSet<E>} are two classes in the
collections framework that implement the {\tt Set<E>} interface. They
both provide fast operations to check whether an element is in a set.
They also provide quick insertion of an element into the set or
removal of an element from a set.  For large sets---those having at
least several thousand elements---where there are large numbers of
insertions, deletions, and tests for whether elements are in a set,
linked lists would be much slower.

When using the {\tt Set<E>} interface for a user-defined class {\tt
E}, you will likely want to override the definition of the {\tt
equals()} method from the {\tt Object} class in {\tt E} because that
\marginnote{Overriding methods}
method is used when computing the value of {\tt
aSet.contains(anElement)}. When using the {\tt TreeSet<E>} class for a
user defined class {\tt E}, you should implement the {\tt compareTo()}
method of the {\tt Comparable} interface because it is used to order
the elements of {\tt E}. In the next section, we will discuss the
specific manner in which elements are ordered.  Finally, when using
the {\tt HashSet<E>} class for a user defined class {\tt E}, you
should override the {\tt hashCode()} method of the {\tt Object} class
because it is used {\tt HashSet<E>}.  {\em Hash codes} are indexes
that are computed from the particular object that is being stored in
the {\tt HashSet}. Given an object's hash code, the object can
be retrieved directly, as we do with object's stored in an array.
However, we will not discuss hash codes in any detail in this text.

\subsection*{Problem Statement}
Let's think about a simple example for using a set data
structure. Suppose that a programmer is developing an application for
a large company for maintaining a {\em no--call} list.  The programmer
has decided to use the {\tt TreeSet<E>} data structure to store
objects of the {\tt PhoneRecord} class that was defined earlier in
this chapter and will use methods of the {\tt Set<E>} interface to
manipulate the data.  

\begin{minipage}{26pc}
A {\tt TreeSet} seems to be an appropriate structure for this problem,
since
\begin{itemize}
\item  {A large amount of data will be involved.}
\item  {The company wants the {\tt PhoneRecord} data stored in 
alphabetical order.}
\item  {The main use of the data will be to test whether names are in the set.}
\end{itemize}
\end{minipage}

\noindent The programmer might write a short method like that in
Figure~\ref{fig-settest} to demonstrate how the {\tt Set<E>} and {\tt
TreeSet<E>} structures will be used.

\begin{figure}[tbh]
\jjjprogstart
\begin{jjjlisting}
\begin{lstlisting}
public static void testSet() {
  Set<PhoneRecord> theSet = new TreeSet<PhoneRecord>();
  // new HashSet<PhoneRecord>(); could also be used.
  theSet.add(new PhoneRecord("Roger M", "090-997-2918"));
  theSet.add(new PhoneRecord("Jane M", "090-997-1987"));
  theSet.add(new PhoneRecord("Stacy K", "090-997-9188"));
  theSet.add(new PhoneRecord("Gary G", "201-119-8765"));
  theSet.add(new PhoneRecord("Jane M", "090-987-6543"));

  System.out.println("Testing TreeSet and Set");
  PhoneRecord ph1 =
      new PhoneRecord("Roger M", "090-997-2918");
  PhoneRecord ph2 =
      new PhoneRecord("Mary Q", "090-242-3344");
  System.out.print("Roger M contained in theSet is ");
  System.out.println(theSet.contains(ph1));
  System.out.print("Mary Q contained in theSet is ");
  System.out.println(theSet.contains(ph2));
  for (PhoneRecord pr : theSet)
      System.out.println(pr);
} // testSet
\end{lstlisting}
\end{jjjlisting}
\jjjprogstop{A method that demonstrates use of the interface {\tt Set<E>}
and the class {\tt TreeSet<E>}.
}{fig-settest}
\end{figure}


In order for the {\tt testSet()} method to work as we would like, we
need to have the {\tt PhoneRecord} class implement the {\tt
Comparable} interface and to override the {\tt equals()} method.  For
this example, it is reasonable to assume that the {\tt name} field of
{\tt PhoneRecord} objects should be unique so that it can be used to
decide if two {\tt PhoneRecord} objects are equal.  The {\tt name}
field of {\tt PhoneRecord} can also be used to define the other two
methods discussed above.  Thus, add the following code to the {\tt
PhoneRecord} class.

\begin{jjjlisting}
\begin{lstlisting}
public boolean equals(Object ob){
    return name.equals(((PhoneRecord)ob).getName());
} //equals()

public int compareTo(Object ob){
   return name.compareTo(((PhoneRecord)ob).getName());
} // compareTo()

public int hashCode(){
    return name.hashCode();
} // hashCode()
\end{lstlisting}
\end{jjjlisting}

The output of the {\tt TestSet()} method is listed below:

\begin{jjjlisting}
\begin{lstlisting}
            Testing TreeSet and Set
            Roger M is contained in theSet is true
            Mary Q is contained in theSet is false
            Gary G 201-119-8765
            Jane M 090-997-1987
            Roger M 090-997-2918
            Stacy K 090-997-9188
\end{lstlisting}
\end{jjjlisting}

\noindent Notice that {\tt Jane M} {\tt PhoneRecord} appears only once in the
listing of elements in the set.

\subsection{Using the {\tt Map<K,V>} Interface.}

The {\tt Map<K,V>} interface is modeled after looking up definitions
for words in a dictionary. In computer science, maps are considered to
be a collection of pairs of elements. A pair consists of a {\bf key}
that corresponds to a word being looked up and a {\bf value}
corresponding to the definition of the word.  Pairs can be added to
maps and can be removed from maps. Maps cannot
\marginfig{chptr16/libmap.eps}{A partial list of methods of {\tt Map<K,V>}.}
{fig-libmap}
\noindent have distinct pairs with the same keys; if you attempt to add a pair
to a map that already contains a pair with the same key, the second
pair will replace the first.  An incomplete listing of the methods of
the {\tt Map<K,V>} interface is given in the UML diagram in
Figure~16.34.  {\tt TreeMap<K,V>} and {\tt HashMap<K,V>} are two
classes in the collections framework that implement the {\tt Map<K,V>}
interface.

Let's now consider a simple example of using a map data
structure. Suppose that our programmer has been hired by a large
company to develop an application that maintains an electronic phone
list for company employees.  The programmer has decided to use the
{\tt TreeMap<E>} data structure to store pairs of names and telephone
numbers and will use methods of the {\tt Map<V,K>} interface to
manipulate the data.  

\begin{minipage}{26pc}
A {\tt TreeMap} seems like an appropriate data
structure for this problem, since

\begin{itemize}
\item  {A large amount of data will be involved.}
\item  {The company wants the {\tt PhoneRecord} data stored in 
alphabetical order.}
\item  {The main use of the data will be to use names to access 
telephone numbers.}
\end{itemize}
\end{minipage}

\noindent The programmer might write a short method like that in
Figure~\ref{fig-maptest} to demonstrate how the {\tt Map<K,V>} and
{\tt TreeMap<K,V>} structures will be used.

\begin{figure}[tbh]
\jjjprogstart
\begin{jjjlisting}
\begin{lstlisting}
public static void testMap() {
  Map<String, String> theMap =
     new TreeMap<String,String>();
  // new HashMap<K,V>(); could also be used
  theMap.put("Roger M", "090-997-2918");
  theMap.put("Jane M", "090-997-1987");
  theMap.put("Stacy K", "090-997-9188");
  theMap.put("Gary G", "201-119-8765");
  theMap.put("Jane M", "090-233-0000");
  System.out.println("Testing TreeMap and Map");
  System.out.print("Stacy K has phone ");
  System.out.print(theMap.get("Stacy K");
  System.out.print("Jane M has phone ");
  System.out.print(theMap.get("Jane M");
} // testList
\end{lstlisting}
\end{jjjlisting}
\jjjprogstop{A method that demonstrates use of the interface {\tt Map<K,V>}
and the class {\tt TreeMap<K,V>}.
}{fig-maptest}
\end{figure}


The output for this test program is:
\begin{jjjlisting}
\begin{lstlisting}
        Stacy K has phone 090-997-9188
        Jane M has phone 090-233-0000
\end{lstlisting}
\end{jjjlisting}

\noindent Notice the the second phone number for {\tt Jane M} is the one that
is stored in the data structure.

\section{ The Binary Search Tree Data Structure}
\label{the-binary-search-tree-data-structure}

To gain some appreciation of what binary search trees are and why they
are useful in implementing the {\tt Set} and {\tt Map} interfaces, let's
make a few comments about implementing very simple versions of these 
structures.

Like a linked list, a {\em binary tree} is a data structure consisting
of a collection of nodes that are linked together by references from
one node to another.  However, unlike a linked list, each node in a
binary tree contains references to two other other nodes, ({\tt left}
and {\tt right}), corresponding to the left- and right-subtrees
growing out of a particular node.  A {\em subtree} is a tree that is
part of larger tree. This creates a tree-like structure, as shown in
Figure~16.36. Note that some of the references to other nodes might
\begin{figure}[h!]
\figaleft{chptr16/btree.eps}{A binary search tree of {\tt PhoneTreeNode}s.
} {fig-btree-ds}
\end{figure}
\noindent be null.  The trunk of the tree corresponds to the node labeled {\tt
root}. In computer science, trees are almost always drawn upside down.
Thus the trunk of the tree, {\tt root}, is at the top of the figure.

If we assume that the objects contained in a tree are from a class
that implements the {\tt Comparable} interface, then a {\bf binary
search tree} is a binary tree in which the objects are ordered so that
the object at a particular node is greater than the objects stored in
its left subtree and less than the objects stored in its right
subtree.

Figure~16.36 shows a binary search tree with the phone list data that
we have used throughout the chapter.  Objects are compared by
comparing the names alphabetically.  From the figure it is easy to see
that searching for a object should start at the root of the tree. At
each node, examining the name at the node will tell you whether you
have found the object there.  Otherwise, by checking the name at the
node, you can decide which subtree the data could be in, and you can
traverse either left or right through each level of the tree. One can
deduce that if the tree is balanced---that is, if at most nodes the
size of the left subtree is about the same size as the right
subtree---searching the tree much faster than searching a linked
list. This is one of the main advantages of using a binary search tree
over a linked list.  

The {\tt TreeSet} and {\tt TreeMap} classes implement sophisticated
algorithms for inserting and removing data from a tree, which
guarantees that the tree remains relatively balanced. The details of
these algorithms are beyond the scope of this book, but would be a
subject of study in a standard {\em Data Structures and Algorithms}
course.

We will conclude this subsection by giving a brief outline of an
implementation of a simple binary search tree that stores our phone
list data.  Like our {\tt LinkedList} example, you need to define a
node class and a tree class.  The node class with unimplemented
methods, would look like:

\begin{jjjlisting}
\begin{lstlisting}
public class PhoneTreeNode {
   private String name;
   private String phone;
   private PhoneTreeNode left;
   private PhoneTreeNode right;

   public PhoneTreeNode(String nam, String pho){ }
   public void setData(String nam, String pho){ }
   public String getName(){ }
   public boolean contains(String nam, String pho){ }
   public void insert(String nam, String pho){ }
   //other methods
} // PhoneTreeNode
\end{lstlisting}
\end{jjjlisting}

\noindent The tree structure itself contains a reference to a node:

\begin{jjjlisting}
\begin{lstlisting}
public class PhoneTree {
  private PhoneTreeNode root;

  public PhoneTree(){ }
  public boolean contains(String nam, String pho){ }
  public void insert(String nam, String pho){ }
  //other methods
} // PhoneTreeNode
\end{lstlisting}
\end{jjjlisting}

\noindent We will implement only the two {\tt contains()} methods.
The {\tt PhoneTree} version is very simple:

\begin{jjjlisting}
\begin{lstlisting}
public boolean contains(String nam, String pho){
    if (root == null) return false;
    else return root.contains(nam, pho);
} // contains() in PhoneTree
\end{lstlisting}
\end{jjjlisting}

\noindent The implementation of the {\tt contains()} method of {\tt
PhoneTreeNode} is only slightly more involved.

\begin{jjjlisting}
\begin{lstlisting}
public boolean contains(String nam, String pho){
    if (name.equals(nam))
        return true;
    else if(name.compareTo(nam) < 0) { // name < nam
        if (right == null) return false;
        else return right.contains(nam, pho);
    } else { {\color{cyan} // name > nam}
        if (left == null) return false;
        else return left.contains(nam, pho);
    }
} // contains() in PhoneTreeNode
\end{lstlisting}
\end{jjjlisting}

\pagebreak
\secSMHleft{Chapter Summary}

%\addcontentsline{toc}{section}{\S~~~ Chapter Summary}

In this chapter, we have given you a brief introduction to the concept of a
dynamic data structure and tried to illustrate how they work and why
they are useful for organizing and managing large amounts of data. We
also introduced you to an important new language feature introduced in
Java 5.0, the concept of generic types.  Obviously, we have only
scratched the surface of the important topic of data structures and
the algorithms used to manage them.  For a broader and deeper
treatment of this subject, you will have to take a {\em Data
Structures and Algorithms} course, which is a fundamental course in
most computer science curricula.

\secKTH{Technical Terms}
\begin{KT}
Abstract Data Type (ADT)

binary search tree

data structure

dequeue

dynamic structure

enqueue

first-in--first-out (FIFO)

generic type

Java collections framework

key

last-in--first-out (LIFO)

link

list

linked list

pop

push

queue

reference

self-referential object

stack

static structure

traverse

value

vector

\end{KT}

\secSMHtwo{Summary of Important Points}

\begin{SMBL}
\item  A {\it data structure} is used to organize
data and make them more efficient to process.  An array is
an example of a {\it static structure}, since its size
does not change during a program's execution.  A vector
is an example of a {\it dynamic structure}, one whose
size can grow and shrink during a program's execution.

\item  A {\it linked list} is a linear structure in which
the individual nodes of the list are joined together by references.  A
{\it reference} is a variable that refers to an object.  Each node in
the list has a {\it link} variable that refers to another node.  An
object that can refer to the same kind of object is said to be {\it
self-referential}.

\item  The {\tt Node} class is an example of a self-referential class.
It contains a link variable that refers to a {\tt Node}. By assigning
references to the link variable, {\tt Node}s can be chained together
into a linked list.  In addition to their link variables, {\tt Node}s
contain data variables, which should be accessible through public
methods.

\item  Depending on the use of a linked list, nodes can be
inserted at various locations in the list: at the head,
the end, or in the middle of the list.

\item  Traversal algorithms must be used to access the elements
of a singly linked list.  To traverse a list you start at the
first node and follow the links of the chain until you
reach the desired node.

\item  Depending on the application, nodes can be
removed from the front, rear, or middle of a linked list.
Except for the front node, traversal algorithms are used to locate the
desired node.

\item  In developing list algorithms, it is important to test
them thoroughly.  Ideally, you should test every possible combination
of insertions and removals that the list can support.  Practically, you
should test every independent case of insertions and removals that the
list supports.

\item  An {\it Abstract Data Type (ADT)} is a concept
that combines two elements: A collection of data, and the operations
that can be performed on the data.  For the list ADT, the data are the
values ({\tt Object}s or {\tt int}s) contained in the nodes that make
up the list, and the operations are insertion, removal, and tests of
whether the list is empty.

\item  In designing an ADT, it's important to provide a public
interface that can be used to access the ADT's data.  The ADT's
implementation details should not matter to the user and should,
therefore, be hidden.  A Java class definition, with its {\tt public}
and {\tt private} aspects, is perfectly suited to implement an ADT.

\item  A {\it stack} is a list that allows insertions and removals
only at the front of the list.  A stack insertion is called a {\it
push} and a removal is called a {\it pop}. The first element in a
stack is usually called the top of the stack.   The {\tt Stack} ADT can
easily be defined as a subclass of {\tt List}. Stacks are used
for managing the method call and return in most programming languages.

\item  A {\it queue} is a list that only allows insertions at the
rear and removals from the front of a list.  A queue insertion is
called {\it enqueue}, and a removal is called {\it dequeue}.  The
{\tt Queue} ADT can easily be defined as a subclass of {\tt List}. Queues
are used for managing the various lists used by the CPU scheduler---such as
the ready, waiting, and blocked queues.

\item  A {\it binary search tree} is a binary tree in which the ordered data
stored at any node is greater than all data stored in the left 
subtree and is less
than all data stored in the right subtree.
\end{SMBL}



\secANSH

%\addcontentsline{toc}{section}{\S~~~ Answers to Self-Study Exercises}
\begin{ANS}
\item  \mbox{ }

\begin{jjjlisting}
\begin{lstlisting}
Node node = new Node(new String("Hello"));
\end{lstlisting}
\end{jjjlisting}


\item  \mbox{ }

\begin{jjjlisting}
\begin{lstlisting}
Node node = new Node(new Student("William"));
\end{lstlisting}
\end{jjjlisting}

\item  \mbox{ }

\begin{jjjlisting}
\begin{lstlisting}
PhoneListNode newNode = 
         new PhoneListNode("Bill C", "111-202-3331");
nodeptr.setNext(newNode);
\end{lstlisting}
\end{jjjlisting}

\item  The following condition is too general.  It will cause the 
loop to exit as soon
as a nonnull node is encountered, whether or not the node matches the
one being sought.

\begin{jjjlisting}
\begin{lstlisting}
((current.getNext() != null) || 
                  (!current. getName().equals(name)))
\end{lstlisting}
\end{jjjlisting}

\item  The {\tt PhoneList} program will generate the
following output, which has been edited slightly
to improve its readability:

\begin{jjjlisting}
\begin{lstlisting}
Phone Directory
---------------
Roger M 997-0020    Roger W 997-0086    Rich P  997-0010
Jane M  997-2101    Stacy K 997-2517
Looking up numbers by name
   Roger M 997-0020
   Rich P 997-0010
   Stacy K 997-2517
   Sorry. No entry for Mary P
Removed Rich P  997-0010
Phone Directory
---------------
Roger M 997-0020    Roger W 997-0086   Jane M  997-2101
Stacy K 997-2517
Removed Roger M 997-0020
Phone Directory
---------------
Roger W 997-0086    Jane M  997-2101    Stacy K 997-2517
Removed Stacy K 997-2517
Phone Directory
---------------
Roger W 997-0086    Jane M  997-2101
Removed Jane M  997-2101
Phone Directory
---------------
Roger W 997-0086
Sorry. No entry for Jane M
Phone Directory
---------------
Roger W 997-0086
Removed Roger W 997-0086
Phone Directory
---------------
Phone list is empty
\end{lstlisting}
\end{jjjlisting}

\item  Executing the following method calls will test
whether it is possible to insert items into a
list after items have been removed:

\begin{jjjlisting}
\begin{lstlisting}
       // Create and insert some nodes
PhoneList list = new PhoneList();
list.insert(new PhoneListNode("Roger M", "997-0020"));
list.insert(new PhoneListNode("Roger W", "997-0086"));
System.out.println(list.remove("Roger M") );
list.insert(new PhoneListNode("Rich P", "997-0010"));
System.out.println(list.remove("Roger W"));
list.insert(new PhoneListNode("Jane M", "997-2101"));
list.insert(new PhoneListNode("Stacy K", "997-2517"));
System.out.println(list.remove("Jane M"));
System.out.println(list.remove("Stacy K"));
list.print();
       // List should be empty
\end{lstlisting}
\end{jjjlisting}

\item  The  List ADT program will produce the
following output:

\begin{jjjlisting}
\begin{lstlisting}
Generic List
---------------
Hello, World!
8647
Roger M 997-0020
Jane M 997-2101
Stacy K 997-2517
  Removed Stacy K 997-2517
Generic List:
Hello, World!
8647
Roger M 997-0020
Jane M 997-2101
  Removed Jane M 997-2101
Generic List:
Hello, World!
8647
Roger M 997-0020
  Removed Hello, World!
Generic List:
8647
Roger M 997-0020

\end{lstlisting}
\end{jjjlisting}

\item  Executing the following method calls will test
whether it is possible to insert items into a
  {\tt List} after items have been removed:

\begin{jjjlisting}
\begin{lstlisting}
       // Create and insert some nodes
List list = new List();
list.insertAtFront(new PhoneRecord("Roger M", "997-0020"));
list.insertAtFront(new PhoneRecord("Roger W", "997-0086"));
System.out.println("Current List Elements");
list.print();
Object o = list.removeLast();   // Remove last element
list.insertAtFront(o); // Insert at the front of the list
System.out.println("Current List Elements");
list.print();
o = list.removeFirst();
System.out.println("Removed " + o.toString());
o = list.removeFirst();
System.out.println("Removed " + o.toString());
list.insertAtRear(o);
System.out.println("Current List Elements");
list.print();       // List should have one element
\end{lstlisting}
\end{jjjlisting}

\item  The {\tt peek()} method should just return the
first node without deleting it:

\begin{jjjlisting}
\begin{lstlisting}
public Object peek() {
     return head;
}
\end{lstlisting}
\end{jjjlisting}

\item  The {\tt peekLast()} method can be modeled after the
{\tt List.re\-moveLast()} method:

\begin{jjjlisting}
\begin{lstlisting}
public Object peekLast() {
  if (isEmpty())
    return null;
  else {
    Node current = head;         // Start at head of list
    while (current.getNext() != null)// Find end of  list
      current = current.getNext();
      return  current;                 // Return last node
    }
} // peekLast()
\end{lstlisting}
\end{jjjlisting}

\item  The following class tests the {\tt java.util.Stack<E>} class:

\begin{jjjlisting}
\begin{lstlisting}
import java.util.*;
public class StackTest{
public static void main( String argv[] ) {
   Stack<Character> stack = new Stack<Character>();
   String string = "Hello this is a test string";

   System.out.println("String: " + string);
   for (int k = 0; k < string.length(); k++)
     stack.push(new Character(string.charAt(k)));

   Character ch = null;
   String reversed = "";
   while (!stack.empty()) {
     ch  = stack.pop();
     reversed = reversed + ch.charValue();
   }
   System.out.println("Reversed String: " + reversed);
} // main()
} // StackTest class
\end{lstlisting}
\end{jjjlisting}


\end{ANS}

\secEXRHtwo{Exercises}
%\addcontentsline{toc}{section}{\S~~~ Exercises}
\begin{EXRtwo}

\item  Explain the difference between each of the following pairs of terms:
\begin{EXRtwoLL}
\begin{multicols}{2}
\item  {\it Stack} and {\it queue}.
\item  {\it Static structure} and {\it dynamic structure}.
\item  {\it Data structure} and {\it Abstract Data Type}.
\item  {\it Push} and {\it pop}.
\item  {\it Enqueue} and {\it dequeue}.
\item  {\it Linked list} and {\it node}.
\end{multicols}
\end{EXRtwoLL}
\marginnote{\raggedright{\bf Note:} For programming 
exercises, {\bf first} draw
a UML class diagram describing all classes and
their inheritance relationships and/or associations.}

\item  Fill in the blanks.
\begin{EXRtwoLL}\baselineskip=12.5pt
\item  An {\it Abstract Data Type} consists of two
main parts: \rule{40pt}{0.5pt} and \rule{40pt}{0.5pt}\,.
\item  An object that contains a variable that refers to an object of
the same class is a \rule{40pt}{0.5pt}\,.
\item  One application for a \rule{40pt}{0.5pt} is to manage the method
call and returns in a computer program.
\item  One application for a \rule{40pt}{0.5pt} is to balance the parentheses
in an arithmetic expression.
\item  A \rule{40pt}{0.5pt} operation is one that starts at the beginning
of a list and processes each element.
\item  A vector is an example of a \rule{40pt}{0.5pt} data structure.
\item  An array is an example of a \rule{40pt}{0.5pt} data structure.
\item  By default, the initial value of a reference variable
is \rule{40pt}{0.5pt}\,.
\end{EXRtwoLL}\baselineskip=11pt

\item  Add a {\tt removeAt()} method to the {\tt List} class to
return the object at a certain index location in the list.  This method
should take an {\tt int} parameter, specifying the object's position
in the list, and it should return an {\tt Object}.

\item  Add an {\tt insertAt()} method to the {\tt List} class that
will insert an object at a certain position in the list.  This method
should take two parameters, the {\tt Object} to be inserted, and an
{\tt int} to designate where to insert it.   It should return a
{\tt boolean} to indicate whether the insertion was successful.


\item  Add a {\tt removeAll()} method to the {\tt List} class.
This {\tt void} method should remove all the members of the list.

\item  Write an {\tt int}  method named {\tt size()} that returns the
number of elements in a {\tt List}.

\item  Write an {\tt boolean}  method named {\tt contains(Object o)} 
that returns
{\tt true} if its {\tt Object} parameter is contained in the list.

\item  The {\it head} of a list is the first element in the list.
The {\it tail} of a list consists of all the elements except the
head.  Write a method named {\tt tail()} that returns a reference to
the tail of the list.  Its return value should be {\tt Node}.

\item  Write a program that uses the {\tt List} ADT
to store a list of 100 random floating-point numbers.   Write methods
to calculate the average of the numbers.

\item  Write a program that uses the {\tt List} ADT
to store a list of {\tt Student} records, using a variation of the Student
class defined in Chapter 11.  Write a method to calculate the mean
grade point average for all students in the list.

\item  Write a program that creates a copy of a {\tt List}.
It is necessary to copy each node of the list.  This will require that
you create new nodes that are copies of the nodes in the original
list.  To simplify this task, define a copy constructor for your node
class and then use that to make copies of each node of the list.

\item  Write a program that uses a {\tt Stack} ADT to determine
if a string is a palindrome---spelled the same way backward and
forward.

\item  Design and write a program that uses a {\tt Stack} to determine
whether a parenthesized expression is well-formed.  Such an expression
is well formed only if there is a closing parenthesis for each opening
parenthesis.

\item  Design and write a program that uses {\tt Stack}s to determine
whether an expression involving both parentheses and square brackets
is well formed.

\item  Write a program that links two lists together, appending
the second list to the end of the first list.

\item  Design a {\tt Stack} class that uses
a {\tt Vector} instead of a linked list to store its elements.  This
is the way Java's {\tt Stack} class is defined.

\item  Design a {\tt Queue} class that uses
a {\tt Vector} instead of a linked list to store its elements.

\item Write a program that uses the {\tt List<E>} and {\tt
LinkedList<E>} classes to store a list of {\tt Student} records, using
a variation of the {\tt Student} class defined in Chapter 11. Write a
method to calculate the mean grade point average for all students in
the list.

\item Write an implementation of the {\tt insert()} method of the {\tt
PhoneTree} class described at the end of this chapter.

\item Write an implementation of the {\tt insert()} method of the {\tt
PhoneTreeNode} class described at the end of this chapter.

\item  {\bf Challenge:} Design a {\tt List} class, similar in
functionality to the one we designed in this chapter, that uses an
{\it array} to store the list's elements.  Set it up so that the middle
of the array is where the first element is placed.  That way you can still
insert at both the front and rear of the list.   One limitation of this
approach is that, unlike a linked list, an array has a fixed
size.  Allow the user to set the initial size of the array in a
constructor, but if the array becomes full, don't allow any further
insertions.

\item  {\bf Challenge:} Add a method to the program in the previous
exercise that lets the user increase the size of the array used to
store the list.

\item  {\bf Challenge:} Recursion is a useful technique for list processing.
Write recursive versions of the {\tt print()} method and the lookup-by-name
method for the {\tt PhoneList}. ({\it Hint}: The base case in processing
a list is the empty list.  The recursive case should handle the head of
the list and then recurse on the tail of the list.  The tail of the
list is everything but the first element.)


\item  {\bf Challenge:} Design an {\tt OrderedList} class.  An ordered
list is one that keeps its elements in order.  For example, if it's an
ordered list of integers, then the first integer is less than or equal
to the second, the second is less than or equal to the third, and so
on.  If it's an ordered list of employees, then perhaps the employees
are stored in order according to their social security numbers.  The
{\tt OrderedList} class should contain an {\tt insert(Object o)}
method that inserts its object in the proper order.  One major
challenge in this project is designing your class so that it will work
for any kind of object.  ({\it Hint:} Define an {\tt Orderable} interface
that defines an abstract {\tt precedes()} method.  Then
define a subclass of {\tt Node} that implements {\tt Orderable}. This
will let you compare any two {\tt Node}s to see which one comes before
the other.)
\end{EXRtwo}
%