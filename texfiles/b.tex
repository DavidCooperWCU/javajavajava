\setcounter{table}{0}
\setcounter{figure}{0}
\renewcommand{\thetable}{\mbox{B.\arabic{table}}}%
\renewcommand{\thefigure}{\mbox{B--\arabic{figure}}}%

\chapter{The Java Development Kit}

\markboth{{\color{cyan}APPENDIX\,B\,\,$\bullet$\,\,}The Java Development Kit}
{{\color{cyan}APPENDIX\,B\,\,$\bullet$\,\,}The Java Development Kit}

\label{appendix-jdk}


\noindent The Java Development Kit (JDK) for Java$^{\rm TM}$\,2 Platform Standard Edition
is a set of command-line tools for developing Java programs. It is
available for free in versions for recent editions of Microsoft
Windows, Linus, Macintosh OS X, and Solaris (Sun Microsystems).

Download information and documentation are available for the entire
range of products associated with the Java$^{\rm TM}$\,2 Platform,
Standard Edition (Java SE) at;

\begin{jjjlisting}
\begin{lstlisting}[commentstyle=\color{black}]
http://java.sun.com/j2se/
\end{lstlisting}
\end{jjjlisting}

\noindent This appendix summarizes some of the primary tools available
in the JDK.  For more detailed information you should consult Sun's
Web site.

Table~B.1 provides a summary of some of the JDK tools.

\begin{table}[htb]
%\hphantom{\caption{Tools included in the Java Development Kit.}}
\TBT{0pc}{Tools included in the Java Development Kit.}
\hspace*{-6pt}\begin{tabular}{ll}
\multicolumn{2}{l}{\color{cyan}\rule{27pc}{1pt}}\\[2pt]
%%\TBCH{{\bf Tool Name}} & \TBCH{{\bf Description}}
{\bf Tool Name} & {\bf Description}
\\[-4pt]\multicolumn{2}{l}{\color{cyan}\rule{27pc}{0.5pt}}\\[2pt]
javac      &  Java compiler. Translates source code into bytecode.\cr
java       &  Java interpreter. Translates and executes bytecode.\cr
javadoc    &  Java documentation generator. Creates HTML pages from\cr
           &  documentation comments embedded in Java programs.\cr
appletviewer  &  Appletviewer. Used instead of a browser to run Java applets.\cr
jar        &  Java archive manager. Manages Java archive (JAR) files.\cr
jdb        &  Java debugger. Used to find bugs in a program.\cr
javap      &  Java disassembler. Translates bytecode into Java source code.
\\[-4pt]\multicolumn{2}{l}{\color{cyan}\rule{27pc}{1pt}}
\end{tabular}
\endTB
\end{table}

\noindent Sun Microsystems provides detailed instructions on how
to install JDK for Java SE on computers running any of the above operating
systems, including how to set the system's PATH and CLASSPATH variables.
Installation instructions can be located using the above link to
downloading information.

\section*{The Java Compiler: {\tt javac}}
\label{the-java-compiler}
\noindent The Java compiler ({\tt javac}) translates Java source files into Java
bytecode.  A Java source file must have the {\tt .java} extension.   The
{\tt javac} compiler will create a bytecode file with the same name
but with the {\tt .class} extension.  The {\tt javac} command takes
the following form:

\vspace{6pt plus3pt minus2pt}
\noindent {\tt javac} \qquad 
 [ {\it options} ] \qquad 
 {\it sourcefiles} \qquad 
 [ {\it files} ]

\vspace{6pt plus3pt minus2pt}\noindent The brackets in this expression indicate optional parts of
the command.   Thus, {\it options} is an optional list of command-line options 
(including the {\tt -classpath} option), and
{\it files} is an optional list of files, each of which contains a
list of Java source files.   The {\it files} option would be used if
you were compiling a very large collection of files, too large to list
each file individually on the command line.

Most of the time you would simply list the {\it sourcefiles} you are
compiling immediately after the word {\tt javac}, as in the
following example:

\begin{jjjlisting}
\begin{lstlisting}
javac MyAppletClass.java MyHelperClass.java
\end{lstlisting}
\end{jjjlisting}

\noindent Given this command, {\tt javac} will read class
definitions contained in {\tt MyAppletClass.java} and {\tt MyHelperClass.java}
in the current working directory
and translate them into bytecode files named {\tt MyAppletClass.class} and
{\tt MyHelperClass.class}.

If a Java source file contains inner classes, these would be compiled
into separate class files.  For example, if {\tt MyAppletClass.java}
contained an inner class named {\tt Inner}, {\tt javac} would compile
the code for the inner class into a file named {\tt
MyAppletClass\$Inner.class}.

If you are writing a program that involves several classes, it is not
necessary to list each individual class on the command line.  You must
list the main class---that is, the class where execution will begin.
The compiler will perform a search for all the other classes used in
the main class.  For example, if {\tt MyAppletClass} uses an instance
of {\tt MyHelperClass}, you can compile both classes with the following
command:

\begin{jjjlisting}
\begin{lstlisting}
javac MyAppletClass.java
\end{lstlisting}
\end{jjjlisting}

\noindent In this case, {\tt javac} will perform a search for
the definition of {\tt MyHelperClass}.

\subsection*{How Java Searches for Class Definitions}
\noindent When compiling a file, {\tt javac} needs a definition for every class
or interface that's used in the source file.  For example, if you are
creating a subclass of {\tt java.applet.Applet}, {\tt javac} will need
definitions for all of {\tt Applet}'s superclasses, including {\tt
Panel}, {\tt Container}, and {\tt Component}. The definitions for
these classes are contained in the {\tt java.awt} package.   Here's how
{\tt javac} will search for these classes.


{\tt Javac} will first search among its library files for 
definitions of classes, such as {\tt Applet}
and {\tt Panel}.  Next, {\tt javac} will search among the files and
directories listed on the user's {\it class path}. The class path is a
\marginnote{The Classpath}
system variable that lists all the user directories and files that
should be searched when compiling a user's program.  
JDK no longer requires a class path variable. The class path can
be set either by using the environment variable CLASSPATH or by using
the {\tt -classpath} option when invoking {\tt javac}. 
By default, JDK will check in the current working directory
for user classes. It doesn't require that the CLASSPATH variable be set. If this variable
{\bf is} set, it must include the current directory.
The preferred way to set the classpath is by using {\tt -classpath} option.
For example,

\begin{jjjlisting}
\begin{lstlisting}
javac -classpath ../source:. MyApplet.java
\end{lstlisting}
\end{jjjlisting}

\noindent will tell {\tt javac} to search in both the current directory (.)
and in the {\tt ../source} directory for user source files. Because the
details for setting the CLASSPATH variable are system dependent, it's best to
consult the online installation documentation to see exactly how this is done on
your system.

During a successful search, {\tt javac} may find a source file, a
class file, or both.   If it finds a class file but not source file,
{\tt javac} will use the class file.  This would be the case for Java
library code.  If {\tt javac} finds a source file but not a class file,
it will compile the source and use the resulting class file.  This
would be the case for the first compilation of one of your source
programs.   If {\tt javac} finds both a source and a class file, it
determines whether the class file is up-to-date.  If so, it uses it.  If
not, it compiles the source and uses the resulting class file.  This
would be the case for all subsequent compilations of one of your
source programs.

As noted earlier, if your application or applet uses several source
files, you need only provide {\tt javac} with the name of the main
application or applet file.   It will find and compile all the source
files, as long as they are located in a directory that's listed in the
class path.

\section*{The Java Interpreter: {\tt java}}
\label{the-java-interpreter}
\noindent The {\tt java} interpreter launches a Java application.   This command
takes one of the following forms:

\vspace{6pt plus3pt minus2pt}\columnsep = 0pt\begin{multicols}{5}
\small
\parindent=0pc
{\tt java} 

{\tt java} 

 [ {\it options} ] 

 [ {\it options} ] 

 classname 

 {\bf -jar} 

 [ {\it argument \dots } ] 

 file.jar 

\mbox{ }

 [ {\it argument \dots } ]
\end{multicols}

\vspace{6pt plus3pt minus2pt}\noindent If the first form is used, {\tt java} starts a Java runtime
environment.   It then loads the specified {\it classname} and runs
that class's {\tt main()} method, which must be declared as follows:

\begin{jjjlisting}
\begin{lstlisting}
public static void main(String args[])
\end{lstlisting}
\end{jjjlisting}

\noindent The {\tt String} parameter {\tt args[]} is an array
of strings, which is used to pass any {\it argument}s listed on the
command line.  Command-line arguments are optional.

If the second form of the {\tt java} command is used, {\tt java} will
load the classes and resources from the specified {\it Java archive
(JAR)}. In this case, the special {\tt -jar} option flag must be
specified. The {\bf options} can also include many other command-line
options, including the {\tt -classpath} option.


\section*{The {\tt appletviewer}}
\label{the}
\noindent The {\tt appletviewer} tool lets you run Java applets without using
a Web browser.   This command takes the following form:

\vspace{6pt plus3pt minus2pt}
\noindent {\tt appletviewer} \qquad 
 [ {\it threads flag} ] \qquad 
 [ {\it options } ] \qquad 
 url \dots

\vspace{6pt plus3pt minus2pt}\noindent The optional {\it threads flag} tells Java which of the various
threading options to use.  This is system dependent.  For details on
this feature and the command line {\it options}, refer to Sun's Web site.

The appletviewer will connect to one or more HTML documents specified
by their {\it Uniform Resource Locators (URLs)}. It will display each
applet referenced in those documents in a separate window.  Some
example commands would be

\begin{jjjlisting}
\begin{lstlisting}
appletviewer http://www.domain.edu/~account/myapplet.html
appletviewer myapplet.html
\end{lstlisting}
\end{jjjlisting}

\noindent In the first case, the document's full path name is
given.  In the second case, since no host computer is mentioned, {\tt
appletviewer} will assume that the applet is located on the local host
and will search the class path for {\tt myapplet.html}.
\marginnote{\vspace{-24pt}AppletViewer tags}

Once {\tt appletviewer} retrieves the HTML document, it will find the
applet by looking for either the {\tt object}, {\tt embed}, or {\tt
applet} tags within the document.  The {\tt appletviewer} ignores all
other HTML tags.   It just runs the applet.  If it cannot find one of
these tags, the appletviewer will do nothing.   If it does locate an
applet, it starts a runtime environment, loads the applet, and then
runs the applet's {\tt init()} method.  The applet's {\tt init()} must
have the following method signature:

\begin{jjjlisting}
\begin{lstlisting}
public void init()
\end{lstlisting}
\end{jjjlisting}

\subsection*{The {\tt applet} Tag}
\noindent 
The {\tt applet} tag is the original HTML 3.2 tag used for embedding
applets within an HTML document.  If this tag is used, the applet will
be run by the browser, using the browser's own implementation of the
Java Runtime Environment (JRE).

\spstrict Note, however, that if your applet uses the latest Java language
features and the browser is not using the latest version of JRE, the
applet may not run correctly.  For example, this might happen if your
applet makes use of Swing features that are not yet supported in the
browser's implementation of the JRE.~In that case, your applet won't
run under that browser.\spnormalstr


To ensure that the applet runs with the latest version of the JRE---the 
one provided by Sun Microsystems---you can also use the {\tt
object} or the {\tt embed} tags.  These tags are used to load the
appropriate version of the JRE into the browser as a {\it plugin}
module.   A plugin is a helper program that extends the browser's
functionality.

The {\tt applet} tag takes the following form:

\begin{jjjlisting}
\begin{lstlisting}
<applet
    code="yourAppletClass.class"
    object="serializedObjectOrJavaBean"
    codebase="classFileDirectory"
    width="pixelWidth"
    height="pixelHeight"
>
   <param name="..." value="...">
    ...
   alternate-text
</applet>
\end{lstlisting}
\end{jjjlisting}

\noindent You would use only the {\tt code} or {\tt object} attribute,
not both.  For the programs in this book, you should always use the
{\tt code} tag.  The {\tt code} tag specifies where the program
will begin execution---that is, in the applet class.

The optional {\tt codebase} attribute is used to specify a relative
path to the applet.  It may be omitted if the applet's class file is in
the same directory as the HTML document.

The {\tt width} and {\tt height} attributes specify the initial
dimensions of the applet's window.  The values specified in the applet
tag can be overridden in the applet itself by using the {\tt
setSize()} method, which the applet inherits from the {\tt
java.awt.Component} class.

The {\tt param} tags are used to specify arguments that can be
retrieved when the applet starts running (usually in the applet's {\tt
init()} method).  The methods for retrieving parameters are defined
in the {\tt java.applet.Applet} class.

Finally, the {\tt alternative-text} portion of the applet tag provides
text that would be displayed on the Web page if the appletviewer or
browser is unable to locate the applet.

Here's a simple example of an applet tag:

\begin{jjjlisting}
\begin{lstlisting}
<applet
    code="HelloWorldApplet.class"
    codebase="classfiles"
    width="200"
    height="200"
>
    <param name="author" value="Java Java Java">
    <param name="date" value="May 1999">

    Sorry, your browser does not seem to be able to
    locate the HelloWorldApplet.
</applet>
\end{lstlisting}
\end{jjjlisting}

\noindent In this case, the applet's code is stored in a
file name {\tt HelloWorld\-Applet.class}, which is stored in the {\tt
classfiles} subdirectory---that is, a subdirectory of the directory
containing the HTML file.   The applet's window will be $200 \times 200$
pixels.  And the applet is passed the name of the program's author and
date it was written.  Finally, if the applet cannot be located, the
``Sorry \dots '' message will be displayed instead.

\subsection*{The {\tt object} Tag}
\noindent The {\tt object} tag is the HTML 4.0 tag for embedding applets and
multimedia objects in an HTML document.  It is also an Internet
Explorer (IE) 4.x extension to HTML.~It allows IE to run a Java applet
using the latest JRE plugin from Sun.  The {\tt object} tag takes the
following form:

\begin{jjjlisting}
\begin{lstlisting}
<object
    classid="name of the plugin program"
    codebase="url for the plugin program"
    width="pixelWidth"
    height="pixelHeight"
>
    <param name="code" value="yourClass.class">
    <param name="codebase" value="classFileDirectory">
     ...
    alternate-text
</object>
\end{lstlisting}
\end{jjjlisting}

\noindent Note that parameters are used to specify your
applet's code and codebase.  In effect, these are parameters
to the plugin module.  An example tag that corresponds
to the {\tt applet} tag for the {\tt HelloWorldApplet}
might be as follows:

\begin{jjjlisting}
\begin{lstlisting}
<object
    classid="clsid:8AD9C840-044E-11D1-B3E9-00805F499D93"
    codebase="http://java.sun.com/products/plugin/"
    width="200"
    height="200"
>
    <param name="code" value="HelloWorldApplet.class">
    <param name="codebase" value="classfiles">
    <param name="author" value="Java Java Java">
    <param name="date" value="May 1999">

    Sorry, your browser does not seem to be able to
    locate the HelloWorldApplet.
</object>
\end{lstlisting}
\end{jjjlisting}

\noindent If the browser has an older version of the plug in than
shown in the {\tt codebase} attribute, the user will be prompted to
download the newer version. If the browser has the same or newer
version, that version will run. In theory Netscape 6 should also work
the same as IE.  For further details on how to use the {\tt object}
tag, see Sun's plugin site at:

\begin{jjjlisting}
\begin{lstlisting}[commentstyle=\color{black}]
http://java.sun.com/products/plugin/
\end{lstlisting}
\end{jjjlisting}


\vspace{-6pt}\subsection*{The {\tt embed} Tag}
\noindent The embed tag is Netscape's version of the {\tt applet} and {\tt
object} tags.  It is included as an extension to HTML 3.2.  It can be used
to allow a Netscape 4.x browser to run a Java applet using the latest
Java plugin from Sun.  It takes the following form:

\begin{jjjlisting}
\begin{lstlisting}
<embed
    type="Type of program"
    code="yourAppletClass.class"
    codebase="classFileDirectory"
    pluginspage="location of plugin file on the web"
    width="pixelWidth"
    height="pixelHeight"
>
    <noembed>
    Alternative text
    </noembed>
</embed>
\end{lstlisting}
\end{jjjlisting}

\noindent The {\tt type} and {\tt pluginspage} attributes
are not used by the appletviewer, but they are necessary for
browsers.  They would just be ignored by the appletviewer.

For example, an {\tt embed} tag for {\tt HelloWorldApplet}
would be as follows:

\begin{jjjlisting}
\begin{lstlisting}
<EMBED
    type="application/x-java-applet;version=1.4"
    width="200"
    height="200"
    code="HelloWorldApplet.class"
    codebase="classfiles"
    pluginspage="http://java.sun.com/products/plugin/"
 >
   <NOEMBED>
       Sorry.  This page won't be able to run this applet.
   </NOEMBED>
</EMBED>
\end{lstlisting}
\end{jjjlisting}

It may be possible to combine the {\tt applet}, {\tt embed}, and {\tt
object} tags in the same HTML file.  Sun provides much more information,
as well as demo programs on its plugin website:

\begin{jjjlisting}
\begin{lstlisting}[commentstyle=\color{black}]
http://java.sun.com/products/plugin/
\end{lstlisting}
\end{jjjlisting}

\vspace{-6pt}\section*{The Java Archiver {\tt jar} Tool}
\label{the-java-archiver-tool}
\noindent The {\tt jar} tool can be used to combine multiple files into a single
JAR archive file.  Although the {\tt jar} tool is a general-purpose
archiving and compression tool, it was designed mainly to facilitate
the packaging of Java applets and applications into a single file.


The main justification for combining files into a single archive and
compressing the archive is to improve download time.  The {\tt jar}
command takes the following format:

\vspace{6pt plus4pt minus2pt}
\noindent {\tt jar} \qquad [ {\it options} ] destination-file \qquad 
 input-file 
\qquad  [ input-files ]

\vspace{6pt plus4pt minus2pt}\noindent For an example of its usage, let's use it to
archive the files involved in the {\tt WordGuess} example in
Chapter~9. As you may recall, this example used classes, such as {\tt
TwoPlayerGame}, and interfaces, such as {\tt IGame}, that were
developed in earlier sections of the chapter.  So, to help manage the
development of {\tt WordGuess}, it would be useful to have a library
containing those files that must be linked together when we compile
{\tt WordGuess}. 

This is a perfect use of a {\tt jar} file. Let's name our library
{\tt nplayerlib.jar}. We choose this name because the library
can be used to create game programs that have {\em N} players,
including two-player games.

For the two-player game, {\tt WordGuess}, we want to combine all the
{\tt .class} files needed by {\tt WordGuess.java} into a single jar
file.  Here's a list of the files we want to archive:

\begin{jjjlisting}
\begin{lstlisting}
CLUIPlayableGame.class
ComputerGame.class
GUIPlayableGame.class
IGame.class
KeyboardReader.class
Player.class
TwoPlayerGame.class
UserInterface.class
\end{lstlisting}
\end{jjjlisting}

\noindent Assuming all of these files are contained in a single
directory (along with other files, perhaps), then the command we use
from within that directory is as follows:

\begin{jjjlisting}
\begin{lstlisting}
jar cf nplayerlib.jar *.class
\end{lstlisting}
\end{jjjlisting}

\noindent In this case, the {\tt cf} options specify that we are
creating a jar file named {\tt animated.jar} that will consist of all
the files having the {\tt .class} suffix.  This will create a
file named {\tt nplayerlib.jar}.  To list the contents of this
file, you can use the following command:

\begin{jjjlisting}
\begin{lstlisting}
jar tf nplayerlib.jar
\end{lstlisting}
\end{jjjlisting}

\noindent Once we have created the jar file, we can copy it into the
directory that contains our source code for the {\tt WordGuess}
program.  We would then the following commands to include the code
contained in the library when we compile and run {\tt WordGuess.java}

\begin{jjjlisting}
\begin{lstlisting}
javac -classpath nplayerlib.jar:. WordGuess.java
java -classpath nplayerlib.jar:. WordGuess
\end{lstlisting}
\end{jjjlisting}

\noindent These commands, which use the {\tt -classpath} option,
tell {\tt javac} and {\tt java} to include code from the {\tt
nplayerlib.jar}. The notation \verb|:.|, links code in the current
directory (\.) with the library code.

Once we have created a library, we can also use it for
Java applets.  For example, suppose we have developed an applet
version of the {\tt WordGuess}  game and linked all of the
applet's code into a jar file named {\tt wordguessapplet.jar}.
The following HTML file would allow users to download the applet
from their web browser:

\begin{jjjlisting}
\begin{lstlisting}
<html>
    <head><title>WordGuess Applet</title></head>
    <body>
    <applet
        archive="wordguessapplet.jar"
        code="WordGuessApplet.class"
        width=350 height=350
    >
        <parameter name="author" value="Java Java Java">
        <parameter name="date" value="April 2005">
    </applet>
    </body>
</html>
\end{lstlisting}
\end{jjjlisting}

\noindent When specified in this way, the browser will take care
of downloading the archive file and extracting the individual files
needed by the applet.  Note that the {\tt code} attribute must still
designate the file where the program will start execution.

\section*{The Java Documentation Tool: {\tt javadoc}}
\label{the-java-documentation-tool}
\noindent The {\tt javadoc} tool parses the declarations and documentation
comments in a Java source file and generates a set of HTML pages that
describes the following elements: public and protected classes, inner
classes, interfaces, constructors, methods, and fields.

The {\tt javadoc} tool can be used on a single file or an entire package
of files.  Recall that a Java documentation comment is one that begins
with \verb|/**| and ends with \verb|*/|.  These are the comments that
are parsed by {\tt javadoc}.

The {\tt javadoc} tool has many features, and it is possible to use
Java {\it doclets} to customize your documentation.  For full details
on using the tool, it is best to consult Sun's Web site.  To illustrate
how it might be used, let's just look at a simple example.

The {\tt FirstApplet} program from Chapter~\ref{ch:intro2}
contains documentation comments.  It was processed using the
following command:

\begin{jjjlisting}
\begin{lstlisting}
javadoc FirstApplet.java
\end{lstlisting}
\end{jjjlisting}

\noindent {\tt javadoc} generated the following HTML documents:

\begin{jjjlistingleft}[30pc]{-4pc}
\begin{lstlisting}[stringstyle=\color{black}]
FirstApplet.html      -The main documentation file
allclasses-frame.html -Names and links to all the classes used in 
                       FirstApplet
overview-tree.html    -A tree showing FirstApplet's place 
                       in the class hierarchy
packages.html         -Details on the packages used in FirstApplet
index.html            -Top-level HTML document for 
                       FirstApplet documentation
index-all.html        -Summary of all methods and variables in
                       FirstApplet
\end{lstlisting}
\end{jjjlistingleft}

\noindent To see how the documentation appears, review the {\tt FirstApplet.java}
source file and the documentation it generated.  Both are available at

\begin{jjjlisting}
\begin{lstlisting}[commentstyle=\color{black}]
http://www.prenhall.com/morelli/
\end{lstlisting}
\end{jjjlisting}



%
