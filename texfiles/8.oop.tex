%%%  Chapter 8: Inheritance and Polymorphism
%%%  3rd Edition

\setcounter{SSTUDYcount}{1}
\setcounter{chapter}{7}
\chapter{Inheritance and Polymorphism}
\label{chapter-inheritance}


\CObegin
\secCOBH{Objectives}
\label{objectives}
\noindent After studying this chapter, you will
\begin{COBL}
\item  Understand the concepts of inheritance and polymorphism.
\item  Know how Java's dynamic binding mechanism works.
\item  Be able to design and use abstract methods and classes.
\item  Be able to design and use polymorphic methods.
\item  Gain a better understanding of object-oriented design.
\end{COBL}

\secCOLH{Outline}
\label{outline}
\begin{COL}
\item Introduction
\item Java's Inheritance Mechanism
\item Abstract Classes, Interfaces, and Polymorphism
\item Example: A Toggle Button
\item[] {{\color{cyan}Special Topic:} Historical Cryptography}
\item Example: The Cipher Class Hierarchy
\item Case Study: A Two Player Game Hierarchy
\item Principles of Object-Oriented Design
\par\small\item[] Chapter Summary
\par\small\item[] Solutions to Self-Study Exercises
\par\small\item[] Exercises
\end{COL}
\COend

\section{Introduction}
\label{introduction}

Among the most important concepts in object oriented programming are
the concepts of inheritance and polymorphism.  We first introduced the
idea of inheritance in Chapter~0. There we compared inheritance to the
natural form of inheritance, in which horses and cows share certain
inherited characteristics, such as being warm-blooded, by virtue of
their being mammals. We also gave the example of a hierarchy of
chess pieces and showed how different kinds of chess pieces, such as
{\tt Pawn} and {\tt Bishop}, inherited certain shared characteristics
from their {\tt ChessPiece} superclass.

We took a more technical look at inheritance in Chapter~3, where we
talked about the {\tt toString()} method and how it is inherited from
the {\tt Object} class. We illustrated there how subclasses of {\tt
Object} could override the inherited {\tt toString()} method in order
to customize it for their purposes. We also introduced the idea of
polymorphism, in which a method call, such as {\tt obj.toString()},
can have different behaviors depending on the type of object, {\tt
obj}, on which it is called.

In Chapter 4, we continued introducing inheritance and polymorphism,
when we learned about Java's Abstract Windowing Toolkit (AWT) and
Swing hierarchies, the class hierarchies that are used to create
Graphical User Interfaces (GUIs).  We also learned how to extend a
class to create our own subclass, and we made limited use of
inheritance in the design of the {\tt SimpleGUI} class.  We were also
introduced to the concept of a Java interface, and we learned how to
use the {\tt ActionListener} interface to enable a {\tt SimpleGUI} to
handle action events while the GUI is running.

In this chapter we will take a much closer look at these important
object-oriented concepts.  We will learn how Java's {\em dynamic
binding} mechanism works and how it makes polymorphism possible.  Most
importantly, we will see why inheritance and polymorphism are
important elements of object-oriented design, and we will learn how to
use these important tools to design several different programs.  In
keeping with our running games example, we will develop a {\tt
TwoPlayerGame} hierarchy and show how it can simplify the
implementation of {\tt OneRowNim} and other two-player games.

\section{Java's Inheritance Mechanism}
\label{javas-inheritance-mechanism}

As we described in Chapter~0, {\bf class inheritance} is the mechanism
whereby a class acquires ({\em inherits}) the methods and variables of
its superclasses.  To remind you of the basic concept, let's repeat an
earlier example: Just as horses inherit the attributes and behaviors
associated with mammals and vertebrates, a Java subclass inherits the
attributes and behaviors of its superclasses.

\marginfig{chptr08/horsehier.eps}{A class hierarchy for horses.}
{fig-horsehier}

Figure~\ref{fig-horsehier} uses a UML diagram to illustrate the relationships among
horses, mammals, vertebrates, and animals. As the root of the
hierarchy, which is always shown at the top, the Animal class contains
the most general attributes, such as being alive and being able to
move. All animals share these attributes. The class of vertebrates is
a somewhat more specialized type of animal, in that vertebrates have
backbones. Similarly, the class of mammals is a further specialization
over the vertebrates in that mammals are warm-blooded and nurse their
young.  Finally, the class of horses is a further specialization over
the class of mammals, in that all horses have four legs. Some mammals,
such as humans and penguins, do not have four legs.  Thus, by virtue
of its class's position in this hierarchy, we can infer that a horse
is a living, moving, four-legged vertebrate, which is warm blooded and
nurses its young.

We have deliberately used an example from the natural world to show
that the concept of inheritance in Java is inspired by its counterpart
in the natural world.  But how exactly does the concept of inheritance
apply to Java (and to other object-oriented languages)? And, more
importantly, how do we use the inheritance mechanism in
object-oriented design?

\subsection{Using an Inherited Method}

In Java, the public and protected instance methods and instance
variables of a superclass are inherited by all of its subclasses.
This means that objects belonging to the subclasses can use the
inherited variables and methods as their own.

We have already seen some examples of this in earlier chapters. For
example, recall that by default all Java classes are subclasses of the
{\tt Object} class, which is the most general class in Java's class hierarchy.
One public method that is defined in the {\tt Object} class is the
{\tt toString()} method.  Because every class in the Java hierarchy is
a subclass of {\tt Object}, every class inherits the {\tt toString()}
method.  Therefore, {\tt toString()} can be used with any Java object.

To illustrate this, suppose we define a {\tt Student} class as follows:

\begin{jjjlisting}
\begin{lstlisting}
public class Student {
    protected String name;
    public Student(String s) {
        name = s;
    }
    public String getName() {
        return name;
    }
}
\end{lstlisting}
\end{jjjlisting}

\noindent Figure~\ref{fig-student1} shows the relationship between this class
\marginfig{chptr08/student1.eps}{The {\tt Student} class hierarchy.}
{fig-student1}
and the {\tt Object} class.  As a subclass of {\tt Object}, the {\tt
Student} class inherits the {\tt toString()} method. Therefore, for a
given {\tt Student} object, we can call its {\tt toString()} as
follows:

\begin{jjjlisting}
\begin{lstlisting}
Student stu = new Student("Stu");
System.out.println(stu.toString());
\end{lstlisting}
\end{jjjlisting}

\noindent How does this work? That is, how does Java know where to
find the {\tt toString()} method, which, after all, is not defined in
the {\tt Student} class?  The answer to this question is crucial to
understanding how Java's inheritance mechanism works.  

Note in this example that the variable {\tt stu} is declared to be of
type {\tt Student} and is assigned an instance of the {\tt Student}
class. When the expression {\tt stu.toString()} is executed, Java will
first look in the {\tt Student} class for a definition of the {\tt
toString()} method. Not finding one there, it will then search up the
{\tt Student} class hierarchy (Fig.~8.2) until it finds a public or
protected definition of the {\tt toString()} method. In this case, it
finds a {\tt toString()} method in the {\tt Object} class and it
executes that implementation of {\tt toString()}.  As you know from
Chapter~3, this would result in the expression {\tt stu.toString()}
returning something like:

\begin{jjjlisting}
\begin{lstlisting}
Student@cde100
\end{lstlisting}
\end{jjjlisting}

\noindent The default implementation of {\tt toString()} returns the
name of the object's class and the address ({\tt cde100}) where the
object is stored in memory. However, this type of result is much too
general and not particularly useful.

\subsection{Overriding an Inherited Method}

In Chapter~3 we pointed out that the {\tt toString()} method is
designed to be {\em overridden}---that is, to be redefined in
subclasses of {\tt Object}. Overriding {\tt toString()} in a subclass
provides a customized string representation of the objects in that
subclass.  We showed that by redefining {\tt toString()} in our {\tt
OneRowNim} class, we customized its actions so that it returned
useful information about the current state of a {\tt
OneRowNim} game.

To override {\tt toString()} for the {\tt Student} class, let's add
the following method definition to the {\tt Student} class:

\begin{jjjlisting}
\begin{lstlisting}
public String toString() {
  return "My name is " + name +  " and I am a Student.";
}
\end{lstlisting}
\end{jjjlisting}

\marginfig{chptr08/student2.eps}{The revised {\tt Student} class hierarchy.}
{fig-student2}


\noindent Given this change, the revised {\tt Student} class hierarchy
is shown in Figure~\ref{fig-student2}.  Note that both {\tt Object} and {\tt
Student} contain implementations of {\tt toString()}. Now when the
expression {\tt stu.toString()} is invoked, the following, more
informative, output is generated:

\begin{jjjlisting}
\begin{lstlisting}
    My name is Stu and I am a Student.
\end{lstlisting}
\end{jjjlisting}

\noindent In this case, when Java encounters the method call {\tt
stu.toString()}, it invokes the {\tt toString()} method that it finds
in the {\tt Student} class (Fig.~8.3).

These examples illustrate two important object-oriented concepts:
inheritance and method overriding.

\JavaTIP[false]{EFFECTIVE DESIGN}{Inheritance.}{The public and protected
instance methods (and variables) in a class can be used 
by objects that belong to the class's subclasses. }

\JavaTIP[false]{EFFECTIVE DESIGN}{Overriding a Method.}{Overriding an inherited
method is an effective way to customize that method for a particular
subclass.}

\subsection{Static Binding, Dynamic Binding \\and Polymorphism}

The mechanism that Java uses in these examples is known as {\bf
dynamic binding}, in which the association between a method call and
the correct method implementation is made at run time. In dynamic
binding a method call is bound to the correct implementation of the
method at run time by the Java Virtual Machine (JVM).

Dynamic binding is contrasted with {\bf static binding}, the mechanism
by which the Java compiler {\em resolves} the association between a
method call and the correct method implementation when the program is
compiled. In order for dynamic binding to work, the JVM needs to
maintain some kind of representation of the Java class hierarchy,
including classes defined by the programmer.  When the JVM encounters
a method call, it uses information about the class hierarchy to {\em
bind} the method call to the correct implementation of that method.

In Java, all method calls use dynamic binding except methods that are
\marginnote{Dynamic binding}
declared {\tt final} or {\tt private}. Final methods cannot be
overridden, so declaring a method as {\tt final} means that the Java
compiler can bind it to the correct implementation.  Similarly,
private methods are not inherited and therefore cannot be overridden
in a subclass. In effect, private methods are final methods and the
compiler can perform the binding at compile time.

Java's dynamic-binding mechanism, which is also called {\em late}
\marginnote{Polymorphism}
binding or {\em run-time} binding, leads to what is know as {\em
polymorphism}.  {\bf Polymorphism} is a feature of object-oriented
languages whereby the same method call can lead to different behaviors
depending on the type of object on which the method call is made.  The
term {\em polymorphism} means, literally, having many (poly) shapes
(morphs).  Here's a simple example:

\begin{jjjlisting}[32pc]
\begin{lstlisting}
 Object obj;                        // Static type: Object
 obj = new Student("Stu");          // Actual type: Student
 System.out.println(obj.toString());// Prints "My name is Stu..."
 obj = new OneRowNim(11);           // Actual type: OneRowNim
 System.out.println(obj.toString());// Prints "nSticks = 11, player = 1"
\end{lstlisting}
\end{jjjlisting}

\noindent The variable {\tt obj} is declared to be of type {\tt
Object}.  This is its {\bf static} or {\em declared} type.  A
variable's static type never changes.  However, a variable also has an
{\bf actual} or {\em dynamic} type.  This is the actual type of the
object that has been assigned to the variable.  As you know, an {\tt
Object} variable can be assigned objects from any {\tt Object}
subclass.  In the second statement, {\tt obj} is assigned a {\tt
Student} object.  Thus, at this point in the program, the actual type
of the variable {\tt obj} is {\tt Student}.  When {\tt obj.toString()}
is invoked in the third line, Java begins its search for the {\tt
toString()} method at the {\tt Student} class, because that is the
variable's actual type.

In the fourth line, we assign a {\tt OneRowNim} object to {\tt obj},
thereby changing its actual type to {\tt OneRowNim}.  Thus, when {\tt
obj.toString()} is invoked in the last line, the {\tt toString()}
method is bound to the implementation found in the {\tt OneRowNim}
class.

Thus, we see that the same expression, {\tt obj.toString()}, is bound
alternatively to two different {\tt toString()} implementations, based
on the actual type of the object, {\tt obj}, on which it is invoked.
This is polymorphism and we will sometimes say that the {\tt
\marginnote{Polymorphic method}
toString()} method is a {\em polymorphic} method. A {\bf polymorphic
method} is a method signature that behaves differently when it is
invoked on different objects.  An overridden method, such as the {\tt
toString()} method, is an example of a polymorphic method,
because its use can lead to different behaviors depending upon the
object on which it is invoked.

The previous example is admittedly somewhat contrived. In some
object-oriented languages, a code segment such as that above would use
static binding rather than dynamic binding. In other words, the
compiler would be able to figure out the bindings.  So let's take an
example where static binding, also called {\em early} binding, is not
possible. Consider the following method definition:

\begin{jjjlisting}
\begin{lstlisting}
public void polyMethod(Object obj) {
  System.out.println(obj.toString()); // Polymorphic
}
\end{lstlisting}
\end{jjjlisting}

\noindent The method call in this method, {\tt obj.toString()}, can't
be bound to the correct implementation of {\tt toString()} until the
method is actually invoked---that is, at run time.  For example,
suppose we make the following method calls in a program:

\begin{jjjlisting}
\begin{lstlisting}
 Student stu = new Student("Stu");
 polyMethod(stu);
 OneRowNim nim = new OneRowNim();
 polyMethod(nim);  
\end{lstlisting}
\end{jjjlisting}

\noindent The first time {\tt polyMethod()} is called, the {\tt
obj.toString()} is invoked on a {\tt Student} object.  Java will use
its dynamic binding mechanism to associate this method call with the
{\tt toString()} implementation in {\tt Student} and output ``My name
is Stu and I am a Student.''  The second time {\tt polyMethod()} is
called, the {\tt obj.toString()} expression is invoked on a {\tt
OneRowNim} object. In this case, Java will bind the method call to the
implementation in the {\tt OneRowNim} class. The output generated in
this case will report how many sticks are left in the game. 

The important point here is that polymorphism occurs when an
overridden method is called on a superclass variable, {\tt obj}. In
such a case, the actual method implementation that is invoked is
determined at run time.  The determination depends on the type of
object that was assigned to the variable.  Thus, we say that the
method call {\tt obj.toString()} is polymorphic because it is bound to
different implementations of {\tt toString()} depending on the actual
type of the object that is bound to {\tt obj}.


\subsection{Polymorphism and Object-Oriented Design}

Now that we understand how inheritance and polymorphism work in Java,
it will be useful to consider an example that illustrates how these
mechanisms can be useful in designing classes and methods.  We have
been using the various {\tt System.out.print()} and {\tt
System.out.println()} methods since Chapter~1.  The {\tt print()} and
{\tt println()} methods are examples of {\bf overloaded}
methods---that is, methods that have the same name but different
\marginnote{Overloaded methods}
parameter lists.  Remember that a method's signature involves its
name, plus the type, number, and order of its parameters. Methods that
have the same name but different parameters are said to be overloaded.

Here are the signatures of some of the different {\tt print()} and
{\tt println()} methods:

\begin{jjjlisting}
\begin{lstlisting}
print(char c);           println(char c);
print(int i);            println(int i);
print(double d);         println(double d);
print(float f);          println(float f);
print(String s);         println(String s);
print(Object o);         println(Object o);
\end{lstlisting}
\end{jjjlisting}

\noindent Basically, there is a {\tt print()} and {\tt println()}
method for every type of primitive data, plus methods for printing any
type of object.  When Java encounters an expression involving {\tt
print()} or {\tt println()} it chooses which particular {\tt print()}
or {\tt println()} method to call. To determine the correct method,
Java relies on the differences in the signatures of the various {\tt
print()} methods. For example, because its argument is an {\tt int},
the expression {\tt print(5)} is associated with the method whose
signature is {\tt print(int i)} be cause its parameter is an {\tt
int}.

Note that there is only one set of {\tt print()} and {\tt println()}
methods for printing {\tt Object}s.  The reason is that polymorphism
is used by the {\tt print(Object o)} and {\tt println(Object o)}
methods to print any type of object.  While we do not have access to
the source code for these methods, we can make an educated guess that
their implementations utilize the polymorphic {\tt toString()} method,
as follows:

\begin{jjjlisting}
\begin{lstlisting}
 public void print(Object o) {
     System.out.print(o.toString());
 }
 
 public void println(Object o) {
     System.out.println(o.toString());
 }    
\end{lstlisting}
\end{jjjlisting}

\noindent Here again we have a case where an expression, {\tt
o.toString()}, is bound dynamically to the correct implementation of
{\tt toString()} based on the type of {\tt Object} that the variable
{\tt o} is bound to. If we call {\tt System.out.print(stu)}, where
{\tt stu} is a {\tt Student}, then the {\tt Student.toString()} method
is invoked. On the other hand, if we call {\tt
System.out.print(game)}, where {\tt game} is a {\tt OneRowNim}, then
the {\tt OneRowNim.toString()} method is invoked.

The beauty of using polymorphism in this way is the flexibility and
\marginnote{Extensibility}
extensibility that it allows.  The {\tt print()} and {\tt println()}
methods can print any type of object, even new types of objects that
did not exist when these library methods were written.

\secEXRHone{Self-Study Exercises}
\label{self-study-exercises}
\begin{SSTUDY}

\item  To confirm that the {\tt print()} and {\tt println()} methods
are implemented along the lines that we suggest here, compile and run
the {\tt TestPrint} program shown here. Describe how it confirms our claim.

\begin{jjjlisting}
\begin{lstlisting}
public class TestPrint {
    public static void main(String args[]) {
        System.out.println(new Double(56));
        System.out.println(new TestPrint());
    }
}
\end{lstlisting}
\end{jjjlisting}

\item Override the {\tt toString()} method in the {\tt TestPrint}
class and rerun the experiment. Describe how this adds further
confirmation to our claim.

\end{SSTUDY}


\subsection{Using {\tt super} to Refer to the Superclass}

One question that might occur to you is: Once you override the default
{\tt toString()} method, is it then impossible to invoke the default
method on a {\tt Student} object?  The default {\tt toString()} method
(and any method from an object's superclass) can be invoked using the
{\tt super} keyword.  For example, suppose that within the {\tt
Student} class, you wanted to concatenate the result of both the
default and the new {\tt toString()} methods. The following expression
would accomplish that:

\begin{jjjlisting}
\begin{lstlisting}
super.toString() + toString()
\end{lstlisting}
\end{jjjlisting}

\noindent The {\tt super} keyword specifies that the first
{\tt toString()} is the one implemented in the superclass.  The second
{\tt toString()} refers simply to the version implemented within the
\marginnote{Keyword {\tt super}}
{\tt Student} class.  We will see additional examples of using the
{\tt super} keyword in the following sections.

\secEXRHone{Self-Study Exercises}
\label{self-study-exercises}
\begin{SSTUDY}

\item  Consider the following class definitions and determine the
output that would be generated by the code segment.

\begin{jjjlisting}
\begin{lstlisting}
 public class A {
     public void method() { System.out.println("A"); }
 }
 public class B extends A {
     public void method() { System.out.println("B"); }
 }

 // Determine the output from this code segment
 A a = new A();
 a.method();
 a = new B();
 a.method();
 B b = new B();
 b.method();
\end{lstlisting}
\end{jjjlisting}

\item  For the class {\tt B} defined in the previous exercise,
modify its {\tt method()} so that it invokes {\tt A}'s version
of {\tt method()} before printing out {\em B}.

\item  Given the definitions of the classes {\tt A} and {\tt B},
which of the following statements are valid? Explain.

\begin{jjjlisting}
\begin{lstlisting}
    A a = new B();
    a = new A();
    B b = new A();
    b = new B();
\end{lstlisting}
\end{jjjlisting}


\end{SSTUDY}

\subsection{Inheritance and Constructors}

Java's inheritance mechanism applies to a class's public and protected
instance variables and methods. It does not apply to a class's
constructors. To illustrate some of the implications of this language
feature, let's define a subclass of {\tt Student} called {\tt
CollegeStudent}:

\begin{jjjlisting}
\begin{lstlisting}
public class CollegeStudent extends Student {
    public CollegeStudent() { }
    public CollegeStudent(String s) {
        super(s);
    }
    public String toString() {
       return "My name is " + name +  
               " and I am a CollegeStudent.";
    }
}
\end{lstlisting}
\end{jjjlisting}

\noindent Because {\tt CollegeStudent} is a subclass of {\tt Student},
it inherits the public and protected instance methods and variables
from {\tt Student}. So, a {\tt CollegeStudent} has an instance
variable for {\tt name} and it has a public {\tt getName()}
\marginfig{chptr08/collstudent.eps}{The class hierarchy for {\tt CollegeStudent} .}
{fig-collstudent}
method. Recall that a {\tt protected} element, such as the {\tt name}
variable in the {\tt Student} class, is accessible only within the
class and its subclasses. Unlike {\tt public} elements, it is not
accessible to other classes.

Note that {\tt CollegeStudent} overrides the {\tt toString()} method,
giving it a more customized implementation.  The hierarchical
relationship between {\tt CollegeStudent} and {\tt Student} is shown
in Figure~\ref{fig-collstudent}.  A {\tt CollegeStudent} is a {\tt Student} and both
are {\tt Object}s. 

Note how we have implemented the {\tt CollegeStudent(String s)}
constructor. Because the superclass's constructors are not inherited,
we have to implement this constructor in the subclass if we want to be
able to assign a {\tt CollegeStudent}'s name during object
construction.  The method call, {\tt super(s)}, is used to invoke the
superclass constructor and pass it {\em s}, the student's name. The
superclass constructor will then assign {\em s} to the {\tt name}
variable.

As we have noted, a subclass does not inherit constructors from its
\marginnote{Constructor chaining} superclasses.  However, if the
subclass constructor does not explicitly invoke a superclass
constructor, Java will automatically invoke the default superclass
constructor---in this case, {\tt super()}.  By ``default superclass
constructor'' we mean the constructor that has no parameters.  For a
subclass that is several layers down in the hierarchy, this automatic
invoking of the {\tt super()} constructor will be repeated upwards
through the entire class hierarchy.  Thus when a {\tt CollegeStudent}
is constructed, Java will automatically call {\tt Student()} and {\tt
Object()}.  Note that if one of the superclasses does not contain a
default constructor, this will result in a syntax error.

If you think about this, it makes good sense. How else will the
inherited elements of the object be created? For example, in order for
a {\tt CollegeStudent} to have a {\tt name} variable, a {\tt Student}
object, where name is declared, must be created. The {\tt
CollegeStudent} constructor then extends the definition of the {\tt
Student} class.  Similarly, in order for a {\tt Student} object to
have the attributes common to all objects, an {\tt Object} instance
must be created and then extended into a {\tt Student}.

Thus, unless a constructor explicitly calls a superclass constructor,
Java will automatically invoke the default superclass constructors.
It does this {\em before} executing the code in its own constructor.
For example, if you had two classes, {\tt A} and {\tt B}, where {\tt
B} is a subclass of {\tt A}, then whenever you create an instance of
{\tt B}, Java will first invoke {\tt A}'s constructor before executing
the code in {\tt B}'s constructor.  Thus, Java's default behavior
during construction of {\tt B} is equivalent to the following
implementation of {\tt B}'s constructor:

\begin{jjjlisting}
\begin{lstlisting}
public B() {
    A();   // Call the superconstructor 
    // Now continue with this constructor's code
}
\end{lstlisting}
\end{jjjlisting}

\noindent Calls to the default constructors are made all the way
up the class hierarchy, and the superclass constructor is always
called before the code in the class's constructor is executed.

\secEXRHone{Self-Study Exercises}
\label{self-study-exercises}
\begin{SSTUDY}

\item  Consider the following class definitions and describe what would
be output by the code segment.

\begin{jjjlisting}
\begin{lstlisting}
 public class A {
     public A() { System.out.println("A"); }
 }
 public class B extends A {
     public B() { System.out.println("B"); }
 }
 public class C extends B {
     public C() { System.out.println("C"); }
 }

 // Determine the output.
 A a = new A();
 B b = new B();
 C c = new C();
\end{lstlisting}
\end{jjjlisting}

\end{SSTUDY}

     \section{Abstract Classes, Interfaces, \\ and Polymorphism}
\label{abstract-classes-interfacesand-polymorphism}

In Java, there are three kinds of polymorphism:

\begin{itemize}
\item Overriding an inherited method.
\item Implementing an abstract method.
\item Implementing a Java interface.
\end{itemize}

\noindent In the previous section we saw examples of the first type of
polymorphism.  All forms of polymorphism are based on Java's dynamic
binding mechanism. In this section we will develop an example that
illustrates the other two types of polymorphism and discuss some of
the design implications involved in choosing one or the other
approach.

\subsection{Implementing an Abstract Method}

An important feature of polymorphism is the ability to invoke a
polymorphic method that has been defined only abstractly in the
superclass. To illustrate this feature, we will develop a hierarchy of
simulated animals that make characteristic animal sounds, an example
that is widely used to illustrate polymorphism.

As we all know from our childhood, animals have distinctive ways of
\marginnote{Extensibility} 
speaking. A cow goes ``moo''; a pig goes ``oink''; and so on.  Let's
design a hierarchy of animals that simulates this characteristic by
printing the characteristic sounds that these animals make.  We want
to design our classes so that any given animal will return something
like ``I am a cow and I go moo,'' when we invoke the {\tt toString()}
method.  Moreover, we want to design this collection of classes so
that it is extensible---that is, so that we can continue to add new
animals to our menagerie without having to change any of the code in
the other classes.

%%\begin{figure}[b]
%%\figa{chptr08/animals.eps}
%%%\figaleft{chptr08/animals.eps}
%%{The {\tt Animal} class hierarchy.
%%\label{fig-animals}
%%\label{pg-fig-animals}}
%%\end{figure}

Figure~\ref{fig-animals} provides a summary of the design we will implement. The
{\tt Animal} class is an {\tt abstract} class. That's why its name is
italicized in the UML diagram. The reason that this class is abstract
is because its {\tt speak()} method is an {\bf abstract method}, which
is a method definition that does not contain an implementation. That
is, the method definition contains just the method's signature, not
its body. Any class that contains an abstract method, must itself be
\marginfig{chptr08/animals.eps}{The {\tt Animal} class hierarchy.}
{fig-animals}
declared abstract. Here is the definition of the {\tt Animal} class:

\begin{jjjlisting}
\begin{lstlisting}
public abstract class Animal {
    protected String kind; // Cow, pig, cat, etc.
         
    public Animal()  {  }
    public String toString() {
        return "I am a " + kind + " and I go " + speak();
    }
    public abstract String speak();   // Abstract method
}
\end{lstlisting}
\end{jjjlisting}

\noindent Note how we declare the abstract method ({\tt speak()}) and
the abstract class.  Because one or more of its methods is not
implemented, an abstract class cannot be instantiated. That is, you
cannot say:

\begin{jjjlisting}
\begin{lstlisting}
Animal animal = new Animal(); // Error: Animal is abstract
\end{lstlisting}
\end{jjjlisting}

\noindent Even though it is not necessary, we give the {\tt Animal} 
class a constructor.  If we had left this off, Java would have
supplied a default constructor that would be invoked when {\tt Animal}
subclasses are created.

Java has the following rules on using abstract methods and
\marginnote{Rules for abstract classes}
classes.

\begin{itemize}

\item Any class containing an {\tt abstract} method must be declared
an \mbox{\tt abstract} class.

\item An {\tt abstract} class cannot be instantiated.  It must be
subclassed.

\item A subclass of an {\tt abstract} class may be instantiated only
if it implements {\it all} of the superclass's {\tt abstract} methods.
A subclass that implements only some of the {\tt abstract} methods
must itself be declared {\tt abstract}.

\item A class may be declared {\tt abstract} even it contains no {\tt
abstract} methods.  It could, for example, contain instance variables
that are common to all its subclasses.
\end{itemize}

Even though an abstract method is not implemented in the superclass,
it can be called in the superclass. Indeed, note how the {\tt
toString()} method calls the abstract {\tt speak()} method. The reason
that this works in Java is due to the dynamic binding mechanism.  The
polymorphic {\tt speak()} method will be defined in the various {\tt
Animal} subclasses. When the {\tt Animal.toString()} method is called,
Java will decide which actual {\tt speak()} method to call based on
what subclass of {\tt Animal} is involved.  

Definitions for two such subclasses are shown in
Figure~\ref{fig-animalsubs}.  
\begin{figure}[htb]
\jjjprogstart
\begin{jjjlisting}
\begin{lstlisting}
public class Cat extends Animal {
    public Cat() {
        kind = "cat";
    }
    public String speak() {
        return "meow";
    }
}

public class Cow extends Animal {
    public Cow() {
        kind = "cow";
    }
    public String speak() {
        return "moo";
    }
}
\end{lstlisting}
\end{jjjlisting}
\jjjprogstop{Two {\tt Animal} subclasses.}
{fig-animalsubs}
\end{figure}
\noindent In each case the subclass extends the {\tt Animal} class and provides
its own constructor and its own implementation of the {\tt speak()}
method. Note that in their respective constructors, we can refer to
the {\tt kind} instance variable, which is inherited from the {\tt
Animal} class.  By declaring {\tt kind} as a {\tt protected} variable,
it is inherited by all {\tt Animal} subclasses but hidden from
all other classes.  On the other hand, if {\tt kind} had been declared
{\tt public}, it would be inherited by {\tt Animal} subclasses, but it
would also be accessible to every other class, which would violate the
information hiding principle.

Given these definitions, we can now demonstrate the power and
flexibility of inheritance and polymorphism.  Consider the following
code segment:

\begin{jjjlisting}
\begin{lstlisting}
 Animal animal = new Cow();
 System.out.println(animal.toString()); // A cow goes moo
 animal = new Cat();
 System.out.println(animal.toString()); // A cat goes meow
\end{lstlisting}
\end{jjjlisting}

\noindent We first create a {\tt Cow} object and then invoke its
(inherited) {\tt toString()} method. It returns, ``I am a cow and I go
moo.''  We then create a {\tt Cat} object and invoke its (inherited)
{\tt toString()} method, which returns, ``I am a cat and I go meow.''
In other words, Java is able to determine the appropriate
implementation of {\tt speak()} at run time in each case. The
invocation of the abstract {\tt speak()} method in the {\tt
Animal.toString()} method is a second form of polymorphism.

What is the advantage of polymorphism here? The main advantage is the
\marginnote{Advantage of polymorphism}
extensibility that it affords our {\tt Animal} hierarchy.  We can
define and use completely new {\tt Animal} subclasses without
redefining or recompiling the rest of the classes in the hierarchy.
Note that the {\tt toString()} method in the {\tt Animal} class does
not need to know what type of {\tt Animal} subclass will be executing
its {\tt speak()} method. The {\tt toString()} method will work
correctly for any subclass of {\tt Animal} because every non-abstract
subclass of {\tt Animal} must implement the {\tt speak()} method.

To get a better appreciation of the flexibility and extensibility of
this design, it might be helpful to consider an alternative design that
does not use polymorphism. One such alternative would be to define each
{\tt Animal} subclass with its own speaking method. A {\tt Cow} would
have a {\tt moo()} method; a {\tt Cat} would have a {\tt meow()} method;
and so forth.  Given this design, we could use a {\tt switch}
statement to select the appropriate method call. For example, consider
the following method definition:

\begin{jjjlisting}
\begin{lstlisting}
public String talk(Animal a) {
  if (a instanceof Cow)
     return "I am a " + kind + " and I go " + a.moo();
  else if (a instanceof Cat)
     return "I am a " + kind + " and I go " + a.meow();
  else
    return "I don't know what I am";
}
\end{lstlisting}
\end{jjjlisting}

\noindent In this example, we introduce the {\tt instanceof} operator,
which is a built-in boolean operator. It returns true if the object on
its left-hand side is an instance of the class on its right-hand side.

The {\tt talk()} method would produce more or less the same result. If
you call {\tt talk(new Cow())}, it will return ``I am a cow and I go
moo.''  However, with this design, it is not possible to extend the
{\tt Animal} hierarchy without rewriting and recompiling the {\tt
talk()} method.

Thus, one of the chief advantages of using polymorphism is the great
\marginnote{Extensibility}
flexibility and extensibility it affords. We can define new {\tt Animal}
subclasses and define their {\tt speak()} methods. These will
all work with the {\tt toString()} method in the {\tt Animal} class,
without any need to revise that method.

Another advantage of using abstract methods is the control that it
gives the designer of the {\tt Animal} hierarchy.  By making it an
abstract class with an abstract {\tt speak()} method, any non-abstract
{\tt Animal} subclass must implement the {\tt speak()} method. This
lends a great degree of predictability to the subclasses in the
hierarchy, making it easier to use them in applications.

\secEXRHone{Self-Study Exercises}
\label{self-study-exercises}
\begin{SSTUDY}

\item  Following the examples in this section, define an {\tt Animal} subclass
named {\tt Pig}, which goes ``oink.''

\item  Show how you would have to modify the {\tt talk()} method
defined above to incorporate the {\tt Pig} class. 

\end{SSTUDY}

\subsection{Implementing a Java Interface}

A third form of polymorphism results through the implementation of
Java {\bf interfaces}, which are like classes but contain only
abstract method definitions and constants ({\tt final}) variables.  An
\marginnote{Java interface}
interface cannot contain instance variables.  We have already seen
interfaces, such as when we encountered the {\tt ActionListener}
interface in Chapter~4.

The designer of an interface specifies what methods will be implemented
by classes that {\em implement} the interface. This is similar to what
we did when we implemented the abstract {\tt speak()} method in the
animal example. The difference between implementing a method from an
interface and from an abstract superclass is that a subclass {\em
extends} an abstract superclass but it {\em implements} an interface.

Java's interface mechanism gives us another way to design polymorphic
methods. To see how this works, we will provide an alternative design
for our animal hierarchy. Rather than defining {\tt speak()} as an
abstract method within the {\tt Animal} superclass, we will define it
as an abstract method in the {\tt Speakable} interface
(Fig.~\ref{fig-speakable}).
\begin{figure}[h!]
\jjjprogstart
\begin{jjjlisting}
\begin{lstlisting}
public interface Speakable {
    public String speak();
}
public class Animal {
    protected String kind; // Cow, pig, cat, etc.
    public Animal()  {  }
    public String toString() {
        return "I am a " + kind + " and I go " + 
               ((Speakable)this).speak();
    }
}
\end{lstlisting}
\end{jjjlisting}
\jjjprogstop{Defining and using the  {\tt Speakable} interface.}
{fig-speakable}
\end{figure}

Note the differences between this definition of {\tt Animal} and the
previous definition. This version no longer contains the abstract {\tt
speak()} method. Therefore, the class itself is not an abstract class.
However, because the {\tt speak()} method is not declared in this
class, we cannot call the {\tt speak()} method in the {\tt toString()}
method, unless we cast this object into a {\tt Speakable} object.

We encountered the cast operation in Chapter~5, where we used it with
\marginnote{Cast operation}
primitive types such as {\tt (int)} and {\tt (char)}.  Here, we use it
to specify the actual type of some object.  In this {\tt toString()}
example, {\tt this} object is some type of {\tt Animal} subclass, such
as a {\tt Cat}. The cast operation, {\tt (Speakable)}, changes the
object's actual type to {\tt Speakable}, which syntactically allows
its {\tt speak()} method to be called.

Given these definitions, {\tt Animal} subclasses will now {\tt extend}
the {\tt Animal} class and {\tt implement} the {\tt Speakable}
interface:

\begin{jjjlisting}
\begin{lstlisting}
public class Cat extends Animal implements Speakable {
    public Cat() { kind = "cat"; }
    public String speak() { return "meow";  }
}
public class Cow extends Animal implements Speakable {
    public Cow() { kind = "cow";  }
    public String speak() { return "moo";  }
}
\end{lstlisting}
\end{jjjlisting}

\noindent To implement a Java interface, one must provide a method
implementation for each of the abstract methods in the interface. In
this case there is only one abstract method, the {\tt speak()} method.

Note, again, the expression from the {\tt Animal.toString()} class 

\begin{jjjlisting}
\begin{lstlisting}
((Speakable)this).speak();
\end{lstlisting}
\end{jjjlisting}

\noindent which casts {\tt this} object into a {\tt Speakable} object. 
The reason that this cast is required is because an {\tt Animal} does
not necessarily have a {\tt speak()} method. A {\tt speak()} method is
not defined in the {\tt Animal} class. However, the {\tt Cat} subclass
of {\tt Animal} does implement a {\tt sleep()} method as part of its
{\tt Speakable} interface.  Therefore, in order to invoke {\tt speak()}
on an object from one of the {\tt Animal} subclasses, the object must
actually be a {\tt Speakable} and we must perform the cast as shown here.

This illustrates, by the way, that a {\tt Cat}, by virtue of extending
\marginnote{Interface inheritance}
the {\tt Animal} class and implementing the {\tt Speakable} interface,
is both an {\tt Animal} and a {\tt Speakable}.  In general, a class
that implements an interface, has that interface as one of its types.
Interface implementation is itself a form of inheritance. A Java class
can be a direct subclass of only one superclass. But it can implement
any number of interfaces. 

Given these definitions of the {\tt Cow} and {\tt Cat} subclasses, the
following code segment will produce the same results as in the previous
section.

\begin{jjjlisting}
\begin{lstlisting}
 Animal animal = new Cow();
 System.out.println(animal.toString()); // A cow goes moo
 animal = new Cat();
 System.out.println(animal.toString()); // A cat goes meow
\end{lstlisting}
\end{jjjlisting}

\noindent Although the design is different, both approaches produce the
same result. We will put off, for now, the question of how one decides
whether to use an abstract method or a Java interface. We will get to this
question when we design the {\tt TwoPlayerGame} class hierarchy later in
this chapter.

\section{Example: A Toggle Button}
\label{sec-toggle}


\noindent The ability to extend an existing class is one of the most
\marginnote{Reusing code} powerful features of object-oriented
programming.  It allows objects to reuse code defined in the
superclasses without having to redefine or recompile the code.  As we
saw in Chapter~4, a programmer-defined JFrame, such as {\tt
GreeterGUI}, uses the {\tt public} methods defined for {\tt
JFrame}s, {\tt Frame}s,  {\tt Window}s, {\tt Container}s, {\tt Component}s, and {\tt
Object}s simply because it is a subclass of {\tt JFrame}
(Fig.~4.11).  By the same token, it can use all of the {\tt public}
and {\tt protected} instance variables and constants defined in these
classes by simply referring to them in its own code.

In this section, we present an example of how inheritance can be used
to extend and customize the functionality of a Java library class. As
we saw in Chapter~4, a {\tt JButton} is a GUI component that can be
associated with a particular action by implementing the {\tt
ActionListener} interface.  For example, we used a {\tt JButton} in
the {\tt GreeterGUI} to generate a greeting to the user.

In this section, we will design a more sophisticated button. 
We will call it a  {\tt ToggleButton} and define it as a {\tt JButton} subclass that
\marginnote{Problem decomposition}
toggles its label whenever it is clicked, in addition to
carrying out some kind of associated action.

A light switch behaves similarly to a {\tt ToggleButton} in this
sense.  Whenever you flick a light switch, it changes its label from
``on'' to ``off,'' but it also turns the lights on or off.  Although
different switches are associated with different lights, every light
switch toggles its label each time it is clicked.  So let's design a
{\tt ToggleButton} that behaves like a light switch.

The main idea in our design is that a {\tt ToggleButton} is a
{\tt JButton} that has two labels. By default, a {\tt JButton} has just
a single label. Thus, because of the type of behavior we want to elicit,
we need to define {\tt ToggleButton} as a subclass of
{\tt JButton} with two {\tt String} variables that will serve as its
alternate labels (Fig.~\ref{fig-toggleuml}). Note that we give it a constructor
%\begin{figure}
%\begin{graphic}
\marginfig{chptr08/togglebutton.eps}{A {\tt ToggleButton} {\it isa} {\tt JButton} with two labels.}
{fig-toggleuml}
%\end{graphic}
%\end{figure}
method that will allow us to provide the initial value of its two
label strings. Another important feature of a {\tt ToggleButton} is
that it should act as its own {\tt ActionListener} so that it can
toggle its label whenever it is clicked. Therefore, it must also
implement the {\tt ActionListener} interface.  

The complete definition of {\tt ToggleButton} is given in
Figure~\ref{fig-togglebutton}. Note how we have defined its
constructor. Recall that the {\tt JButton} class has a constructor
method with the signature {\tt JButton(String)}, which allows us to
set a {\tt JButton}'s label during instantiation.  We need to do the
same thing with one of {\tt ToggleButton}'s two labels.  That is, when
we create a {\tt ToggleButton}, we want to initialize its label to one
of its two alternative labels (here, ``On'' or ``Off'').

\begin{figure}[b!]
\jjjprogstart
\begin{jjjlisting}
\begin{lstlisting}
import java.awt.*;
import java.awt.event.*;
import javax.swing.*;

public class ToggleButton extends JButton 
                        implements ActionListener {   
  private String label1;   // Toggle between two labels
  private String label2;
   
  public ToggleButton(String l1, String l2) {// Constructor
    super(l1);          // Use l1 as the default label
    label1 = l1; 
    label2 = l2;
    addActionListener(this);
  } 
  public void actionPerformed(ActionEvent e) {
    String tempS = label1;  // Swap the labels
    label1 = label2;
    label2 = tempS;
    setText(label1);
  } // actionPerformed()
} // ToggleButton
\end{lstlisting}
\end{jjjlisting}
\jjjprogstop{Definition of the {\tt ToggleButton} class.}
{fig-togglebutton}
\end{figure}

Because constructor methods are {\it not} inherited by the subclass,
we want to invoke the superclass's constructor in the {\tt ToggleButton()}
constructor using the {\tt super} keyword. This must be done as the
first statement in the {\tt ToggleButton()} constructor.
By passing {\tt l1} to the super constructor we are making the first string
that the user gives us the default label for our {\tt ToggleButton}.
This will be the label that appears on the button when it is first
displayed in a {\tt Component}.

Notice also in the {\tt ToggleButton()} constructor that the {\tt
ToggleButton} is designated as its own {\tt ActionListener}, so
whenever it is clicked, its
%\begin{marginalnote}\it Swapping algorithm\end{marginalnote}
\marginnote{Swapping algorithm} 
{\tt actionPerformed()} method will be invoked. The {\tt
actionPerformed()} method exchanges the button's current label for its
other label.  Swapping two values in memory is a standard programming
practice used in lots of different algorithms.  In order to do it
properly, you must use a third variable to temporarily store one of
the two values you are swapping. The comments in {\tt
actionPerformed()} provide a step-by-step trace of the values of the
three variables involved.

\JavaTIP{PROGRAMMING TIP}{Swapping Values.}{It is necessary to
use a temporary variable whenever you are swapping two values,
of any type, in memory.  The temporary variable holds the
first value while you overwrite it with the second value.}

\noindent The first statement in {\tt actionPerformed()} creates a temporary
%\begin{marginalnote}\it Swapping values requires a temporary variable\end{marginalnote}
\marginnote{Swapping values requires a temporary variable}
{\tt String} variable named {\tt tempS} and assigns it the value of
{\tt label1}. Recall that \mbox{\tt label1} was the button's initial
label.  To make this example easier to follow, let's suppose that
initially {\tt label1} is ``off'' and that {\tt label2} is ``on.'' 
After line 1 is executed, both {\tt tempS} and {\tt label1}
contain ``off'' as their value.  Line 2 then assigns {\tt label2}'s
value to {\tt label1}.  Now both {\tt label1} and {\tt label2}
store ``on'' as their values.   In line 3 we assign {\tt tempS}'s value
to {\tt label2}. Now {\tt label2} stores ``off'' and {\tt label1}
stores ``on,'' and we have effectively swapped their original values.

The next time we invoke {\tt actionPerformed()}, {\tt label1}
and {\tt label2} will have their opposite values initially.  
Swapping them a second time will assign them their initial values
again.  We can continue toggling their values in this way
indefinitely.  To complete the method, the last statement in
{\tt actionPerformed()} assigns {\tt label1}'s current value as the new
{\tt ToggleButton}'s label.

Now that we have seen that a {\tt ToggleButton} toggles its label
between two values, what about performing an associated action?
\marginnote{Multiple event handlers} To do this, we need a design
involving multiple event handlers, one to handle the toggling of the
button's label and the other to handle its associated action
(Fig~\ref{fig-p204}).
\begin{figure}[h!]
\figa{chptr04/p204.eps}{The {\tt ToggleButton} has two {\tt
ActionListener}s.  When the button is clicked, the JVM will call each
listener's {\tt actionPerformed()} method, and each listener will take
its own independent action.}
{fig-p204}
\end{figure}
In this design, {\tt lightSwitch} has two listeners that respond to
its events: the {\tt lightSwitch} itself, as a result of the {\tt
actionPerformed()} method in its class, and the {\tt ToggleFrame}, as
a result of {\tt actionPerformed()} method in this class.

The implementation of this design is given by {\tt ToggleFrame}, a
program that uses a {\tt ToggleButton} (Fig.~\ref{fig-toggletest}).
Like the GUI we designed in Chapter~4, this program extends the {\tt
JFrame} class and implements the {\tt ActionListener} interface.  In
this example we use a {\tt ToggleButton} to simulate a light switch.
Note that we assign the program itself as an {\tt ActionListener} for
the {\tt lightSwitch}, so that

\begin{figure}[tb]
\jjjprogstart
\begin{jjjlisting}
\begin{lstlisting}
import java.awt.*;
import java.awt.event.*;
import javax.swing.*;

public class ToggleFrame extends JFrame
                            implements ActionListener {
  private ToggleButton lightSwitch;  

  public ToggleFrame() {
    lightSwitch = new ToggleButton ("off","on");
    getContentPane().add(lightSwitch);
    lightSwitch.addActionListener(this);
  } // init()

  public void actionPerformed(ActionEvent e)  {
    setTitle("The light is " + lightSwitch.getText());
  } // actionPerformed()

   public static void main(String args[]) 
   {  
      JFrame f = new ToggleFrame();
      f.setSize(200,200);
      f.setVisible(true);
   }
} // ToggleFrame
\end{lstlisting}
\end{jjjlisting}
\jjjprogstop{Definition of the {\tt ToggleFrame} class.}
{fig-toggletest}
\end{figure}

When {\tt lightSwitch} is clicked, the program displays the message,
``The light is on,'' or ``The light is off,'' in the program's title
bar (Fig.~8.12). This is a somewhat trivial action but it illustrates
that a {\tt ToggleButton} both toggles its own label {\it and} carries
out some associated action.

%\begin{figure}
\begin{figure}[h!]
\figaleft{chptr04/4f23.eps}{When clicked, {\tt ToggleFrame}
button causes ``The light is on'' or ``The light is off'' to appear in
the window's title bar.
} {fig-toggledump}

\end{figure}
%\end{figure}

The {\tt ToggleButton} design satisfies several key design
principles of object-oriented\index{object-oriented design}
programming.  First and foremost, it uses inheritance to extend the
functionality of the predefined {\tt JButton} class---the
\marginnote{Object oriented design  principles}
extensibility principle. Secondly, it encapsulates a {\tt
ToggleButton}'s essential behavior within the {\tt ToggleButton} class
itself---the modularity principle. Finally, it hides the mechanism by
which a {\tt ToggleButton} manages its labels---the information-hiding
principle.

\JavaTIP[false]{EFFECTIVE DESIGN}{Inheritance.}{Inheritance enables you to
specialize an object's behavior.  A {\tt ToggleButton} does everything
that a {\tt JButton} does, plus it can toggle its own label.}



\secEXRHone{Self-Study Exercises}
\label{self-study-exercises}
\begin{SSTUDY}

\item  Write a code segment (not a whole method) to swap
two boolean variables, {\tt b1} and {\tt b2}.

\item  Suppose you are designing an GUI that plays a
card game, and you want a single button that can
be used both to deal the cards and to collect the cards.
Write a code segment that creates this type of button,
adds it to the JFrame, and designates the JFrame as its
{\tt ActionListener}.

\end{SSTUDY}

\pagebreak
\section*{{\color{cyan}Special Topic:} Historical Cryptography}

{\bf Cryptography}, the study of secret writing,
has had a long and interesting history.   Modern-day cryptographic
techniques employ sophisticated mathematics to
{\it encrypt\index{encryption}} and {\it decrypt} messages.  Today's most
\mbox{secure} encryption schemes are safe from attack by even the most
powerful computers.   Given our widespread dependence on computers and
the Internet, secure encryption has become an important application
area within computer science.  While the cryptographic techniques used up
through World War II are too simple to serve as the basis for modern-day encryption 
schemes, they can provide an interesting and accessible
introduction to this important area of computer science.\\
\hspace*{1pc}One of the earliest and simplest {\it ciphers\index{cipher}} is the
Caesar\index{Caesar cipher} cipher, used by Julius Caesar during the
Gallic wars.  According to this scheme, letters of the alphabet are
{\it shifted} by three letters, wrapping around at the end of the
alphabet:

\begin{jjjlisting}
\begin{lstlisting}
PlainText:     abcdefghijklmnopqrstuvwxyz
CaesarShifted: defghijklmnopqrstuvwxyzabc
\end{lstlisting}
\end{jjjlisting}

\noindent When encrypting a message, you take each letter
of the message and replace it with its corresponding letter from the
shifted alphabet.  To decrypt a secret message, you perform the
operation in reverse---that is, you take the letter from the shifted
alphabet and replace it with the corresponding letter from the
{\bf plaintext} alphabet\index{plaintext alphabet}.  Thus, ``hello'' would be
Caesar encrypted as ``khoor.''\\
\hspace*{1pc}The Caesar cipher is a {\bf substitution cipher}\index{substitution cipher},
because each letter in the plaintext message is replaced with
a substitute letter from the {\bf ciphertext}
alphabet\index{ciphertext alphabet}. A more general form of a
substitution cipher uses a {\it keyword\index{keyword, cipher}} to
create a ciphertext alphabet:

\begin{jjjlisting}
\begin{lstlisting}
PlainText:  abcdefghijklmnopqrstuvwxyz
Ciphertext: xylophneabcdfgijkmqrstuvwz
\end{lstlisting}
\end{jjjlisting}

\noindent In this example, the keyword ``xylophone,''
(with the second {\it o} removed) is used to set up a substitution
alphabet.  According to this cipher, the word ``hello'' would be encrypted
as ``epddi.''  Substitution ciphers of this form are found frequently in
cryptogram puzzles in the newspapers.\\
\hspace*{1pc}Another type of cipher is known as a {\bf transposition
cipher}.\index{transposition cipher} In this type of cipher, the
letters in the original message are rearranged in some methodical way.
A simple example would be if we reversed the letters in each word so
that ``hello'' became ``olleh.'' 

\section{Example: The Cipher Class Hierarchy}
\label{example-the-cipher-class-hierarchy}

Suppose we wish to design a collection of cipher classes, including a
Caesar cipher and a transposition cipher.  Because the basic
operations used in all forms of encryption are the same, both the 
\marginnote{Problem decomposition}
{\tt Caesar} class and the {\tt Transpose} class will have methods to {\tt
encrypt()} and {\tt decrypt()} messages, where each message is assumed
to be a string of words separated by spaces.  These methods will take
a {\tt String} of words and translate each word using the encoding
method that is appropriate for that cipher.  Therefore, in addition to
{\tt encrypt()} and {\tt decrypt()}, each cipher class will need
polymorphic {\tt encode()} and {\tt decode()} methods, which take a
single word and encode or decode it according to the rules of that
particular cipher.

From a design perspective the {\tt encrypt()} and {\tt decrypt()}
methods will be the same for every class: They simply break the
message into words and encode or decode each word.  However, the
{\tt encode()} and {\tt decode()} methods will be different for each
different cipher.  The {\tt Caesar.encode()} method should replace each
letter of a word with its substitute, whereas the
{\tt Transpose.encode()} method should rearrange the letters of the
word.  Given these considerations, how should we design this set of
classes?

Because all of the various ciphers will have the same methods, it will
\marginfig{chptr07/p380f1.eps}{A hierarchy of cipher classes.
The {\tt Cipher} class implements operations common to all ciphers.
The {\tt Caesar} and {\tt Transpose} classes implement functions
unique to those kinds of ciphers.}
{fig-cipherhier}
be helpful to define a common {\tt Cipher} superclass
(Fig.~\ref{fig-cipherhier}).  {\tt Cipher} will encapsulate those
features that the individual cipher classes have in common---the
{\tt encrypt()}, {\tt decrypt()}, {\tt encode()}, and {\tt decode()}
methods.

Some of these methods can be implemented in the {\tt Cipher} class
itself.  For example, the {\tt encrypt()} method should take a message
in a {\tt String} parameter, encode each word in the message, and
return a {\tt String} result.  The following method definition will
work for any cipher:

\begin{jjjlisting}[26.5pc]
\begin{lstlisting}
public String encrypt(String s) {
  StringBuffer result = new StringBuffer("");          
  StringTokenizer words = new StringTokenizer(s);// Tokenize
  while (words.hasMoreTokens()) {        // Encode each word 
    result.append(encode(words.nextToken()) + " "); 
  }
  return result.toString();        // Return result
} // encrypt()
\end{lstlisting}
\end{jjjlisting}

\noindent This method creates a local {\tt StringBuffer} variable,
{\tt result}, and uses {\tt StringTokenizer} to break the original
{\tt String} into its component words.  It uses the {\tt encode()} method to
encode the word, appending the result into {\tt result}. The result is
converted back into a {\tt String} and returned as the encrypted
translation of {\it s}, the original message.

If we define {\tt encrypt()} in the superclass, it will be inherited
%\begin{marginalnote}\it Inheritance\end{marginalnote}
\marginnote{Inheritance}
by all of {\tt Cipher}'s subclasses.  Thus, if we define {\tt Caesar}
and {\tt Transpose} as

\begin{jjjlisting}
\begin{lstlisting}
public class Caesar extends Cipher { ... }
public class Transpose extends Cipher { ... }
\end{lstlisting}
\end{jjjlisting}

\noindent instances of these classes will be able to use
the {\tt encrypt()} method.

On the other hand, the polymorphic {\tt encode()} method cannot be
implemented
%\begin{marginalnote}\it Abstract method\end{marginalnote}
within {\tt Cipher}. This is because unlike the {\tt encrypt()}
method, which is the same for every {\tt Cipher} subclass, the {\tt
encode()} method will be different for every subclass.  However, by
declaring the {\tt encode()} method as {\tt abstract}, we can leave
its implementation up to the {\tt Cipher} subclasses. Thus, within the
{\tt Cipher} class, we would define {\tt encode()} and {\tt decode()}
as follows:

\begin{jjjlisting}
\begin{lstlisting}
           // Abstract methods
public abstract String encode(String word); 
public abstract String decode(String word);
\end{lstlisting}
\end{jjjlisting}

\noindent These declarations within the {\tt Cipher} class tell the
compiler that these methods will be implemented in {\tt Cipher}'s
subclasses. By defining it as {\tt abstract}, {\tt encode()} can be
used in the {\tt Cipher} class, as it is within the {\tt encrypt()}
method.

\subsection{Class Design: {\tt Caesar} }
\noindent Figure~\ref{fig-cipher} provides the full definition of the
{\tt Cipher} class.  The \mbox{\tt encode()} and {\tt decode()} methods are
declared abstract.  They are intended to be implemented by
{\tt Cipher}'s subclasses.

%% proglist ch7/cipher/Cipher.java
\begin{figure}[h!]
\jjjprogstart
\begin{jjjlisting}[35pc]
\begin{lstlisting}
import java.util.*;

public abstract class Cipher {
  public String encrypt(String s) {
    StringBuffer result = new StringBuffer("");         // Use a StringBuffer
    StringTokenizer words = new StringTokenizer(s); // Break s into its words
    while (words.hasMoreTokens()) {                     // For each word in s
      result.append(encode(words.nextToken()) + " ");   //  Encode it
    }
    return result.toString();                            // Return the result
  } // encrypt()

  public String decrypt(String s) {
    StringBuffer result = new StringBuffer("");        // Use a StringBuffer
    StringTokenizer words = new StringTokenizer(s);    // Break s into words
    while (words.hasMoreTokens()) {                    // For each word in s
      result.append(decode(words.nextToken()) + " ");  //  Decode it
    }
    return result.toString();                       // Return the decryption
  } // decrypt()

  public abstract String encode(String word);            // Abstract methods
  public abstract String decode(String word);
} // Cipher
\end{lstlisting}
\end{jjjlisting}
\jjjprogstop{The {\tt abstract Cipher} class.}
{fig-cipher}
\end{figure}


Note again that {\tt encrypt()} and {\tt decrypt()}, which are
implemented in \mbox{\tt Cipher}, invoke {\tt encode()} and {\tt
decode()}, \marginnote{\raggedright encode() and decode() are
polymorphic} respectively, which are declared in {\tt Cipher} but
implemented in {\tt Cipher}'s subclasses.  Java's dynamic binding
mechanism will take care of invoking the appropriate implementation of
{\tt encode()} or {\tt decode()}, depending on what type of object is
involved.  For example, if {\tt caesar} and {\tt transpose} are {\tt
Caesar} and {\tt Transpose} objects, respectively, then the following
calls to {\tt encrypt()} will cause their respective {\tt encode()}
methods to be invoked:

\begin{jjjlisting}
\begin{lstlisting}
// Invokes caesar.encode()
caesar.encrypt("hello world");     
// Invokes transpose.encode()
transpose.encrypt("hello world");  
\end{lstlisting}
\end{jjjlisting}

\noindent When {\tt caesar.encrypt()} is called, it will in turn
invoke {\tt caesar.en\-code()}---that is, it will call the
{\tt encode()} method implemented in the {\tt Caesar} class.  When
{\tt transpose.encrypt()} is invoked, it will in turn invoke
{\tt trans\-pose.encode()}. In this way, each object can perform the encoding
algorithm appropriate for its type of 
\marginnote{Method polymorphism}
cipher.

\subsection{Algorithm Design: Shifting Characters}
\noindent The {\tt Caesar} class is defined as an extension of {\tt Cipher}
(Fig.~\ref{fig-caesar}). The only methods implemented in {\tt Caesar}
are {\tt encode()} and {\tt decode()}. The {\tt encode()} method takes
a {\tt String} parameter and returns a {\tt String} result.  It takes
each character of its parameter ({\tt word.charAt(k)}) and performs a
Caesar shift on the character.  Note how the shift is done:

\begin{jjjlisting}
\begin{lstlisting}
ch = (char)('a' + (ch -'a'+ 3) % 26);// Caesar shift
\end{lstlisting}
\end{jjjlisting}

\noindent Recall from Chapter~5 that {\tt char} data in Java are
represented as 16-bit integers.  This enables us to manipulate
characters as numbers.  Thus, to shift a character by 3, we simply add
3 to its integer representation.

%% proglist ch7/cipher/Caesar.java
\begin{figure}[h]
\jjjprogstart
\begin{jjjlisting}[35pc]
\begin{lstlisting}
public class Caesar extends Cipher {
  public String encode(String word) {
    StringBuffer result = new StringBuffer(); // Initialize a string buffer
    for (int k = 0; k < word.length(); k++) { // For each character in word
      char ch = word.charAt(k);               //  Get the character
      ch = (char)('a' + (ch -'a'+ 3) % 26);   //  Perform caesar shift
      result.append(ch);                   //  Append it to new string
    }
    return result.toString();              // Return the result as a string
  } // encode()

  public String decode(String word) {
    StringBuffer result = new StringBuffer(); // Initialize a string buffer
    for (int k = 0; k < word.length(); k++) { // For each character in word
    char ch = word.charAt(k);                 //  Get the character
       ch = (char)('a' + (ch - 'a' + 23) % 26); //  Perform reverse shift
       result.append(ch);                     //  Append it to new string
    }
    return result.toString();            // Return the result as a string
  } // decode()
} // Caesar
\end{lstlisting}
\end{jjjlisting}
\jjjprogstop{The {\tt Caesar} class.}
{fig-caesar}
\end{figure}

For example, suppose that the character ({\tt ch}) is {\it h}, which has
%\begin{marginalnote}\it Character conversions\end{marginalnote}
\marginnote{Character conversions}
an ASCII code of 104 (see Table~5.13). We want to shift it
by 3, giving {\it k}, which has a code of 107. In this case, we could
simply add 3 to 104 to get the desired result.  However, suppose
that {\tt ch} was the character {\it y}, which has an ASCII code of
121. If we simply add 3 in this case, we get 124, a code that
corresponds to the symbol ``|,'' which is not our desired result.
Instead, we want the shift in this case to ``wrap around'' to the
beginning of the alphabet, so that {\it y} gets shifted into {\it b}.
In order to accomplish this we need to do some modular arithmetic.

Let's suppose the 26 characters {\it a} to {\it z} were numbered 0 through 25,
so that {\it a} corresponds to 0, {\it b} to 1, and so on up to {\it z} as 25.  If
we take any number {\it N} and divide it (modulo 26), we would get a
number between 0 and 25. Suppose, for example, {\it y} were numbered
24. Shifting it by 3 would give us 27, and {\tt 27 \% 26} would
give us 1, which corresponds to {\it b}. So, if the {\it a} to {\it z} were
numbered 0 through 25, then we can shift any character within that
range by using the following formula:

\begin{jjjlisting}
\begin{lstlisting}
(ch + 3) % 26         // Shift by 3 with wraparound
\end{lstlisting}
\end{jjjlisting}

\noindent To map a character in the range {\it a} to {\it z} onto the
integers 0 to 25, we can simply subtract {\it a} from it:


\begin{jjjlisting}
\begin{lstlisting}
'a' - 'a' = 0
'b' - 'a' = 1
'c' - 'a' = 2
...
'z' - 'a' = 25
\end{lstlisting}
\end{jjjlisting}

\noindent Finally, we simply map the numbers 0 through 25 back to the
characters {\it a} to {\it z} to complete the shift operation:

\begin{jjjlisting}
\begin{lstlisting}
(char)('a' + 0) = 'a'
(char)('a' + 1) = 'b'
(char)('a' + 2) = 'c'
...
(char)('a' + 25) = 'z'
\end{lstlisting}
\end{jjjlisting}

\noindent Note the use here of the cast operator {\tt (char)} to
covert an integer into a {\tt char}.

To summarize, we can shift any character by 3 if we map it into the
%\begin{marginalnote}\it Modular arithmetic\end{marginalnote}
range 0 to 25, then add 3 to it mod 26, then map that result back into
\marginnote{Modular arithmetic}
the~range {\it a} to {\it z}. Thus, shifting {\it y} would go as follows:

\begin{jjjlisting}
\begin{lstlisting}
(char)('a' + (ch -'a'+ 3) % 26)   //  Perform Caesar shift
(char)('a' + ('y' - 'a' +3) % 26) //   on 'y'
(char)(97 + (121 - 97 + 3) % 26)      //  Map 'y' to 0..25
(char)(97 + (27 % 26))     //  Shift by 3, wrapping around
(char)(97 + 1)           //  Map result back to 'a' to 'z'
(char)(98)                    //  Convert from int to char
'b'
\end{lstlisting}
\end{jjjlisting}

\noindent  Note that in {\tt decode()} a reverse
Caesar shift is done by shifting by 23, which is $26-3$. If the original
shift is 3, we can reverse that by shifting an additional 23. Together
this gives a shift of 26, which will give us back our original string.

\subsection{Class Design: {\tt Transpose} }
\noindent The {\tt Transpose} class (Fig.~\ref{fig-transpose}) is
structured the same as the {\tt Caesar} class.  It implements both the
{\tt encode()} and {\tt decode()} methods.  The key element here is
the transpose operation, which in this case is a simple reversal of
the letters in the word.  Thus, ``hello'' becomes ``olleh''.  This is
very easy to do when using the {\tt StringBuffer.reverse()} method.  The
{\tt decode()} method is even simpler, because all you need to do in
this case is call {\tt encode()}. Reversing the reverse of a string
gives you back the original string.

%% proglist ch7/cipher/Transpose.java
\begin{figure}[h!]
\jjjprogstart
\begin{jjjlisting}
\begin{lstlisting}
public class Transpose extends Cipher {
  // encode() reverses and returns a word
  public String encode(String word) {
    StringBuffer result = new StringBuffer(word);
    return result.reverse().toString(); 
  } // encode()

  public String decode(String word) {
    return encode(word);         // Just call encode
  } // decode
} // Transpose
\end{lstlisting}
\end{jjjlisting}
\jjjprogstop{The {\tt Transpose} class\index{Transpose class}.}
{fig-transpose}
\end{figure}

\subsection{Testing and Debugging}
\noindent Figure~\ref{fig-testcipher} provides a simple test program for testing
{\tt Cipher} and its subclasses.  It creates a {\tt Caesar} cipher and
\begin{figure}[tb]
\jjjprogstart
\begin{jjjlisting}[35pc]
\begin{lstlisting}
public class TestEncrypt {
  public static void main(String argv[]) {
    Caesar caesar = new Caesar();
    String plain = "this is the secret message";   // Here's the message
    String secret = caesar.encrypt(plain);         // Encrypt the message
    System.out.println(" ********* Caesar Cipher Encryption *********");
    System.out.println("PlainText: " + plain);    // Display the results
    System.out.println("Encrypted: " + secret);
    System.out.println("Decrypted: " + caesar.decrypt(secret));// Decrypt

    Transpose transpose = new Transpose();
    secret = transpose.encrypt(plain);
    System.out.println("\n ********* Transpose Cipher Encryption *********");
    System.out.println("PlainText: " + plain);    // Display the results
    System.out.println("Encrypted: " + secret);
    System.out.println("Decrypted: " + transpose.decrypt(secret));// Decrypt
  } // main()
} // end TestEncrypt

\end{lstlisting}
\end{jjjlisting}
\jjjprogstop{The {\tt TestEncrypt} class\index{TestEncrypt class}.}
{fig-testcipher}
\end{figure}
a {\tt Transpose} cipher and then encrypts and decrypts the same
sentence using each cipher.  If you run this program, it will produce
the following output:

\begin{jjjlisting}
\begin{lstlisting}
     ********* Caesar Cipher Encryption *********
    PlainText: this is the secret message
    Encrypted: wklv lv wkh vhfuhw phvvdjh
    Decrypted: this is the secret message

     ********* Transpose Cipher Encryption *********
    PlainText: this is the secret message
    Encrypted: siht si eht terces egassem
    Decrypted: this is the secret message
\end{lstlisting}
\end{jjjlisting}

\secEXRHone{Self-Study Exercises}
\label{self-study-exercises}
\begin{SSTUDY}

\item Modify the {\tt Caesar} class so that
it will allow various sized shifts to be used.  ({\it Hint}: Use an
instance variable to represent the shift.)

\item Modify {\tt Transpose.encode()} so that it uses a rotation
instead of a reversal.  That is, a word like ``hello'' should be
encoded as ``ohell'' with a rotation of one character.
\end{SSTUDY}

\section{Case Study: A Two Player Game Hierarchy}
\label{case-study-a-two-player-game-hierarchy}

In this section we will redesign our {\tt OneRowNim} game to fit
within a hierarchy of classes of two-player games, which are games
that involve two players.  Many games that this characteristic:
checkers, chess, tic-tac-toe, guessing games, and so forth. However,
there are also many games that involve just 1 player: blackjack,
solitaire, and others. There are also games that involve two or more
players, such as many card games.  Thus, our redesign of {\tt
OneRowNim} as part of a two-player game hierarchy will not be our
last effort to design a hierarchy of game-playing classes. We will
certainly re-design things as we learn new Java language constructs
and as we try to extend our game library to other kinds of games.

This case study will illustrate how we can apply inheritance and
polymorphism, as well as other object-oriented design principles.  The
justification for revising {\tt OneRowNim} at this point is to make it
easier to design and develop other two-player games.  As we have seen,
one characteristic of class hierarchies is that more general
attributes and methods are defined in top-level classes. As one
proceeds down the hierarchy, the methods and attributes become more
specialized.  Creating a subclass is a matter of specializing a
given class.

\subsection{Design Goals}

One of our design goals is to revise the {\tt OneRowNim} game so that
it fits into a hierarchy of two-player games.  One way to do this is
to generalize the {\tt OneRowNim} game by creating a superclass that
contains those attributes and methods that are common to all
two-player games.  The superclass will define the most general and
\marginnote{Generic superclass}
generic elements of two-player games.  All two-player games, including
{\tt OneRowNim}, will be defined as subclasses of this top-level
superclass and will inherit and possibly override its public and
protected variables and methods. Also, our top-level class will
contain certain abstract methods, whose implementations will be given
in {\tt OneRowNim} and other subclasses.

A second goal is to design a class hierarchy that makes it possible
for computers to play the game, as well as human users. Thus, for a
given two-player game, it should be possible for two humans to play
each other, or for two computers to play each other, or for a human
to play against a computer.  This design goal will require that our
design exhibit a certain amount of flexibility.  As we shall see, this
is a situation in which Java interfaces will come in handy.

Another important goal is to design a two-player game hierarchy that
can easily be used with a variety of different user interfaces,
including command-line interfaces and GUIs.  To handle this feature,
we will develop Java interfaces to serve as interfaces
between our two-player games and various user interfaces.

\subsection{Designing the {\tt TwoPlayerGame} Class}

To begin revising the design of the {\tt OneRowNim} game, we first
need to design a top-level class, which we will call the {\tt
TwoPlayerGame} class.  What variables and methods belong in this
class?  One way to answer this question is to generalize our current
version of {\tt OneRowNim} by moving any variables and methods that
apply to all two-player games up to the {\tt TwoPlayerGame} class.
All subclasses of {\tt TwoPlayerGame}---which includes the {\tt
OneRowNim} class---would inherit these elements.  Figure~\ref{fig-onerownim} shows
the current design of {\tt OneRowNim}.
\marginfig{chptr08/onerownim.eps}{The current {\tt OneRowNim}
class.}
{fig-onerownim}

What variables and methods should we move up to the {\tt
TwoPlayer\-Game} class?  Clearly, the class constants, {\tt PLAYER\_ONE}
and {\tt PLAYER\_TWO}, apply to all two-player games. These
should be moved up.  On the other hand, the {\tt MAX\_PICKUP} and {\tt
MAX\_STICKS} constants apply just to the {\tt OneRowNim} game. They
should remain in the {\tt OneRowNim} class.

The {\tt nSticks} instance variable is a variable that only applies to
the {\tt OneRowNim} game, but not to other two-player games. It should
stay in the {\tt OneRowNim} class. On the other hand, the {\tt
onePlaysNext} variable applies to all two-player games, so we will
move it up to the {\tt TwoPlayerGame} class.

Because constructors are not inherited, all of the constructor methods
\marginnote{Constructors are not inherited}
will remain in the {\tt OneRowNim} class. The instance methods, {\tt
takeSticks()} and {\tt getSticks()}, are particular to {\tt
OneRowNim}, so they should remain there. However, the other methods,
{\tt getPlayer()}, {\tt gameOver()}, {\tt getWinner()}, and {\tt
reportGameState()}, are methods that would be useful to all two-player
games. Therefore these methods should be moved up to the superclass.
Of course, while these methods can be defined in the superclass, some
of the methods can only be implemented in subclasses.  For example,
the {\tt reportGameState()} method reports the current state of the
game, so it has to be implemented in {\tt OneRowNim}. Similarly, the
{\tt getWinner()} method defines how the winner of the game is
determined, a definition that can only occur in the subclass. Every
two-player game needs methods such as these. Therefore, we will define
these methods as abstract methods in the superclass.  The intention is
that {\tt TwoPlayerGame} subclasses will provide game-specific
implementations for these methods.

Given these considerations, we come up with the design shown in
Figure~\ref{fig-twoplayergame}.  The design shown in this figure
is much more complex than designs we have used in earlier chapters.
However, the complexity comes from combining ideas already
discussed in previous sections of this chapter, so don't be put
off by it.

\begin{figure}[h!]
\figaleft{chptr08/twoplayergame.eps}{{\tt TwoPlayerGame} is the
superclass for {\tt OneRowNim} and other two player games.
} {fig-twoplayergame}

\end{figure}

To begin with, notice that we have introduced two Java interfaces into
our design in addition to the {\tt TwoPlayerGame} superclass.  As we
will show, these interfaces lead to a more flexible design and one
that can easily be extended to incorporate new two-player games. Let's
take each element of this design separately.

\subsection{The {\tt TwoPlayerGame} Superclass}

As we have stated, the purpose of the {\tt TwoPlayerGame} class is to
serve as the superclass for all two-player games. Therefore, it should
define those variables and methods that are shared by
two-player games. 

The {\tt PLAYER\_ONE}, {\tt PLAYER\_TWO}, and {\tt onePlaysNext}
variables and the {\tt getPlayer()}, {\tt setPlayer()}, and {\tt
changePlayer()} methods have been moved up from the {\tt OneRowNim}
class. Clearly, these variables and methods apply to all two-player
games.  Note that we have also added three new variables:  {\tt
nComputers}, {\tt computer1}, {\tt computer2} and their corresponding
methods, {\tt getNComputers()} and {\tt addComputerPlayer()}.  We will
use these elements to give our games the ability to be played by
computer programs. Because we want all of our two-player games to have
this capability, we define these variables and methods in the
superclass rather than in {\tt OneRowNim} and subclasses of {\tt
TwoPlayerGame}.

Note that the {\tt computer1} and {\tt computer2} variables are declared
to be of type {\tt IPlayer}.  {\tt IPlayer} is an interface, which 
contains a single method declaration, the {\tt makeAMove()} method:

\begin{jjjlisting}
\begin{lstlisting}
 public interface IPlayer {
     public String makeAMove(String prompt);
 }
\end{lstlisting}
\end{jjjlisting}

\noindent Why do we use an interface here rather than some type of
game-playing object?  This is a good design question.  Using an
interface here makes our design more flexible and extensible because
it frees us from having to know the names of the classes that
implement the {\tt makeAMove()} method.  The variables {\tt computer1}
and {\tt computer2} will be assigned objects that implement {\tt
IPlayer} via the {\tt addComputerPlayer()} method.

The algorithms used in the various implementations of {\tt
\marginnote{Game-dependent algorithms}
makeAMove()} are {\em game-dependent}---they depend on the particular
game being played.  It would be impossible to define a game-playing
object that would suffice for all two-player games. Instead, if we
want an object that plays {\tt OneRowNim}, we would define a {\tt
OneRowNimPlayer} and have it implement the {\tt IPlayer}
interface. Similarly, if we want an object that plays checkers, we
would define a {\tt CheckersPlayer} and have it implement the {\tt
IPlayer} interface.  By using an interface here, our {\tt
TwoPlayerGame} hierarchy can deal with a wide range of
\marginnote{The {\tt IPlayer} interface}
differently named objects that play games, as long as they implement
the {\tt IPlayer} interface.  So, using the {\tt IPlayer} interface
adds flexibility to our game hierarchy and makes it easier to extend
it to new, yet undefined, classes.  We will discuss the details of how
to design a game player in one of the following sections.

Turning now to the methods defined in {\tt TwoPlayerGame}, we have
already seen implementations of {\tt getPlayer()}, {\tt setPlayer()},
and {\tt changePlayer()} in the {\tt OneRowNim} class.  We will just
move those implementations up to the superclass. The {\tt
getNComputers()} method is the accessor method for the {\tt
nComputers} variable, and its implementation is routine.  The {\tt
addComputerPlayer()} method adds a computer player to the
game. Its implementation is as follows:

\begin{jjjlisting}
\begin{lstlisting}
public void addComputerPlayer(IPlayer player) {
   if (nComputers == 0)
      computer2 = player;
   else if (nComputers == 1)
      computer1 = player;
   else 
      return; // No more than 2 players
   ++nComputers;
}
\end{lstlisting}
\end{jjjlisting}

\noindent As we noted earlier, the classes that play the various
{\tt TwoPlayerGame}s must implement the {\tt IPlayer} interface.
The parameter for this method is of type {\tt IPlayer}.  The
algorithm we use checks the current value of {\tt nComputers}.
If it is 0, which means that this is the first {\tt IPlayer}
added to the game, the {\tt player} is assigned to {\tt computer2}.
This allows the human user to be associated with {\tt PLAYER\-ONE}, if 
this is a game between a computer and a human user.

If {\tt nComputers} equals 1, which means that we are adding a second
{\tt IPlayer} to the game, we assign that {\tt player} to {\tt
computer1}.  In either of these cases, we increment {\tt nComputers}.
Note what happens if {\tt nComputers} is neither 1 nor 2. In that
case, we simply return without adding the {\tt IPlayer} to the game
and without incrementing {\tt nComputers}. This, in effect, limits the
number of {\tt IPlayer}s to two.  (A more sophisticated design would
throw an exception to report an error. but we will leave that for
a subsequent chapter.)

The {\tt addComputerPlayer()} method is used to initialize a 
game after it is first created. If this method is not called,
the default assumption is that {\tt nComputers} equals zero
and that {\tt computer1} and {\tt computer2} are both {\tt null}.
Here's an example of how it could be used:

\begin{jjjlisting}
\begin{lstlisting}
 OneRowNim nim = new OneRowNim(11); // 11 sticks
 nim.add(new NimPlayer(nim));       // 2 computer players
 nim.add(new NimPlayerBad(nim));
\end{lstlisting}
\end{jjjlisting}

\noindent Note that the {\tt NimPlayer()} constructor takes a
reference to the game as its argument.  Clearly, our design should not
assume that the names of the {\tt IPlayer} objects would be known to
the {\tt TwoPlayerGame} superclass.  This method allows the objects to
be passed in at run time.  We will discuss the details of {\tt
NimPlayerBad} in a subsequent section.

The {\tt getRules()} method is a new method whose purpose is to return
\marginnote{Overriding a method}
a string that describes the rules of the particular game. This method
is implemented in the {\tt TwoPlayerGame} class with the intention
that it will be overridden in the various subclasses. For example, its
implementation in {\tt TwoPlayerGame} is:

\begin{jjjlisting}
\begin{lstlisting}
 public String getRules() {
     return "The rules of this game are: ";
 }
\end{lstlisting}
\end{jjjlisting}

\noindent and its redefinition in {\tt OneRowNim} is:

\begin{jjjlisting}
\begin{lstlisting}
public String getRules() {
  return "\n*** The Rules of One Row Nim ***\n" +
  "(1) A number of sticks between 7 and " + MAX_STICKS + 
          " is chosen.\n" + 
  "(2) Two players alternate making moves.\n" + 
  "(3) A move consists of subtracting between 1 and\n\t" + 
  MAX_PICKUP + 
  " sticks from the current number of sticks.\n" + 
  "(4) A player who cannot leave a positive\n\t" + 
  " number of sticks for the other player loses.\n";
}
\end{lstlisting}
\end{jjjlisting}

\noindent The idea is that each {\tt TwoPlayerGame} subclass will take
responsibility for specifying its own set of rules in a form that can
be displayed to the user.  

You might recognize that defining {\tt getRules()} in the superclass
and allowing it to be overridden in the subclasses is a form of
\marginnote{Polymorphism}
polymorphism. It follows the design of the {\tt toString()} method,
which we discussed earlier.  This design will allow us to use code
that takes the following form:

\begin{jjjlisting}
\begin{lstlisting}
  TwoPlayerGame game = new OneRowNim();
  System.out.println(game.getRules());
\end{lstlisting}
\end{jjjlisting}

\noindent In this example the call to {\tt getRules()} is polymorphic.
The dynamic binding mechanism is used to invoke the {\tt getRules()}
method that is defined in the {\tt OneRowNim} class.

The remaining methods in {\tt TwoPlayerGame} are defined abstractly.
The {\tt gameOver()} and {\tt getWinner()} methods are both methods
that are {\em game dependent}. That is, the details of their
implementations depend on the particular {\tt TwoPlayerGame} subclass
in which they are implemented.

This is good example of how abstract methods should be used in
designing a class hierarchy. We give abstract definitions in the
superclass and leave the detailed implementations up to the individual
subclasses.  This allows the different subclasses to tailor the
implementations to their particular needs, while allowing all
subclasses to share a common signature for these tasks.  This
allows us to use polymorphism to create flexible, extensible class
hierarchies.

Figure~\ref{fig-twoplayercode} shows the complete implementation of
the abstract {\tt TwoPlayerGame} class. We have already discussed the
most important details of its implementation.  

\JavaTIP[false]{EFFECTIVE DESIGN}{Abstract Methods.}{Abstract methods allow
you to give general definitions in the superclass and to leave the
implementation details to the different subclasses.}

\begin{figure}[h!]
\jjjprogstart
\begin{jjjlisting}
\begin{lstlisting}
public abstract class TwoPlayerGame {   
  public static final int PLAYER_ONE = 1;
  public static final int PLAYER_TWO = 2;

  protected boolean onePlaysNext = true;
  protected int nComputers = 0;     // How many computers
                                // Computers are IPlayers
  protected IPlayer computer1, computer2; 
 
  public void setPlayer(int starter) {   
    if (starter == PLAYER_TWO)
      onePlaysNext = false;
    else onePlaysNext = true;
  } //setPlayer()
  public int getPlayer() {   
    if (onePlaysNext) 
      return PLAYER_ONE;
    else return PLAYER_TWO;
  } //getPlayer()
  public void changePlayer() {   
    onePlaysNext = !onePlaysNext;
  } //changePlayer()
  public int getNComputers() {
    return nComputers;
  }
  public String getRules() {
    return "The rules of this game are: ";
  }
  public void addComputerPlayer(IPlayer player) {
    if (nComputers == 0)
      computer2 = player;
    else if (nComputers == 1)
      computer1 = player;
    else 
      return;  // No more than 2 players
    ++nComputers;
  }
    public abstract boolean gameOver();// Abstract Methods
    public abstract String getWinner();
} //TwoPlayerGame
\end{lstlisting}
\end{jjjlisting}
\jjjprogstop{The {\tt TwoPlayerGame} class}
{fig-twoplayercode}
\end{figure}


\subsection{The {\tt CLUIPlayableGame} Interface}

Let's turn now to the two interfaces shown in
Figure~\ref{fig-twoplayergame}.  Taken together, the purpose of these
interfaces is to create a connection between any two-player game and a
command-line user interface (CLUI).  The interfaces provide method
signatures for the methods that will implement the details of the
interaction between a {\tt TwoPlayerGame} and a {\tt UserInterface}.
Because the details of this interaction vary from game to game, it is
best to leave the implementation of these methods to the games
themselves.

Note that {\tt CLUIPlayableGame} extends the {\tt IGame} interface.
\marginnote{Extending an interface}
The {\tt IGame} interface contains two methods that are used to define
a standard form of communication between the CLUI and the game.  The
{\tt getGamePrompt()} method defines the prompt that is used to signal
the user for some kind of move---for example, ``How many sticks do you
take (1, 2, or 3)?''  And the {\tt reportGameState()} method defines
how that particular game will report its current state---for example,
``There are 11 sticks remaining.''  {\tt CLUIPlayableGame} adds the
{\tt play()} method to these two methods. As we will see shortly, the
{\tt play()} method will contain the code that will control the
playing of the game.

\noindent The source code for these interfaces is very simple:

\begin{jjjlisting}
\begin{lstlisting}
public interface CLUIPlayableGame extends IGame {
    public abstract void play(UserInterface ui);
}
public interface IGame {
    public String getGamePrompt();
    public String reportGameState();
} //IGame
\end{lstlisting}
\end{jjjlisting}

\noindent Notice that the {\tt CLUIPlayableGame} interface extends the
{\tt IGame} interface. A {\tt CLUIPlayableGame} is a game that can be
played through a CLUI. The purpose of its {\tt play()} method is to
contain the {\em game dependent} control loop that determines how the
game is played via some kind of user interface (UI).  In pseudocode, a
typical control loop for a game would look something like the
following:

\begin{jjjlisting}[27pc]
\begin{lstlisting}[stringstyle=\color{black}]
Initialize the game.
While the game is not over 
  Report the current state of the game via the UI.
  Prompt the user (or the computer) to make a move via the UI.
  Get the user's move via the UI.
  Make the move.
  Change to the other player.
\end{lstlisting}
\end{jjjlisting}

\noindent The play loop sets up an interaction between the game and
the UI.  The {\tt UserInterface} parameter allows the game to connect
directly to a particular UI.  To allow us to play our games through a
variety of UIs, we define {\tt UserInterface} as the following Java
interface:

\begin{jjjlisting}
\begin{lstlisting}
 public interface UserInterface {
     public String getUserInput();
     public void report(String s);
     public void prompt(String s);
 }
\end{lstlisting}
\end{jjjlisting}

\noindent Any object that implements these three methods can serve as
a UI for one of our {\tt TwoPlayerGame}s.  This is another example of
the flexibility of using interfaces in object-oriented design.  

To illustrate how we use {\tt UserInterface}, let's attach it to our
{\tt KeyboardReader} class, thereby letting a {\tt KeyboardReader}
serve as a CLUI for {\tt TwoPlayerGame}s. We do this simply by
implementing this interface in the {\tt KeyboardReader} class, as
follows:

\begin{jjjlisting}
\begin{lstlisting}
 public class KeyboardReader implements UserInterface
\end{lstlisting}
\end{jjjlisting}

\noindent As it turns out, the three methods listed in {\tt
UserInterface} match three of the methods in the current
version of {\tt KeyboardReader}.  This is no accident. The design of
{\tt UserInterface} was arrived at by identifying the minimal number
of methods in {\tt KeyboardReader} that were needed to interact with a
{\tt TwoPlayerGame}.

\JavaTIP{EFFECTIVE DESIGN}{Flexibility of Java Interfaces.}{A Java
interface provides a means of associating useful methods with a
variety of different types of objects, leading to a more flexible
object-oriented design.}

\noindent The benefit of defining the parameter more generally as a
\marginnote{Generality principle}
{\tt User\-Interface}, instead of as a {\tt KeyboardReader}, is that
we will eventually want to allow our games to be played via other
kinds of command-line interfaces. For example, we might later define
an Internet-based CLUI that could be used to play {\tt OneRowNim}
among users on the Internet. This kind of extensibility---the ability
to create new kinds of UIs and use them with {\tt TwoPlayerGame}s---
is another important design feature of Java interfaces.

\JavaTIP{EFFECTIVE DESIGN}{Extensibility and  Java Interfaces.}
{Using interfaces to define useful method signatures increases the
extensibility of a class hierarchy.}

As Figure~\ref{fig-twoplayergame} shows, {\tt OneRowNim} implements
the {\tt CLUIPlayable\-Game} interface, which means it must supply
implementations of all three abstract methods: {\tt play()}, {\tt
getGamePrompt()}, and {\tt reportGame\-State()}.

\subsection{Object Oriented Design: Interfaces or Abstract Classes}

Why are these methods defined in interfaces? Couldn't we just as
easily define them in the {\tt TwoPlayerGame} class and use
inheritance to extend them to the various game subclasses?  After all,
isn't the net result the same, namely, that {\tt OneRowNim} must
implement all three methods.

These are very good design questions, exactly the kinds of questions
\marginnote{Interfaces vs. abstract methods}
one should ask when designing a class hierarchy of any sort. As we
pointed out in the {\tt Animal} example earlier in the chapter, you
can get the same functionality from a abstract interface and from an
abstract superclass method.  When should we put the abstract method in
the superclass and when does it belong in an interface? A very good
discussion of these and related object-oriented design issues is
available in {\em Java Design, 2nd Edition}, by Peter Coad and
Mark Mayfield (Yourdan Press, 1999). Our discussion of these issues
follows many of the guidelines suggested by Coad and Mayfield. 

We have already seen that using Java interfaces increases the
flexibility and extensibility of a design.  Methods defined in an
interface exist independently of a particular class hierarchy. By
their very nature, interfaces can be attached to any class, which
\marginnote{Flexibility of interfaces}
makes them very flexible to use.

Another useful guideline for answering this question is that the
superclass should contain the basic common attributes and methods that
define a certain type of object.  It should not necessarily contain
methods that define certain {\em roles} that the object plays.  For
example, the {\tt gameOver()} and {\tt getWinner()} methods are
fundamental parts of the definition of a {\tt TwoPlayerGame}.  One
cannot define a game without defining these methods.  By contrast,
methods such as {\tt play()}, {\tt getGamePrompt()}, and {\tt
reportGameState()} are important for playing the game but they do not
contribute in the same way to the game's definition. Thus these
methods are best put into an interface. So, one important design
guideline is:

\JavaTIP{EFFECTIVE DESIGN}{Abstract Methods.}{Methods defined 
abstractly in a superclass should contribute in a fundamental way
toward the basic definition of that type of object, not merely toward
one of its roles or its functionality.}

\subsection{The Revised {\tt OneRowNim} Class}

Figure~\ref{fig-revisednim} provides a listing of the revised 
{\tt OneRowNim} class, one that fits into the {\tt TwoPlayerGame}
class hierarchy. Our discussion in this section will focus on just
those features of the game that are new or revised.

\begin{figure}[h!]
\jjjprogstart
\begin{jjjlisting}[32pc]
\begin{lstlisting}
public class OneRowNim extends TwoPlayerGame 
                                 implements CLUIPlayableGame {   
  public static final int MAX_PICKUP = 3;
  public static final int MAX_STICKS = 11;
  private int nSticks = MAX_STICKS;

  public OneRowNim() { }        // Constructors
  public OneRowNim(int sticks) {   
    nSticks = sticks;
  }  // OneRowNim()
  public OneRowNim(int sticks, int starter) {   
    nSticks = sticks;
    setPlayer(starter);
  } // OneRowNim()
  public boolean takeSticks(int num) {   
    if (num < 1 || num > MAX_PICKUP || num > nSticks) 
      return false;                // Error
    else                             // Valid move
    {   nSticks = nSticks - num;
        return true;
    } //else
  } // takeSticks()
  public int getSticks() {   
    return nSticks;
  } // getSticks()
  public String getRules() {
    return "\n*** The Rules of One Row Nim ***\n" +
    "(1) A number of sticks between 7 and " + MAX_STICKS + 
         " is chosen.\n" + 
    "(2) Two players alternate making moves.\n" + 
    "(3) A move consists of subtracting between 1 and\n\t" + 
        MAX_PICKUP + " sticks from the current number of sticks.\n" + 
    "(4) A player who cannot leave a positive\n\t" + 
    " number of sticks for the other player loses.\n";
  } // getRules()
  public boolean gameOver() {   /** From TwoPlayerGame */
    return (nSticks <= 0);
  }  // gameOver()
  public String getWinner()  {       /** From TwoPlayerGame */
    if (gameOver()) //{
      return "" + getPlayer() + " Nice game.";
    return "The game is not over yet.";   // Game is not over
  } // getWinner()
\end{lstlisting}
\end{jjjlisting}
\jjjprogstop{The revised  {\tt OneRowNim} class, Part I.}
{fig-revisednim}
\end{figure}

\begin{figure}[h!]
\jjjprogstart
\begin{jjjlistingleft}[32pc]{-6pc}
\begin{lstlisting}
  /** From CLUIPlayableGame */
  public String getGamePrompt() {
    return "\nYou can pick up between 1 and " + 
                    Math.min(MAX_PICKUP,nSticks) + " : ";
  } // getGamePrompt()
  public String reportGameState() {   
    if (!gameOver())
       return ("\nSticks left: " + getSticks() + 
          " Who's turn: Player " + getPlayer());
    else
       return ("\nSticks left: " + getSticks() + 
        " Game over! Winner is Player "  + getWinner() +"\n");
  }   // reportGameState()
  public void play(UserInterface ui) {// From CLUIPlayableGame interface
    int sticks = 0;
    ui.report(getRules());
    if (computer1 != null) 
      ui.report("\nPlayer 1 is a " + computer1.toString());
    if (computer2 != null)
      ui.report("\nPlayer 2 is a " + computer2.toString());
    while(!gameOver()) {  
      IPlayer computer = null;     // Assume no computers
      ui.report(reportGameState());
      switch(getPlayer()) {
      case PLAYER_ONE:             // Player 1's turn
        computer = computer1;
        break;
      case PLAYER_TWO:             // Player 2's turn
        computer = computer2;
        break;
      } // cases
      if (computer != null) {      // If computer's turn
        sticks = Integer.parseInt(computer.makeAMove(""));
        ui.report(computer.toString() + " takes " + 
                                   sticks + " sticks.\n");
      } else {                  // otherwise, user's turn
        ui.prompt(getGamePrompt());
        sticks = 
          Integer.parseInt(ui.getUserInput()); // Get user's move 
      }
      if (takeSticks(sticks))      // If a legal move
        changePlayer();
    } // while
    ui.report(reportGameState());  // The game is now over
  } // play()
} // OneRowNim class
\end{lstlisting}
\end{jjjlistingleft}
\jjjprogstop{The revised  {\tt OneRowNim} class, continued from previous page.}
{fig-revisednim2}
\end{figure}

The {\tt gameOver()} and {\tt getWinner()} methods, which are now
inherited from the {\tt TwoPlayerGame} superclass, are virtually the
same as in the previous version. One small change is that {\tt
getWinner()} now returns a {\tt String} instead of an {\tt int}.  This
makes that method more generally useful as a way of identifying the
winner for all {\tt TwoPlayerGame}s.  

Similarly, the {\tt getGamePrompt()} and {\tt reportGameState()}
methods merely encapsulate functionality that was present in the
earlier version of the game. In our earlier version the prompts to the
user were generated directly by the main program.  By encapsulating
\marginnote{Inheritance and generality}
this information in an inherited method, we make it more generally
useful to all {\tt TwoPlayerGame}s.

The major change to {\tt OneRowNim} comes in the {\tt play()} method,
which controls the playing of the {\tt OneRowNim}
(Fig.~\ref{fig-revisednim2}).  Because this version of the game
incorporates computer players, the play loop is a bit more complex
than in earlier versions of the game.  The basic idea is still the
same: The method loops until the game is over.  On each iteration of
the loop, one or the other of the two players, {\tt PLAYER\_ONE} or
{\tt PLAYER\_TWO}, takes a turn making a move---that is, deciding how
many sticks to pick up. If the move is a legal move, then it becomes
the other player's turn.

Let's look now at how the code distinguishes between whether it is a
computer's turn to move or a human player's turn.  Note that at the
beginning of the {\tt while} loop, it sets the {\tt computer} variable
to null. It then assigns {\tt computer} a value of either {\tt
computer1} or {\tt computer2}, depending on whose turn it is. But
recall that one or both of these variables may be {\tt null},
depending on how many computers are playing the game. If there are no
computers playing the game, then both variables will be null. If only
one computer is playing, then {\tt computer1} will be null. This is
determined during initialization of the game, when the {\tt addComputerPlayer()}
is called. (See above.)

In the code following the {\tt switch} statement, if {\tt computer} is
not {\tt null}, then we call {\tt computer.makeAMove()}. As we know,
the {\tt makeAMove()} method is part of the {\tt IPlayer}
interface. The {\tt makeAMove()} method takes a {\tt String} parameter
that is meant to serve as a prompt, and returns a {\tt String} that is
meant to represent the {\tt IPlayer}'s move:

\begin{jjjlisting}
\begin{lstlisting}
 public interface IPlayer {
     public String makeAMove(String prompt);
 }
\end{lstlisting}
\end{jjjlisting}

\noindent In {\tt OneRowNim} the ``move'' is an integer, representing
the number of sticks the player picks. Therefore, in {\tt play()}
{\tt OneRowNim} has to convert the {\tt String} into an {\tt int}, which
represents the number of sticks the {\tt IPlayer} picks up.  

On the other hand, if {\tt computer} is {\tt null}, this means that it
is a human user's turn to play. In this case, {\tt play()} calls {\tt
ui.getUserInput()}, employing the user interface to input a value from the
keyboard.  The user's input must also be converted from {\tt String}
to {\tt int}.  Once the value of {\tt sticks} is set, either from the
user or from the {\tt IPlayer}, the {\tt play()} method calls {\tt
takeSticks()}. If the move is legal, then it changes whose turn it is,
and the loop repeats.

There are a couple of important points to notice about the design of
\marginnote{Encapsulation of game-dependent knowledge}
the {\tt play()} method. First, the {\tt play()} method has to know
what to do with the input it receives from the user or the {\tt
IPlayer}.  This is game-dependent knowledge. The user is inputting the
number of sticks to take in {\tt OneRowNim}. For a tic-tac-toe game,
the ``move'' might represent a square on the tic-tac-toe board. This
suggests that {\tt play()} is a method that should be implemented in
{\tt OneRowNim}, as it is here, because {\tt OneRowNim} encapsulates
the knowledge of how to play the One Row Nim game.

A second point is to notice that the method call {\tt
computer.make\-AMove()} is another example of polymorphism at work.
\marginnote{Polymorphism}
The {\tt play()} method does not know what type of object the {\tt
computer} is, other than that it is an {\tt IPlayer}---that is, an
object that implements the {\tt IPlayer} interface. As we will show in
the next section, the {\tt OneRowNim} game can be played by two
different {\tt IPlayer}s: one named {\tt NimPlayer} and another named
{\tt NimPlayerBad}.  Each has its own game-playing strategy, as
implemented by their own versions of the {\tt makeAMove()} method.
Java uses dynamic binding to decide which version of {\tt makeAMove()}
to invoke depending on the type of {\tt IPlayer} whose turn it is.
Thus, by defining different {\tt IPlayer}s with different {\tt
makeAMove()} methods, this use of polymorphism makes it possible to
test different game-playing strategies against each other.

\subsection{The {\tt IPlayer} Interface}

The last element of our design is the {\tt IPlayer} interface, which,
as we just saw, consists of the {\tt makeAMove()} method.  To see how
we use this interface, let's design a class to play the game of {\tt
OneRowNim}.  We will call the class {\tt NimPlayerBad} and give it a
very weak playing strategy. For each move it will pick a random number
between 1 and 3, or between 1 and the total number of sticks left, if
there are fewer than 3 sticks.  (We will leave the task of defining
{\tt NimPlayer}, a good player, as an exercise.)
\marginfig{chptr08/nimplayer.eps}{Design of the {\tt NimPlayerBad} class.}
{fig-nimplayerbad}

As an implementer of the {\tt IPlayer} interface, {\tt NimPlayerBad}
will implement the {\tt makeAMove()} method. This method will contain
{\tt NimPlayerBad}'s strategy (algorithm) for playing the game.  The
result of this strategy will be the number of sticks that the player
will pick up.

What other elements (variables and methods) will a {\tt NimPlayerBad}
need?  Clearly, in order to play {\tt OneRowNim}, the player must know
the rules and the current state of the game.  The best way
to achieve this is to give the Nim player a reference to the {\tt
OneRowNim} game.  Then it can call {\tt getSticks()} to determine how
many sticks are left, and it can use other public elements of the {\tt
OneRowNim} game. Thus, we will have a variable of type {\tt OneRowNim},
and we will assign it a value in a constructor method.

Figure~8.23 shows the design of {\tt NimPlayerBad}. Note that we have
added an implementation of the {\tt toString()} method.  This will
be used to give a string representation of the {\tt NimPlayerBad}.
Also, note that we have added a private helper method named
{\tt randomMove()}, which will simply generate an appropriate
random number of sticks as the player's move.

\begin{figure}[h]
\jjjprogstart
\begin{jjjlisting}[28pc]
\begin{lstlisting}
public class NimPlayerBad implements IPlayer {   
  private OneRowNim game;
  public NimPlayerBad (OneRowNim game) {  
    this.game = game;
  } // NimPlayerBad()
  public String makeAMove(String prompt) {   
    return "" + randomMove();
  } // makeAMove()
  private int randomMove() {   
    int sticksLeft = game.getSticks();
    return 1 + (int)(Math.random() * 
           Math.min(sticksLeft, game.MAX_PICKUP));
  } // randomMove()
  public String toString() { 
    String className = 
        this.getClass().toString(); // Gets 'class NimPlayerBad'
    return className.substring(5);  // Cut off the word 'class'
  } // toString()
} // NimPlayerBad
\end{lstlisting}
\end{jjjlisting}
\jjjprogstop{The {\tt NimPlayerBad} class.}
{fig-nimbad}
\end{figure}

The implementation of {\tt NimPlayerBad} is shown in
Figure~\ref{fig-nimbad}. The {\tt makeAMove()} method converts the
{\tt randomMove()} to a {\tt String} and returns it, leaving it up to
{\tt OneRowNim}, the calling object, to convert that move back into an
{\tt int}.  Recall the statement in {\tt OneRowNim} where {\tt
makeAMove()} is invoked:

\begin{jjjlisting}
\begin{lstlisting}
 sticks = Integer.parseInt(computer.makeAMove(""));
\end{lstlisting}
\end{jjjlisting}

\noindent In this context, the {\tt computer} variable, which is of
type {\tt IPlayer}, is bound to a {\tt NimPlayerBad} object.  In order
for this interaction between the game and a player to work, the {\tt
OneRowNim} object must know what type of data is being returned by
{\tt NimPlayerBad}.  This is a perfect use for a Java interface, which
specifies the signature of {\tt makeAMove()} without committing to any
particular implementation of the method.  Thus, the association between
{\tt OneRowNim} and {\tt IPlayer} provides a flexible and effective model
for this type of interaction.

\JavaTIP[false]{EFFECTIVE DESIGN}{Interface Associations.}{Java interfaces 
provide a flexible way to set up associations between two different
types of objects.}

Finally, note the details of the {\tt randomMove()} and {\tt
toString()} methods.  The only new thing here is the use of the {\tt
getClass()} method in {\tt toString()}.  This is a method that is
defined in the {\tt Object} class and inherited by all Java
objects. It returns a {\tt String} of the form ``class X'' where {\em
X} is the name of that object's class. Note here that we are removing
the word ``class'' from this string before returning the class name.
This allows our {\tt IPlayer} objects to report what type of players
they are, as in the following statement from {\tt OneRowNim}:

\begin{jjjlisting}
\begin{lstlisting}
ui.report("\nPlayer 1 is a " + computer1.toString());
\end{lstlisting}
\end{jjjlisting}

\noindent If {\tt computer1} is a {\tt NimPlayerBad}, it would
report ``Player1 is a NimPlayerBad.''

\pagebreak
\secEXRHone{Self-Study Exercises}
\label{self-study-exercises}
\begin{SSTUDY}

\item Define a class {\tt NimPlayer} that plays the optimal strategy
for {\tt OneRowNim}. This strategy was described in Chapter~5. 

\end{SSTUDY}

\subsection{Playing {\tt OneRowNim}}

Let's now write a {\tt main()} method to play {\tt OneRowNim}:

\begin{jjjlisting}[26.5pc]
\begin{lstlisting}
public static void main(String args[]) {
  KeyboardReader kb = new KeyboardReader();
  OneRowNim game = new OneRowNim();
  kb.prompt("How many computers are playing, 0, 1, or 2? ");
  int m = kb.getKeyboardInteger();
  for (int k = 0; k < m; k++) {
    kb.prompt("What type of player, " + 
                  "NimPlayerBad = 1, or NimPlayer = 2 ? ");
    int choice = kb.getKeyboardInteger();
    if (choice == 1) {
      IPlayer computer = new NimPlayerBad(game);
      game.addComputerPlayer(computer);
    } else {
      IPlayer computer = new NimPlayer(game);
      game.addComputerPlayer(computer);
    }
  }
  game.play(kb);
 } // main()
\end{lstlisting}
\end{jjjlisting}

\noindent After creating a {\tt KeyboardReader} and then creating an
instance of {\tt OneRowNim}, we prompt the user to determine how many
computers are playing. We then repeatedly prompt the user to identify
the names of the {\tt IPlayer} and use the {\tt addComputerPlayer()}
method to initialize the game.  Finally, we get the game started by
invoking the {\tt play()} method, passing it a reference to the {\tt
KeyboardReader}, our {\tt UserInterface}.

Note that in this example we have declared a {\tt OneRowNim} variable
to represent the game.  This is not the only way to do things.  For
example, suppose we wanted to write a {\tt main()} method that could
be used to play a variety of different {\tt TwoPlayerGame}s.  Can we
\marginnote{Generality}
make this code more general? That is, can we rewrite it to work with
any {\tt TwoPlayerGame}?

A {\tt OneRowNim} object is also a {\tt TwoPlayerGame}, by virtue of
inheritance, and it is also a {\tt CLUIPlayableGame}, by virtue of
implementing that interface.  Therefore, we can use either of these
types to represent the game.  Thus, one alternative way of coding this
is as follows:

\begin{jjjlisting}
\begin{lstlisting}
    TwoPlayerGame game = new OneRowNim();
    ...
    IPlayer computer = new NimPlayer((OneRowNim)game);
    ...
    ((CLUIPlayableGame)game).play(kb);
\end{lstlisting}
\end{jjjlisting}

\noindent Here we use a {\tt TwoPlayerGame} variable to represent the
game. However, note that we now have to use a cast expression, {\tt
(CLUIPlayableGame)}, in order to call the {\tt play()} method.  If we
don't cast {\tt game} in this way, Java will generate the following
syntax error:

\begin{jjjlisting}
\begin{lstlisting}
 OneRowNim.java:126: cannot resolve symbol
 symbol  : method play (KeyboardReader)
 location: class TwoPlayerGame
        game.play(kb);
           ^
\end{lstlisting}
\end{jjjlisting}

\noindent The reason for this error is that {\tt play()} is not a
method in the {\tt TwoPlayerGame} class, so the compiler cannot find
the {\tt play()} method. By using the cast expression, we are telling
the compiler to consider {\tt game} to be a {\tt
CLUIPlayableGame}. That way it will find the {\tt play()} method. Of
course, the object assigned to {\tt nim} must actually implement the
{\tt CLUIPlayableGame} interface in order for this to work at run
time.  We also need a cast operation in the {\tt NimPlayer()}
constructor in order to make the argument ({\tt computer}) compatible
with that method's parameter.

Another alternative for the {\tt main()} method would be the
following:

\begin{jjjlisting}
\begin{lstlisting}
 CLUIPlayableGame game = new OneRowNim();
 ...
 IPlayer computer = new NimPlayer((OneRowNim)game);
 ((TwoPlayerGame)game).addComputerPlayer(computer);
 ...
 game.play(kb);
  nim.play(kb);
\end{lstlisting}
\end{jjjlisting}

\noindent By representing the game as a {\tt
CLUIPlayableGame} variable, we don't need the cast expression to call
{\tt play()}, but we do need a different cast expression, {\tt
(TwoPlayerGame)}, to invoke {\tt addComputerPlayer()}.  Again, the
reason is that the compiler cannot find the {\tt add\-Computer\-Player()}
method in the {\tt CLUIPlayableGame} interface, so we must tell it to
consider {\tt game} as a {\tt TwoPlayerGame}, which of course it
is. We still need the cast operation for the call to the {\tt
NimPlayer()} constructor.

All three of the code options that we have considered will generate
something like the interactive session shown in
Figure~\ref{fig-playnim} for a game in which two {\tt IPlayer}s play
each other.


\begin{figure}[h]
\jjjprogstart
\begin{jjjlistingleft}[28.5pc]{-2.5pc}
\begin{lstlisting}[stringstyle=\color{black}]
How many computers are playing, 0, 1, or 2? {\color{cyan}2}

*** The Rules of One Row Nim ***
(1) A number of sticks between 7 and 11 is chosen.
(2) Two players alternate making moves.
(3) A move consists of subtracting between 1 and
        3 sticks from the current number of sticks.
(4) A player who cannot leave a positive
         number of sticks for the other player loses.

Player 1 is a  NimPlayerBad
Player 2 is a  NimPlayer
Sticks left: 11 Who's turn: Player 1 NimPlayerBad takes 2 sticks.
Sticks left: 9 Who's turn: Player 2 NimPlayer takes 1 sticks.
Sticks left: 8 Who's turn: Player 1 NimPlayerBad takes 2 sticks.
Sticks left: 6 Who's turn: Player 2 NimPlayer takes 1 sticks.
Sticks left: 5 Who's turn: Player 1 NimPlayerBad takes 3 sticks.
Sticks left: 2 Who's turn: Player 2 NimPlayer takes 1 sticks.
Sticks left: 1 Who's turn: Player 1 NimPlayerBad takes 1 sticks.
Sticks left: 0 Game over! Winner is Player 2 Nice game.
\end{lstlisting}
\end{jjjlistingleft}
\jjjprogstop{A typical run of the {\tt OneRowNim} using a command-line
user interface. }
{fig-playnim}
\end{figure}

Given our object-oriented design for the {\tt TwoPlayerGame}
hierarchy, we can now write generalized code that can play any {\tt
TwoPlayerGame} that implements the {\tt CLUIPlayableGame}
interface. We will give a specific example of this in the next
section.

\subsection{Extending the {\tt TwoPlayerGame} Hierarchy}

Now that we have described the design and the details of the
{\tt TwoPlayerGame} class hierarchy, let's use it to develop
a new game. If we've gotten the design right, developing
new two-player games and adding them to the hierarchy should 
be much simpler than developing them from scratch. 

The new game is a guessing game in which the two players take turns
guessing a secret word.  The secret word will be generated randomly
from a collection of words maintained by the game object. The letters
of the word will be hidden with question marks, as in ``????????.''
On each turn a player guesses a letter. If the letter is in the secret
word, it replaces one or more question marks, as in ``??????E?.''  A
player continues to guess until an incorrect guess is made and then it
becomes the other player's turn.  Of course, we want to develop a
version of this game that can be played either by two humans, or by
one human against a computer---that is, against an {\tt IPlayer}---or
by two different {\tt IPlayer}s.

\begin{figure}[h]
\figaright{chptr08/wordguess.eps}{Design of the {\tt WordGuess}
class as part of {\tt TwoPlayerGame} hierarchy.
} {fig-wordguessuml}

\end{figure}

Let's call the game class {\tt WordGuess}. Following the design of
{\tt OneRowNim}, we get the design shown in 
Figure~\ref{fig-wordguessuml}.  The {\tt WordGuess} class extends the
{\tt TwoPlayerGame} class and implements the {\tt CLUIPlayableGame}
interface. We don't show the details of the interfaces and the {\tt
TwoPlayerGame} class, as these have not changed.  Also, following the
design of {\tt NimPlayerBad}, the {\tt WordGuesser} class implements
the {\tt IPlayer} interface.  Note how we show the association between
{\tt WordGuess} and zero or more {\tt IPlayer}s.  A {\tt WordGuess}
uses between zero and two instances of {\tt IPlayer}s, which in
this game are implemented as {\tt WordGuesser}s.

\begin{figure}[h!!]
\jjjprogstart
\begin{jjjlisting}
\begin{lstlisting}[basicstyle=\scriptsize]
public class WordGuess extends TwoPlayerGame implements CLUIPlayableGame {
    private String secretWord;
    private StringBuffer currentWord;
    private StringBuffer previousGuesses;
    private int unguessedLetters;

    public WordGuess() {   
        secretWord = getSecretWord();
        currentWord = new StringBuffer(secretWord);
        previousGuesses = new StringBuffer();
        for (int k = 0; k < secretWord.length(); k++)
           currentWord.setCharAt(k,'?');
        unguessedLetters = secretWord.length();
    } // WordGuess()
    public String getPreviousGuesses() {
        return previousGuesses.toString();
    } // getPreviousGuesses()
    public String getCurrentWord() {
        return currentWord.toString();
    } // getCurrentWord()
    private String getSecretWord() {   
        int num = (int)(Math.random()*10);
        switch (num)
        {   case 0: return "SOFTWARE";
            case 1: return "SOLUTION";
            case 2: return "CONSTANT";
            case 3: return "COMPILER";
            case 4: return "ABSTRACT";
            case 5: return "ABNORMAL";
            case 6: return "ARGUMENT";
            case 7: return "QUESTION";
            case 8: return "UTILIZES";
            case 9: return "VARIABLE";
            default: return "MISTAKES";
        } //switch
    } // getSecretWord()
    private boolean guessLetter(char letter) {   
       previousGuesses.append(letter);
       if (secretWord.indexOf(letter) == -1)
            return false; // letter is not in secretWord
        else // find positions of letter in secretWord
        {   for (int k = 0; k < secretWord.length(); k++)
            {   if (secretWord.charAt(k) == letter)
                {   if (currentWord.charAt(k) == letter)
                        return false; ////already guessed
                    currentWord.setCharAt(k,letter);
                    unguessedLetters--; //one less to find
                } //if
            } //for
            return true;
        } //else
    } //guessLetter()

    public String getRules() {      // Overridden from TwoPlayerGame
        return "\n*** The Rules of Word Guess ***\n" +
        "(1) The game generates a secret word.\n" +
        "(2) Two players alternate taking moves.\n" +
        "(3) A move consists of guessing a letter in the word.\n" +
        "(4) A player continues guessing until a letter is wrong.\n" +
        "(5) The game is over when all letters of the word are guessed\n" +
        "(6) The player guessing the last letter of the word wins.\n";
    } //getRules()
\end{lstlisting}
\end{jjjlisting}
\jjjprogstop{The {\tt WordGuess} class, Part I.}
{fig-wordguess-oop}
\end{figure}



\begin{figure}[h!]
\jjjprogstart
\begin{jjjlisting}[29pc]
\begin{lstlisting}[basicstyle=\scriptsize]
    public boolean gameOver() {   // From TwoPlayerGame
        return (unguessedLetters <= 0);
    } // gameOver()
    public String getWinner() {  // From TwoPlayerGame
        if (gameOver())
            return "Player " + getPlayer();
        else return "The game is not over.";
    } // getWinner()
    public String reportGameState() {   
        if (!gameOver())
            return "\nCurrent word " + currentWord.toString() + " Previous guesses " 
                + previousGuesses + "\nPlayer " + getPlayer() + " guesses next.";
        else
	    return "\nThe game is now over! The secret word is " + secretWord 
              + "\n" + getWinner() + " has won!\n";
    } // reportGameState()
    public String getGamePrompt() {    // From CLUIPlayableGame
        return "\nGuess a letter that you think is in the secret word: ";
    } // getGamePrompt()
    public String move(String s) {   
        char letter = s.toUpperCase().charAt(0);
        if (guessLetter(letter)) {   //if correct 
            return "Yes, the letter " + letter +
                   " IS in the secret word\n";
         } else {   
             changePlayer();
             return "Sorry, " + letter + " is NOT a " +
                    "new letter in the secret word\n";
         } 
    } // move()
    public void play(UserInterface ui) {  // From CLUIPlayableGame
        ui.report(getRules());
        if (computer1 != null) 
            ui.report("\nPlayer 1 is a " + computer1.toString());
        if (computer2 != null)
            ui.report("\nPlayer 2 is a " + computer2.toString());

        while(!gameOver()) {  
           IPlayer computer = null;       // Assume no computers playing
           ui.report(reportGameState());
           switch(getPlayer()) {
           case PLAYER_ONE:             // Player 1's turn
               computer = computer1;
               break;
           case PLAYER_TWO:             // Player 2's turn
               computer = computer2;
               break;
           } // cases

           if (computer != null) {      // If computer's turn
               ui.report(move(computer.makeAMove("")));
           } else {                     // otherwise, user's turn
               ui.prompt(getGamePrompt());
               ui.report(move(ui.getUserInput()));
           }
        } // while
        ui.report(reportGameState());  // The game is now over
    } //play()
} //WordGuess class
\end{lstlisting}
\end{jjjlisting}
\jjjprogstop{The {\tt WordGuess} class, continued.}
{fig-wordguess2-oop}
\end{figure}

Let's turn now to the details of the {\tt WordGuess} class, whose
source code is shown in Figures~\ref{fig-wordguess-oop} and
\ref{fig-wordguess2-oop}.  The game needs to have a supply of words from
which it can choose a secret word to present to the players.  The {\tt
getSecretWord()} method will take care of this task. It calculates a
random number and then uses that number, together with a {\tt switch}
statement, to select from among several words that are coded right into
the {\tt switch} statement.  The secret word is stored in the {\tt
secretWord} variable.  The {\tt currentWord} variable stores the
partially guessed word. Initially, {\tt currentWord} consists entirely
of question marks.  As the players make correct guesses, {\tt
currentWord} is updated to show the locations of the guessed letters.
Because {\tt currentWord} will change as the game progresses, it is
stored in a {\tt StringBuffer}, rather than in a {\tt String}. Recall
that {\tt String}s are immutable in Java, whereas a {\tt StringBuffer}
contains methods to insert letters and remove letters.

The {\tt unguessedLetters} variable stores the number of letters
remaining to be guessed.  When {\tt unguessedLetters} equals 0, the
game is over.  This condition defines the {\tt gameOver()}
method, which is inherited from {\tt TwoPlayerGame}.  The winner of
the game is the player who guessed the last letter in the secret word.
This condition defines the {\tt getWinner()} method, which
is also inherited from {\tt TwoPlayerGame}. The other methods that are
inherited from {\tt TwoPlayerGame} or implemented from the {\tt
CLUIPlayableGame} are also implemented in a straightforward manner.

A move in the {\tt WordGuess} game consists of trying to guess a
letter that occurs in the secret word. The {\tt move()} method
processes the player's guesses. It passes the guessed letter to the
{\tt guessLetter()} method, which checks whether the letter is a new,
secret letter. If so, {\tt guessLetter()} takes care of the various
housekeeping tasks. It adds the letter to {\tt previousGuesses},
which keeps track of all the players' guesses.  It decrements the
number of {\tt unguessedLetters}, which will become 0 when all the
letters have been guessed. And it updates {\tt currentWord} to show
where all occurrences of the secret letter are located.  Note how {\tt
guessLetter()} uses a for-loop to cycle through the letters in the
secret word. As it does so, it replaces the question marks in {\tt
currentWord} with the correctly guessed secret letter.  The {\tt
guessLetter()} method returns false if the guess is incorrect.  In
that case, the {\tt move()} method changes the player's turn.  When
correct guesses are made, the current player keeps the turn.

The {\tt WordGuess} game is a good example of a string-processing
\marginnote{Reusing code}
problem.  It makes use of several of the {\tt String} and {\tt
StringBuffer} methods that we learned in Chapter~7.  The
implementation of {\tt WordGuess}, as an extension of {\tt
TwoPlayerGame}, is quite straight forward. One advantage of
the {\tt TwoPlayerGame} class hierarchy is that it decides many
of the important design issues in advance. Developing a new game
is largely a matter of implementing methods whose definitions have
already been determined in the superclass or in the interfaces.  This
greatly simplifies the development process.

Let's now discuss the details of {\tt WordGuesser} class
(Fig.~\ref{fig-wordguesser-oop}). Note that the constructor takes a {\tt
WordGuess} parameter. This allows {\tt WordGuesser} to be passed a
reference to the game, which accesses the game's {\tt public}
methods, such as {\tt getPreviousGuesses()}.  The {\tt toString()}
method is identical to the {\tt toString()} method in the {\tt
NimPlayerBad} example.  The {\tt makeAMove()} method, which is part of
the {\tt IPlayer} interface, is responsible for specifying the
algorithm that the player uses to make a move.  The strategy in this
case is to repeatedly pick a random letter from {\em A} to {\em Z}
until a letter is found that is not contained in {\tt
previousGuesses}.  That way, the player will not guess letters that
have already been guessed. 

\begin{figure}[t]
\jjjprogstart
\begin{jjjlisting}
\begin{lstlisting}
public class WordGuesser implements IPlayer {   
  private WordGuess game;
  public WordGuesser (WordGuess game) {   
    this.game = game;
  }
  public String makeAMove(String prompt)  {   
    String usedLetters = game.getPreviousGuesses();
    char letter;
    do {  // Pick one of 26 letters
      letter = (char)('A' + (int)(Math.random() * 26));
    } while (usedLetters.indexOf(letter) != -1);
    return "" + letter;
  }
  public String toString() { // returns 'NimPlayerBad'
    String className = this.getClass().toString(); 
    return className.substring(5);        
  }
} // WordGuesser
\end{lstlisting}
\end{jjjlisting}
\jjjprogstop{The {\tt WordGuesser} class.}
{fig-wordguesser-oop}
\end{figure}

\section{Principles Of Object-Oriented Design}
\label{principles-of-object-oriented-design}

To conclude this chapter, it will be helpful to focus briefly on how
the examples we've seen address the various object-oriented design
(OOD) principles we set out at the beginning of the book.

\begin{itemize}
\item {\bf Divide-and-Conquer Principle.}  Notice how all of the
problems tackled in this chapter have been solved by dividing them
into several classes, with each of the classes divided into separate
methods.  The very idea of a class hierarchy is an application of
this principle.

\item {\bf Encapsulation Principle.} The superclasses in our designs,
{\tt Cipher} and {\tt TwoPlayerGame}, encapsulate those features of the
class hierarchy that are shared by all objects in the hierarchy. The
subclasses, {\tt CaesarCipher} and {\tt OneRowNim}, encapsulate features
that make them distinctive with the class hierarchy. 

\item {\bf Interface Principle.}  The several Java interfaces we've
designed, {\tt IPlayer}, {\tt CLUIPlayableGame} and {\tt UserInterface},
specify clearly how various types of related objects will interact
with each other through the methods contained in the interfaces.
Clean interfaces make for clear communication among objects.

\item {\bf Information Hiding Principle.} We have continued to make
consistent use of the {\tt private} and {\tt public} qualifiers, and have
now introduced the {\tt protected} qualifier to extend this concept. 
The inheritance mechanism gives subclasses access to protected and
public elements of their superclasses.  

\item {\bf Generality Principle.}  As you move down a well-designed
class hierarchy, you go from the more general to the more specific
features of the objects involved. The abstract {\em encode()} method
specifies the general form that encoding will take while the various
implementations of this method in the subclasses provide the
specializations necessary to distinguish, say, Caesar encoding from
Transpose encoding.  Similarly, the abstract {\tt makeAMove()} method
in the {\tt IPlayer} interface  provides a general format for
a move in a two-player game, while its various implementations provide
the specializations that distinguish one game from another.

\item {\bf Extensibility Principle.}  Overriding inherited methods and
implementing abstract methods from either an abstract superclass or a
Java interface provide several well-designed ways to extend the
functionality in an existing class hierarchy. Extending a class is a
form of specialization of the features inherited from the superclass.

\item {\bf Abstraction Principle.} Designing a class hierarchy is an
exercise in abstraction, as the more general features of the objects
involved are moved into the superclasses. Similarly, designing a Java
interface or an abstract superclass method is a form of abstraction,
whereby the signature of the method is distinguished from its various
implementations. 

\end{itemize}

These, then, are some of the ways that the several examples we have
considered and this chapter's discussion have contributed
to a deepening of our understanding of object-oriented design.

\pagebreak
\secSMHleft{Chapter Summary} 
\secKTH{Technical Terms}
\label{technical-terms}

\begin{KT}
abstract method

actual type (dynamic type)

ciphertext

class inheritance 

cryptography

dynamic binding (late binding)

interface

overloaded method

plaintext

polymorphic method

polymorphism 

static binding (early binding)

static type (declared type)

substitution cipher

transposition cipher

\end{KT}


\secKTH{Summary of Important Points}
\label{summaryof-important-points}
\begin{SMBL}
\item Inheritance is an object-oriented mechanism whereby subclasses inherit
the public and protected instance variables and methods from their superclasses.

\item Dynamic binding (or late binding) is the mechanism by which a
method call is bound to (associated with) the correct implementation
of the method at run time. In Java, all method calls, except for {\tt
final} or {\tt private} methods, are resolved using dynamic binding.

\item Static binding (or early binding) is the association of a method
call with its corresponding implementation at compile time.

\item Polymorphism is an object-oriented language feature in which a
method call can lead to different actions depending on the object on
which it is invoked.  A polymorphic method is a method signature that
is given different implementation by different classes in a class
hierarchy.

\item A static type is a variable's declared type. A dynamic type, or actual type,
is the type of object assigned to that variable at a given point in a running program.

\item An {\tt abstract} method is a method definition that lacks an
implementation. An {\tt abstract} class is one that contains one or
more {\tt abstract} methods.  An {\tt abstract} class can be
subclassed but not instantiated.

\item A Java interface is a class that contains only method signatures and
(possibly) constant declarations, but no variables.  An interface can be
implemented by a class by providing implementations for all of its abstract
methods.

\end{SMBL}


\secANSHleft
\begin{ANS}

\item Running the {\tt TestPrint} program will produce the 
output shown here. It is clear that the inherited {\tt toString()}
method is used by {\tt println()} when printing a {\tt TestPrint}
object.  

\begin{jjjlisting}
\begin{lstlisting}
 56.0
 TestPrint@be2d65
\end{lstlisting}
\end{jjjlisting}

\item If you override the {\tt toString()} method in {\tt TestPrint},
you should get something like the output shown here, depending on how
you code {\tt toString()}.  It is clear that the {\tt toString()}
method is used polymorphously by {\tt println()}.

\begin{jjjlisting}
\begin{lstlisting}
 56.0
 Hello from TestPrint
\end{lstlisting}
\end{jjjlisting}

\item The output produced when constructing objects of type {\em A}
and {\tt B} in the order shown in the exercise would be as follows,
with each letter occurring on a separate line:

\begin{jjjlisting}
\begin{lstlisting}
 A B B
\end{lstlisting}
\end{jjjlisting}

\item The new implementation of {\tt B}'s {\tt method()} will
invoke {\tt A}'s version of the method before printing {\em B}.
This will print ``A A B A B''.

\begin{jjjlisting}
\begin{lstlisting}
 void method () { 
     super.method();  
     System.out.println("B");
 }
\end{lstlisting}
\end{jjjlisting}

\item Give the definitions of classes {\tt A} and {\tt B} in the
exercise, the marked statements would be invalid:

\begin{jjjlisting}
\begin{lstlisting}
A a = new B();   // Valid a B is an A
a = new A();     // Ok
B b = new A();   // Invalid. An A is not necessarily a B
b = new B();     // OK
\end{lstlisting}
\end{jjjlisting}

\item Given the class definitions and code segment in this exercise, the
output would be, {\tt A A B A B C}, with each letter printing on a separate
line.

\item Definition of an {\tt Pig} subclass of {\tt Animal}:

\begin{jjjlisting}
\begin{lstlisting}
 public class Pig extends Animal {
    public Pig() {
        kind = "pig";
    }
    public String speak() {
        return "oink";
    }
}
\end{lstlisting}
\end{jjjlisting}

\item If polymorphism was not used in our design, the {\tt talk()}
method would have to be modified to the following in order to accommodate
a {\tt Pig} subclass:

\begin{jjjlisting}
\begin{lstlisting}
public String talk(Animal a) {
  if (a instanceof Cow)
     return "I am a " + kind + " and I go " + a.moo();
   else if (a instanceof Cat)
     return "I am a " + kind + " and I go " + a.meow();
   else if (a instanceof Pig)
     return "I am a " + kind + " and I go " + a.oink();
   else
     return "I don't know what I am";
}
\end{lstlisting}
\end{jjjlisting}

\item Code to swap two {\tt boolean} variables:

\begin{jjjlisting}
\begin{lstlisting}
 boolean temp = b1; // Save b1's value
 b1 = b2;           // Change b1 to b2
 b2 = temp;         // Change b2 to b1's original value
\end{lstlisting}
\end{jjjlisting}

\item Creating a {\tt ToggleButton} that can be used to deal or
collect cards:

\begin{jjjlisting}
\begin{lstlisting}
 private ToggleButton dealer = 
            new ToggleButton("deal","collect");
 add(dealer);
 dealer.addActionListener(this);
\end{lstlisting}
\end{jjjlisting}



%{Exercise 7.20}
\item  Modify the {\tt Caesar} class so that it will allow various-sized
shifts to be used.

\begin{jjjlisting}
\begin{lstlisting}
private int shift;                               
public void setShift(int n) { shift = n;    }
public int getShift()       { return shift; }
// Modification to encode():
ch = (char)('a' + (ch -'a'+ shift) % 26); // Shift
// Modification to decode():
ch = (char)('a' + (ch -'a'+ (26-shift)) % 26); // Shift
\end{lstlisting}
\end{jjjlisting}


%{Exercise 7.21}
\item  Modify {\tt Transpose.encode()} so that it uses a rotation instead of
a reversal.  The operation here is very similar to the shift operation
in the Caesar cipher.  It uses modular arithmetic to rearrange the letters
in the word. For example, suppose the word is ``hello''.  Its letters are
indexed from 0 to 4. The following table shows how the expression \verb|((k+2) % 5)|
will rearrange the letters as {\it k} varies from 0 to 4:

\begin{jjjlisting}
\begin{lstlisting}
    k  charAt(k)  (k+2) % 5   charAt((k+2) % 5)
    -------------------------------------------
    0  'h'          2          'l'
    1  'e'          3          'l'
    2  'l'          4          'o'
    3  'l'          0          'h'
    4  'o'          1          'e'

// Modification to encode():
public String encode(String word) {
    StringBuffer result = new StringBuffer();
    for (int k=0; k < word.length(); k++)
        result.append(word.charAt((k+2) % word.length()));
    return result.toString();
}
\end{lstlisting}
\end{jjjlisting}

\item A {\tt NimPlayer} class that plays the optimal {\tt OneRowNim} game would
be identical to the {\tt NimPlayerBad} class except the {\tt move():int} method
would be replaced with the following implementation:

\begin{jjjlisting}
\begin{lstlisting}
 public int move() {   
    int sticksLeft = game.getSticks();
    if (sticksLeft % (game.MAX_PICKUP + 1) != 1)
        return (sticksLeft - 1) % (game.MAX_PICKUP +1);
    else {
        int maxPickup = Math.min(game.MAX_PICKUP, sticksLeft);
         return 1 + (int)(Math.random() * maxPickup);
    }
 }
\end{lstlisting}
\end{jjjlisting}

\end{ANS}

\secEXRHtwo{Exercises}
\label{exercises}
\marginnote{\vspace{12pt}\raggedright{\bf Note:} For programming exercises, {\bf first} draw 
a UML class diagram describing all classes and
their inheritance relationships and/or associations.}
\begin{EXRtwo}

\item  Fill in the blanks in each of the following sentences:

\begin{EXRtwoLL}
\baselineskip=13pt
\item  A method that lacks a body is an \rule{40pt}{0.5pt} method.
\item  An \rule{40pt}{0.5pt} is like a class except that it contains
only instance methods, no instance variables.
\item  Two ways for a class to inherit something in Java is to  \rule{40pt}{0.5pt}
a class and \rule{40pt}{0.5pt}  an interface.
\item  Instance variables and instance methods that are declared
 \rule{40pt}{0.5pt}  or \rule{40pt}{0.5pt}  are inherited by the subclasses.
\item  An object can refer to itself by using the \rule{40pt}{0.5pt} keyword.
\item  If a GUI class intends to handle {\tt ActionEvent}s, it must
implement the \rule{40pt}{0.5pt} interface.
\item   A \rule{40pt}{0.5pt} method is one 
that  does different things depending upon the object that invokes it.
\end{EXRtwoLL}

\baselineskip=11pt\item  Explain the difference between the following pairs of
concepts:

\begin{EXRtwoLL}
\item  {\it Class} and {\it interface}.
\item  {\it Stub method} and {\it abstract method}.
\item  {\it Extending a class} and {\it instantiating an object}.
\item  {\it Defining a method} and {\it implementing a method}.
\item  A {\tt protected} method and a {\tt public} method.
\item  A {\tt protected} method and a {\tt private} method.
\end{EXRtwoLL}

\item  Draw a hierarchy to represent the following situation.
There are lots of languages in the world.  English, French, Chinese,
and Korean are examples of natural languages.  Java, C, and C++ are
examples of formal languages.  French and Italian are considered
romance languages, while Greek and Latin are considered classical
languages.

\item  Look up the documentation for the {\tt JButton} class
on Sun's Web site:

\begin{jjjlisting}
\begin{lstlisting}[commentstyle=\color{black}]
http://java.sun.com/j2se/1.5.0/docs/api/
\end{lstlisting}
\end{jjjlisting}

\noindent List the names of all the methods that would be inherited by
the {\tt ToggleButton} subclass that we defined in this chapter.

\item Design and write a {\tt toString()} method for the {\tt
ToggleButton} class defined in this chapter. The {\tt toString()}
method should return the {\tt ToggleButton}'s current label.

\item Design a class hierarchy rooted in the class {\tt Employee} that
includes subclasses for {\tt HourlyEmployee} and {\tt SalaryEmployee}.
The attributes shared in common by these classes include the name, and
job title of the employee, plus the accessor and mutator methods
needed by those attributes.  The salaried employees need an attribute
for weekly salary, and the corresponding methods for accessing and
changing this variable.  The hourly employees should have a pay rate
and an hours worked variable.  There should be an abstract method
called {\tt calculateWeeklyPay()}, defined abstractly in the
superclass and implemented in the subclasses. The salaried worker's
pay is just the weekly salary. Pay for an hourly employee is simply
hours worked times pay rate.

\item Design and write a subclass of {\tt JTextField} called {\tt
Integer\-Field} that is used for inputting integers but behaves in all
other respects like a {\tt JTextField}.  Give the subclass a public
method called {\tt getInteger()}.

\item Implement a method that uses the following variation of the
Caesar cipher.  The method should take two parameters, a {\tt String}
and an {\tt int} {\it N}. The result should be a {\tt String} in which
the first letter is shifted by {\it N}, the second by $N+1$, the third
by $N+2$, and so on.  For example, given the string ``Hello,'' and an
initial shift of 1, your method should return ``Igopt.''

\noindent Write a method that converts its {\tt String}
parameter so that letters are written in blocks five characters long.

\item  Design and implement an GUI that lets the user type a document
into a {\tt TextArea} and then provides the following analysis of the
document: the number of words in the document, the number of
characters in the document,~and the percentage of words that have more
than six letters.

\item  Design and implement a {\tt Cipher} subclass to
implement the following substitution cipher: Each letter in the alphabet
is replaced with a letter from the opposite end of the alphabet:
{\it a} is replaced with {\it z}, {\it b} with {\it y}, and so forth.

\item  One way to design a substitution alphabet for a cipher
is to use a keyword to construct the alphabet.  For example,
suppose the keyword is ``zebra.'' You place the keyword
at the beginning of the alphabet, and then fill out the
other 21 slots with remaining letters, giving the following
alphabet:

\begin{jjjlisting}
\begin{lstlisting}
Cipher alphabet:   zebracdfghijklmnopqstuvwxy
Plain alphabet:    abcdefghijklmnopqrstuvwxyz
\end{lstlisting}
\end{jjjlisting}

\noindent Design and implement an {\tt Alphabet} class for
constructing these kinds of substitution alphabets.  It should have a
single public method that takes a keyword {\tt String} as an argument
and returns an alphabet string.  Note that an alphabet cannot contain
duplicate letters, so repeated letters in a keyword like ``xylophone''
would have to be removed.

\item  Design and write a {\tt Cipher} subclass for a
substitution cipher that uses an alphabet from the {\tt Alphabet}
class created in the previous exercise.

\item  {\bf Challenge:} Find a partner and concoct your own encryption scheme.
Then work separately with one partner writing {\tt encode()} and the
other writing {\tt decode()}. Test to see that a message can be
encoded and then decoded to yield the original message.

\item Design a {\tt TwoPlayerGame} subclass called {\tt
Multi\-plication\-Game}.  The rules of this game are that the game
generates a random multiplication problem using numbers between 1 and
10, and the players, taking turns, try to provide the answer to
the problem. The game ends when a wrong answer is given. The winner
is the player who did not give a wrong answer. 

\item Design a class called {\tt MultiplicationPlayer} that plays the
multiplication game described in the previous exercise.  This class
should implement the {\tt IPlayer} interface.

\item Design a {\tt TwoPlayerGame} subclass called {\tt
Rock\-Paper\-Scis\-sors}.  The rules of this game are that each player, at
the same time, picks either a rock, a paper, or a scissors. For each
round, the rock beats the scissors, the scissors beats the paper, and
the paper beats the rock. Ties are allowed.  The game is won in a best 
out of three fashion when one of the players wins two rounds.

\item Design a class called {\tt RockPaperScissorsPlayer} that plays
the the game described in the previous exercise.  This class should
implement the {\tt IPlayer} interface.

\end{EXRtwo}

