%%%  Chapter 6: Control Structures
%%%  3rd Edition

\setcounter{chapter}{5}
\setcounter{SSTUDYcount}{1}
\chapter{Control Structures}
\label{chapter-loops}


\CObegin
\secCOBH{Objectives}

\noindent After studying this chapter, you will
\begin{COBL}
\item  Be able to solve problems involving repetition.
\item  Understand the differences among various loop structures.
\item  Know the principles used to design effective loops.
\item  Improve your algorithm design skills.
\item  Understand the goals and principles of structured programming.
\end{COBL}

\secCOLH{Outline}
\begin{COL}
\item Introduction
\item Flow of Control: Repetition Structures
\item Counting Loops
\item Example: Car Loan
\item Graphics Example: Drawing a Checkerboard
\item Conditional Loops
\item Example: Computing Averages
\item Example: Data Validation
\item Principles of Loop Design
\item The {\tt switch} Multiway Selection Structure
\item Object-Oriented Design: Structured Programming
\item[] {{\color{cyan}Special Topic:} What Can Be Computed?}
\par\small\item[] Chapter Summary
\par\small\item[] Solutions to Self-Study Exercises
\par\small\item[] Exercises
\end{COL}
\COend

\section{Introduction}
\label{introduction}
\noindent As we learned in Chapter~3, a control structure is a language 
element that changes the flow of control of a program. Thus far, we
have used the {\tt if} and {\tt if/else} statements to select between
two or more alternate paths in a program.  We have used the {\tt
while}-loop structure to repeat statements. And we have used
method-call-and-return to invoke methods that carry out certain
well-defined tasks in a program.

In this chapter we will extend our repertoire of control structures.
We will introduce the {\it for} and {\it do-while} statements, both of
which are used in programs that require calculations to be repeated.
We will also introduce the {\tt switch} statement, which will give us
another way, in addition to {\tt if/else}, to select from among
several alternate paths in a program.

We begin by introducing the idea of a {\it counting loop}, which is
used for repetitive tasks when you know beforehand exactly how many
repetitions are necessary.  This type of loop is most often
implemented using a {\tt for} statement.

We then distinguish two kinds of {\it conditional loops}, which are
used for performing repetitive tasks where the number of repetitions
depends on some kind of non-counting condition. These kinds of loops
are usually implemented using Java's {\tt while} and {\tt do-while}
statements.  We give examples of several kinds of {\em loop bounds}
and use them to identify several useful principles of loop design.
Finally, we introduce some of the key principles of the {\it
structured programming} approach, a disciplined design approach that
preceded the object-oriented approach.

\section{Flow of Control: Repetition Structures}
\label{flowof-control-repetition-structures}
\noindent As we saw in Chapter~3, a {\bf repetition structure} is a control
structure that repeats a statement or sequence of statements.  Many
programming tasks require a repetition structure.  Consider some
examples.

\begin{BL}
\item  You're working for the National Security Agency trying
to decipher secret messages intercepted from foreign spies, and you
want to count the number of times a certain letter, ``a,'' occurs in
a document containing {\it N} characters.  In this case, you would want to
employ something like the following (pseudocode) algorithm:


\begin{jjjlisting}
\begin{lstlisting}
initialize totalAs to 0
for each character in the document
    if the character is an 'a'
        add 1 to totalAs
return totalAs as the result
\end{lstlisting}
\end{jjjlisting}

\item You're working for a caterer who wants to number the invitations
to a client's wedding, so you need to print the numbers between 1 and
5000 on the invitation cards (it's a big wedding)! In this case, you
want to go through each number, 1 to 5000, and simply print it out:

\begin{jjjlisting}
\begin{lstlisting}
for each number, N, from 1 to 5000
    print N on the invitation card
\end{lstlisting}
\end{jjjlisting}

\item  You are helping the forest service in Alaska to track  the
number of black bear sightings, and you want to compute the average
number of~sightings per month.   Suppose the user enters each month's
count at the keyboard, and uses a special number, say, 9999, to signify
the end of the sequence. However, 9999 should not be figured into the average.
This example differs a bit from the preceding ones, because here you
don't know exactly how many numbers the user will input:

\begin{jjjlisting}
\begin{lstlisting}
initialize sumOfBears to 0
initialize numOfMonths to 0
repeat the following steps
    read a number from the keyboard
    if the number is NOT 9999
        add it to the sumOfBears
        add 1 to numOfMonths
until the number read is 9999
divide sumOfBears by numOfMonths giving average
return average as the result
\end{lstlisting}
\end{jjjlisting}

\noindent We repeat the process of reading numbers and
adding them to a running total ``until the number read is 9999.''

\item  Student records are stored in a file and you want to
calculate Erika Wilson's current GPA.~Here we need to perform a
repetitive process---searching through the file for Erika Wilson's
record---but again we don't know exactly how many times to repeat
the process:

\begin{jjjlisting}
\begin{lstlisting}[stringstyle=\color{black}]
repeat the following steps
    read a record from the file
until Erika Wilson's record is read
compute Erika Wilson's GPA
return gpa as the result
\end{lstlisting}
\end{jjjlisting}
\end{BL}

\noindent As these examples suggest, two types of loops are used:
counting loops and non-counting loops.  Counting loops are used
whenever you know in advance exactly how many times an action must be
performed.  Non-counting loops are used when the number of repetitions
depends on some condition---for example, the number of data items
input from the keyboard or the input of a particular record from a
file.

\section{Counting Loops}
\label{counting-loops}
\noindent A {\bf counting loop}, or {\it counter-controlled loop}, is
a loop in which you know beforehand how many times it will be
repeated. Among the preceding examples, the first two are counting
loops.

Because you know the exact number of times the loop repeats 
 beforehand, a counting loop can be made dependent on the value of a
counter. For example, if you want to print the word ``Hello'' 100
times, you can use the following {\tt while}  structure:

\begin{jjjlisting}
\begin{lstlisting}
int k = 0;
while (k < 100) {
    System.out.println("Hello");
    k++;
}
\end{lstlisting}
\end{jjjlisting}

\noindent In this case, the counter is the variable {\it k}, which counts
from 0 through 99---that is, it counts 100 times.  Note that we start
counting from 0 instead of 1. Counting from 0 is known as {\bf zero
indexing} and it is a common programming convention for counting
loops.  Although it doesn't really make any practical difference in
this case, later on we will use loops to process
%\begin{marginalnote}\it Zero indexing\end{marginalnote}
\marginnote{Zero indexing}
structures, such as strings and arrays, which use zero indexing. It
will be easier to process these structures if our loop counter also
starts at~0.

The variable {\it k} in this example is called a
%\begin{marginalnote}\it Loop counter\end{marginalnote}
\marginnote{Loop counter}
{\it counter variable} or {\it loop counter}.  Although it is
certainly possible to name the counter variable anything we like, it
is customary to use single letters like {\it i, j}, and {\it k} as
loop counters.  The fundamental feature of a counting loop is that we
must know beforehand exactly how many iterations the loop will take.

\JavaTIP{EFFECTIVE DESIGN}{Loop Design.}{A {\it counting loop} can be
used whenever you know exactly how many times a process must be
repeated.}

\noindent Although we can use a {\tt while}-structure to
code a counting loop, Java's {\tt for} statement is ideally suited for
this purpose.  For example, the following {\tt for} statement will also
print the word ``Hello'' 100 times:

\begin{jjjlisting}
\begin{lstlisting}
 for (int k = 0; k < 100; k++)
     System.out.println("Hello");
\end{lstlisting}
\end{jjjlisting}

\noindent In fact, this {\tt for} statement is equivalent
to the preceding {\tt while} structure. 

The {\tt for} statement has the following syntax:

\JavaRule{For Statement.}{The {\it for statement} has the
following syntax: \\
$
\matrix{
\hspace{1.1in}{\tt for} & ( \;{\it initializer}\;;\; {\it loop\;entry\;condition} \;;\; {\it updater}\;  ) \cr
\hspace{1.1in}\;\;\;\; & {\it for\;loop\;body } \;  ;          \cr
}
$}

\noindent The for statement begins with the keyword {\tt for}, which
is followed by a parenthesized list of three expressions separated by
semicolons: an {\bf initializer}, a {\bf loop entry condition}, and an
{\bf updater}.  Following the parenthesized list is the for {\bf loop
body}, which is either a single statement or a sequence of statements
contained in curly brackets, \verb|{|\dots\verb|}|.

\subsection{The For Structure}
%\begin{figure}
\begin{figure}[tb]
\figaright{chptr06/6f1.eps}{Flowchart of the {\tt for} statement.
} {fig-forloop}

\end{figure}
%\end{figure}

Figure~\ref{fig-forloop} shows how the {\tt for} statement works.  It
might be useful to compare this flowchart with the flowchart for the the
{\tt while} structure (Fig.~\ref{fig-whileflow6}), which was
introduced in Chapter~3.  You see that it has exactly the same
structure.  First, the initializer is evaluated.  Thus, the initializer
sets the integer variable {\it k} to 0.  Then the loop entry
condition, which must be a boolean expression, is evaluated.  If it is
true, the body of the loop is executed; if it is false, the body of
the loop is skipped and control passes to the next statement following
the {\tt for} statement.  The updater is evaluated after the loop body is
executed.  After completion of the updater, the loop entry condition
is reevaluated and the loop body is either executed again or not,
depending on the truth value of the loop entry condition.  This
process is repeated until the loop entry condition becomes false.

\begin{figure}[h!]
\figa{chptr03/whileflow.eps}{Flowchart of the {\tt while} statement and {\tt while}
structure.}
{fig-whileflow6}
\end{figure}

Tracing the order in which the {\tt for} loop components are
evaluated gives this sequence:

\begin{jjjlisting}
\begin{lstlisting}
 evaluate initializer
 evaluate loop entry condition ==> True
 execute for loop body;
 evaluate updater
 evaluate loop entry condition ==> True
 execute for loop body;
 evaluate updater
 evaluate loop entry condition ==> True
 execute for loop body;
 evaluate updater
 .
 .
 .
 evaluate loop entry condition ==> False
\end{lstlisting}
\end{jjjlisting}

\noindent As this trace shows, the loop entry condition controls
entry to the body of the loop and will, therefore, be the last thing
done before the loop terminates.

We have followed the standard convention of declaring the counter
variable in the header of the {\tt for} statement.  This restricts the
%\begin{marginalnote}\it Loop variable scope\end{marginalnote}
\marginnote{Loop variable scope} variable's {\it scope} to the {\tt
for} statement itself.  It would be a syntax error to use {\it k}
outside the scope of the {\tt for} loop, as in this example:

\begin{jjjlisting}
\begin{lstlisting}
for (int k = 0; k < 100; k++)
  System.out.println("Hello");
        // Syntax error, k undeclared
System.out.println("k = " + k);
\end{lstlisting}
\end{jjjlisting}

\noindent For some problems, it might be necessary to use the loop
variable outside the scope of the {\tt for} statement, in which case
the variable should be declared before the {\tt for} statement. For
example, if we want to print the value of the loop variable,
{\tt k}, after the loop completes, we have to declare it before
the loop:

\begin{jjjlisting}
\begin{lstlisting}
int k = 0;         // Declare the loop variable here
for (k = 0; k < 100; k++)
    System.out.println("Hello");
System.out.println("k = " + k); // To use it here
\end{lstlisting}
\end{jjjlisting}

\noindent In this example, the loop will exit when {\tt k} becomes
100, so ``k = 100'' will be printed.

\subsection{Loop Bounds}
\noindent A counting loop starts at some initial value and counts 0 or more
%\begin{marginalnote}\it Loop bound\end{marginalnote}
iterations.  A {\bf loop bound} is a value that controls how many times
a loop is repeated. A loop will repeat until its loop bound is reached.
\marginnote{Loop bound}
In a counting loop, the {\it loop entry condition} should be a boolean
expression that tests whether the loop's bound has been reached.
Similarly, in a counting loop, the {\it updater} should modify the loop
counter so that it makes progress toward reaching its bound.  Counting
loops often increment or decrement their counter by 1, depending on
whether the loop is counting forward or backward.  The following
method contains a countdown loop, which prints {\tt 10 9 8 7 6 5 4 3
2 1 BLASTOFF}.  In this case, progress toward the loop bound is made by
decrementing the loop counter:


\begin{jjjlisting}
\begin{lstlisting}
public void countdown() {
    for (int k = 10; k > 0; k--)
        System.out.print(k + " ");
    System.out.println("BLASTOFF");
} // countdown()
\end{lstlisting}
\end{jjjlisting}

\noindent Note in this case that we are using {\bf unit indexing}
%\begin{marginalnote}\it Unit indexing\end{marginalnote}
\marginnote{Unit indexing}
instead of zero indexing, because countdowns repeat, or {\it iterate},
from 10 down to 1, not from 10 down to 0.

\subsection{Infinite Loops}
\noindent If the loop bound is never reached, the loop entry condition will
never become false and the loop will repeat forever.  This is known as
an {\bf infinite loop}.  Can you see why each of the following
%\begin{marginalnote}\it Infinite loop\end{marginalnote}
\marginnote{Infinite loop}
{\tt for} statements will result in an infinite loop?

\begin{jjjlisting}
\begin{lstlisting}
for (int k = 0; k < 100; k--)       // Infinite loop
    System.out.println("Hello");

for (int k = 1; k != 100; k+=2)     // Infinite loop
    System.out.println("Hello");

for (int k = 98; k < 100; k = k / 2)// Infinite loop
    System.out.println("Hello");
\end{lstlisting}
\end{jjjlisting}

\noindent In the first example, {\it k} starts out at 0 and is decremented
on each iteration, taking on values $-1, -2, -3$, and so on, so
{\it k} will never reach its loop bound.

In the second example, {\it k} starts out at 1 and is incremented by 2 on
each iteration, taking on the values 3, 5, 7, and so on.  Because all
these values are odd, {\it k} will never equal 100. A much safer loop
bound in this case would be {\it k $<$= 100}.

In the third example, {\it k} starts out at 98 and is halved on each
iteration, taking on the values 49, 24, 12, 6, 3, 1, 0, 0, and so on,
forever.   Thus, it too will be stuck in an infinite loop.

Encountering an unintended infinite loop when developing a program can
be very frustrating.   If the program is stuck in a loop 
\marginnote{Stuck in a loop}
that generates output, it will be obvious that it is looping, but if
no output is being generated, the computer will appear to ``freeze,''
no longer responding to your keyboard or mouse commands.  Some
programming environments allow you to break out of a looping program
by typing a special keyboard command such as {\tt CONTROL-C} or {\tt
CTRL-ALT-DELETE} or {\tt CONTROL-APPLE-ESCAPE}, but if that doesn't
work you will have to reboot the computer, possibly causing a loss of
data.  The best way to avoid infinite loops is to determine that the
loop's updater expression will cause the loop to eventually reach its
bound.

\JavaTIP{EFFECTIVE DESIGN}{Loop Design.}{To guard against infinite
loops, make sure that the loop bound will eventually be reached.}

\subsection{Loop Indentation}
\noindent Note how indentation is used to distinguish the loop body
from both the heading and from the statement that follows the loop:

\begin{jjjlisting}
\begin{lstlisting}
for (int k = 10; k > 0; k--)     // Loop heading
    System.out.print (k + " ");  //  Indent the body
System.out.println( "BLASTOFF" ) // After the loop
\end{lstlisting}
\end{jjjlisting}

\noindent Indenting the loop body is a stylistic convention intended
to make the code more readable.  However, the indentation itself has no
effect on how the code is interpreted by Java.  Each of the following
code segments would still produce the same countdown:

\begin{jjjlisting}
\begin{lstlisting}
for (int k = 10; k > 0; k--)
System.out.print (k + " ");
System.out.println("BLASTOFF");

for (int k = 10; k > 0; k--) System.out.print(k +" ");
System.out.println("BLASTOFF");

for
(int k = 10; k > 0; k--)
System.out.print (k + " ");
System.out.println("BLASTOFF");
\end{lstlisting}
\end{jjjlisting}

\noindent In each case the statement, \verb|System.out.println("BLASTOFF")|,
is not part of the {\tt for} loop body and is executed only once when the
loop terminates.

\JavaTIP{PROGRAMMING TIP}{Loop Indentation.}{To
make loops more readable, indent the loop body to set it off from the
heading and to highlight which statement(s) will be repeated.}

\JavaTIP{DEBUGGING TIP}{Loop Indentation.}{Loop indentation 
has no effect on how Java interprets the loop.  The loop body is
determined entirely by the syntax of the {\tt for} statement.}

\noindent Note that up to this point the loop body has consisted of a
single statement, such as a {\tt println()} statement. But the loop
body may consist of any Java statement, including an {\tt if} or {\tt
if-else} statement or a compound statement, which contains a sequence
of statements enclosed within braces.  Consider the following
examples.  The first example prints the sequence 0, 5, 10, 15, $\dots$
95. Its loop body consists of a single {\tt if} statement:


\begin{jjjlisting}
\begin{lstlisting}
for (int k = 0; k < 100; k++)// Print 0 5 10...95
  if (k % 5 == 0) // Loop body: single if statement
     System.out.println("k= " + k);
\end{lstlisting}
\end{jjjlisting}

\noindent The next example prints the lowercase letters of the
alphabet.  In this case, the loop counter is of type {\tt char},
and it counts the letters of the alphabet.  The loop body
consists of a single {\tt print()} statement:

\begin{jjjlisting}
\begin{lstlisting}
for (char k = 'a'; k <= 'z'; k++)// Print 'a' 'b'...'z'
  System.out.print (k + " ");  // Loop body: single print()
\end{lstlisting}
\end{jjjlisting}


\noindent The next example prints the sequence  5, 10, 15, $\dots$ 50,
but it uses several statements within the loop body:


\begin{jjjlisting}
\begin{lstlisting}
for (int k = 1; k <= 10; k++) {// Print 5 10 15...50
   int m = k * 5;                  // Begin body
   System.out.print (m + " ");
}                                  // End body
\end{lstlisting}
\end{jjjlisting}

\noindent In this example, the scope of the local variable {\it m},
declared within the loop body, is limited to the loop body and
cannot be used outside of that scope.

\JavaRule{Loop Body.}{The body of a
{\tt for} statement consists of the statement that immediately follows
the {\tt for} loop heading.  This statement can be either a simple
statement or a compound statement---a sequence of statements enclosed
within braces, \{\dots \}.}

\noindent Although braces are not required when the body of a {\tt
for} loop consists of a single statement, some coding styles recommend
that braces should always be used for the body of a loop statement.
For example, it is always correct to code the {\tt for} loop as

\begin{jjjlisting}
\begin{lstlisting}
for (int k = 1; k <= 10; k++) {// Print 1 2 ... 10
  System.out.print (k + " ");     // Begin body
}                                // End body
\end{lstlisting}
\end{jjjlisting}

\noindent Another advantage of this coding style is that you
can easily place additional statements in the loop body by placing
them within the braces.

\JavaTIP[false]{DEBUGGING TIP}{Missing Braces.}{A common programming error for
novices is to forget to use braces to group the statements they intend
to put in the loop body.  When braces are not used only the first
statement after the loop heading will be iterated.}

\secEXRHone{Self-Study Exercises}
\begin{SSTUDY}
\item  Identify the syntax error in the following {\tt for} loop \mbox{statements.}
\begin{EXRLL}
\item 

\begin{jjjlisting}
\begin{lstlisting}
 for (int k = 5, k < 100, k++)
     System.out.println(k);
\end{lstlisting}
\end{jjjlisting}

\item 

\begin{jjjlisting}
\begin{lstlisting}
 for (int k = 0; k < 12 ; k--;)
     System.out.println(k);
\end{lstlisting}
\end{jjjlisting}

\end{EXRLL}

\item  Identify those statements that result in infinite loops.
\begin{EXRLL}
\item 

\begin{jjjlisting}
\begin{lstlisting}
 for (int k = 0; k < 100; k = k )
     System.out.println(k);
\end{lstlisting}
\end{jjjlisting}


\item 

\begin{jjjlisting}
\begin{lstlisting}
 for (int k = 1; k == 100; k = k + 2 )
     System.out.println(k);
\end{lstlisting}
\end{jjjlisting}


\item 

\begin{jjjlisting}
\begin{lstlisting}
 for (int k = 1; k >= 100; k = k - 2 )
     System.out.println(k);
\end{lstlisting}
\end{jjjlisting}

\end{EXRLL}

\item  Suppose you're helping your little sister learn to count by
fours.  Write a {\tt for} loop that prints the following sequence of numbers:
1, 5, 9, 13, 17, 21, 25.

\item  What value will {\it j} have when the following loop \mbox{terminates?}
\begin{jjjlisting}
\begin{lstlisting}
for (int i = 0; i < 10; i++) {
    int j;
    j = j + 1;
}
\end{lstlisting}
\end{jjjlisting}

\end{SSTUDY}

\subsection{Nested Loops}
\noindent A {\bf nested loop} is a structure in which one loop is contained
inside the body of another loop, such as when a {\tt for} loop body
contains a {\tt for} loop.  For example, suppose you are working for
Giant Auto Industries, and your boss wants you to print a table 
for buyers to figure the cost of buying multiple quantities
of a certain part.  The cost of individual parts ranges from
\$1 to \$9.  The cost of {\it N} items is simply the unit price times the
quantity.  Thus, you'll want to print something like the following
table of numbers, where the prices per unit are listed in the top row,
and the prices for 2, 3 and 4 units are listed in subsequent rows:

\begin{jjjlisting}
\begin{lstlisting}
 1  2  3  4  5  6  7  8  9
 2  4  6  8  10 12 14 16 18
 3  6  9  12 15 18 21 24 27
 4  8  12 16 20 24 28 32 36
\end{lstlisting}
\end{jjjlisting}

\noindent To produce this multiplication table, we could use the following
nested {\tt for} loops:

\begin{jjjlisting}[28pc]
\begin{lstlisting}[numberstyle=\small, numbers=left]
for (int row = 1; row <= 4 ; row++) { // For each of 4 rows      
  for (int col = 1; col <= 9; col++) // For each of 9 columns   
    System.out.print(col * row + "\t" );      // Print number
  System.out.println();              // Start a new row         
} // for row
\end{lstlisting}
\end{jjjlisting}

\noindent Note how indenting is used here to distinguish the
levels of nesting and to make the code more readable. In this example,
the {\it outer loop}
%\begin{marginalnote}\it Inner and outer loop\end{marginalnote}
\marginnote{Inner and outer loop}
controls the number of rows in the table, hence, our choice of {\tt
row} as its loop counter.  The {\tt println()} statement is executed
after the {\it inner loop} is done iterating, which allows us to print
a new row on each iteration of the outer loop.  The inner loop prints
the nine values in each row by printing the expression {\it
col\hbox{\tt *}row}. Obviously, the value of this expression depends
on both loop variables.

Let's dissect this example a bit.  How many times is the {\tt for}
statement in line 2 executed? The inner loop is executed once for each
iteration of the outer loop.  Thus, it is executed four times, which
is the same number of times that line 4 is executed.  How many times
is the statement on line 3 executed? The body of the inner loop is
executed 36 times---9 times for each execution of line 2.

Sometimes it is useful to use the loop variable of the outer loop as
%\begin{marginalnote}\it Algorithm design\end{marginalnote}
\marginnote{Algorithm design}
the bound for the inner loop.  For example, consider the following
pattern:

\begin{jjjlisting}
\begin{lstlisting}
 # # # # #
 # # # #
 # # #
 # #
 #
\end{lstlisting}
\end{jjjlisting}

\noindent Note that the number of \# symbols in each row
varies inversely with the row number.  In row 1, we have five symbols;
in row 2 we have four; and so on down to row 5, where we have one \#.

To produce this kind of two-dimensional pattern, we need two counters:
one to count the row number, and one to count the number of \# symbols
in each row.  Because we have to print each row's symbols before moving on
to the next row, the outer loop will count row numbers, and the inner
loop will count the symbols in each row.  But note that the inner loop's
bound will depend on the row number.  Thus, in row 1 we want five symbols;
in row 2 we want four symbols; and so on.  If we let {\tt row} be the row
number, then in each row we want to print 6 $-$ {\it row} symbols.  The
following table shows the relationship we want:

\begin{table}[h!]
\hspace*{3.5pc}
\begin{tabular}{cll} 
\multicolumn{3}{l}{\color{cyan}\rule{19pc}{1pt}}\\
[2pt] {Row}\hspace*{12pt}&{Bound (6-row)}\hspace*{12pt}&{Number of Symbols} \\
[-6pt]\multicolumn{3}{l}{\color{cyan}\rule{19pc}{0.5pt}}\\ [2pt]
1 & 6-1 & 5   \\
2 & 6-2 & 4 \\
3 & 6-3 & 3 \\
4 & 6-4 & 2 \\
5 & 6-5 & 1 
\\[-8pt]\multicolumn{3}{l}{\color{cyan}\rule{19pc}{1pt}}
\end{tabular}
\endTB
\end{table}

%%%RAM\begin{jjjlisting}
%%%RAM\begin{lstlisting}
%%%RAMRow   Bound (6-row)    Number of # Symbols
%%%RAM---   -------------    -------------------
%%%RAM 1         6-1               5
%%%RAM 2         6-2               4
%%%RAM 3         6-3               3
%%%RAM 4         6-4               2
%%%RAM 5         6-5               1\
%%%RAMend{lstlisting}
%%%RAM\end{jjjlisting}

\noindent If we let {\it j} be the counter for the inner loop,
then {\it j} will be bound by the expression 6 $-$ {\it row}.
This leads to the following nested loop structure:


\begin{jjjlisting}
\begin{lstlisting}
for (int row = 1; row <= 5; row++) { // For each row
  for (int j = 1; j <= 6 - row; j++)// Print the row
    System.out.print('#');
  System.out.println();    // Start a new line
} // for row
\end{lstlisting}
\end{jjjlisting}

\noindent Note that the bound of the inner loop varies according
to the value of {\it row}, the loop counter for the outer loop.

\secEXRHone{Self-Study Exercise}
\begin{SSTUDY}
\item  As the engineer hired to design ski jumps, 
write a nested {\tt for} loop to print the following pattern:

\begin{jjjlisting}
\begin{lstlisting}
#
# #
# # #
# # # #
# # # # #
\end{lstlisting}
\end{jjjlisting}

\end{SSTUDY}

\section{Example: Car Loan}
\label{example-car-loan}
\noindent Recall the self-study exercise from Chapter~5 that
calculated the value of a bank CD ({\it a}) given its initial
principle ({\it p}), interest rate ({\it r}), and number of years
({\it n}), using the formula $a = p(1 + r)^n$.  This section explains
how to use the same formula to calculate the cost of a car loan for
different interest rates over various time periods.

\subsection*{Problem Description}
\noindent For example, suppose you are planning on buying a car that costs
\$20,000. You find that you can get a car loan ranging anywhere from
8 to 11 percent, and you can have the loan for periods as short as two
years and as long as eight years.  Let's use our loop constructs to
create a table to show what the car will actually cost you, including
financing.  In this case, {\it a} will represent the total cost of the
car, including the financing, and {\it p} will represent the price
tag on the car (\$20,000):

\begin{jjjlisting}
\begin{lstlisting}
            8%          9%          10%         11%
Year 2  $23,469.81  $23,943.82  $24,427.39  $24,920.71
Year 3  $25,424.31  $26,198.42  $26,996.07  $27,817.98
Year 4  $27,541.59  $28,665.32  $29,834.86  $31,052.09
Year 5  $29,835.19  $31,364.50  $32,972.17  $34,662.19
Year 6  $32,319.79  $34,317.85  $36,439.38  $38,692.00
Year 7  $35,011.30  $37,549.30  $40,271.19  $43,190.31
Year 8  $37,926.96  $41,085.02  $44,505.94  $48,211.60
\end{lstlisting}
\end{jjjlisting}

\subsection*{Algorithm Design}
\noindent The key element in this program is the nested {\tt for} loop 
that generates the table.  Because the table contains seven rows, the
outer loop should 
%\begin{marginalnote}\it Nested loop design\end{marginalnote}
\marginnote{Nested loop design}
iterate seven times, through the values $2, 3, \dots 8$:

\begin{jjjlisting}
\begin{lstlisting}
for (int years = 2; years <= 8; years++) 
\end{lstlisting}
\end{jjjlisting}

\noindent The inner loop should iterate through each of the interest rates,
8 through 11:


\begin{jjjlisting}
\begin{lstlisting}
for (int years = 2; years <= 8; years++) { 
   for (int rate = 8; rate <= 11; rate++) {
   } // for rate
} // for years
\end{lstlisting}
\end{jjjlisting}

\noindent The financing calculation should be placed in the body of
the inner loop together with a statement to print one cell (not row)
of the table.  Suppose the variable we use for {\it a} in the formula
is {\tt carPriceWithLoan}, and the variable we use for the actual
price of the car is {\tt carPrice}. Then our inner loop body is


\begin{jjjlisting}
\begin{lstlisting}
carPriceWithLoan = carPrice * 
     Math.pow(1 + rate/100.0/365.0, years * 365.0);
System.out.print(dollars.format(carPriceWithLoan) +"\t");
\end{lstlisting}
\end{jjjlisting}

\noindent Note that the rate is divided by both 100.0 (to make it
a percentage) and by 365.0 (for daily compounding), and the year
is multiplied by 365.0 before these values are passed to the
{\tt Math.pow()} method.   It is important here to use 100.0 and
not 100 so that the resulting value is a {\tt double} and not
the {\tt int} 0.

\subsection*{Implementation}
\noindent The program must also contain statements to print the row and column
headings.  Printing the row headings should be done within the outer
loop, because it must be done for each row.  Printing the column
headings should be done before the outer loop is entered.  Finally, our
%\begin{marginalnote}\it Formatting output\end{marginalnote}
\marginnote{Formatting output}
program should contain code to format the dollar and cents values
properly.  For this we use the {\tt java.text.NumberFormat} class that
was described in Chapter~5. The complete program is shown in
Figure~\ref{fig-carloan}.
%% proglist ch6/carloan/CarLoan.java
\begin{figure}[h]
\jjjprogstart
\begin{jjjlistingleft}[33pc]{-7pc}
\begin{lstlisting}
import java.text.NumberFormat;   // For formatting $nn.dd or n%

public class CarLoan {
  public static void main(String args[]) {
    double carPrice = 20000;  // Car's actual price
    double carPriceWithLoan;  // Total cost of the car plus financing
                                              // Number formatting code
    NumberFormat dollars = NumberFormat.getCurrencyInstance(); 
    NumberFormat percent = NumberFormat.getPercentInstance();
    percent.setMaximumFractionDigits(2);
                                            // Print the table
    for (int rate = 8; rate <= 11; rate++)  // Print the column heading
      System.out.print("\t" + percent.format(rate/100.0) + "\t" );
    System.out.println();

    for (int years = 2; years <= 8; years++) {  // For years 2 through 8
      System.out.print("Year " + years + "\t"); // Print row heading
      for (int rate = 8; rate <= 11; rate++) {  // Calc and print CD value
        carPriceWithLoan = carPrice * 
                       Math.pow(1 + rate / 100.0 / 365.0, years * 365.0);
        System.out.print( dollars.format(carPriceWithLoan)  + "\t");
      } // for rate
      System.out.println();                     // Start a new row
    } // for years
  } // main()
} // CarLoan
\end{lstlisting}
\end{jjjlistingleft}
\jjjprogstop{The {\tt CarLoan} application.}
{fig-carloan}
\end{figure}
%$
\newpage
\section{Graphics Example: Drawing a Checkerboard}
\label{graphics-example-drawinga-checkerboard}

In this section we will combine some of the graphics methods we have
learned with the nested for-loop structure to draw a checkerboard with
checkers on it (Fig.~6.4). For this example, we will just
concentrate on drawing the checkerboard.  We will not worry about
providing a full checkerboard representation, of the sort we would
need if we were writing a program to play the game of checkers. So,
our design will not involve instance
\marginfig{chptr06/checkerboard.eps}{A checkerboard with checkers.}
{fig-checkerboard}
variables for whose turn it is, how many checkers each player has,
where the checkers are located on the board, and other elements of the
game's state. Still, our visible representation of the board should
be designed so that it will eventually be useful when we do develop a
checkers game in Chapter~\ref{chapter-inheritance}.

\subsection*{Problem Description and Specification}

The specification for this problem is to develop a program that will
draw a checkerboard and place upon it the checkers in their
appropriate starting positions.  As with many of our programs, our
design will involve two classes, one which defines the user interface
and the other which represents the computational object.  For this
problem, the computational object will be defined in the {\tt
CheckerBoard} class. The details of its design are described in the
next section.

Because the purpose of this example is to focus on how we use loops
and drawing methods, we will employ a very simple JFrame interface,
whose implementation is given in Figure~\ref{fig-checkerapplet}.  As
\begin{figure}[h]
\jjjprogstart
\begin{jjjlisting}
\begin{lstlisting}
import java.awt.*;
import javax.swing.*;

public class CheckerBoardFrame extends JFrame {
   private CheckerBoard theBoard;

   public CheckerBoardFrame() {
       theBoard = new CheckerBoard();
   }
   public  void paint(Graphics g) {
       super.paint(g);
       theBoard.draw(g);
   } // paint()

   public static void main(String[] args){
        CheckerBoardFrame c = new CheckerBoardFrame();
        c.setSize(500,500);
        c.setVisible(true);
    }
} // CheckerBoardFrame
\end{lstlisting}
\end{jjjlisting}
\jjjprogstop{The {\tt Check\-er\-Board\-Frame} class.}
{fig-checkerapplet}
\end{figure}
shown there, the program simply creates a {\tt CheckerBoard} instance
in its {\tt CheckerBoardFrame()} constructor, and then invokes the {\tt CheckerBoard}'s
{\tt draw} method in its {\tt paint()} method. The reason we invoke
the {\tt draw()} method in {\tt paint()} is because we need to have
access to the JFrame's {\tt Graphics} context, which is passed as an
argument to the {\tt draw()} method. Recall that the {\tt
  main()} method is invoked when CheckerBoardFrame is run,and
then the {\tt paint()} methods are invoked automatically by calling 
the {\tt setVisible(true)} method in {\tt main()}.  Thus, the action
taken by this program is simply to draw a visual representation of the
checkerboard.


\subsection*{Class Design: {\tt CheckerBoard}}
\marginfig{chptr06/checkeruml.eps}{Design of the {\tt CheckerBoard}
class.}
{fig-checkeruml}
\noindent Because the program will invoke its {\tt draw()} method, this
method must be part of the {\tt CheckerBoard}'s interface.  Hence, it
must be declared {\tt public}.  The task of drawing a checkerboard
involves two distinct subtasks: (1) drawing the board itself, which
will involve drawing a square with smaller squares of alternating
colors; and, (2) drawing the checkers on the checkerboard.  A good
design for the {\tt draw()} method would be simply to invoke helper
methods that encapsulate these two subtasks. This is good method
design because it results in relatively small methods, each of which
performs a very well-defined task.  Let's call these methods {\tt
drawBoard()} and {\tt drawCheckers()}, respectively.  Their signatures
are shown in Figure~\ref{fig-checkeruml}, which summarizes the design of the {\tt
CheckerBoard} class.

Before gettinginto the details of the {\tt drawBoard} and {\tt
drawCheckers()} methods, we must first discuss {\tt CheckerBoard}'s
several instance variables.  The first two variables {\tt LEFT\_X} and
{\tt UPPER\_Y}, give the absolute position of the upper left corner of
the checkerboard on the JFrame's drawing panel. The {\tt SQ\_SIDE}
variable gives the size of the checkerboard's individual squares.
{\tt N\_ROWS} and {\tt N\_COLS} give the number of rows and columns in
the checkerboard (typically, 8 by 8). All of these variables are
integers.  The final four variables, {\tt SQ\_COLOR1}, {\tt SQ\_COLOR2},
{\tt CHECKER\_COLOR1}, and {\tt CHECKER\_COLOR2}, specify the colors
of the checkerboard and the checkers. 

Note that the names of all the instance variables are written in
uppercase letters. This is to identify them as symbolic
constants---that is, as final variables whose values do not chage once
they are initialized. Because their actual values are not important,
we do not show them in the UML diagram and we won't discuss them here.
Recall that the advantage of defining class constants, rather than
sprinkling literal values throughout the program, is that they make it
easy to modify the program if we decide to change the size, location,
or color of the checkerboard.
%%\marginnote{Symbolic constants} 

\subsection*{Method Design}
Returning now to the design of {\tt CheckerBoard}'s instance methods,
the complete definition of the {\tt CheckerBoard} class is given in
Figure~\ref{fig-checkerclass}. Note how simple its {\tt draw()} method
is.  As we noted earlier, in order to using Java's drawing commands,
it is necessary to have a reference to a {\tt Graphics} object. This
is passed to the {\tt draw()} method when the {\tt draw()} method is
invoked in the program.  Because the {\tt draw()} method delegates the
details of the drawing algorithm to its helper methods, {\tt
drawBoard()} and {\tt drawCheckers()}, it has to pass them a reference
to the {\tt Graphics} object.

\begin{figure}[h!]
\jjjprogstart
\begin{jjjlisting}[28pc]
\begin{lstlisting}
import java.awt.*;

public class CheckerBoard {
             // Default values for a standard checkerboard
  private final int LEFT_X = 10;        // Position of left
  private final int UPPER_Y = 10;           // upper corner
  private final int SQ_SIDE = 40;     //Size of each square
  private final int N_ROWS = 8;        // Checkerboard rows
  private final int N_COLS = 8;     // Checkerboard columns
  private final Color SQ_COLOR1 = Color.lightGray; //Colors
  private final Color SQ_COLOR2 = Color.gray;  //of squares
  private final Color CHECKER_COLOR1 = Color.white;  // and
  private final Color CHECKER_COLOR2 = Color.black; //checkers

  private void drawBoard(Graphics g) {
    for(int row = 0; row < N_ROWS; row++)        // For each row
      for(int col = 0; col < N_COLS; col++) { // For each square
        if ((row + col) % 2 == 0)            // Alternate colors
          g.setColor(SQ_COLOR1);                        // Light
        else
          g.setColor(SQ_COLOR2);                      // or dark
        g.fillRect(LEFT_X+col*SQ_SIDE,
              UPPER_Y+row*SQ_SIDE,SQ_SIDE,SQ_SIDE);
      } //for
  } //drawBoard()
  private void drawCheckers(Graphics g) {   // Place checkers
    for(int row = 0; row < N_ROWS; row++)   // For each row
      for(int col = 0; col < N_COLS; col++)// For each square
        if ((row + col)%2 == 1) {// One player has top 3 rows
          if (row < 3) {
              g.setColor(CHECKER_COLOR1);
              g.fillOval(LEFT_X+col*SQ_SIDE,
                  UPPER_Y+row*SQ_SIDE,SQ_SIDE-2,SQ_SIDE-2);
          }//if
          if (row >= N_ROWS - 3) { // Other has bottom 3 rows
              g.setColor(CHECKER_COLOR2);
              g.fillOval(LEFT_X+col*SQ_SIDE,
                   UPPER_Y+row*SQ_SIDE,SQ_SIDE-2,SQ_SIDE-2);
          }//if
        }//if
  }//drawCheckers()
  public void draw(Graphics g) { // Draw board and checkers
      drawBoard(g);
      drawCheckers(g);
  }//draw()
} //CheckerBoard
\end{lstlisting}
\end{jjjlisting}
\jjjprogstop{The {\tt CheckerBoard} class.}
{fig-checkerclass}
\end{figure}

The {\tt drawBoard()} method uses a nested {\tt for} loop to draw an
$8 \times 8$ array of rectangles of alternating colors. The loop
variables, {\tt row} and {\tt col}, both range from 0 to 7.  The
expression used to determine alternating colors tests whether the sum
of the row and column subscripts is even: $((row + col) \% 2 == 0)$.
If their sum is even, we use one color; if odd, we use the other
color. 


As the table in the margin shows for a $4 \times 4$ board, the sum of
a board's row and column subscripts alternates between even and odd
values. Thus, in row 2 column 3, the sum of the subscripts is 5.

\marginpar{
\small
\tt
  \hspace*{1em}| 0  1  2  3\\
  -----------------\\
   0 | 0  1  2  3\\
   1 | 1  2  3  4\\
   2 | 2  3  4  5\\
   3 | 3  4  5  6\\
}

To switch from one color to the other, we use the {\tt
Graphics} {\tt setColor()} method to alternate between the two colors
designated for the checkerboard, {\tt SQ\_COLOR1} and {\tt
SQ\_COLOR2}.  We then use the following method call to draw the
colored squares:

\begin{jjjlisting}
\begin{lstlisting}
g.fillRect(LEFT_X+col*SQ_SIDE,UPPER_Y+row*SQ_SIDE,
                                        SQ_SIDE,SQ_SIDE);
\end{lstlisting}
\end{jjjlisting}

\noindent Note how we use the loop variables, {\tt row} {\tt col},
together with the constants specifying the top left corner of the
board ({\tt UPPER\_Y} and {\tt LEFT\_X}) and the size of the squares
({\tt SQ\_SIDE}) to calculate the location and size of each square.
The calculation here is illustrated in Figure~\ref{fig-checkerlayout}.
The first two parameters in {\tt fillRect(left,top,width,height)}
specify the coordinates for the rectangle's top-left corner. These
are calculated as a function of the rectangle's row and column
position within the checkerboard and the rectangle's width and
height, which are equal for the squares of a checkerboard.

%\begin{figure}
\begin{figure}[hbt]
\figaleft{chptr06/checkerlayout.eps}{Calculating the
locations of the checkerboard squares.
} {fig-checkerlayout}

\end{figure}
%\end{figure}

The {\tt drawCheckers()} method also uses a nested {\tt for} loop to
trace through the checkerboard's rows and columns. In this case,
however, we draw checkers on just the dark-colored squares---that is,
those that satisfy the expression $(row+col) \% 2 == 1)$---on the
first three rows of each player's side of the board. So, each player's
checkers initially are located in the first three rows and last three
rows of the checker board:

\begin{jjjlisting}
\begin{lstlisting}
if ((row + col)%2 == 1) {// One player has top 3 rows
    if (row < 3){
        g.setColor(CHECKER_COLOR1);
        g.fillOval(LEFT_X+col*SQ_SIDE,
           UPPER_Y+row*SQ_SIDE,SQ_SIDE-2,SQ_SIDE-2);
    }//if
    if (row >= N_ROWS - 3) { // Other has bottom 3 rows
        g.setColor(CHECKER_COLOR2);
        g.fillOval(LEFT_X+col*SQ_SIDE,
           UPPER_Y+row*SQ_SIDE,SQ_SIDE-2,SQ_SIDE-2);
    }//if
}//if
\end{lstlisting}
\end{jjjlisting}

\noindent Because the checkers are circles, we use the {\tt
fillOval()} method to draw them. Note that the parameters for {\tt
fillOval(left, top, width, height)} are identical to those for {\tt
fillRect()}. The parameters specify an enclosing rectangle in which
the oval is inscribed.  In this case, of course, the enclosing
rectangle is a square, which causes {\tt fillOval()} to draw a
circle.

Our design of the {\tt CheckerBoard} class illustrates an important
principle of method design. First, rather than placing all of the
commands for drawing the checkerboard and the checkers into one
method, we broke up this larger task into distinct subtasks. This
resulted in small methods, each of which has a well defined purpose.

\JavaTIP{EFFECTIVE DESIGN}{Method Decomposition.}{Methods should be
designed to have a clear focus. If you find a method becoming too
long, you should break its algorithm into subtasks and define a
separate method for each subtask.}

\section{Conditional Loops}
\label{conditional-loops}
\noindent Unlike the problems in the previous sections, not all loops can be
coded as counting loops. Here's a problem that can't be solved by a
counting loop.

Mathematicians, especially number theorists, have found that certain
operations on numbers lead to interesting sequences.  For example, the
{\it 3N+1 problem} is a conjecture in number theory, which says that
if {\it N} is any positive integer, then the sequence generated by the
following rules will always terminate at 1.


\begin{jjjlisting}
\begin{lstlisting}
Case          Operation
----          ---------
N is odd      N = 3 * N + 1
N is even     N = N / 2
\end{lstlisting}
\end{jjjlisting}

\noindent In other words, start with any positive integer, {\it N}. If {\it N} 
is odd, multiply it by 3 and add 1.  If {\it N} is even, divide it by
2. In either case, assign the result back to {\it N}. The conjecture
states that {\it N} will eventually equal 1. For example, if {\it N}
is initially 26, then the sequence generated is 26, 13, 40, 20, 10, 5,
16, 8, 4, 2, 1.

The 3{\it N}+1 problem is an example of a noncounting loop. Because for any
\marginnote{Sentinel bound}
given {\it N} we don't know how long the 3{\it N}+1 sequence will be,
we need a loop that terminates when the loop variable reaches a
certain value, called a {\it sentinel} value---when {\it N} equals
1. This is an example of a loop that is terminated by a {\bf sentinel
bound}.  With the exception of infinite loops, all loops are bounded
by some condition, which is why they are sometimes referred to as {\bf
conditional loop} structures.  The count and sentinel bounds are just
special cases of the conditional loop structure.

\subsection{The While Structure, Revisited}
\noindent Consider the following pseudocode algorithm for the 3{\it N}+1 problem:

\begin{jjjlisting}
\begin{lstlisting}
Algorithm for computing the 3N+1 sequence
    While N is not equal to 1, do: {
        Print N.
        If N is even, divide it by 2.
        If N is odd, multiply N by 3 and add 1.
    }
    Print N
\end{lstlisting}
\end{jjjlisting}

\noindent In this structure, the body of the loop prints {\it N} and
then updates {\it N}'s value, using the 3{\it N}+1 rules.  Suppose
{\it N} equals 5 when this code segment begins.  It will print the
following sequence: 5, 16, 8, 4, 2, 1. Note that the loop body is
entered as long as {\it N} is not equal to 1. So the loop entry
condition in this case is {\it N != 1}. Conversely, the loop will
terminate when {\it N} equals 1. Also note that in this code
segment the loop bound is tested {\it before} the body of the loop
is executed.

We can implement this algorithm using Java's {\bf {\tt while} statement},
whose flowchart is shown in Figure~\ref{fig-whileflow6}:

\begin{jjjlisting}
\begin{lstlisting}
while (N != 1) {         // While N not equal to 1
  System.out.print(N + " ");// Print N
  if (N % 2 == 0)         // If N is even
      N = N / 2;          //  divide it by 2
  else                    // If N is odd
      N = 3 * N + 1;      //  multiply by 3 and add 1
}
System.out.println(N);    // Print N
\end{lstlisting}
\end{jjjlisting}

\noindent Recall that unlike the {\tt for} statement, the {\tt while} 
statement does not contain syntax for the initializer and the
updater. These must be coded separately. As we pointed out in
Chapter~3, the {\tt while} structure (as opposed to the {\tt while}
statement) is a segment of code built by the programmer that 
satisfies the following design principle:

\JavaTIP{EFFECTIVE DESIGN}{Loop Structure.}{A properly designed
{\it loop structure} must include an {\it initializer}, a {\it
loop-entry condition}, and an {\it updater}. The updater should
guarantee that the loop-entry condition will eventually become false,
thereby causing the loop to terminate.}

\noindent The {\tt while} structure has the following form:
\begin{jjjlisting}
\begin{lstlisting}
InitializerStatements;         // Initializer
while (loop entry condition) { // Bound test
    Statements;                // Loop body
    UpdaterStatements;         // Updater
}
\end{lstlisting}
\end{jjjlisting}

\noindent As its form suggests, the {\tt while} structure is designed
so that on some conditions the loop body will never be
executed. Because it tests for the loop bound {\it before} the loop
body, it is possible that the loop body is never executed.  We might
say that it is designed to perform 0 or more iterations.

For example, going back to the 3{\it N}+1 problem, what if {\it N}
equals 1 initially? In that case, the loop body will be skipped,
because the loop entry condition is false to begin with. No iterations
will be performed, and the algorithm will simply print the value 1.

The {\tt while} structure would be an appropriate control structure
for the following type of problem:

\begin{jjjlisting}
\begin{lstlisting}
write the problems on the assignment sheet // Initializer
while there are problems on the sheet      // Bound test
    do a problem                            // Loop body
    cross it off the assignment sheet         // Updater
\end{lstlisting}
\end{jjjlisting}

\noindent It is possible that the assignment sheet contains
no homework problems to begin with. In that case, there's no work for
the body of the loop to do and it should be skipped.

\secEXRHone{Self-Study Exercises}
\begin{SSTUDY}
\item  Identify the syntax error in the following while structures:
\begin{EXRLL}
\item

\begin{jjjlisting}
\begin{lstlisting}
int k = 5;
while (k < 100) {
    System.out.println(k);
    k++
}
\end{lstlisting}
\end{jjjlisting}

\item 

\begin{jjjlisting}
\begin{lstlisting}
int k = 0;
while (k < 12 ;) {
   System.out.println(k);
   k++;
}
\end{lstlisting}
\end{jjjlisting}
\end{EXRLL}


\item  Determine the output and/or identify the error in each of the following
while structures:
\begin{EXRLL}
\item

\begin{jjjlisting}
\begin{lstlisting}
int k = 0;
while (k < 100)
    System.out.println(k);
\end{lstlisting}
\end{jjjlisting}

\item

\begin{jjjlisting}
\begin{lstlisting}
while (k < 100) {
    System.out.println(k);
    k++;
}
\end{lstlisting}
\end{jjjlisting}
\end{EXRLL}


\item  Your younger sister is now learning how to count by sixes. Write a
while loop that prints the following sequence of numbers: 0, 6, 12,
18, 24, 30, 36.

\item  Here's another number theory 
problem. Start with any positive integer, {\it N}.  If {\it N} is
even, divide it by 2. If {\it N} is odd, subtract 1 and then divide it
by 2. This will generate a sequence that is guaranteed to terminate at
0.  For example, if {\it N} is initially 15, then you get the
sequence: 15, 7, 3, 1, 0. Write a method that implements this
sequence using a while statement.

\end{SSTUDY}

\subsection{The \mbox{Do-While} Structure}
\noindent Here's another problem that can't be solved with a counting loop. Your
%\begin{marginalnote}\it Problem description\end{marginalnote}
\marginnote{Problem description}
father has been fretting about the bare spots on the front lawn and is
considering hiring the ChemSure Lawn Service to fertilize.  However,
your scientifically minded younger sister wants to reassure him that
at the rate the grass is dying, there will be enough to last through
the summer. Using techniques she learned in biology, your sister
estimates that the grass is dying at the rate of 2 percent per day. How
many weeks will it take for half the lawn to disappear?

One way to solve this problem would be to keep subtracting 2 percent from the
%\begin{marginalnote}\it Algorithm design\end{marginalnote}
\marginnote{Algorithm design}
current amount of grass until the amount dipped below 50 percent, all
the while counting the number of iterations required. Consider
the following pseudocode algorithm:

\begin{jjjlisting}
\begin{lstlisting}
Algorithm for calculating grass loss
    Initialize amtGrass to 100.0
    Initialize nDays to 0
    Repeat the following statements
        amtGrass -= amtGrass * 0.02;
        ++nDays;
    As long as amtGrass > 50.0
    Print nDays / 7
\end{lstlisting}
\end{jjjlisting}

\noindent We begin by initializing {\tt amtGrass} to
100.0, representing 100 percent. And we initialize our counter, {\tt
nDays} to 0. Then we repeatedly subtract 2 percent of the amount and
increment the counter until the amount drops below 50 percent. In
other words, in this case, we repeat the loop body as long as the
amount of grass remains above 50 percent of the original. When the
loop finishes, we report the number of weeks it took by dividing the
number of days by 7.

The loop bound in this case is known as a {\bf limit bound}. The loop
%\begin{marginalnote}\it Limit bound\end{marginalnote}
\marginnote{Limit bound}
will terminate when a certain limit has been reached---in this case,
when the amount of grass dips below 50 percent of the original amount.
Note that in this case the loop bound is tested {\it after} the loop
body. This is appropriate for this problem, because we know in advance
that the loop will iterate at least once. We can implement this
algorithm using Java's {\it do-while statement}:

\begin{jjjlisting}
\begin{lstlisting}
public int losingGrass(double perCentGrass) {
  double amtGrass = 100.0; // Initialize amount grass
  int nDays = 0;            // Initialize day counter
  do {                               // Repeat
    amtGrass -= amtGrass * LOSSRATE; // Update amount
    ++nDays;               //   Increment the counter
  } while (amtGrass > perCentGrass);   
                   // As long as enough grass remains
  return nDays / 7;     // Return the number of weeks
} // losingGrass()
\end{lstlisting}
\end{jjjlisting}

\noindent The {\bf do-while statement} is a loop statement in which the loop
entry condition occurs after the loop body. It has the following
general form:

\JavaRule{Do-while Statement.}{The {\it do-while statement} has the
following syntax: \\
$
\matrix{
\hspace{1.1in}{\tt do} & & \cr
\hspace{1.1in}         & {\it loop\;body } & \cr
\hspace{1.1in}{\tt while \;( } &  {\it loop\;entry\;condition} & {\tt ) \; ; } \cr
}
$}

\noindent Note, again, that unlike the {\tt for} statement, the {\tt do-while}
statement does not contain syntax for the initializer and the updater.
These must be coded separately.

To further highlight the difference between a loop statement and a
loop structure, the {\tt do-while} structure takes the following
form:

\begin{jjjlisting}
\begin{lstlisting}
InitializerStatements1;     // Initializer
do {                        // Beginning of loop body
  InitializerStatements2;   //   Another initializer
  Statements;               //   Loop body
  UpdaterStatements         //   Updater
} while (loop entry condition); // Bound test
\end{lstlisting}
\end{jjjlisting}

\noindent Note that initializer statements may be placed
before the loop body, at the beginning of the loop body, or in
both places, depending on the particular problem. Like the other loop
structures, updater statements occur within the body of the loop.  A
flowchart of the {\tt do-while} structure is shown in
Figure~\ref{fig-dowhileflow}.

%\begin{figure}
\begin{figure}[t]
\figaleft{chptr06/6f4.eps}{Flowchart of the {\tt do-while} statement
and {\tt do-while} structure.
} {fig-dowhileflow}

\end{figure}
%\end{figure}

The {\tt do-while} structure would be an appropriate control structure
for the following type of problem:

\begin{jjjlisting}
\begin{lstlisting}[stringstyle=\color{black}]
do
  dial the desired telephone number  // Initializer
  if you get a busy signal
    hang up                          // Updater
while there is a busy signal         // Bound test
\end{lstlisting}
\end{jjjlisting}

\noindent In this case, you want to perform the actions in
the body of the loop at least once and possibly more than
once (if you continue to receive a busy signal).

\JavaTIP[false]{EFFECTIVE DESIGN}{Do-While Loops.}{The  {\tt do-while} loop is designed
for solving problems in which at least one iteration must
occur.\index{loop design}}

\JavaTIP[false]{EFFECTIVE DESIGN}{While versus Do-While Structures.}{For problems where
a noncounting loop is required, the {\tt while} loop structure is more
general and, therefore, preferable to the {\tt do-while} structure.
Use {\tt do-while} only when at least one iteration must occur.}


\secEXRHone{Self-Study Exercises}
\begin{SSTUDY}
\item Identify the syntax error in the
following {\tt do-while} \mbox{structures:}
\begin{EXRLL}
\item

\begin{jjjlisting}
\begin{lstlisting}
int k = 0;
do while (k < 100)
{    System.out.println(k);
     k++
}
\end{lstlisting}
\end{jjjlisting}

\item

\begin{jjjlisting}
\begin{lstlisting}
int k = 0;
do {
    System.out.println(k);
    k++;
} while (k < 12)
\end{lstlisting}
\end{jjjlisting}
\end{EXRLL}


\item Your sister has moved on to counting by sevens. Write a {\tt
do-while} loop that prints the following sequence of numbers: 1, 8,
15, 22, 29, 36, 43.


\item  As the owner of Pizza Heaven, every night at the close of business you
quickly enter the price of every pizza ordered that day. You take the
data from the servers' receipts. Pizzas cost \$8, \$10, or (the
Heavenly Special) \$15. You enter the data without dollar signs, and
use 99 to indicate you're finished for the day.  Write a Java method to
input and validate a single pizza data item. If an incorrect price is
entered, the program should print an error message and prompt for
corrected input.  Correct input is used to compute a daily total.

\item  Because the pizza prices in the previous exercise are fixed, change
the method so you can save time on keyboarding. Instead of entering
the price, you'll enter codes of 1, 2, or 3 (corresponding to the
\$8, \$10, and \$15 pizzas), and 0 to indicate that you're finished. Validate that the
data value entered is correct and then convert it to the
corresponding price before returning it.

\end{SSTUDY}


\section{Example: Computing Averages}
\label{example-computing-averages}
\index{Compute Average Example}
\noindent Suppose you want to compute the average of your exam grades in a
%\begin{marginalnote}\it Algorithm design: what kind of loop?\end{marginalnote}
\marginnote{Algorithm design: what kind of loop?}  course.  Grades,
represented as real numbers, will be input from the keyboard using our
{\tt KeyboardReader} class. To signify the end of the list, we will
use a sentinel value---9999 or $-1$ or some other value that won't be
confused with a legitimate grade. Because we do not know exactly how
many grades will be entered, we will use a noncounting loop in this
algorithm. Because there could be no grades to average, we will use a
{\tt while} structure so it is possible to skip the loop entirely in
the case that there are no grades to average.

The algorithm should add each grade to a running total, keeping track
%\begin{marginalnote}Algorithm design\end{marginalnote}
\marginnote{Algorithm design}
of the number of grades entered. Thus, this algorithm requires two
variables: one to keep track of the running total and the other to
keep track of the count. Both should be initialized to 0.  After the
last grade has been entered, the total should be divided by the count
to give the average.  In pseudocode, the algorithm for this problem is
as follows:

\begin{jjjlisting}
\begin{lstlisting}
initialize runningTotal to 0         // Initialize
initialize count to 0
prompt and read the first grade      // Priming read
while the grade entered is not 9999 {// Sentinel test
    add it to the runningTotal
    add 1 to the count
    prompt and read the next grade   // Update
}
if (count > 0)           // Guard against divide by 0
    divide runningTotal by count
output the average as the result
\end{lstlisting}
\end{jjjlisting}

\noindent Note that in this problem our loop variable, {\tt grade},
%\begin{marginalnote}\it Priming read\end{marginalnote}
\marginnote{Priming read} is read before the loop test is made. This
is known as a {\bf priming read}. It is necessary in this case,
because the loop test depends on the value that is read.  Within the
loop's body, the updater reads the next value for grade. This is a
standard convention for coding {\tt while} structures that involve
input, as this problem does. Note also that we must make sure that
{\tt count} is not 0 before we attempt to compute the
average because dividing by 0 would cause a divide-by-zero error.

Translating the pseudocode algorithm into Java raises several issues.
Suppose we store each grade that is input in a {\tt double} variable
named {\tt grade}.  The loop will terminate when {\tt grade} equals
9999, so its entry condition will be {\tt (grade != 9999)}. Because
this condition uses {\tt grade}, it is crucial that the {\tt grade}
variable be initialized before the bound test is made.  This requires
a priming read. By reading the first value of {\tt grade} before the
loop entry condition is tested, ensures that the loop will be skipped
if the user happens to enter the sentinel (9999) on the very first
prompt. In addition to reading the first exam score, we must
initialize the variables used for the running total and the
counter. Thus, for our initialization step, we get the following code:
\marginnote{Initialization step}

\begin{jjjlisting}
\begin{lstlisting}
double runningTotal = 0;
int count = 0;
reader.prompt("Input a grade (e.g., 85.3) " +
      "or 9999 to indicate the end of the list >> ");
double grade = 
     reader.getKeyboardDouble(); // Priming input
\end{lstlisting}
\end{jjjlisting}

\noindent Within the body of the loop we must add the grade to the
running total and increment the counter. Since these variables are not
tested in the loop entry condition, they will not affect the loop
control.  Our loop updater in this case must read the next
grade. Placing the updater statement at the end of the
%\begin{marginalnote}\it Updater step\end{marginalnote}
\marginnote{Updater step}
loop body will ensure that the loop terminates immediately after the
user enters the sentinel value:

\begin{jjjlisting}
\begin{lstlisting}
while (grade != 9999) {    // Loop test: sentinel
  runningTotal += grade;                            
  count++;                                          
  reader.prompt("Input a grade (e.g., 85.3) " +
      "or 9999 to indicate the end of the list >> ");
  grade = reader.getKeyboardDouble(); // Update:input
} // while
\end{lstlisting}
\end{jjjlisting}

\noindent You can see that it is somewhat redundant to repeat the same 
statements needed to do the initializating and the updating of
the {\tt grade} variable. A better design would be to encapsulate
%\begin{marginalnote}\it Modularity\end{marginalnote}
\marginnote{Modularity}
these into a method and then call the method both before and within
the loop.  The method should take care of prompting the user, reading
the input, converting it to {\tt double}, and returning the input
value. The method doesn't require a parameter:

\begin{jjjlisting}
\begin{lstlisting}
private double promptAndRead() {
  reader.prompt("Input a grade (e.g., 85.3) " +
     "or 9999 to indicate the end of the list >> ");
  double grade = reader.getKeyboardDouble();
                          // Confirm input
  System.out.println("You input " + grade + "\n"); 
  return grade;
}
\end{lstlisting}
\end{jjjlisting}

\noindent Note that we have declared this as a {\tt private}
method. It will be used to help us perform our task but won't be
available to other objects. Such private methods are frequently called
{\em helper methods}.

This is a much more modular design. In addition to cutting down on
redundancy in our code, it makes the program easier to maintain. For
example, there is only one statement to change if we decide to change
the prompt message.  It also makes the program easier to debug. Input
errors are now localized to the {\tt promptAndRead()} method.

\JavaTIP{EFFECTIVE DESIGN}{Modularity.}{Encapsulating code
in a method is a good way to avoid redundancy in a program.}

\JavaTIP{DEBUGGING TIP}{Localization.}{Encapsulating code in a method
removes the need to have the same code at several locations in a
program. By localizing the code in this way, you make it easier to
modify and debug.}

\noindent Another advantage of encapsulating the input task in a
separate method is that it simplifies the task of calculating the
average. This task should also be organized into a separate method:

\begin{jjjlisting}
\begin{lstlisting}
public double inputAndAverageGrades() {
  double runningTotal = 0;
  int count = 0;
  double grade = promptAndRead();// Priming initializer
  while (grade != 9999) {      // Loop test: sentinel
    runningTotal += grade;                           
    count++;                                          
    grade = promptAndRead();// Update: get next grade
  } // while
  if (count > 0)      // Guard against divide-by-zero
    return runningTotal / count;// Return the average
  else
    return 0;         // Special (error) return value
}
\end{lstlisting}
\end{jjjlisting}

\noindent Note that we have declared this as a {\tt public}
method. This will be the method you call to calculate your course
average. 

Because we have decomposed the problem into its subtasks, each subtask
\marginnote{Method decomposition}
is short and simple, making it easier to read and understand. As we saw
in the checkerboard example, the use of small, clearly-focused methods
is a desireable aspect of designing a program.

The complete {\tt Average.java} application is shown in
Figure~\ref{fig-avg}. Its overall design is similar to application
programs we designed in previous chapters. The only instance variable
it uses is the {\tt KeyboardReader} variable. The other variables are
declared locally, within the methods. In this case, declaring them
locally makes the algorithms easier to read.

%% proglist ch6/average/Average.java
\begin{figure}[h!]
\jjjprogstart
\begin{jjjlistingleft}[30pc]{-4pc}
\begin{lstlisting}
import java.io.*;
public class Average {                            // Console I/O
  private KeyboardReader reader = new KeyboardReader();  

  private double promptAndRead() {
    reader.prompt("Input a grade (e.g., 85.3) " +
           "or 9999 to indicate the end of the list >> ");
    double grade = reader.getKeyboardDouble();
    System.out.println("You input " + grade + "\n");// Confirm input
    return grade;
  }
  public double inputAndAverageGrades() {
    double runningTotal = 0;
    int count = 0;
    double grade = promptAndRead();  // Initialize: priming input
    while (grade != 9999) {          // Loop test: sentinel
      runningTotal += grade;                            
      count++;                                          
      grade = promptAndRead();       // Update: get next grade
    } // while

    if (count > 0)         // Guard against divide-by-zero
      return runningTotal / count;   // Return the average
    else
      return 0;            // Special (error) return value
 }
 public static void main(String argv[]) {
   System.out.println("This program calculates average grade."); 
   Average avg = new Average();
   double average = avg.inputAndAverageGrades();
   if (average == 0)                             // Error check
      System.out.println("You didn't enter any grades.");
   else
      System.out.println("Your average is " + average);        
 } // main()
} // Average
\end{lstlisting}
\end{jjjlistingleft}
\jjjprogstop{A program to compute average grade using a {\tt while}
structure.}
{fig-avg}
\end{figure}

One final point about this program is to note the care taken in the
design of the user interface to explain the program to the user, to
prompt the user before a value is input, and to confirm the user's
input after the program has read it.

\JavaTIP[false]{EFFECTIVE DESIGN}{User Interface.}{Whenever you ask a user
for input, the user should know {\it why} you are asking and {\it
what} you are asking for. Prompts should be used for this purpose. It
is also a good idea to confirm that the program has received the
correct input.}

\section{Example: Data Validation}
\label{example-data-validation}
\index{Data Validation Example}
\noindent One frequent programming task is {\it data validation}. This
task can take different forms depending on the nature of the
program. One use for data validation occurs when accepting input from
the user.

In the program in the preceding section, suppose the user types $-10$
by mistake when asked to input an exam grade. Obviously this is not a
valid exam grade and should not be added to the running total. How
should a program handle this task?

Because it is possible that the user may take one or more attempts
\marginnote{Algorithm design} 
to correct an input problem, we should use a do-while structure for
this problem.  The program should first input a number from the
user. The number should then be checked for validity. If it is valid,
the loop should exit and the program should continue computing the
before getting the input average grade. If it is not valid, the
program should print an error message and input the number again. A
flowchart for this algorithm is shown in Figure~6.11.

For example, suppose only numbers between 0 and 100 are considered
valid. The data validation algorithm would be as follows:

\begin{jjjlisting}
\begin{lstlisting}
do
 Get the next grade // Initialize: priming input
 if the grade < 0 or grade > 100 and grade != 9999
   print an error message // Error case
            // Sentinel test
while the grade < 0 or grade > 100 and grade != 9999 
\end{lstlisting}
\end{jjjlisting}

\noindent Note here that initialization and updating of the loop
\marginfig{chptr06/6f6.eps}{Do-while is a good structure for 
the data validation algorithm.}
{fig-datavalid}
variable are performed by the same statement. This is acceptable
because we must update the value of {\tt grade} on each iteration {\it
before} checking its validity. Note also that for this problem the
loop-entry condition is also used in the if statement to check for an
error.  This allows us to print an appropriate error message if the
user makes an input error.

Let's incorporate this data validation algorithm into the {\tt
prompt\-And\-Read()} method that we designed in the previous section
(Fig.~\ref{fig-avg}). The revised method will handle and validate all
input and return a number between 0 and 100 to the calling method. To
reflect its expanded purpose, we will change the method's name to {\tt
getAndValidateGrade()}, and incorporate it into a revised application,
which we name {\tt Validate} (Fig.~\ref{fig-validate}).

%% proglist ch6/validate/Validate.java
\begin{figure}[p]
\jjjprogstart
\begin{jjjlistingleft}[34pc]{-8pc}
\begin{lstlisting}
import java.io.*;
public class Validate {                            // Console input
  private KeyboardReader reader = new KeyboardReader();  

  private double getAndValidateGrade() {
    double grade = 0;
    do {
       reader.prompt("Input a grade (e.g., 85.3) " +
                     "or 9999 to indicate the end of the list >> ");
       grade = reader.getKeyboardDouble();
       if ((grade != 9999) && ((grade < 0) || (grade > 100))) // If error
          System.out.println("Error: grade must be between 0 and 100 \n");  
       else
          System.out.println("You input " + grade + "\n");  // Confirm input
    }  while ((grade != 9999) && ((grade < 0) || (grade > 100)));
    return grade;
  }
  public double inputAndAverageGrades() {
    double runningTotal = 0;
    int count = 0;
    double grade = getAndValidateGrade();  // Initialize: priming input
    while (grade != 9999) {                // Loop test: sentinel
       runningTotal += grade;                            
       count++;                                          
       grade = getAndValidateGrade();     // Update: get next grade
    } // while

    if (count > 0)                        // Guard against divide-by-zero
       return runningTotal / count;       // Return the average
    else
       return 0;                          // Special (error) return value
  }
  public static void main(String argv[]) {
    System.out.println("This program calculates average grade."); // Explain
    Average avg = new Average();
    double average = avg.inputAndAverageGrades();
    if (average == 0)                           // Error check
       System.out.println("You didn't enter any grades.");
    else
       System.out.println("Your average is " + average);        
  } // main()
} // Validate
\end{lstlisting}
\end{jjjlistingleft}
\jjjprogstop{A program to compute average grade using a {\tt while}
structure. This version validates the user's input.}
{fig-validate}
\end{figure}

\section{Principles of Loop Design}
\label{principlesof-loop-design}
\noindent Before moving on, it will be useful to summarize the main
principles\index{loop design principles} involved in correctly
constructing a loop.

\begin{BL}
\item  A {\it counting\index{counting loop} loop} can
be used whenever you know in advance exactly how many iterations are
needed.  Java's {\it for statement} is an appropriate structure for
coding a counting loop.

\item  A {\it while structure}\index{while structure} should
be used when the problem suggests that the loop body may be skipped
entirely. Java's {\it while statement} is specially designed for the
while structure.

\item  A {\it do-while structure}\index{do-while structure} should
be used only when a loop requires one or more iterations. Java's
{\it do-while statement} is specially designed for the do-while structure.

\item  The {\it loop variable} is used to specify the
{\it loop-entry condition}. It must be initialized to an appropriate
initial value, and it must be updated on each iteration of the loop.

\item  A loop's {\it bound\index{bound}} may be a {\it count}, a
{\it sentinel}, or, more generally, a {\it conditional bound}.  It
must be correctly specified in the loop-entry expression, and progress
toward the bound must be made in the {\it updater}.

\item  An {\it infinite\index{infinite loop} loop} may result if the
initializer, loop-entry expression, or updater expression is not
correctly specified.
\end{BL}


\noindent The loop types are also summarized in Table~6.1.

\begin{table}[h]
%\hphantom{\caption{A summary of the design decisions required when coding a loop}}
\TBT{0pc}{A summary of the design decisions required when coding a loop}
\hspace*{-6pt}\begin{tabular}{lll}
\multicolumn{3}{l}{\color{cyan}\rule{27pc}{1pt}}\\[2pt]
%%%\TBCH{Use}                 &\TBCH{If}  &\TBCH{Java Statement}
\bf{Use}                 &\bf{If}  &\bf{Java Statement}
\\[-4pt]\multicolumn{3}{l}{\color{cyan}\rule{27pc}{0.5pt}}\\[2pt]
{\it Counting loop}       &{\it Number of iterations known in advance}\hspace*{6pt} &{\it for}      
\\[-4pt]\multicolumn{3}{l}{\color{cyan}\rule{27pc}{0.5pt}}\\[2pt]
{\it While structure}     &{\it Number of iterations not known}        &{\it while}    \cr
                          &{\it Loop may not be entered at all}        &               
\\[-4pt]\multicolumn{3}{l}{\color{cyan}\rule{27pc}{0.5pt}}\\[2pt]
{\it Do-while structure}\hspace*{6pt}  &{\it Number of iterations not known}        &{\it do-while} \cr
                          &{\it Loop must be entered at least once}    &               
\\[-4pt]\multicolumn{3}{l}{\color{cyan}\rule{27pc}{1pt}}
\end{tabular}
\endTB
\end{table}

\vspace*{-12pt}
\secEXRHone{Self-Study Exercise}
\begin{SSTUDY}
\item  For each of the following problems, decide whether a counting loop
structure, a {\tt while} structure, or a {\tt do-while} structure
should be used, and write a pseudocode algorithm.


\begin{itemize}
\vspace*{-4pt}
\item  Print the names of all visitors to your Web site.
\vspace*{-4pt}
\item  Validate that a number input by the user
is positive.
\vspace*{-4pt}
\item  Change all the backslashes ($\backslash$) in a
Windows Web page address to the slashes (/) used in
a Unix Web page address.
\vspace*{-4pt}
\item Find the car with the best miles-per-gallon ratio among the cars
in the {\it Consumer Reports} database.
\end{itemize}

\end{SSTUDY}


\section{The {\tt switch} Multiway Selection Structure}
\label{the-multiway-selection-structure}

\noindent Another selection structure to add to our repertoire is the {\bf
switch/break structure}.  It is meant
to provide a
shorthand way of coding the following type of multiway selection
structure:

\begin{jjjlisting}
\begin{lstlisting}
if (integralVar == integralValue1)
     // some statements
else if (integralVar == integralValue2)
     // some statements
else if (integralVar == integralValue3)
     // some statements
else // some statements
\end{lstlisting}
\end{jjjlisting}

\noindent Note that each of the conditions in this case involves the
equality of an integral variable and an integral value.  This type of
structure occurs so frequently in programs that most languages contain
statements specially designed to handle it.  In Java, we use a
combination of the {\tt switch} and {\tt break} statements to
implement multiway selection.

The {\tt switch} is designed to select one of several actions
depending on the value of some integral expression:

\begin{jjjlisting}
\begin{lstlisting}
switch (integralExpression)
{  case integralValue1:
       // some statements
   case integralValue2:
       // some statements
   case integralValue3:
       // some statements
   default:
       some statements
}
\end{lstlisting}
\end{jjjlisting}

\noindent The {\it integralExpression} must evaluate to a primitive integral
value of type {\tt byte}, {\tt short}, {\tt int}, {\tt char},
\marginnote{\it\color{cyan}Integral expression}
or {\tt boolean}. It may not be a {\tt long}, {\tt float}, {\tt
double}, or a class type.  The {\it integralValues} must be literals
or {\tt final} variables.  They serve as labels in the one or more
case clauses that make up the {\tt switch} statement body.  The {\tt
default} clause is optional, but it is a good idea to include it.

A {\tt switch} statement\index{switch statement} is executed according
to the following rules:

\begin{NL}
\item[Rule 1.]  The {\it integralExpression} is evaluated.
\vspace*{6pt}
\item[Rule 2.]  Control passes to the statements following the {\tt case}
label whose value equals the {\it integralExpression} or,
if no cases apply, to the {\tt default} clause.
\vspace*{6pt}

\item[Rule 3.]  Beginning at the selected label or at the default, all
of the statements up to the end of the {\tt switch} are executed.
\end{NL}

\noindent Consider the following example:

\begin{jjjlisting}
\begin{lstlisting}
int m = 2;
switch (m)
{  case 1:
      System.out.print(" m = 1");
   case 2:
      System.out.print(" m = 2");
   case 3:
      System.out.print(" m = 3");
   default:
      System.out.print(" default case");
}
\end{lstlisting}
\end{jjjlisting}

\noindent In this case, because {\it m} equals 2, the following output
would be produced:

\begin{jjjlisting}
\begin{lstlisting}
 m = 2 m = 3 default case
\end{lstlisting}
\end{jjjlisting}

\noindent Obviously, this output does not match
the following if-else multiway selection
structure\index{multiway-selection structure}, which would output,
simply, {\tt m = 2}:


\begin{jjjlisting}
\begin{lstlisting}
int m = 2;
if (m == 1)
     System.out.print(" m = 1");
else if (m == 2)
     System.out.print(" m = 2");
else if (m == 3)
     System.out.print(" m = 3");
else
     System.out.print(" default case");
\end{lstlisting}
\end{jjjlisting}

\noindent The reason for this disparity is that the {\tt switch}
executes {\it all} statements following the label that matches the
value of the {\it integralExpression} (see again Rule 3 on the
previous page).

In order to use the switch as a multiway selection, you must force it
to {\tt break\index{break}} out of the case clause after executing
that clause's statements:

\begin{jjjlisting}
\begin{lstlisting}
int m = 2;
switch (m)
{  case 1:
       System.out.print(" m = 1");
       break;
   case 2:
       System.out.print(" m = 2");
       break;
   case 3:
       System.out.print(" m = 3");
       break;
   default:
       System.out.print(" default case");
}
\end{lstlisting}
\end{jjjlisting}


\noindent In this example, the {\tt break} statement causes control to
pass to the end of the {\tt switch}, with the effect being that one
and only one case will be executed within the {\tt switch}. Thus, the
output of this code segment will be simply \mbox{\tt m = 2}, matching
exactly the behavior of the multiway if-else selection structure
(Fig.~\ref{fig-multiwayswitch}).

\begin{figure}[hbt]
\figa{chptr03/switchfl.eps}{Flowchart of the multiway switch structure.
Note that because of the {\tt break} statement, one and only one case is
executed.}
{fig-multiwayswitch}

\end{figure}


\JavaTIP{PROGRAMMING TIP}{Multiway Selection.}{A typical use for the
{\tt switch} statement is to use it together with {\tt break} to code a
multiway selection structure.}

\JavaRule{{\tt break}.}{The {\tt break} statement transfers
control out of its enclosing {\it block}, where a block is any
sequence of statements contained within curly brackets \{ and \}.}


\JavaTIP{DEBUGGING TIP}{Switch without {\tt break}.}{A
common error in coding the switch-based multiway selection is
forgetting to put a {\tt break} statement at the end of each clause.
This may cause more than one case to be executed.}

\newpage
\secEXRHone{Self-Study Exercises}
\begin{SSTUDY}
\item  Identify any errors in the
following switch structures (if
there is no error, specify the output):

\begin{jjjlisting}
\begin{lstlisting}
(a) int k = 0;
    switch (k)
    case 0:
        System.out.println("zero");
        break;
    case 1:
        System.out.println("one");
        break;
    default:
        System.out.println("default");
        break;
\end{lstlisting}
\end{jjjlisting}

\begin{jjjlisting}
\begin{lstlisting}
(b) int k = 0;
    switch (k + 1)
    {   case 0:
            System.out.println("zero");
            break;
        case 1:
            System.out.println("one");
            break;
        default:
            System.out.println("default");
            break;
    }
\end{lstlisting}
\end{jjjlisting}
\begin{jjjlisting}
\begin{lstlisting}
(c) int k = 6;
    switch (k / 3.0)
    {   case 2:
            System.out.println("zero");
            break;
        case 3:
            System.out.println("one");
            break;
        default:
            System.out.println("default");
            break;
    }
\end{lstlisting}
\end{jjjlisting}

\item  Flavors of ice cream are represented as integers where 0 is
vanilla, 1 is chocolate, and 2 is strawberry.  Write a {\tt switch}
statement that checks an integer variable {\tt flavor} and prints
out the name of the ice cream flavor or prints ``Error'' in the
default case.

\item  Modify your solution to the previous exercise to use
constants (final variables) to represent the ice cream flavors.
\end{SSTUDY}

\markright{{\color{cyan}SECTION\,\,\arabic{section}\,\,$\bullet$}\,\, OOD: Structured Programming}

\section{OBJECT-ORIENTED DESIGN: \\Structured Programming}
\label{objectorienteddesign-structured-programming}

\markright{{\color{cyan}SECTION\,\,\arabic{section}\,\,$\bullet$}\,\, OOD: Structured Programming}


%%\secHfourleft{\mbox{OBJECT-ORIENTED} DESIGN: \\Structured Programming}
\index{structured programming}
\noindent {\it Structured programming} is the practice of writing programs that
are built up from a small set of predefined control structures. As an
overall approach to programming, structured programming has largely
been superseded by the object-oriented approach. Nevertheless, its
design principles are still relevant to the design of the
algorithms and methods that make up a program's objects.

The principles of structured programming seem so obvious today that it
may be difficult to appreciate their importance. In the 1960s and 1970s,
one of the main controls used in programs was the infamous
{\it go to\index{goto statement}} statement, which could be used to transfer
control of a program to any arbitrary location within it, and
from there to any other arbitrary location, and so on. This led to
incredibly complex and ill-formed programs---so called ``spaghetti
\marginnote{Spaghetti code}
code''---that were almost impossible to understand and modify.

Structured programming evolved in reaction to the unstructured
software development practices of the 1960s, which were fraught with
budget overruns, costly delays, and failed products. One of the classic
research results of that era was a 1966 paper by Boehm\index{Boehm} and
Jacopini\index{Jacopini} that showed that any program using go to's
could be represented by an equivalent program that used a sequence of
two types of controls: if/else and while structures. Another
influential paper by Edgar Dikjstra\index{Dikjstra} (``GoTo Statement
Considered Harmful'') pointed out the various ways in which the go to statement
could lead to impossibly complex programs.

The Pascal language, introduced by Nicklaus Wirth\index{Wirth} in 1971, was
designed to promote structured programming techniques and became the
language of choice within academic institutions because of its
suitability as a teaching language. In Pascal, the go to was replaced
with the four structures that control the flow of execution
in a program (Fig.~\ref{fig-controlstructs}):


\begin{figure}[hbt]
\figa{chptr06/6f10.eps}{Flowcharts of the four types of
control structures. Each small rectangle represents a single executable
statement.\index{control structures}}
{fig-controlstructs}

\end{figure}

\begin{BL}
\item  {\it Sequence}---The statements in a program are executed in
sequential order unless their flow is interrupted by one of the
following control structures.

\item  {\it Selection}---The {\tt if}, {\tt if/else}, and {\tt switch}
statements are {\it branching} statements that allow choice through
the forking of the control path into two or more alternatives.

\item  {\it Repetition}---The {\tt for}, {\tt while}, and {\tt do-while}
statements are {\it looping} statements that allow the program to
repeat a sequence of statements.

\item  {\it Method Call}---Invoking a method transfers control temporarily
to a named method. Control returns to the point of invocation when the
method is completed.

\end{BL}

\noindent No matter how large or small a program you write, its flow of
control can be constructed as a combination of these four basic types
of structures.


%% Footnote: Boehm, C. and Jacopini, G. (1966) ``Flow Diagrams,
%% Turing Machines, and Languages with Only Two Formation Rules,''
%% Communications of the ACM.

%% Footnote: Dijkstra, E. (1968) ``GoTo Statement Considered Harmful,''
%% Communications of the ACM, 11,3,1968:147-148.

%% Footnote: Wirth, N. (1971) ``The Programming Language Pascal,''
%% Acta Informatica, 1,1,1971:35-63.


\subsection*{Preconditions and Postconditions}
%<sec_prepost:\thiseqno>
%<pg_sec_prepost:\thispageno>
\noindent The Java language supplies us with a good collection of control
structures, and Java's syntax constrains the way we can use them. One of
the features of the four control structures is that each has a single
entry point and exit (Fig.~\ref{fig-controlstructs}). This is an
extremely important property. To grasp its importance, consider the
following debugging problem:

\begin{jjjlisting}
\begin{lstlisting}
k = 0;                        // 1. Unstructured code
System.out.println("k= " + k);// 2. k should equal 0 here
goto label1;                  // 3.
label2:
System.out.println("k= " + k);// 4. k should equal 1 here
\end{lstlisting}
\end{jjjlisting}

\noindent 
In this example a {\it goto} statement is used to jump to {\tt
label1}, a label that marks a section of code somewhere else in the
program. Suppose we're trying to determine how {\it k} has acquired an
erroneous value and that its value is correct in line 2 of this
sequence. Given the go to statement on line 3, there's no guarantee
that control will ever return to the {\tt println()} statement on line
4. Thus, in unstructured code it is very difficult to narrow the scope
of an error to a fixed segment of code. Because the go to statement
can transfer control anywhere in the program, with no guarantee of
return, any segment of code can have multiple entry points and
multiple exits.

Now contrast the unstructured code with the following well-structured code:


\begin{jjjlisting}
\begin{lstlisting}
k = 0;                         // 1. Structured code
System.out.println("k= " + k); // 2. k should equal 0 here
someMethod();                  // 3.
System.out.println("k= " + k); // 4. k should equal 1 here
\end{lstlisting}
\end{jjjlisting}

\noindent In this case, we can be certain that control will eventually
return to line 4. If {\it k}'s value is erroneous on line 4, we can trace
\marginnote{Debugging with {\tt println()}}
through {\tt someMethod()} to find the error. Because any segment
of a structured program has a single entry and exit, we can use
a pair of {\tt println()} statements in this way to converge
on the location of the program bug.

An important implication of the single-entry/single-exit property is
that we can use {\bf preconditions} and {\bf postconditions} to help
us design and debug our code. The previous example provided a simple
example: The precondition is that {\it k} should equal 0 on line 2,
and the postcondition is that {\it k} should equal 1 on line
4. Figure~\ref{fig-preposts} shows some additional examples.

\begin{figure}[htb]
\jjjprogstart
\mbox{$\bullet$} \vspace{-18pt}
\begin{jjjlisting}
\begin{lstlisting}
  int k = 0;   // Precondition: k == 0
  k = 5;       // Assignment to k
 // Postcondition: k == 5
\end{lstlisting}
\end{jjjlisting}


\mbox{$\bullet$} \vspace{-18pt}
\begin{jjjlisting}
\begin{lstlisting}
  int k = 0;           // Precondition: k == 0
  while (k < 100) {    // While loop
      k = 2 * k + 2;
  }
  // Postcondition: k >= 100
\end{lstlisting}
\end{jjjlisting}


\mbox{$\bullet$} \vspace{-18pt}
\begin{jjjlisting}
\begin{lstlisting}
  /*
   * factorial(n):
   *   factorial(n) is 1 if n is 0
   *   factorial(n) is n * n-1 * n-2 * ... * 1 if n > 0
   * Precondition:  n >= 0
   * Postcondition: 
   *  factorial(n) = 1 if n = 0
   *               = n * n-1 * n-2 * ... * 1 if n > 0
   */
  public int factorial(int n) {
      if (n == 0)
          return 1;
      else {
          int f = 1;         // Init a temporary variable
          for (int k = n; k >= 1; k--) // For n down to 1
              f = f * k;      //   Accumulate the product
          return f;               // Return the factorial
      } // else
  } // factorial()
\end{lstlisting}
\end{jjjlisting}
\jjjprogstop{Using pre- and postconditions to document code.}
{fig-preposts}
\end{figure}

In the first example, we use pre- and postconditions to define
the semantics of an assignment statement. No matter what value {\it k} has
before the assignment, the execution of the assignment ({\tt k = 5}) will make
the postcondition ({\tt k == 5}) true.

In the second example, the postcondition follows from the semantics of
the while loop. Because the loop-entry condition is \verb|k < 100|,
when the loop exits the postcondition (\verb|k >= 100|) must be true.

The third example shows how pre- and postconditions can be used
to design and document methods. The {\it factorial(n)} is defined
for $ n \geq 0 $ as follows:


\begin{jjjlisting}
\begin{lstlisting}
factorial(n) is 1, if n == 0
factorial(n) is n * n-1 * n-2 * ... * 1, if n > 0
\end{lstlisting}
\end{jjjlisting}

\noindent In other words, the factorial of {\it N} is defined as
the cumulative product of multiplying 1 times 2, times 3, and so on up
to {\it N}.  For example, if {\it N} is 5, then {\tt factorial(5)} is 1 * 2 * 3 *
4 * 5 = 120.


Note how the factorial computation is done in the method.  The
variable {\it f}, which is used to accumulate the product, is
initialized to 1. Then on each iteration of the {\tt for} loop, {\it f} is
multiplied by {\it k} and the product is assigned back to {\it f}.
This is similar to the way we accumulate a sum, except in this case we
are accumulating a product.




The precondition on the {\tt factorial()} method represents the
condition that must be true in order for the method to work
correctly. Factorial is undefined for $n < 0$, so it is important that
{\it n} be greater than or equal to 0 whenever this method is called. Given
that the precondition holds, the postcondition gives a precise
specification of what must be true when the~method is finished.

\subsection*{Design: Defensive Programming}
\noindent The pre- and postconditions for a method can be used to design
defensive code---that is, code that guards against errors. For
example, what action should {\tt factorial()} take if its precondition
fails to hold?  In Java, the best way to handle this situation is to
{\em throw} an {\tt IllegalArgumentException}, as the following example
illustrates:

\begin{jjjlisting}
\begin{lstlisting}
public int factorial(int n) {
  if (n < 0)                  // Precondition failure
   throw new IllegalArgumentException("Factorial:"+ n);
  if (n == 0)
    return 1;
  else {
    int f = 1;           // Init a temporary variable
    for (int k = n; k >= 1; k--)   // For n down to 1
      f = f * k;        //     Accumulate the product
    return f;                 // Return the factorial
  }
} // factorial()
\end{lstlisting}
\end{jjjlisting}

\noindent An {\bf exception} is an erroneous condition (an error) that
arises during the running of a program. An {\tt Exception} is an
object that encapsulates information about the erroneous condition.  A
program can {\em throw} an {\tt Exception}, thereby stopping the
program, when an erroneous condition is detected. In this example, we
create a new {\tt IllegalArgumentException} that would report the
illegal value of {\em n} with something like the following error
message:

\begin{jjjlistingleft}[28pc]{-2pc}
\begin{lstlisting}
Exception in thread "main" java.lang.IllegalArgumentException: 
          Factorial: -1
   at Test.factorial(Param.java:5)
   at Test.main(Param.java:18)
\end{lstlisting}
\end{jjjlistingleft}

\noindent You have undoubtedly already encountered thrown exceptions
during program development.  Java has an extensive hierarchy of {\tt
Exception}s, which we will cover in some depth in Chapter~11.  For
now, however, we just note how to use the {\tt
IllegalArgumentException}. As its name implies, an {\tt
IllegalArgumentException} is used when an argument in a method call is
not legal.

Rather than continuing the program with an erroreous data value,
throwing an exception causes the program to stop and print an error
message.  Determining whether an argument is legal or illegal is an
important use of the method's preconditions.  The failure of the
precondition in {\tt factorial()} points to a problem elsewhere in the
program, because it is doubtful that the program deliberately passed a
negative value to {\tt factorial()}. The discovery of this error
should lead to modifications in that part of the program where {\tt
factorial()} was invoked---perhaps to some validation of the user's
input:

\begin{jjjlisting}
\begin{lstlisting}
int num = Integer.parseInt(textIn.getText());
if (num >= 0)    // If factorial() precondition valid
    factNum = factorial(num);   // Compute factorial
else
    System.out.println("Error");// Report input error}
\end{lstlisting}
\end{jjjlisting}

\noindent This would be the traditional way to handle this kind
of error. 

\subsection*{Using \mbox{Pre-} and Postconditions}
\noindent The use of preconditions and postconditions in the ways we've
described can help improve a program's design at several distinct
stages of its \mbox{development:}

\begin{BL}
\item  Design stage: Using pre- and postconditions in design helps
to clarify the design and provides a precise measure of correctness.

\item  Implementation and testing stage: Test data can be designed
to demonstrate that the preconditions and postconditions hold for
any method or code segment.

\item  Documentation stage: Using pre- and postconditions to document
the program makes the program more readable and easier to modify and
maintain.

\item  Debugging stage: Using the pre- and postconditions provides
precise criteria that can be used to isolate and locate bugs. A
method is incorrect if its precondition is true and its postcondition
is false. A method is improperly invoked if its precondition is false.
\end{BL}

\noindent Like other programming skills and techniques, learning how
to use pre- and postconditions effectively requires practice. One way
to develop these skills is to incorporate pre- and postconditions into
the documentation of the methods you write for laboratories and
programming exercises.  Appendix~A provides guidelines on how to
incorporate pre- and postconditions into your program's documentation.
However, it would be a mistake to get in the habit of leaving the
identification of pre- and postconditions to the documentation
stage. The method's documentation, including its pre- and
postconditions, should be developed during the design stage and
should play a role in all aspects of program development.


\subsection*{Effective Program Design}
\noindent What we're really saying here is that using pre- and postconditions
forces you to analyze your program's logic.  It is not enough to know
that a single isolated statement within a program works correctly at
the present time. You have to ask yourself: Will it continue to work
if you change some other part of the program? Will other parts of the
program continue to work if you revise it? No matter how clever you
are, it is not possible to keep an entire model of a good-sized
program in your head at one time. It is always necessary to focus on a
few essential details and leave aside certain others. Ideally, what
you hope is that the details you've left aside for the moment
aren't the cause of the current bug you're trying to fix.  Using pre-
and postconditions can help you determine the correctness of the
details you choose to set aside.

\JavaTIP{EFFECTIVE DESIGN}{Pre- and Postconditions.}{Pre- and postconditions
are an effective way of analyzing the logic of your
program's loops and methods. They should be identified at the earliest
stages of design and development. They should play a role in the
testing and debugging of the program. Finally, they should be
included, in a systematic way, in the program's documentation.}

\JavaTIP{PROGRAMMING TIP}{}{Develop your program's documentation at
the same time that you develop its code and include the pre- and
postconditions in the documentation.}

\noindent As the programs you write become longer and more complex,
the chances that they contain serious errors increase dramatically.
There's no real way to avoid this complexity. The only hope is to try
to manage it. In addition to analyzing your program's structure,
another important aspect of program design is the attempt to reduce
its complexity.

\JavaTIP{EFFECTIVE DESIGN}{Reducing Complexity.}{Design your programs
with an aim toward reducing their complexity.}

\noindent Perhaps the best way to reduce complexity is to build your programs
using a small collection of standard structures and techniques.  The
basic control structures (Fig.~\ref{fig-controlstructs}) help reduce
the potential complexity of a program by constraining the kinds of
branching and looping structures that can be built. The control
structures help to manage the complexity of your program's algorithms.
In the same way, the following practices can help reduce and manage
the complexity in a program.

\JavaTIP[false]{PROGRAMMING TIP}{Standard Techniques.}{Acquire and use standard
programming techniques for standard programming problems. For example,
using a temporary variable to swap the values of two variables is
a standard technique.}

\JavaTIP[false]{PROGRAMMING TIP}{Encapsulation.}{Use methods wherever
appropriate in your own code to encapsulate important sections of code
and thereby reduce complexity.}

\JavaTIP[false]{PROGRAMMING TIP}{Code Reuse.}{Instead of reinventing the wheel,
use library classes and methods whenever possible. These have been
carefully designed by experienced programmers. Library code has been
subjected to extensive testing.}

%\bigskip\begin{minipage}{\hsize}\section*{What Can Be Computed?}{./art/2/2b1.eps}


\secEXRHone{Self-Study Exercises}
\begin{SSTUDY}
\item  Identify the pre- and postconditions on {\it j} and {\it k}
where indicated in the following code segment:

\begin{jjjlisting}
\begin{lstlisting}
int j = 0; k = 5;
do {
    if (k % 5 == 0)  {
                       // Precondition
        j += k;
        k--;
    }
    else k *= k;
} while (j <= k);     
                      // Postcondition
\end{lstlisting}
\end{jjjlisting}

\item  Identify the pre- and postconditions for the following method, which
computes $x^n$ for  $n \geq 0$:

\begin{jjjlisting}
\begin{lstlisting}
public double power(double x, int n) {
    double pow = 1;
    for (int k = 1; k <= n; k++)
        pow = pow * x;
    return pow;
} // power()
\end{lstlisting}
\end{jjjlisting}

\end{SSTUDY}


\pagebreak
\section*{{\color{cyan}Special Topic:} What Can Be Computed?}
%% Footnote: Turing, A.M. (1936) ``On computable numbers...'',
%% Proceedings of the London Mathematical Society, 42: 230-265.
{Did you} ever wonder whether there are problems that cannot be solved
by a computer, no matter what kind of control structures are used?
Well, back in 1939, in his seminal paper titled ``On Computable
Numbers,'' Alan Turing proved that indeed there are an infinite number
of unsolvable problems. Prior to this, mathematicians and logicians
thought all problems could be solved. So Turing's proof was quite
a blow!

To help him prove this point, Turing defined an abstract computer,
which has come to be known as a Turing machine. A Turing machine has
an alphabet of symbols; a read/write head; an infinitely long tape on
which the read/write head can write symbols, and from which it can also
read symbols; and a control unit, which controls the movement and
action of the read/write head. Note that the Turing machine
elements correspond to key components of a real computer---although
Turing invented this concept a decade before the first computers were
developed.  The read/write head corresponds to a computer's central
processing unit (CPU).  The tape corresponds to the computer's
memory. And the control unit corresponds to the computer program.

A Turing machine represents a purely abstract concept of computation.
It represents the pure idea of an algorithmic solution to a problem.
Equipped with this concept, Turing was able to prove that there are
unsolvable problems---that is, problems for which no algorithm can
arrive at a solution.

One such problem is the {\it halting problem}. This problem asks
whether an algorithm can be devised to determine whether an arbitrary
program will eventually halt. If there were such an algorithm, it
could be used to detect programs that contain infinite loops, a
service that might be really helpful in an introductory computing lab,
among other places!  But, alas, there can be no such algorithm.

Here's an outline of a proof that shows that the halting
problem is unsolvable. (This particular version of the proof was
suggested by J.~Glenn Brookshear in {\it Computer Science: An
Overview}, Benjamin-Cummings, 1985.)

Suppose you had a program, {\it P}, that solves the halting problem.
That is, whenever {\it P} is given a self-halting program, it
sets a variable {\it isTerminating} to true, and otherwise it
sets {\it isTerminating} to false. Now let's create a new version of
{\it P}, named $P\prime$, which is identical to {\it P} except that
right after where {\it P} sets {\it isTerminating} to true or false,
$P\prime$ contains the following loop:

\begin{jjjlisting}[29pc]
\begin{lstlisting}
while (isTerminating == true);  // Infinite if isTerminating true
\end{lstlisting}
\end{jjjlisting}

\noindent In other words, if the input to $P\prime$  is a self-terminating
program, then $P\prime$ will enter an infinite loop and it won't
terminate.  Otherwise, if a non-self-terminating program is input to
$P\prime$, $P\prime$ will skip the loop and will terminate.

Now what if we give a representation of $P\prime$ to itself. Will it halt?
The answer generates a contradiction: If $P\prime$  is a self-terminating
program, then when it is input to itself, it will not terminate.  And
if $P\prime$ is not self-terminating, when it is input to itself, it will
terminate. Because our assumption that {\it P} solves the halting problem
has led to a contradiction, we have to conclude that it wasn't a very
good assumption in the first place. Therefore, there is no program
that can solve the halting problem.

The topic of computability is a fundamental part of the computer
science curriculum, usually taught in a sophomore- or junior-level
theory of computation course.

\secSMHleft{Chapter Summary}
\secKTH{Technical Terms}

\begin{KT}
conditional loop

counting loop

do-while statement

infinite loop

initializer

limit bound

loop body

loop bound

loop entry condition

nested loop

postcondition

precondition 

priming read

repetition structure

sentinel bound

unit indexing

updater

while statement

zero indexing

\end{KT}


\secSMHtwo{Summary of Important Points}
\begin{SMBL}
\item  A {\it repetition structure} is a control structure\index{repetition structure}
that allows a statement or sequence of statements to be repeated.

\item  All loop structures involve three elements---an {\it initializer}, 
a {\it loop entry condition} or a {\it loop boundary condition}, and
an {\it updater}.

\item When designing a loop, it is important to analyze the loop structure
to make sure that the loop bound will eventually be satisfied.

\item The {\tt for} statement\index{for statement} has the following syntax:

\begin{alltt}
for ( {\it initializer} ; {\it loop entry condition} ; {\it updater} )
    {\it for loop body} ; 
\end{alltt} 

\marginpar{
\TBT{0.5pc}{A summary \hfill\break\hspace*{0.5pc}of various loop bounds}
\hspace*{0pc}\begin{tabular}{ll}
\multicolumn{2}{l}{\color{cyan}\rule{10pc}{1pt}}\\[2pt]
%%%\TBCH{Bound}\hspace*{20pt}        &\TBCH{Example}
\bf{Bound}\hspace*{20pt}        &\bf{Example}
\\[-4pt]\multicolumn{2}{l}{\color{cyan}\rule{10pc}{0.5pt}}\\[2pt]
{\it Counting}     &{\it k $<$ $100$}\cr
{\it Sentinel}     &{\it input} != 9999\cr
{\it Flag}         &{\it done} != {\it true}\cr
{\it Limit}        &{\it amount $<$ $0.5$}
\\[-4pt]\multicolumn{2}{l}{\color{cyan}\rule{10pc}{1pt}}
\end{tabular}
\endTB
}

\item The {\tt while} statement takes the following form:

\begin{jjjlisting}
\begin{alltt}
{\tt while (} {\it loop entry condition} {\tt)}   
    {\it loop body} ;
\end{alltt}
\end{jjjlisting}

\item The {\tt do-while} statement has the following general form:

\begin{jjjlisting}
\begin{alltt}
{\tt do} 
    {\it loop body} ;
{\tt while (} {\it loop entry condition} {\tt )} ;
\end{alltt}
\end{jjjlisting}

\item When designing a loop, it is important to analyze the loop
structure to make sure that the loop bound will eventually be
satisified. Table~6.2 summarizes the types of loop bounds that
we have identified.

\item  {\it Structured programming\index{structured programming}}
is the practice of writing programs that are built up from a small set
of predefined control structures---the {\it sequence},
{\it selection}, {\it repetition}, and {\it method-call} structures. An
important feature of these structures is that each has a single entry
and exit.

\item A {\it precondition} is a condition that must be true before a
certain code segment executes. A {\it postcondition} is a condition
that must be true when a certain code segment is
finished. Preconditions and postconditions should be used in the
design, coding, documentation, and debugging of algorithms and
methods.

\end{SMBL}

\pagebreak
\secANSH
\begin{ANS}
\item  Identify the syntax error in the following {\tt for} loop statements:
\begin{enumerate}

\item[a.]

Commas are used instead of semicolons in the header.
\begin{jjjlisting}
\begin{lstlisting}[basicstyle=\scriptsize]
 for (int k = 5; k < 100; k++)
     System.out.println(k);
\end{lstlisting}
\end{jjjlisting}

\item[b.]

There shouldn't be 3 semicolons in the header
\begin{jjjlisting}
\begin{lstlisting}[basicstyle=\scriptsize]
 for (int k = 0; k < 12 ; k--)
     System.out.println(k);
\end{lstlisting}
\end{jjjlisting}

\end{enumerate}


\item  Identify those statements that result in infinite loops:
\begin{enumerate}

\item[a.]

Infinite loop because k is never incremented.


\item[b.]

Infinite loop because k is always odd and thus never equal to 100.

\end{enumerate}

\item  Your sister is learning to count by fours. Write
a {\tt for} loop that prints the following sequence of numbers: 1, 5, 9, 13,
17, 21, 25.

\begin{jjjlisting}
\begin{lstlisting}[basicstyle=\scriptsize]
for (int k = 1; k <= 25; k = k+4)
    System.out.print(k + " ");
\end{lstlisting}
\end{jjjlisting}

\item  What value will {\it j} have when the following loop terminates?
{\it Answer}: {\it j} will be undefined when the loop terminates. It
is a local variable whose scope is limited to the loop body.

\begin{jjjlisting}
\begin{lstlisting}[basicstyle=\scriptsize]
for (int i = 0; i < 10; i++)
{
   int j;
   j = j + 1;
}
\end{lstlisting}
\end{jjjlisting}


\item  Write a nested {\tt for} loop to print the following geometric pattern:

\begin{jjjlisting}
\begin{lstlisting}[basicstyle=\scriptsize]
#
# #
# # #
# # # #
# # # # #

for (int row = 1; row <= 5; row++)  {    // For each row
  for (int col = 1; col <= row; col++) // Columns per row
    System.out.print('#');
  System.out.println();                // New line
} // row
\end{lstlisting}
\end{jjjlisting}

\item  Identify the syntax error in the following {\tt while} structures:
\begin{enumerate}\baselineskip=10.5pt

\item[a.]

\begin{jjjlisting}
\begin{lstlisting}[basicstyle=\scriptsize]
int k = 5;
while (k < 100) {
    System.out.println(k);
    k++                     << Missing semicolon
}
\end{lstlisting}
\end{jjjlisting}

\item[b.]

\begin{jjjlisting}
\begin{lstlisting}[basicstyle=\scriptsize]
int k = 0;
while (k < 12;) {           << Extra semicolon
    System.out.println(k);
    k++;
}
\end{lstlisting}
\end{jjjlisting}

\end{enumerate}
%%\baselineskip=11pt


\item  Determine the output and/or identify the error in each of the following
{\tt while} structures.
\begin{enumerate}

\item[a.]

\begin{jjjlisting}
\begin{lstlisting}[basicstyle=\scriptsize]
int k = 0;
while (k < 100)
  System.out.println(k);  << Missing updater in loop body
\end{lstlisting}
\end{jjjlisting}
Output: infinite loop prints 0 0 0 0 0...


\item[b.]

\begin{jjjlisting}
\begin{lstlisting}[basicstyle=\scriptsize]
while (k < 100) {           << Missing initializer
  System.out.println(k);
  k++;
}
\end{lstlisting}
\end{jjjlisting}
Output: unpredictable since k's initial value is not known

\end{enumerate}


\item  Your younger sister is now learning how to count by sixes.
Write a {\tt while} loop that prints the following sequence of numbers: 0,
6, 12, 18, 24, 30, 36.

\begin{jjjlisting}
\begin{lstlisting}[basicstyle=\scriptsize]
   int k = 0;                 // Initializer
   while (k <= 36) {          // Loop-entry condition
       System.out.println(k);
       k += 6;                  // Updater
   }
\end{lstlisting}
\end{jjjlisting}


\item  If {\it N} is even, divide it by 2. If {\it N} is odd, subtract 1 and
then divide it by 2. This will generate a sequence that is guaranteed
to terminate at 0.  For example, if {\it N} is initially 15, then you
get the sequence 15, 7, 3, 1, 0. Write a method that implements this
sequence using a {\tt while} statement.

\begin{jjjlisting}
\begin{lstlisting}[basicstyle=\scriptsize]
public static void sub1Div2(int N) {
    while(N != 0) {
        System.out.print(N + " ");
        if (N % 2 == 0)
            N = N / 2;
        else
            N = (N - 1) / 2;
    }
    System.out.println( N );
} // sub1Div2()
\end{lstlisting}
\end{jjjlisting}

\item  Identify the syntax error in the following {\tt do-while} structures:
\begin{enumerate}

\item[a.]

\begin{jjjlisting}
\begin{lstlisting}[basicstyle=\scriptsize]
int k = 0;
do while (k < 100) << Misplaced condition
{
     System.out.println(k);
     k++;
}                    << Belongs here
\end{lstlisting}
\end{jjjlisting}

\item[b.]

\begin{jjjlisting}
\begin{lstlisting}[basicstyle=\scriptsize]
int k = 0;
do {
    System.out.println(k);
    k++;
} while (k < 12) << Missing semicolon
\end{lstlisting}
\end{jjjlisting}

\end{enumerate}


\item  Your sister has moved on to counting by sevens. Write a {\tt do-while}
loop that prints the following sequence of numbers: 1, 8, 15, 22, 29,
36, 43.

\begin{jjjlisting}
\begin{lstlisting}[basicstyle=\scriptsize]
n = 1;                           // Initializer
do {
    System.out.print(n + " ");
    n += 7;                      // Updater
} while (n <= 43);                // Loop entry condition
\end{lstlisting}
\end{jjjlisting}

\item  Write a method to input and validate pizza sales.
\begin{jjjlisting}
\begin{lstlisting}[basicstyle=\scriptsize]
public int getAndValidatePizzaPrice() {// Uses KeyboardReader
  int pizza = 0;
  do {
    reader.prompt("Input a pizza price (8, 10, or 15) ");
    reader.prompt("or 99 to end the list >> ");
    pizza = reader.getKeyboardInteger();
    if ((pizza != 99) && (pizza != 8) && (pizza != 10) && 
                       (pizza != 15))
      System.out.println("Error: you've entered an "
       + "invalid pizza price\n");    // Error input
    else                               // OK input
        System.out.println("You input " + pizza + "\n");                 
  } while ((pizza != 99) && (pizza != 8) && 
                   (pizza != 10) && (pizza != 15));
  return pizza;
} // getAndValidatePizzaPrice()
\end{lstlisting}
\end{jjjlisting}

\item  Write a method to input and validate pizza sales using the numbers
1, 2, and 3 to represent pizzas at different price levels.

\begin{jjjlisting}
\begin{lstlisting}[basicstyle=\scriptsize]
public int getAndValidatePizzaPrice() { // Uses KeyboardReader
  int pizza = 0;
  do {
    reader.prompt("Input a 1,2 or 3 to indicate pizza" 
              + "price ( 1($8), 2($10), or 3($15) ) ");
    reader.prompt("or 0 to end the list >> ");
    pizza = reader.getKeyboardInteger();
    if ((pizza < 0) || (pizza > 3))  // Error check
        System.out.println("Error: you've entered an " 
                                + "invalid value\n");  
    else                                // OK input
        System.out.println("You input " + pizza + "\n"); 
  } while ( (pizza < 0) || (pizza > 3) );
  if (pizza == 1)
    return 8;
  else if (pizza == 2)
    return 10;
  else if (pizza == 3)
    return 15;
  else
    return 0;
} // getAndValidatePizzaPrice()
\end{lstlisting}
\end{jjjlisting}
 %$

\item  For each of the following problems, decide whether
a counting loop structure, a {\tt while} structure, or a
{\tt do-while} structure should be used, and write a
pseudocode algorithm.

\begin{ANSBL}
\item  Printing the names of all the visitors to
a Web site could use a counting loop because the
exact number of visitors is known.

\begin{jjjlisting}
\begin{lstlisting}[basicstyle=\scriptsize]
for each name in the visitor's log
    print the name
\end{lstlisting}
\end{jjjlisting}


\item  Validating that a user has entered a positive
number requires a {\tt do-while} structure in which
you repeatedly read a number and validate it.

\begin{jjjlisting}
\begin{lstlisting}[basicstyle=\scriptsize]
do
    read a number
    if number is invalid, print error message
while number is invalid
\end{lstlisting}
\end{jjjlisting}

\item  Changing all the backslashes ($\backslash$) in a
Windows Web page address, to the slashes (/) used in
a Unix Web page address.

\begin{jjjlisting}
\begin{lstlisting}[basicstyle=\scriptsize]
for each character in the Web page address
    if it is a backslash replace it with slash
\end{lstlisting}
\end{jjjlisting}


\item  Finding the largest in a list of  numbers
requires a {\tt while} loop to guard against
an empty list.

\begin{jjjlisting}
\begin{lstlisting}[basicstyle=\scriptsize]
initialize maxMPG to smallest possible number
while there are more cars in the database
    if current car's MPG is greater than maxMPG
        replace maxMPG with it
\end{lstlisting}
\end{jjjlisting}


\end{ANSBL}

\item  Identify any errors in the following switch structures (if
there is no error, specify the output):

\begin{enumerate}
\item[a.]

\begin{jjjlisting}
\begin{lstlisting}[basicstyle=\scriptsize]
   int k = 0;
   switch (k)  // Syntax error: missing braces
   case 0:
       System.out.println("zero");
       break;
   case 1:
       System.out.println("one");
       break;
   default:
       System.out.println("default");
       break;
\end{lstlisting}
\end{jjjlisting}

\item[b.]

\begin{jjjlisting}
\begin{lstlisting}[basicstyle=\scriptsize]
   int k = 0;
   switch (k + 1) 
   {   
       case 0:
           System.out.println("zero");
           break;
       case 1:
           System.out.println("one"); // Output "one"
           break;
       default:
           System.out.println("default");
           break;
   }
\end{lstlisting}
\end{jjjlisting}

\item[c.]

\begin{jjjlisting}
\begin{lstlisting}[basicstyle=\scriptsize]
   int k = 6;
   switch (k / 3.0) // Syntax error: not an integral value
   {   
       case 2:
           System.out.println("zero");
           break;
       case 3:
           System.out.println("one");
           break;
       default:
           System.out.println("default");
           break;
   }
\end{lstlisting}
\end{jjjlisting}

\end{enumerate}

\item  A {\tt switch} statement to print ice cream flavors:
\verb| |

\begin{jjjlisting}
\begin{lstlisting}[basicstyle=\scriptsize]
switch (flavor) 
{   
    case 1:
        System.out.println("Vanilla");
        break;
    case 2:
        System.out.println("Chocolate");
        break;
    case 3:
        System.out.println("Strawberry");
        break;
    default:
        System.out.println("Error");
}
\end{lstlisting}
\end{jjjlisting}

\item  \verb| |

\begin{jjjlisting}
\begin{lstlisting}[basicstyle=\scriptsize]
public final int VANILLA = 0,
                 CHOCOLATE = 1,
                 STRAWBERRY = 2;
switch (flavor) 
{   
    case VANILLA:
        System.out.println("Vanilla");
        break;
    case CHOCOLATE:
        System.out.println("Chocolate");
        break;
    case STRAWBERRY:
        System.out.println("Strawberry");
        break;
    default:
        System.out.println("Error");
}
\end{lstlisting}
\end{jjjlisting}

\item  Identify the pre- and postconditions on {\it j} and {\it k} where indicated in
the following code segment:

\begin{jjjlisting}
\begin{lstlisting}[basicstyle=\scriptsize]
int j = 0; k = 5;
do {
    if (k % 5 == 0) {
                      // Precondition: j <= k
        j += k;
        k--;
    }
    else k *= k;
} while (j <= k);
                      // Postcondition: j > k
\end{lstlisting}
\end{jjjlisting}


\item  Identify the pre- and postconditions for the following method, which
computes $x^n$ for $n >= 0$.

\mbox{ }
\begin{jjjlisting}
\begin{lstlisting}[basicstyle=\scriptsize]
 // Precondition: N >= 0
 // Postcondition: power(x,n) == x to the n
 public double power(double x, int n ) {
     double pow = 1;
     for (int k = 1; k <= n; k++)
         pow = pow * x;
     return pow;
 } // power()
\end{lstlisting}
\end{jjjlisting}
\end{ANS}

\secEXRHtwoleft{Exercises}

\begin{EXRtwo}

\item  Explain the difference between the following pairs of terms:
\marginnote{\vspace{12pt}\raggedright{\bf Note:} For programming exercises, {\bf first} draw 
a UML class diagram describing all classes and
their inheritance relationships and/or associations.}
\begin{EXRtwoLL}
\item  {\it Counting loop} and {\it conditional loop}.
\item  {\it For statement} and {\it while statement}.
\item  {\it While statement} and {\it do-while statement}.
\item  {\it Zero indexing} and {\it unit indexing}.
\item  {\it Sentinel bound} and {\it limit bound}.
\item  {\it Counting bound} and {\it flag bound}.
\item  {\it Loop initializer} and {\it updater}.
\item  {\it Named constant} and {\it literal}.
\item  {\it Compound statement} and {\it null statement}.
\end{EXRtwoLL}

\item  Fill in the blank.
\begin{EXRtwoLL}\baselineskip=14pt
\item  The process of reading a data item before entering a loop is
known as a  \rule{30pt}{0.5pt}\,.
\item  A loop that does nothing except iterate is an example of \rule{40pt}{0.5pt}\,.
\item  A loop that contains no body is an example of a \rule{40pt}{0.5pt} statement.
\item  A loop whose entry condition is stated as $( k < 100\; ||\; k >= 0)$
would be an example of an \rule{40pt}{0.5pt} loop.
\item  A loop that should iterate until the user types in a special value
should use a  \rule{40pt}{0.5pt} bound.
\item  A loop that should iterate until its variable goes from 5 to 100
should use a  \rule{40pt}{0.5pt} bound.
\item  A loop that should iterate until the difference between two
values is less than 0.005 is an example of a \rule{40pt}{0.5pt} bound.
\end{EXRtwoLL}\baselineskip=11pt


\item  Identify the syntax errors in each of the following:
\begin{EXRtwoLL}
\item

for (int k = 0; k < 100; k++)
    System.out.println(k)


\item

for (int k = 0; k < 100; k++);
    System.out.println(k);


\item

int k = 0
while  k < 100
{
   System.out.println(k);  k++;
}


\item

int k = 0;
do
{
   System.out.println(k);  k++;
}
while  k < 100 ;

\end{EXRtwoLL}

\item  Determine the output and/or identify the error in
each of the following code segments:
\begin{EXRtwoLL}
\item

for (int k = 1; k == 100; k += 2)
    System.out.println(k);


\item

int k = 0;
while (k < 100)
    System.out.println(k);
    k++;


\item

for (int k = 0; k < 100; k++) ;
    System.out.println(k);

\end{EXRtwoLL}

\item  Write pseudocode algorithms for the following activities,
paying particular attention to the {\it initializer},
{\it updater}, and {\it boundary condition} in each case.
\begin{EXRtwoLL}
\item  {\tt a softball game}
\item  {\tt a five-question quiz}
\item  {\tt looking up a name in the phone book}
\end{EXRtwoLL}

\item  Identify the pre- and postconditions for each of the
 statements that follow. Assume that all variables are {\tt int}
and have been properly declared.
\begin{EXRtwoLL}
\item

 int result = x / y;


\item

 int result = x % y;


\item

 int x = 95;
do
    x /= 2;
while(x >= 0);

\end{EXRtwoLL}


\item  Write three different loops---a {\tt for} loop, a {\tt while} loop, and a {\tt do-while}
loop---to print all the multiples of 10, including 0, up to and
including 1,000.

\item  Write three different loops---a {\tt for} loop, a {\tt while} loop, and a {\tt do-while}
loop---to print the following sequence of numbers: 45, 36, 27, 18, 9, 0,
$-9$, $-18$, $-27$, $-36$, $-45$.


\item  Write three different loops---a {\tt for} loop, a while {\tt loop}, and a {\tt do-while}
loop---to print the following ski-jump design:

\begin{jjjlisting}
\begin{lstlisting}
#
# #
# # #
# # # #
# # # # #
# # # # # #
# # # # # # #
\end{lstlisting}
\end{jjjlisting}


\item  The Straight Downhill Ski Lodge in Gravel Crest, Vermont, gets
lots of college students on breaks. The lodge likes to keep track of
repeat visitors. Straight Downhill's database includes an integer
variable, {\it visit}, which gives the number of times a guest has
stayed at the lodge (1 or more). Write the pseudocode to catch those
visitors who have stayed at the lodge at least twice and to send them
a special promotional package (pseudocode = send promo).  ({\it Note:}
The largest number of stays recorded is eight. The number nine is used
as an end-of-data flag.)

\item  Modify your pseudocode in the previous exercise. In addition to
every guest who has stayed at least twice at the lodge receiving a
promotional package, any guest with three or more stays should also get a
\$40 coupon good for lodging, lifts, or food.

\item  Write a method that is passed a single parameter, {\it N}, and displays
all the even numbers from 1 to {\it N}.

\item  Write a method that is passed a single parameter, {\it N}, that prints
all the odd numbers from 1 to {\it N}.

\item  Write a method that is passed a single parameter, {\it N}, that prints
all the numbers divisible by 10 from {\it N} down to 1.

\item  Write a method that is passed two parameters---a {\tt char} {\it Ch} and
an {\tt int} {\it N}---and prints a string of {\it N Chs}.

\item  Write a method that uses a nested {\tt for} loop to print the
following multiplication table:

\begin{jjjlisting}
\begin{lstlisting}
   1  2  3  4  5  6  7  8  9
1  1
2  2  4
3  3  6  9
4  4  8 12 16
5  5 10 15 20 25
6  6 12 18 24 30 36
7  7 14 21 28 35 42 48
8  8 16 24 32 40 48 56 64
9  9 18 27 36 45 54 63 72 81
\end{lstlisting}
\end{jjjlisting}


\item  Write a method that uses nested {\tt for} loops to print the 
patterns that follow. Your method should use the following statement
to print the patterns: \verb|System.out.print('#')|.

\begin{jjjlisting}
\begin{lstlisting}[basicstyle=\scriptsize]
# # # # # # # #     # # # # # # # #   # # # # # # # #   # # # # # # # #
  # # # # # # #     # # # # # # #       #         #                 #
    # # # # # #     # # # # # #           #     #                 #
      # # # # #     # # # # #               # #                 #
        # # # #     # # # #                 # #               #
          # # #     # # #                 #     #           #
            # #     # #                 #         #       #
              #     #                 # # # # # # # #   # # # # # # # #
\end{lstlisting}
\end{jjjlisting}

\item  Write a program that asks the user for the number
of rows and the number of columns in a box of asterisks. Then
use nested loops to generate the box.

\item  Write a Java application that lets the user input a sequence
of consecutive numbers. In other words, the program should let the
user keep entering numbers as long as the current number is one
greater than the previous number.

\item  Write a Java application that lets the user input
a sequence of integers terminated by any negative value. The
program should then report the largest and smallest values
that were entered.

\item How many guesses does it take to guess a secret number between 1
and {\it N}? For example, I'm thinking of a number between 1 and
100. I'll tell you whether your guess is too high or too
low. Obviously, an intelligent first guess would be 50. If that's too
low, an intelligent second guess would be 75.  And so on. If we
continue to divide the range in half, we'll eventually get down to one
number. Because you can divide 100 seven times (50, 25, 12, 6, 3, 1,
0), it will take at most seven guesses to guess a number between 1 and
100. Write a Java Swing program that lets the user input a positive integer,
{\it N}, and then reports how many guesses it would take to guess a
number between 1 and {\it N}.

\item  Suppose you determine that the fire extinguisher in your
kitchen loses {\it X} percent of its foam every day. How long before
it drops below a certain threshold ({\it Y} percent), at which point
it is no longer serviceable?  Write a Java Swing program that lets the user
input the values {\it X} and {\it Y} and then reports how many weeks
the fire extinguisher will last.

\item  Leibnitz's method for computing $\pi$ is based on the following
convergent series:

\begin{displaymath}
  \frac{\pi}{4} \; = 1 \; - \; \frac{1}{3} \; + \; \frac{1}{5} \; - \; \frac{1}{7} + \; \cdots
\end{displaymath}

\noindent How many iterations does it take to compute $\pi$ to 
a value between 3.141 and 3.142 using this series? Write a Java
program to find out.

\item  Newton's method for calculating the square root of {\it N} starts
by making a (nonzero) guess at the square root. It then
uses the original guess to calculate a new guess, according to
the following formula:

\begin{jjjlisting}
\begin{lstlisting}
guess = (( N / guess) + guess) / 2;
\end{lstlisting}
\end{jjjlisting}

\noindent No matter how wild the original guess is, if we repeat
this calculation, the algorithm will eventually find the square
root.  Write a square root method based on this algorithm. Then write
a program to determine how many guesses are required to find the
square roots of different numbers. Uses {\tt Math.sqrt()} to determine
when to terminate the guessing.

\item  Your employer is developing encryption software and wants you
to develop a Java Swing Program that will display all of the primes less than
{\it N}, where {\it N} is a number to be entered by the user. In
addition to displaying the primes themselves, provide a count of how
many there are. 

\item  Your little sister asks you to help her with her
multiplication and you decide to write a Java application that tests
her skills. The program will let her input a starting number, such as
5. It will generate multiplication problems ranging from from $5
\times 1$ to $5 \times 12$. For each problem she will be prompted to
enter the correct answer. The program should check her answer and
should not let her advance to the next question until the correct
answer is given to the current question.

\item  Write an application that prompts the user for
four values and draws corresponding bar graphs using an
ASCII character. For example, if the user entered
15, 12, 9, and 4, the program would draw

\begin{jjjlisting}
\begin{lstlisting}
******************
************
*********
****
\end{lstlisting}
\end{jjjlisting}

\item  Revise the application in the previous problem so that
the bar charts are displayed vertically. For example, if the
user inputs 5, 2, 3, and 4, the program should display

\begin{jjjlisting}
\begin{lstlisting}
 **
 **       **
 **    ** **
 ** ** ** **
 ** ** ** **
-------------
\end{lstlisting}
\end{jjjlisting}

\item  The Fibonacci\index{Fibonacci sequence} sequence
(named after the Italian mathematician Leo\-nardo of Pisa, ca. 1200)
consists of the numbers $0,1,1,2,3,5,8,13,\dots$ in which each number
(except for the first two) is the sum of the two preceding
numbers. Write a method {\tt fibonacci(N)} that prints the first {\it N}
Fibonacci numbers.

\item  The Nuclear Regulatory Agency wants you to write a
program that will help determine how long certain radioactive
substances will take to decay. The program should let the user input
two values: a string giving the substance's name and its half-life in
years. (A substance's half-life is the number of years required for the
disintegration of half of its atoms.) The program should report how
many years it will take before there is less than 2 percent of the
original number of atoms remaining.

\item  Modify the {\tt CarLoan} program so that it calculates a
user's car payments for loans of different interest rates and
different loan periods. Let the user input the amount of the loan.
Have the program output a table of monthly payment schedules.

\end{EXRtwo}

\vspace*{6pt}
\noindent{\it\normalsize The next chapter also contains a number of loop exercises.}
