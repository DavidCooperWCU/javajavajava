\setcounter{table}{0}
\setcounter{figure}{0}
\renewcommand{\thetable}{\mbox{F.\arabic{table}}}%
\renewcommand{\thefigure}{\mbox{F--\arabic{figure}}}%

\chapter{Java Inner Classes}

\markboth{{\color{cyan}APPENDIX\,F\,\,$\bullet$\,\,}Java Inner Classes}
{{\color{cyan}APPENDIX\,F\,\,$\bullet$\,\,}Java Inner Classes}

\noindent This appendix describes basic features of some advanced
elements of the Java language.  As for many language features, there
are details and subtleties involved in using these features that are
not covered here.  For further details, you should consult Sun's online
references or other references for a more comprehensive description.

\section*{What Are Inner Classes?}
\label{what-are-inner-classes}
\noindent {\it Inner classes} were introduced in Java 1.1. This features lets
you define a class as part of another class, just as fields and
methods are defined within classes.   Inner classes can be used to
support the work of the class in which they are contained.

Java defines four types of inner classes.   A {\it nested top-level}
class or interface is a {\it static} member of an enclosing top-level
class or interface.  Such classes are considered top-level classes
by Java.

A {\it member class} is a nonstatic inner class.  It is not a
top-level class.  As a full-fledged member of its containing class, a
member class can refer to the fields and methods of the containing
class, even the {\tt private} fields and methods.  Just as you would
expect for the other instance fields and methods of a class, all
instances of a member class are associated with an instance of the
enclosing class.

A {\it local class} is an inner class that's defined within a block of
Java code, such as within a method or within the body of a loop.  Local
classes have local scope---they can only be used within the block in
which they are defined.  Local classes can refer to the methods and
variables of their enclosing classes.   They are used mostly to implement
{\it adapters}, which are used to handle events.

When Java compiles a file containing a named inner class, it creates
separate class files for them with names that include the nesting
class as a qualifier.  For example, if you define an inner class named
{\tt Metric} inside a top-level class named {\tt Converter}, the
compiler will create a class file named {\tt Converter\$Metric.class}
for the inner class.   If you wanted to access the inner class from
some other class (besides {\tt Converter}), you would use a qualified
name: {\tt Converter.Metric}.


An {\it anonymous class} is a local class whose definition and use are
combined into a single expression.  Rather than defining the class in
one statement and using it in another, both operations are combined
into a single expression.  Anonymous classes are intended for one-time
use.  Therefore, they don't contain constructors.  Their bytecode files
are given names like {\tt ConverterFrame\$1.class}.

\section*{Nested Top-Level Versus Member Classes}
\label{nested-top-level-versus-member-classes}
\noindent The {\tt Converter} class (Figure~F--1) shows the
differences between a nested top-level class and a member class.   The
program is a somewhat contrived example that performs various kinds of
\begin{figure}[h!]
\jjjprogstart
\begin{jjjlistingleft}[33pc]{-7pc}
\begin{lstlisting}
public class Converter {
  private static final double INCH_PER_METER = 39.37;
  private final double LBS_PER_KG = 2.2;

  public static class Distance { // Nested Top-level class
    public double metersToInches(double meters) {
      return meters * INCH_PER_METER;
    }
  } // Distance

  public class Weight {          // Member class
    public double kgsToPounds(double kg) {
      return kg * LBS_PER_KG;
    }
  } //Weight
} //Converter

public class ConverterUser {
  public static void main(String args[]) {
    Converter.Distance distance = new Converter.Distance();
    Converter converter = new Converter();
    Converter.Weight weight = converter.new Weight();
    System.out.println( "5 m = " + distance.metersToInches(5) + " in");
    System.out.println( "5 kg = " + weight.kgsToPounds(5) + " lbs");
  }
} // ConverterUser
\end{lstlisting}
\end{jjjlistingleft}
\jjjprogstop{A Java application containing a top-level nested class. }
{fig-converter}
\end{figure}
metric conversions.  The outer {\tt Converter} class serves as a
container for the inner classes, {\tt Distance} and {\tt Weight},
which perform specific conversions.


The {\tt Distance} class is declared {\tt static}, so it is a
top-level class.  It is contained in the {\tt Converter} class
itself.  Note the syntax used in {\tt ConverterUser.main()} to create
an instance of the {\tt Distance} class:

\begin{jjjlisting}
\begin{lstlisting}
Converter.Distance distance = new Converter.Distance();
\end{lstlisting}
\end{jjjlisting}

\noindent A fully qualified name is used to refer to the {\tt static}
inner class via its containing class.


The {\tt Weight} class is not declared {\tt static}.  It is, therefore,
associated with {\it instances} of the {\tt Converter} class.  Note
the syntax used to create an instance of the {\tt Weight} class:

\begin{jjjlisting}
\begin{lstlisting}
Converter converter = new Converter();
Converter.Weight weight = converter.new Weight();
\end{lstlisting}
\end{jjjlisting}

\noindent Before you can create an instance of {\tt Weight}, you
have to declare an instance of {\tt Converter}. In this example, we
have used two statements to create the {\tt weight} object,
which requires using the temporary variable, {\tt converter},
as a reference to the {\tt Converter} object.  We could also
have done this with a single statement by using the following
syntax:

\begin{jjjlisting}
\begin{lstlisting}
Converter.Weight weight = new Converter().new Weight();
\end{lstlisting}
\end{jjjlisting}

\noindent Note that in either case
the qualified name {\tt Converter.Weight} must be used to access the
inner class from the {\tt ConverterUser} class.

There are a couple of other noteworthy features in this example.
First, an inner top-level class is really just a programming
convenience.   It behaves just like any other top-level class in Java.
One restriction on top-level inner classes is that they can only be
contained within other top-level classes, although they can be nested
one within the other.  For example, we could nest additional converter
classes within the {\tt Distance} class.   Java provides special syntax
for referring to such nested classes.

Unlike a top-level class, a member class is nested within an instance
of its containing class.  Because of this, it can refer to instance
variables (\verb|LBS_PER_KG|) and instance methods of its containing
class, even to those declared {\tt private}. By contrast, a top-level
inner class can only refer to class variables (\verb|INCH_PER_METER|)---that 
is, to variables that are declared {\tt static}. So you would
use a member class if it were necessary to refer to instances of the
containing class.

There are many other subtle points associated with member classes,
including special language syntax that can be used to refer to
nested member classes and rules that govern inheritance and scope
of member classes.  For these details you should consult the
{\it Java Language Specification}, which can be accessed online
at

\begin{jjjlisting}
\begin{lstlisting}[commentstyle=\color{black}]
http://java.sun.com/docs/books/jls/html/index.html
\end{lstlisting}
\end{jjjlisting}


\section*{Local and Anonymous Inner Classes}
\label{localand-anonymous-inner-classes}

In this next example, {\tt ConverterFrame}, a local class is used to
create an {\tt ActionEvent} handler for the application's two buttons
(Fig.~F--2).

\begin{figure}[h!]
\jjjprogstart
\begin{jjjlistingleft}[36.5pc]{-10.5pc}
\begin{lstlisting}
import javax.swing.*;
import java.awt.*;
import java.awt.event.*;

public class ConverterFrame extends JFrame {
  private Converter converter = new Converter();       // Reference to app
  private JTextField inField = new JTextField(8);
  private JTextField outField = new JTextField(8);
  private JButton metersToInch;
  private JButton kgsToLbs;

  public ConverterFrame() {
    metersToInch = createJButton("Meters To Inches");
    kgsToLbs = createJButton("Kilos To Pounds");
    getContentPane().setLayout( new FlowLayout() );
    getContentPane().add(inField);
    getContentPane().add(outField);
    getContentPane().add(metersToInch);
    getContentPane().add(kgsToLbs);
  } // ConverterFram()

  private JButton createJButton(String s) { // A method to create a JButton
    JButton jbutton = new JButton(s);
    class ButtonListener implements ActionListener { // Local class

      public void actionPerformed(ActionEvent e) {
        double inValue = Double.valueOf(inField.getText()).doubleValue();
        JButton button = (JButton) e.getSource();
        if (button.getText().equals("Meters To Inches"))
          outField.setText(""+ converter.new Distance().metersToInches(inValue));
        else
          outField.setText(""+ converter.new Weight().kgsToPounds(inValue));
      } // actionPerformed()
    } // ButtonListener

    ActionListener listener = new ButtonListener(); // Create a listener
    jbutton.addActionListener(listener);      // Register buttons with listener
    return jbutton;
  } // createJButton()

  public static void main(String args[]) {
    ConverterFrame frame = new ConverterFrame();
    frame.setSize(200,200);
    frame.setVisible(true);
  } // main()
} // ConverterFrame
\end{lstlisting}
\end{jjjlistingleft}
\jjjprogstop{The use of a local class as an {\tt ActionListener}
  adapter.} {fig-converterframe}
\end{figure}
\noindent As we have seen, Java's event-handling model uses predefined
interfaces, such as the {\tt ActionListener} interface, to handle
events.  When a separate class is defined to implement an interface, it
is sometimes called an {\it adapter} class.  Rather than defining
adapter classes as top-level classes, it is often more convenient to
define them as local or anonymous classes.   


The key feature of the {\tt ConverterFrame} program is the {\tt
createJButton()} method.  This method is used instead of the {\tt
JButton()} constructor to create buttons and to create action
listeners for the buttons. It takes a single {\tt String} parameter
for the button's label.  It begins by instantiating a new {\tt
JButton}, a reference to which is passed back as the method's {\tt
return} value.  After creating an instance button, a local inner class
named {\tt ButtonListener} is defined.

The local class merely implements the {\tt ActionListener} interface
by defining the {\tt actionPerformed} method.  Note how {\tt
actionPerformed()} uses the containing class's {\tt converter}
variable to acquire access to the {\tt meters\-ToInches()} and {\tt
kgsToPounds()} methods, which are inner class methods of the {\tt
Converter} class (Fig.~F--1). A local class can use
instance variables, such as {\tt converter}, that are defined in its
containing class.

After defining the local inner class, the {\tt createJButton()} method
creates an instance of the class ({\tt listener}) and registers it as
the button's action listener.  When a separate object is created to
serve as listener in this way, it is called an {\it adapter}. It
implements a listener interface and thereby serves as adapter between
the event and the object that generated the event.  Any action events
that occur on any buttons created with this method will be handled by
this adapter.   In other words, for any buttons created by the {\tt
createJButton()} method, a listener object is created and assigned as
the button's event listener.   By using local classes, the code for
doing this is much more compact and efficient.

Local classes have some important restrictions.  Although an instance
of a local class can use fields and methods defined within the class
itself or inherited from its superclasses, it cannot use local
variables and parameters defined within its scope unless these are
declared {\tt final}. The reason for this restriction is that {\tt
final} variables receive special handling by the Java
compiler.  Because the compiler knows that the variable's value won't
change, it can replace uses of the variable with their values at
compile time.


\pagebreak
\subsection*{Anonymous Inner Classes}
\noindent An anonymous inner class is just a local class without a name.  Instead
of using two separate statements to define and instantiate the local
class, Java provides syntax that let's you do it in one expression.
The following code illustrates how this is done:

\begin{jjjlistingleft}[36pc]{-10pc}
\begin{lstlisting}
private JButton createJButton(String s) { // A method to create a JButton
  JButton jbutton = new JButton(s);

  jbutton.addActionListener( new  ActionListener() {     // Anonymous class
    public void actionPerformed(ActionEvent e) {
      double inValue = Double.valueOf(inField.getText()).doubleValue();
      JButton button = (JButton) e.getSource();
      if (button.getLabel().equals("Meters To Inches"))
        outField.setText("" + converter.new Distance().metersToInches(inValue));
      else
        outField.setText("" + converter.new Weight().kgsToPounds(inValue));
      } // actionPerformed()
  }); // ActionListener
  return jbutton;
} // createJButton()
\end{lstlisting}
\end{jjjlistingleft}

\noindent Note that the body of the class definition is put right after
the {\tt new} operator.  The result is that we still create an instance
of the adapter object, but we define it on the fly.  If the name
following {\tt new} is a class name, Java will define the anonymous
class as a subclass of the named class.  If the name following {\tt
new} is an interface, the anonymous class will implement the
interface.  In this example, the anonymous class is an implementation
of the {\tt ActionListener} interface.

Local and anonymous classes provide an elegant and convenient way to
implement adapter classes that are intended to be used once and have
relatively short and simple implementations.  The choice of local
versus anonymous should largely depend on whether you need more than
one instance of the class.  If so, or if it's important that the class
have a name for some other reason (readability), then you should use a
local class.  Otherwise, use an anonymous class.  As in all design
decisions of this nature, you should use whichever approach or style
makes your code more readable and more understandable.
%
