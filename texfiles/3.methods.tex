%%%  Chapter 3: Methods: Communicating with Objects
%%%  3rd Edition

\setcounter{chapter}{2}
\setcounter{SSTUDYcount}{1}
\chapter{Methods: Communicating with Objects}
\label{chapter-methods}
\label{pg-chapter-methods}

\CObegin
\secCOBH{Objectives}
\noindent After studying this chapter, you will

\begin{COBL}
\item  Understand the role that methods play in an
  object-oriented program.
\item  Know how to use parameters and arguments to pass data to an object.
\item  Understand how constructor methods are used to instantiate objects.

\item  Know the difference between passing a value and passing a reference
   to an object.
\item  Be able to design your own methods.
\item  Know how to use the {\tt if-else} and {\tt while} control structures.
\end{COBL}

\secCOLH{Outline}
\begin{COL}
\item Introduction
\item Passing Information to an Object
\item Constructor Methods
\item Retrieving Information from an Object
\item Passing a Value and Passing a Reference
\item Flow of Control: Control Structures
\item Testing an Improved {\tt OneRowNim}
\item[] {{\color{cyan}Special Topic:} Intelligent Agents}
\item From the Java Library: {\tt java.lang.Object}
\item Object-Oriented Design: Inheritance and Polymorphism
\item Drawing Lines and Defining Graphical Methods (Optional)
\par\small\item[] Chapter Summary
\par\small\item[] Solutions to Self-Study Exercises
\par\small\item[] Exercises
\end{COL}
\COend

\section{Introduction}

\noindent In this chapter, we take a look at Java methods and parameters.
Methods and parameters are the primary mechanisms for passing
information into and out of an object.  We will once again focus on
the {\tt OneRowNim} simulation that we designed in the previous chapter.
That version was sufficient to introduce us to Java
objects and classes, but it was limited in its ability to
communicate with other objects.

In this chapter, we want to expand {\tt OneRowNim} to make our
simulation more realistic.  We begin by learning how to pass
information to an object.  That will enable us to specify the number of
sticks to remove using a single method.  We then consider special
methods called constructors, which
are used to initialize an object's state when it is created.   We also
learn how to retrieve information from an object.  That will enable us
to request a {\tt OneRowNim} object for several different bits of information.
Then we consider the {\tt if-else} and {\tt while} control structures
which allow us to define more useful methods and write more realistic
test programs.

\section{Passing Information to an Object}
\noindent  One convention of object-oriented programming is to
provide {\tt public} methods to {\it set} and {\it get} the values of
some of its {\tt private} instance variables. Methods that set or
modify an object's instance variables are called {\bf mutator methods}.
Methods that get or retrieve the value of an instance variable are
called {\bf accessor methods}.

\JavaTIP{EFFECTIVE DESIGN}{Accessor and Mutator
Methods.}{An {\bf accessor method}\index{accessor method} is a public
method used to {\it get} the value of an object's instance
variable. Such methods are often named {\it getVarName()} where {\it
VarName} is the name of the variable that's being accessed.  A {\bf
mutator method}\index{mutator method} is a public method used to
modify the value of one or more instance variables.  The special type
of mutator method that {\it sets} or assigns a variable a specified
value is often called {\it setVarName()}.}


\noindent It is up to the designer of the class to determine
which private
variables require accessor and mutator methods.  If you were designing
a {\tt BankAccount} class, you might want a public {\tt
getAccountNumber()} method, so that clients could retrieve information
about their bank accounts, but you would probably not want a public
{\tt getAccountPassword()} method or a public {\tt setAccountBalance()}
method.

In the remainder of this section, we will be concerned
with mutator methods. We defined three mutator methods named {\tt takeOne()},
{\tt takeTwo()}, and {\tt takeThree} as part of the {\tt OneRowNim} class
in the previous chapter.  All three of these method change the values of the
instance variables {\tt nSticks} and {\tt player}.  All three methods have
very similar bodies.  The definition of the {\tt takeOne()} is:

%%%\newpage
\begin{jjjlisting}
\begin{lstlisting}
public void takeOne()
{   nSticks = nSticks - 1;
    player = 3 - player;
 }
\end{lstlisting}
\end{jjjlisting}

\noindent The only difference in the bodies of the other two methods is 
that they subtract 2 and 3 from {\tt nSticks} instead of 1.  Instead
of having three, virtually identical methods, It would be a more
efficient design to define a single method where the number to be
subtracted from {\tt nSticks} would be supplied as an argument when
the method is called. In order to be able to handle such an argument,
we must design a new method that uses a {\em parameter} to handle
the argument. 

A {\bf formal parameter}, or more simply, {\bf parameter}, is a
variable used to pass
%\begin{marginalnote}\it Parameter\end{marginalnote}
\marginnote{Formal parameter}
information into a method when the method is invoked.  The {\it type}
and {\it variable name} of the formal parameter must appear in the
{\it formal parameter list} that follows the method's name in the
method header.  The formal parameter is used to hold a value that it
is passed while the method is executing.

\JavaRule{Formal Parameter.}{A {\bf formal
parameter}\index{formal parameter} is a variable that serves as a
storage location for information that is passed to a method.  To
specify a formal parameter, you must provide a type identifier
followed by variable identifier, and you must place this declaration
inside the parentheses that follow the method's name.}

\noindent Consider the following definition for a {\tt takeSticks()} method:

\begin{jjjlisting}
\begin{lstlisting}
public void takeSticks(int num)
{    nSticks = nSticks - num;
     player = 3 - player;
}
\end{lstlisting}
\end{jjjlisting}

\noindent Notice that executing the body of {\tt takeSticks()}
when the parameter {\tt num} stores the value 1 accomplishes precisely
the same task as executing {\tt takeOne()}. If, instead, a value of 2 or
3 is stored in {\tt num}, then calling the method acts like {\tt takeTwo()}
or {\tt takeThree()} respectively.  Thus, using parameters enables us to
design methods that are more general in what they do, which is an important
principle of good program design.

Another example of a mutator method is one in which define a {\it set
method} to allow the starting number of sticks to be set for an
instance of {\tt OneRowNim}. For this, we could define:

\label{meth-setsticks}

\begin{jjjlisting}
\begin{lstlisting}
public void setSticks(int sticks)
{    nSticks = sticks;
} // setSticks()
\end{lstlisting}
\end{jjjlisting}

\noindent As we will see in Section~\ref{sec-constructors}, 
we can also define a constructor method that can be used, when the
game is created, to set the initial value of {\tt nSticks}. It is
often desirable to have more than one method that sets the values of
an objects' instance variables.

If a method uses more than one parameter, use a comma to separate the
individual parameter declarations in the method header.  For example,
if we wanted a method for {\tt OneRowNim} that specified both the
number of sticks for the start of a game and which player takes a turn
first, it could be defined: 

\begin{jjjlisting}
\begin{lstlisting}
public void setGame(int sticks, int starter) 
{   nSticks = sticks; 
    player = starter; 
} // setGame()
\end{lstlisting}
\end{jjjlisting}

\subsubsection*{The Scope of Parameters, Variables, and Methods}

\noindent The bodies of the mutator methods in the previous
section make use of both instance variables and parameters.  It is
important to note that there is a difference in where these two types
of variables can be used. The {\bf scope}\index{scope} of a variable
\marginnote{Scope}
or method refers to where it can be used in a program.

A parameter's scope\index{parameter scope} is limited to the body of
the method in which it is declared. Variables that are declared in the
body of a method have scope which extends from the point where they
are declared to the end of the block of code in which they are
declared. Parameters are local variables which are declared in the
parameter list of a method's header and which have initial values
specified by the arguments in a method call. The scope of a parameter
is the same as the scope of a variable declared at the very beginning
of the body of a method.
\marginnote{Local scope}
Once the flow of execution leaves a method, its parameters and other
local variables cease to exist. The scope of local variables is
referred to as {\bf local scope}\index{local scope}.

\JavaRule{Scope.}{Local variables, that is,
parameters and variables declared in the body of a method, have {\bf
local scope}\index{local scope} which extends from the point at which
they are defined to the end of the block of code in which they are
defined. In particular, the scope of a parameter is the entire body of
the method in which it is declared.}

By contrast, instance variables, class variables, and all methods have
scope that extends throughout the entire class, that is, {\bf class
scope}.  They can be used in the body of any method and in the
expressions that assign initial values to class level variables.
There are two restrictions to remember.  First, instance variables and
instance methods cannot be used in the body of a class method, one
modified with {\tt static}, unless an instance of the class is created
there and then the dot notation of qualified names must be used to
\marginnote{Class scope}
refer to the variable or method.  This is because class methods are
called without reference to a particular instance of the class.  The
{\tt main()} method of the {\tt OneRowNim} class that we defined in
the previous chapter is an example of such a class method.  In that
case, to test the instance methods of {\tt OneRowNim} we first created an 
instance of {\tt OneRowNim} and used it to call its instance methods:

\begin{jjjlisting}
\begin{lstlisting}
OneRowNim game = new OneRowNim(); // Create instance
game.report();   // Call an instance method
\end{lstlisting}
\end{jjjlisting}

\noindent The second restriction involved in class scope is
that one class level variable can be used in the expression that
initializes a second class level variable only if the first is
declared before the second.  There is no similar restriction on
methods.

\JavaRule{Scope.}{Class level variables, that is,
instance variables and class variables have {\bf class
scope}\index{class scope}, which extends throughout the class.
Methods also have class scope.}

Except for the restrictions noted above, methods and class level
variables can be referred to within the same class by their simple
names, with just the method (or variable) name itself,
\marginnote{Simple vs. qualified names}
rather than by their qualified names, with the dot operator.  Thus, in
{\tt OneRowNim}, we can refer to {\tt nSticks} and {\tt report()}
in the bodies of other instance methods. In a class method, such as
{\tt main()}, we would have to create an instance of {\tt OneRowNim}
with a name like {\tt game} and refer to {\tt game.report()}.

\JavaRule{Qualified Names.}{Within the same class, references
to class methods or class variables can be made in terms of simple
names.  Within the bodies of instance methods, references to instance
variables and references to other instance methods can also be made in
terms of simple names.  However, within the bodies of class methods,
qualified\index{qualified name} names, or dot notation, must be
used to refer to instance methods or instance variables just like how
they are referred to in other classes.}

%%%REW FIGURE 3-3 page 119 and related discussion does not seem to me to
%%%REW give precise enough info TO BE VERY HELPFUL
%%%\begin{figure}[tb]
%%%\figa{chptr03/nimlocal.eps}{Class scope extends through a class whereas
%%%local scope is confined to individual methods.
%%%\label{fig-scope}
%%%\label{pg-fig-scope}}
%%%\end{figure}

%%%\vspace{-3pt}
%%%\noindent One way to visualize the difference between
%%%local and class scope is 
%%%\marginnote{Class scope vs.  local scope}
%%%to draw boxes around portions of a program to indicate an identifier's
%%%scope (Fig.~\ref{fig-scope}). For class scope you would draw a box
%%%around the entire class.  An instance variable or method can be used
%%%anywhere within that box.  Similarly, draw boxes around each of the
%%%methods in the class.  The scope of a method's parameters, if it has
%%%any, would be confined within these smaller boxes.  From within a
%%%smaller box you can refer to an instance variable, because the smaller
%%%box is completely contained within the class's box.  But from within
%%%one small box, you cannot refer to anything within one of the other
%%%small boxes.  This is yet another sense in which both a method and a
%%%\marginnote{Encapsulation}
%%%class encapsulate their elements.

\JavaTIP{DEBUGGING TIP}{Scope Error.}{It would be a syntax error
to refer to a method's parameters or other local variables from
outside the method.}

\subsection{Arguments and Parameters}

\noindent The new class definition for {\tt OneRowNim} is
given in Figure~\ref{fig-ornclass2}. Note that now that we have a
\begin{figure}[h!]
\jjjprogstart
\begin{jjjlisting}[26.5pc]
\begin{lstlisting}
public class OneRowNim
{ private int nSticks = 7; // Start with 7 sticks
  private int player = 1;  //Player 1 plays first

  public void takeSticks(int num)
  { nSticks = nSticks - num;
    player = 3 - player;
  }  // takeSticks()

  public void report()
  { System.out.println("Number of sticks left: " + nSticks);
    System.out.println("Next turn by player " + player);
  }   // report()
} //OneRowNim1 class
\end{lstlisting}
\end{jjjlisting}
\jjjprogstop{The {\tt OneRowNim} class definition with {\tt takeSticks()} method.}
{fig-ornclass2}
\end{figure}
single method, {\tt takeSticks()}, that can be used to take away a
variable number of sticks, we have removed the three methods we wrote
in the previous chapter, {\tt takeOne()}, {\tt takeTwo()}, and {\tt
takeThree()}, from {\tt OneRowNim}.  Using a single method, with a
parameter, is clearly a better design.  To see this, just imagine what
we would have to do if we didn't use a parameter and we wanted to be
able to take away four sticks, or five, or more. If we didn't have
parameters, we'd have to write a separate method for each case, which
is clearly a bad idea. Using parameters in this way leads to a more
general useful method and thus is an example of the generality
principle.


Now let's consider how we would create a {\tt OneRowNim} instance and
use the new method in the {\tt main()} method or in a different
class. If we want to have an instance of {\tt OneRowNim} object to
remove 3 sticks on the first move by using the {\tt takeSticks()}
method, we need to pass the {\tt int} value 3 to the method. In order
to effect this action, we would use the following statements:

\begin{jjjlisting}
\begin{lstlisting}
OneRowNim game = new OneRowNim();
game.takeSticks(3);
\end{lstlisting}
\end{jjjlisting}

\noindent Because the definition of {\tt takeSticks()} includes a
single {\tt int} parameter, we must supply a single {\tt int} value
(such as 3), when we invoke it.  When the method is invoked, its
formal parameter ({\tt num}) will be set to the value we supply (3).
The value we supply does not have to be a literal {\tt int} value.  We
can supply any expression or variable that evaluates to an {\tt int}
value. For example: 

\begin{jjjlisting}
\begin{lstlisting}
int val = 7 - 5;
game.takeSticks(val);
\end{lstlisting}
\end{jjjlisting}

\noindent In this case, the value being passed to {\tt takeSticks()} is
2, the value that {\tt val} has at the time the method call is made.

It would be an error to try to pass a value that was not a
{\tt int} to {\tt takeSticks()}. For example, each of the following
invocations of {\tt takeSticks()} results in a syntax error:

\begin{jjjlisting}
\begin{lstlisting}
game.takeSticks();     // no argument is supplied
game.takeSticks("3");  // "3" is a String, not an int
game.takeSticks(int);  // int not is an int value
\end{lstlisting}
\end{jjjlisting}

\noindent As you recall from Chapter 0, the value that is 
passed to a method when it is invoked is called 
%\begin{marginalnote}\it Parameter vs.  argument\end{marginalnote}
\marginnote{Parameter vs. argument}
an argument. Even though the terms argument and
parameter\index{parameter} are sometimes used interchangeably, it will
be useful to observe a distinction.  We will use the term {\it
parameter} to refer to the formal parameter---the variable used to
pass data to a method---that occurs in the method definition.  We use
the term {\it argument} to refer to the actual value that is supplied
when the method is invoked.

\JavaTIP{DEBUGGING TIP}{Type Error.}{It would be a syntax error
to use an argument whose type doesn't match the type of
its corresponding \mbox{parameter.}}

\noindent The distinction between parameter and argument is related
%\begin{marginalnote}\it Defining vs. calling a method\end{marginalnote}
\marginnote{Defining vs. calling a method}
to the difference between {\it defining} a method and {\it invoking} a
method.  Defining a method is a matter of writing a method definition,
such as

\begin{jjjlisting}
\begin{lstlisting}
public void printStr(String s)
{   System.out.println(s);
}
\end{lstlisting}
\end{jjjlisting}

\noindent This definition defines a method that takes a
single {\tt String} parameter, {\tt s}, and simply prints the value of
its parameter.  On the other hand, invoking a method is a matter of
\marginnote{Invoking a method}
writing a method call statement, such as

\begin{jjjlisting}
\begin{lstlisting}
printStr("HelloWorld");
\end{lstlisting}
\end{jjjlisting}

\noindent This statement calls the {\tt printStr()} method and
passes it the string ``HelloWorld''.  This notation assumes that the
call to the instance method {\tt printStr()} is made within the body
of another instance method of the same class.

\subsection{Passing an {\tt int} value to a {\tt OneRowNim} method.}
\noindent To get a clearer picture of the interaction that takes place when we
invoke {\tt takeSticks()} and pass it an {\tt int} value, let's write a
{\tt main()} method to test our new version of {\tt OneRowNim}.

Our first version of {\tt main()} is shown in
Figure~\ref{fig-testorn1}.  We will use it to trace how the parameter
\begin{figure}[h!]
\jjjprogstart
\begin{jjjlisting}
\begin{lstlisting}
public static void main (String argv[])
{  OneRowNim game;        // Declare a OneRowNim object
   game = new OneRowNim(); // Instantiate the references
   game.takeSticks(3);     // remove 3 sticks
} // main()
\end{lstlisting}
\end{jjjlisting}
\jjjprogstop{A {\tt main()} method to test the {\tt takeSticks()} method.}
{fig-testorn1}
\end{figure}
of {\tt takeSticks()} interacts with the instance variables {\tt
nSticks} and {\tt player}.  The statements in the {\tt main()} program
simply create an instance of {\tt OneRowNim} that is referenced by
{\tt game} and invoke the {\tt setSticks()} method with an argument
of~3.

To get a clearer understanding of how a parameter works, it will be
instructive to trace through the code in {\tt main()}.  Figure~3.3
displays how the states of the instance variables of {\tt game} and
the parameter of {\tt takeSticks()} interact.

\begin{figure}[tb]
\figaleft{chptr03/param.eps}{Tracing the state of {\tt game} (a) Just
before calling {\tt takeSticks(3)}. (b) Just before executing the body
of {\tt takeSticks(3)}. (c) Just after executing the body of {\tt
takeSticks(3)}. (d) After flow of control leaves {\tt takeSticks()}.}
\label{fig-parameps}
\end{figure}

Executing the first two statements of {\tt main()} creates the
instance {\tt game} of {\tt OneRowNim}. Figure~3.3(a)
shows the initial state of {\tt game}. When the {\tt takeSticks(3)}
method call is made, a parameter (which is a local variable) named
{\tt num} is created and the value 3 is stored in it.  The state of
the instance variables and the parameter are shown in (b). Then the
body of {\tt takeSticks()} is executed. The new state of {\tt
game} is shown in (c). After the flow of control leaves the body of {\tt
takeSticks()} and returns to {\tt main()}, the memory location which
stored the value of the parameter {\tt num} is released for other
uses. The state of {\tt game} at this point is shown in (d). Notice
that the value of {\tt nSticks} has been reduced to 4.

\subsection{Passing keyboard input to {\tt takeSticks()}}

\noindent To complete this section, let's modify our {\tt main()} method
in Figure~\ref{fig-testorn1} so that it prompts the user to input an
integer from the keyboard and then uses a Scanner object, introduced
in the previous chapter, to read the integer. That integer will then
be used as the argument in a call to {\tt takeSticks()}. These
modifications have been incorporated into the revised version of the
{\tt main()} method shown in Figure~\ref{fig-testorn2}.  If we now run
\begin{figure}[h!]
\jjjprogstart
\begin{jjjlisting}
\begin{lstlisting}
import java.util.Scanner;
    
public static void main (String argv[])
{ Scanner sc;             // Declare a Scanner variable
  sc = Scanner.create(System.in); // Instantiate it
  OneRowNim game;         // Declare a OneRowNim variable
  game = new OneRowNim(); // Instantiate it
  game.report();          // Report state of game
  System.out.println("Input 1, 2, or 3 and hit enter:");
  int num = sc.nextInt(); // Read an int from keyboard
  game.takeSticks(num);   // Use the value read
  game.report();          // Report state of game
} // main()
\end{lstlisting}
\end{jjjlisting}
\jjjprogstop{A {\tt main()} method with keyboard input for {\tt One\-Row\-Nim}.}
{fig-testorn2}
\end{figure}
this program the following output will be generated in the console
window before waiting for keyboard input:

\begin{jjjlisting}
\begin{lstlisting}
Number of sticks left: 7
Next turn by player 1
Input 1, 2, or 3 and hit enter:
\end{lstlisting}
\end{jjjlisting}

\noindent If the user then inputs a 2 from the keyboard, that input
will be read and the {\tt takeSticks()} method will remove 2 sticks.
The output in the console window will now look like:

\begin{jjjlisting}
\begin{lstlisting}
Number of sticks left: 7
Next turn by player 1
Input 1, 2, or 3 and hit enter:2
Number of sticks left: 5
Next turn by player 2
\end{lstlisting}
\end{jjjlisting}

\secEXRHone{Self-Study Exercises}
\begin{SSTUDY}
\item  Explain the difference between a {\it  method declaration} and
a {\it method invocation}\index{method invocation}.

\item  Explain the difference between a {\it formal parameter} and
an {\it argument}.

\item  Modify the {\tt OneRowNim} class of Figure~\ref{fig-testorn2}
by adding two instance variables of type {\tt String} to store names
of the two players. Choose names for the instance variables that would
be appropriate for storing names for player one and player two.  Write
a method named {\tt setNames()} with two string parameters which
assigns the first parameter to the instance variable that you created
for the name of player one.  The second parameter should be assigned
to the other new instance variable.

\item  Write a statement that calls the {\tt setName()} method of
the previous exercise and sets the name of player one of {\tt game}
to ``Xena'' and sets the name of player two to ``Yogi''.
\end{SSTUDY}

\section{Constructor Methods}
\label{sec-constructors}
\noindent In the previous section, we looked at several examples of 
mutator methods that change the values of private instance variables
of an object.  It is possible to define mutator methods to set the
initial values of instance variables after an object is created, but
initial values can also be set by constructors.

As you recall from Chapter 0, a {\bf constructor\index{constructor}}
method is used to create an
%\begin{marginalnote}\it Constructor names\end{marginalnote}
\marginnote{Constructor names}
instance (or object) of a class and to assign initial values to
instance variables.  A constructor declaration looks just like a
method definition except it must have the same name as the class, and
it cannot declare a result type.  Unlike the class level variables and
methods of a class, constructors are not considered members of the
class.  Therefore, they are not inherited by a class's subclasses.
Access to constructors is governed by the access modifiers {\tt
public} and {\tt private}.  Here is a simple constructor for our {\tt
OneRowNim} class:

\begin{jjjlisting}
\begin{lstlisting}
public OneRowNim()
{   nSticks = 7;
    player = 1;
}
\end{lstlisting}
\end{jjjlisting}

\noindent This constructor merely sets the initial values of the instance
%\begin{marginalnote}{\it Constructing an object\end{marginalnote}
\marginnote{Constructing an object}
variables, {\tt nSticks} and {\tt player}. In our current version of
{\tt OneRowNim} these variables are given initial values by using
initializer statements when they are first declared:

\begin{jjjlisting}
\begin{lstlisting}
 private int nSticks = 7;
 private int player = 1;
\end{lstlisting}
\end{jjjlisting}

\noindent So we now have two ways to initialize a class's instance
\marginnote{Initializing variables}
variables.  In the {\tt OneRowNim} class it doesn't really matter which
way we do it.  However, the constructor provides more flexibility
because it allows the state of the object to be initialized at runtime.  Of
course, it would be somewhat redundant (though permissible) to
initialize the same variable twice, once when it is declared and again
in the constructor, so we should choose one or the other way to do
this.   For now, let's stick with initializing the instance variables
when they are declared.

\JavaTIP{EFFECTIVE DESIGN}{Constructors.}{Constructors provide a flexible way
to initialize an object's instance variables when the object is created.}

\noindent A constructor cannot return a value and, therefore, its
%\begin{marginalnote}\it Constructors can't return a value\end{marginalnote}
\marginnote{Constructors can't return a value}
declaration cannot include a return type.  Because they cannot return
values, constructors cannot be invoked by a regular method
invocation.  Instead, constructors are invoked as part of an
{\it instance creation expression} when instance objects are created.  An
instance creation expression involves the keyword {\tt new} followed
by the constructor invocation:

\begin{jjjlisting}
\begin{lstlisting}
OneRowNim game // Declare
    = new OneRowNim(); // and instantiate game1
OneRowNim game2 // Declare
    = new OneRowNim(); // and instantiate game2
\end{lstlisting}
\end{jjjlisting}

\noindent Note here that we have combined variable declaration and
instantiation into a single statement, whereas in some previous examples we
used separate declaration and instantiation statements.  Either way is
acceptable.

\JavaRule{Constructors.}{Constructors cannot return a
value.  Therefore, no return type should be declared when the
constructor is defined.}

\JavaTIP{DEBUGGING TIP}{When to Use Return.}{All method definitions
except constructors must declare a return type.}

\noindent Constructors should be used to perform the necessary initialization
operations during object creation.   In the case of a {\tt OneRowNim}
object, what initializations could be performed?
%\begin{marginalnote}\it State initialization\end{marginalnote}
\marginnote{State initialization}
One initialization that would seem appropriate is to initialize the
initial number of sticks to a number specified.  In order to do this,
we would need a constructor with a single {\tt int} parameter:

\begin{jjjlisting}
\begin{lstlisting}
public OneRowNim(int sticks)
{   nSticks = sticks;
}
\end{lstlisting}
\end{jjjlisting}

\noindent Now that we have this constructor we can use it
when we create instances of {\tt OneRowNim}:

\begin{jjjlisting}
\begin{lstlisting}
OneRowNim game1 = new OneRowNim(21);
OneRowNim game2 = new OneRowNim(13);
\end{lstlisting}
\end{jjjlisting}

\noindent The effect of these statements is the same as if we
had used the {\tt setSticks()} method that was discussed briefly on
page~\pageref{meth-setsticks}.  The difference is that we can now set
the number of sticks when we create the object.

Should we keep the preceding constructor, or keep the {\tt
setSticks()} method or keep both in our class definition?  The
constructor can only be invoked as part of a {\tt new} statement when
the object is created but the {\tt setSticks()} method could be called
anytime we want. In many cases, having redundant methods for doing the
same task in different ways would be an asset, because it allows for
more flexibility in how the class could be used.  However, for a game
like One Row Nim, a major concern is that the two instance variables
get changed only in a manner consistent with the rules for One Row
Nim.  The best way to guarantee this is to have {\tt takeSticks()} as
the only method that changes the instance variables {\tt nSticks} and
{\tt player}.  The only time that it should be possible to set the
number of sticks for a game is when a constructor is used to create a
new instance of {\tt OneRowNim}.


\secEXRHone{Self-Study Exercises}
\begin{SSTUDY}
\item  What's wrong with the following
constructor definition?

\begin{jjjlisting}
\begin{lstlisting}
public void OneRowNim(int sticks)
{   nSticks = sticks;
}
\end{lstlisting}
\end{jjjlisting}

\item  Change the {\tt OneRowNim(int sticks)} constructor so
that it sets the number of sticks and also have it also set player two
as the player who takes the first turn.

\end{SSTUDY}

\subsection{Default Constructors}
\label{sec-defaultconstructors}
\noindent As we noted in Chapter 2, Java automatically provides a 
{\it default constructor} when a class does not contain a constructor.

\JavaRule{Default Constructor.}{If a class contains no
constructor declarations, Java will automatically supply a default
constructor\index{constructor default}. The default constructor takes
no parameters.  If the class is {\tt public}, the default constructor
will also be {\tt public} and, hence, accessible to other objects.}


\noindent The default constructor's role is simply to create an
instance (an object) of that class.  It takes no parameters.  In
terms of what it does, the default constructor for {\tt OneRowNim}
would be equivalent to a {\tt public} constructor method with an
empty body:

\begin{jjjlisting}
\begin{lstlisting}
public OneRowNim() { }
\end{lstlisting}
\end{jjjlisting}

\noindent This explains why the following statement
was valid when a class definition of {\tt OneRowNim} contained no
explicit definition of a constructor:

\begin{jjjlisting}
\begin{lstlisting}
OneRowNim game = new OneRowNim();
\end{lstlisting}
\end{jjjlisting}

\subsection{Constructor Overloading and Method Signatures}

\noindent It is often quite useful to have more than one constructor for
\marginnote{Flexible design}
a given class.  For example, consider the following two {\tt
OneRowNim} constructors:

\begin{jjjlisting}
\begin{lstlisting}
public OneRowNim() {} // Constructor #1

public OneRowNim(int sticks)   // Constructor #2
{   nSticks = sticks;
}
\end{lstlisting}
\end{jjjlisting}

\noindent The first is an explicit representation
of the default constructor.  The second is the constructor we defined
earlier to initialize the number of sticks in a {\tt OneRowNim}
object. Having multiple constructors lends flexibility to the design
of a class. In this case, the first constructor merely accepts {\tt
OneRowNim}'s default initial state. The second enables the user to
initialize the number of sticks to something other than the default
value.

In Java, as in some other programming languages, when two different
\marginnote{Method overloading}
methods have the same name, it is known as {\bf method overloading}.
In this case, {\tt OneRowNim} is used as the name for two distinct
constructor methods.  What distinguishes one constructor from another
is its {\it signature}, which consists of its name together with the
number and types of formal parameters it takes.  Thus, our {\tt
OneRowNim} constructors have the following distinct signatures:

\begin{jjjlisting}
\begin{lstlisting}
OneRowNim()
OneRowNim(int)
\end{lstlisting}
\end{jjjlisting}

\noindent Both have the same name, but the first takes no
parameters, whereas the second takes a single {\tt int} parameter.

The same point applies
\marginnote{Methods are known by their \\signatures}
to methods in general.  Two methods can have the same name as long as
they have distinct signatures.  A {\bf method signature} consists of
its name, and the number, types, and order of its formal parameters.
A class may not contain two methods with the same signature, but it
may contain several methods with the same name, provided each has a
distinct signature.

\JavaRule{Method Signature.}{A {\bf method signature}
consists of the method's name, plus the number, types, and order of
its formal parameters.  A class may not contain two methods with the
same signature.}

\noindent There is no limit to the amount of overloading
that can be done in designing constructors and methods.  The only
restriction is that each method have a distinct signature.  For
example, suppose in addition to the two constructors we have already
defined, we want a constructor that would let us set both the number
of sticks and the player who starts first.  The following constructor
will do what we want:

\begin{jjjlisting}
\begin{lstlisting}
public OneRowNim(int sticks, int starter)
{   nSticks = sticks; // Set the number of sticks
    player = starter; // Set who starts
}
\end{lstlisting}
\end{jjjlisting}

\noindent   When calling this constructor, we would have
to take care to pass the number of sticks as the value of the first
argument and either 1 or 2 as the value of the second argument:

\begin{jjjlisting}
\begin{lstlisting}
OneRowNim game3 = new OneRowNim(14, 2);
OneRowNim game4 = new OneRowNim(31, 1);
\end{lstlisting}
\end{jjjlisting}

\noindent  If we mistakenly reversed 14 and 2 in the first
of these statements, we would end up with a {\tt OneRowNim} game that
starts with 2 sticks and has player 14 as the player with the first
move.

We have now defined three constructor methods for the {\tt OneRowNim}
class.  Each constructor has the name {\tt OneRowNim}, but each has a
distinct signature:

\begin{jjjlisting}
\begin{lstlisting}
OneRowNim()
OneRowNim(int)
OneRowNim(int, int)
\end{lstlisting}
\end{jjjlisting}

\subsection{Constructor Invocation}

\noindent A constructor method is invoked only as part of a {\tt new}
expression when an instance object is first created.  Each of
\marginnote{A constructor is invoked once to create an object}
these is a valid invocation of a {\tt OneRowNim} constructor:

\begin{jjjlisting}
\begin{lstlisting}
                  // Default constructor
OneRowNim game1 = new OneRowNim();   
                  // Sets number of sticks
OneRowNim game2 = new OneRowNim(21); 
                  // Sets both instance variables
OneRowNim game3 = new OneRowNim(19, 2); 
\end{lstlisting}
\end{jjjlisting}

The following constructor invocations are invalid because there are no
matching constructor definitions:

\begin{jjjlisting}
\begin{lstlisting}
                  // No matching constructors
OneRowNim game4 = new OneRowNim("21");  
OneRowNim game5 = new OneRowNim(12, 2, 5);
\end{lstlisting}
\end{jjjlisting}

\noindent In the first case, there is no constructor method that takes
a {\tt String} parameter, so there's no matching constructor.  In the
second case, there is no constructor that takes three {\tt int}
arguments.  In both cases, the Java compiler would complain that there
is no constructor method that matches the invocation. 

\JavaTIP{DEBUGGING TIP}{Method Call.}{The signature of
the method call---its name and the number, types, and order of its
arguments---must exactly match the signature of the method definition.}

\section{Retrieving Information from an Object}

\noindent The modifications we've made to the {\tt OneRowNim} class
allow us to set the instance variables of a {\tt OneRowNim} object
with a constructor, but there is no way for us to retrieve their
values other than to use the {\tt report()} method to write a message
to the console. We will want to be able to ask a {\tt OneRowNim}
object to provide us with the number of sticks remaining and who plays
next when we develop a graphical user interface for {\tt OneRowNim} in
the next chapter.  We declared {\tt nSticks} and {\tt player} as {\tt
private} variables, so we cannot access them directly.  Therefore, we
will need accessor methods to {\it get} the values of each of the
instance variables.  Consider the following method definitions:

\begin{jjjlisting}
\begin{lstlisting}
public int getSticks()
{   return nSticks;
}

public int getPlayer()
{   return player;
}
\end{lstlisting}
\end{jjjlisting}

\noindent Recall that a method's {\it ResultType} is specified just
in front of the {\it MethodName}. We want the two methods to return
{\tt int} values that represent {\tt OneRowNim}'s instance
variables. Therefore, their result types are both declared {\tt int}.

Before we discuss how the value that is returned by a method is used
when the method is called, let's consider one more method definition.
Many methods that return a value do a computation rather than simply
returning the value of an instance variable.  For example, suppose we
wish to define a method for the {\tt OneRowNim} class that will notify
the user of an instance of the class whether the game is over.  Thus
we want a method that, when called, returns a {\tt true} or {\tt
false} depending on whether or not all the sticks have been taken.
{\tt gameOver()} is a descriptive name of such a method and the method
should have a {\tt boolean} result type.  This method should return
{\tt true} when the instance variable {\tt nSticks} no longer contains
a positive {\tt int} value.  Thus we can define:

\begin{jjjlisting}
\begin{lstlisting}
public boolean gameOver()
{   return (nSticks <= 0);
}
\end{lstlisting}
\end{jjjlisting}

\noindent The expression {\tt (nSticks <= 0)} evaluates to a {\tt false} 
value if {\tt nSticks} stores a positive value and it evaluates to
{\tt true} otherwise.  Thus the value returned is precisely what is
required.

\subsection{Invoking a Method That Returns a Value}

\noindent When we invoke a method that returns a value, the invocation
\marginnote{Retrieving information}
expression takes on, or is replaced by, the value that is returned.
For example, if we execute the statements

\begin{jjjlisting}
\begin{lstlisting}
OneRowNim game1 = new OneRowNim(11);
int sticksLeft = game1.getSticks();
\end{lstlisting}
\end{jjjlisting}

\noindent the expression {\tt game1.getSticks()} will take
on the value 11 after the {\tt getSticks()} method is finished
executing.  At that point, the second statement above can be treated
as if expression \mbox{\tt game1.getSticks()} is replaced with the
value 11, which is assigned to {\tt sticksLeft}. In effect, the
second statement is equivalent to the following statement:

\begin{jjjlisting}
\begin{lstlisting}
int sticksLeft = 11;
\end{lstlisting}
\end{jjjlisting}


\JavaRule{Evaluating Method Calls.}{A nonvoid method call is an
expression that has a value of a particular type.  After the method is
executed, the method call expression becomes the value returned.}

We can use a value returned by a method call the same way we use a
literal value of the same type.  It can be assigned to variables, be
part of a numerical expression, or be an argument of another
method. All of the following statements involve valid calls of methods
that return values:

\begin{jjjlisting}
\begin{lstlisting}
int fewerSticks = game1.getSticks() - 1;
boolean done = game1.gameOver();
System.out.println(game1.getPlayer());
game1.getSticks();
\end{lstlisting}
\end{jjjlisting}


\noindent In each statement, the method call can be replaced
with the value it returns.  Notice that the last statement is valid
but does nothing useful. In Java and some other languages like C and
C++, methods that return a value can simply be called without making
use of the value returned.  This may be useful to do if the method
changes the state of instance variables or sends a message to another
object or an output device. The method {\tt getSticks()} does nothing
but return the value of {\tt nSticks}, so simply calling the method
accomplishes nothing.

\subsection{An Expanded {\tt OneRowNim} Class}

\noindent Let's add the new methods that return values to our
{\tt OneRowNim} class.  We might note that the {\tt report()} method
from the previous chapter displays the values of {\tt nSticks} and
{\tt player} in the console window which now could be done by using
the methods {\tt getSticks()} and {\tt getPlayer()} with {\tt
System.out.println()}.  However, calling {\tt report()} is an easy way
to display the values of both instance variables but it cannot provide
access to either variable as an {\tt int} value.
%\begin{marginalnote}\it Redundancy and flexibility\end{marginalnote}
\marginnote{Redundancy and flexibility}
So let's keep all three methods in our class definition.  The
inconvenience of a small amount of redundancy is outweighed by the
added flexibility of being able to call all three methods.

\JavaTIP[false]{EFFECTIVE DESIGN}{Using Redundancy.}{Incorporating some
redundancy into a class, such as providing more than one way to access
the value of an instance variable, makes the class more widely usable.}

\noindent Figure~\ref{fig-ornuml2} provides a UML class diagram of the 
expanded {\tt OneRowNim} class.
\begin{figure}[h!]
\begin{center}
\begin{tabular}{|l|} \hline
OneRowNim\\ \hline
$-$ nSticks: int  \\
$-$ player: int  \\ \hline
$+$ OneRowNim() \\
$+$ OneRowNim(in sticks:int) \\
$+$ OneRowNim(in sticks:int,in starter:int) \\
$+$ takeSticks(in num:int) \\
$+$ getSticks():int \\
$+$ getPlayer():int \\
$+$ gameOver():boolean \\
$+$ report()\\ \hline
\end{tabular}
\end{center}
\caption{A UML class diagram for the expanded {\tt OneRowNim}.}
\label{fig-ornuml2}
\end{figure}

Let's also consider a new {\tt main()} method to test the new methods
of the class.  A very short list of statements that call each of the
three new methods returning values is given in the {\tt main()} method
in Figure~\ref{fig-testorn3}
\begin{figure}[h!]
\jjjprogstart
\begin{jjjlisting}
\begin{lstlisting}
public static void main(String[] args) 
{ OneRowNim game = new OneRowNim(13,2);
  game.takeSticks(2);
  System.out.print("The game is over is: ");
  System.out.println(game.gameOver());
  System.out.print("The next turn is by player: ");
  System.out.println(game.getPlayer());
  System.out.print("Sticks remaining: ");
  System.out.println(game.getSticks());
} //main()
\end{lstlisting}
\end{jjjlisting}
\jjjprogstop{A {\tt main()} method that tests the new methods for {\tt OneRowNim}}
{fig-testorn3}
\end{figure}

\noindent The output to the console  when this program is run will
be:

\begin{jjjlisting}
\begin{lstlisting}
  The game is over is: false
  The next turn is by player: 1
  Sticks remaining: 11
\end{lstlisting}
\end{jjjlisting}

\noindent Note that the constructor sets {\tt player} to 2, so
{\tt player} stores the value 1 after one turn.

\secEXRHone{Self-Study Exercises}
\begin{SSTUDY}
\item  What would these segments of Java
code display on the screen?

\begin{jjjlisting}
\begin{lstlisting}
OneRowNim myGame = new OneRowNim(10,2);
System.out.println(myGame.getPlayer());
System.out.println(2 * myGame.getSticks());
System.out.println(myGame.gameOver());
\end{lstlisting}
\end{jjjlisting}


\item  Suppose that an {\tt int} instance variable named {\tt nMoves} is added
to the {\tt OneRowNim} class that counts the number of moves taken in a
One Row Nim game.  Write a Java method for the {\tt OneRowNim} class
to {\it get}
the value stored in {\tt nMoves}.

\item  Write a method for the {\tt OneRowNim} class called {\tt
player\-One\-Goes\-Next()}
that returns a {\tt boolean} value.  The value returned should be
true if and only
if player one has the next turn.

\end{SSTUDY}

\section{Passing a Value and Passing a Reference}
\label{sec-passby}
\label{pg-sec-passby}

\noindent The effect of passing arguments to a method differs depending on
whether you are passing a value of primitive type\index{primitive
type} (such as 5 or {\tt true}) or a value of reference
type\index{reference type} (such as ``Hello'' or {\tt game1}).  When
an argument of primitive type is
\marginnote{Passing a primitive value}
passed to a method, a copy of the argument is passed to the formal
\begin{figure}[h!]
\jjjprogstart
\begin{jjjlisting}
\begin{lstlisting}
public class PrimitiveCall
{
   public static void myMethod(int n)
   {   System.out.println("myMethod: n= " + n);
       n = 100;
       System.out.println("myMethod: n= " + n);
   } // myMethod()

   public static void main(String argv[])
   {   int k = 5;
       System.out.println("main: k= " + k);
       myMethod(k);
       System.out.println("main: k= " + k);
   } // main()
} // PrimitiveCall
\end{lstlisting}
\end{jjjlisting}
\jjjprogstop{Passing a primitive value to a method.}
{fig-primcall}
\end{figure}
parameter.  For example, consider the \mbox{\tt PrimitiveCall} class
shown in Figure~\ref{fig-primcall}. Note that we have an {\tt int}
%% proglist ch3/methodcalls/PrimitiveCall.java
variable {\tt k}, which initially stores the value 5, and a method
{\tt myMethod()}, which takes an {\tt int} parameter {\tt n}.  In this
case, when we invoke {\tt myMethod(k)}, {\tt k}'s value (5) is copied
into {\tt n} and stored there during the method.


One implication of passing a copy of a primitive value to a method is that the
original value of {\tt k} in {\tt main()} cannot be altered from
inside the method.  Thus, the output generated by {\tt
PrimitiveCall} would be

\begin{jjjlisting}
\begin{lstlisting}
  main: k= 5
  myMethod: n= 5
  myMethod: n= 100
  main: k= 5
\end{lstlisting}
\end{jjjlisting}

\noindent Note that in {\tt main()}, {\tt k}'s value is
printed both before and after {\tt myMethod()} is called, but that its
value remains unaffected even though {\tt n}'s value is changed within
the method. This is because {\tt myMethod()} contains just a {\it
copy} of {\tt k}'s value, not {\tt k} itself. Any changes to the copy
within {\tt myMethod()} leave {\tt k} unaltered (See Fig.~3.8).

\JavaRule{Passing a Primitive Value.}{When a value of
a primitive type, like {\tt boolean} or {\tt int}, is passed to a method, a
copy of the value is passed. That's why its original value remains
unchanged outside the method, even if the copy is changed inside the method.}

\begin{figure}[b!]
\figaright{chptr03/vparam.eps}{Tracing the state of variables k and n 
in {\tt PrimitiveCall} (a) Just
before calling {\tt myMethod(k) in main}. (b) Just before executing the body
of {\tt myMethod()}. (c) Just after executing the body of {\tt
myMethod()}. (d) After flow of control returns to {\tt main()}.}
\label{fig-vparameps}
\end{figure}
\noindent In contrast to this, when an argument of a reference type
is passed to a method, a copy of the reference to the object itself is
assigned to the parameter. For example, in the case of a {\tt String}
parameter or a {\tt OneRowNim} parameter, the method would be given a
reference to the object--that is, the address of the object. The
object itself is {\it not} passed, because it would be too inefficient
to copy the entire object with all its data and methods.  However,
because the object's reference gives the object's location in memory,
the method will have access to the object and can make changes to the
original object from within the method.


For example, consider the {\tt ReferenceCall} class
(Fig.~\ref{fig-refcall}). In this case, {\tt myMethod()} takes a
%% proglist ch3/methodcalls/ReferenceCall.java
\begin{figure}[h!]
\jjjprogstart
\begin{jjjlisting}
\begin{lstlisting}
public class ReferenceCall
{
    public static void myMethod(OneRowNim g)
    {   System.out.print("myMethod: Number of sticks: ");
        System.out.println(g.getSticks());
        g.takeSticks(3);
        System.out.print("myMethod: Number of sticks: ");
        System.out.println(g.getSticks());
    } // myMethod()

    public static void main(String argv[])
    {   OneRowNim game = new OneRowNim(10);
         System.out.print("main: Number of sticks: ");
         System.out.println(game.getSticks());
         myMethod(game);
         System.out.print("main: Number of sticks: ");
         System.out.println(game.getSticks());
    }// main()
} // ReferenceCall
\end{lstlisting}
\end{jjjlisting}
\jjjprogstop{Passing a reference value to a method.}
{fig-refcall}
\end{figure}
parameter {\tt g} of type {\tt OneRowNim}. Because a \mbox{\tt
OneRowNim} instance is an object, {\tt g} is a reference variable.  So
when {\tt myMethod(game)} is invoked in {\tt main()}, a reference to
{\tt game} is passed to {\tt myMethod()}. Note that in {\tt
myMethod()}, we use {\tt takeSticks(3)} to change the number of sticks
of {\tt g} from 10 to 7 and that this change persists even after the
method returns control to {\tt main()}. The reason is that during the
method's execution, both {\tt game} and {\tt g} refer to the exact
same object (see Fig.~3.10).  The output generated 
by {\tt ReferenceCall} would be

\begin{jjjlisting}
\begin{lstlisting}
  main: Number of sticks: 10
  myMethod: Number of sticks: 10
  myMethod: Number of sticks: 7
  main: Number of sticks: 7
\end{lstlisting}
\end{jjjlisting}

%%% RAM: Why isn't this figure replaced.
%%%\begin{figure}[b]
%%%\vspace{12pt}\figa{chptr03/p136f1.eps}{In (a), when {\tt
%%%myMethod(pet1)} is called, {\tt p} points to {\tt pet1}.  In (b), {\tt
%%%myMethod()} has changed {\tt pet1}'s name to ``Mary.'' This change
%%%will persist when the method returns.
%%%\label{fig-refparam}
%%%\label{pg-fig-refparam}}
%%%\end{figure}

\begin{figure}[tb]
\figascaled{chptr03/rparam.eps}{0.8}{Tracing the state of {\tt OneRowNim} object
in {\tt ReferenceCall} (a) Just
before calling {\tt myMethod(game)}. (b) Just before executing the body
of {\tt myMethod()}. (c) Just after executing the body of {\tt
myMethod()}. (d) After flow of control returns to {\tt main()}.}
{fig-rparameps}
\end{figure}

\noindent This illustrates that when passing a reference variable
to a method, it is possible for the method to change the state of the
object associated with the reference variable.  In subsequent chapters
we will see ways to make use of this feature of reference parameters.

\JavaRule[false]{Passing a Reference.}{When a reference to an
object is passed to a method, any changes made to the object from
within the method will persist when the method is finished executing.}

\JavaTIP[false]{DEBUGGING TIP}{Side Effects.}{An unintended
change to an object is called a {\bf side effect}. Care should be
taken in designing methods that the method does not produce unwanted
side effects in objects passed as reference parameters.}

\section{Flow of Control: Control Structures}

\noindent We have been ignoring a couple of problems with the
definition of the {\tt OneRowNim} class.  One problem is that we would
describe a One Row Nim game as two players taking turns until there
are no more sticks. An object using {\tt OneRowNim} would need a way
to repeatedly execute a group of statements. One command in Java that
controls the repetition of a block of statements is called a {\it
while loop}.  We will consider it later in this section.

A second problem is with the definition of {\tt takeSticks()}:

\begin{jjjlisting}
\begin{lstlisting}
  public void takeSticks(int num)
  {   nSticks - num;
      player = 3 - player;
  }
\end{lstlisting}
\end{jjjlisting}

\noindent It is possible to call this method with an argument greater
than 3 or less than 1.  The call {\tt game.takeSticks(5)} will remove
5 sticks even though the rules of One Row Nim say that you must remove
1, 2, or 3.  While one might assume that the user interface should
prevent the user from breaking this rule, it is a far better design if
it was dealt with in {\tt OneRowNim}.  To do this we need a Java
structure that executes different statements depending on whether the
parameter is greater than 3, less than 1, or between 1 and 3.  The
Java {\it if-else} statement has this capability.  A fuller treatment
of control structures appears in Chapter 6, but in this section, we
will briefly introduce a couple of simple control structures. This will
enable us to write programs that take more interesting actions.

\subsection{The Simple If Statement}
\label{sec-if}
\label{pg-sec-if}
\noindent A {\bf selection} control structure, allows a program 
to select between two or more alternative paths of execution. The if
statement is the most basic selection control
structure in Java. Most programming languages have its equivalent.
\marginnote{Simple {\tt if} statement}

\JavaRule{If Statement.}{The {\it if statement} has the following syntax: \\
$
\matrix{
\hspace{1.1in}\mbox{\tt if} & ( \; {\it boolean\;expression} \; ) \cr
\hspace{1.1in}\;\;\;\;         & {\it contained statement} \; ;  \cr
}
$}

\noindent The statement contained in the if statement can be
any valid Java statement, including a compound statement. (Recall from
Chapter 1 that a compound statement is a set of statements contained
within curly braces.)  The {\tt boolean expression} is an expression
that is either {\tt true} or {\tt false}. We have seen examples of
boolean expressions that involve {\tt int} variables, {\tt int}
values, and the inequality or equality operators.  A method call to a
method with a {\tt boolean} result type is another example of a {\tt
boolean} expression.  Given this description of if statement syntax,
the following are examples of valid if statements:

\begin{jjjlisting}
\begin{lstlisting}
if (true) System.out.println("Hello");
if (nSticks <= 0) System.out.println("game is over");
\end{lstlisting}
\end{jjjlisting}

\noindent For readability, we usually write an if statement with
its contained statement indented on the next line:

\begin{jjjlisting}
\begin{lstlisting}
  if (true)
      System.out.println("Hello");
  if (nSticks <= 0)
      System.out.println("game is over");
\end{lstlisting}
\end{jjjlisting}

\noindent The following are all examples of syntax errors involving
the if statement:

\begin{jjjlisting}
\begin{lstlisting}
if true                   // Parentheses are missing
    System.out.println("Hello");

if (nSticks <= 0) return   // Semicolon missing

if ("true") return;  // "true" is not a boolean

if (true) "Hello";    // "Hello" is  not a statement
\end{lstlisting}
\end{jjjlisting}

\noindent Semantically, the if statement\index{if statement} has the following
interpretation: First, the boolean condition is evaluated.  If it is
true, then the contained statement is executed; if it is false, then
the contained statement is not executed.  This is shown in
Figure~3.11. The flowchart clearly shows that program flow will take
\marginfig{chptr03/3f12.eps}{Flowchart of the if statement.
Diamond-shaped symbols at  the branch points contain {\tt boolean}
expressions.  Rectangular  symbols can only contain executable
statements.  Circles act  simply as connectors, to connect two
or more paths.}
{fig-ifflowchart}
one or the other of the alternative paths coming out of the
diamond-shaped boolean condition box.  The branch through the
rectangular statement box will be taken when the boolean condition is
true; otherwise the statement will be skipped.

As another example, consider the definition of a {\tt
getPlayerString()} method for the {\tt OneRowNim} class:

\begin{jjjlisting}
\begin{lstlisting}
public String getPlayerString()
{
  if (player == 1)
    return "Player One"; // Exit the method
  if (player == 2)
    return "Player Two"; // Exit the method
  return "Player error";   // Exit the method
}
\end{lstlisting}
\end{jjjlisting}

\noindent The flowchart in Figure~\ref{fig-getplayerflow} shows the program flow
of the entire {\tt getPlayerString()} method.  It is important to note
\begin{figure}[htb]
\figascaled{chptr03/player.eps}{0.9}{Flowchart of the {\tt getPlayerString()} method.}
%%\figa{chptr03/player.eps}{Flowchart of the {\tt getPlayerString()} method.
{fig-getplayerflow}
\end{figure}
that when a {\tt return}\index{return statement} statement is executed
in a method, control is returned immediately to the calling method.
Thus, if {\tt player == 1 } is true, the string ``Player One'' is
returned to the calling method and the {\tt getPlayerString()} method
exits at this point.  If it is false, then {\tt player == 2} should be
true (if we have a consistent state) and the string ``Player Two''
should be returned and the method exited.  Thus, if we have a
consistent state ---that is, if {\tt player} has value 1 or 2---then
the third {\tt return} statement should never be reached.

The following example shows the more common case where the statement
contained in an if statement can be a compound statement:
\marginnote{Compound statement}

\begin{jjjlisting}
\begin{lstlisting}
if (player == 1)
{   String s = "Player One";
    System.out.print (s);
    System.out.println (" plays next");
    System.out.println (" The game is not over");
}
\end{lstlisting}
\end{jjjlisting}

\noindent If {\tt player == 1} is true, then all four statements
in the contained compound statement will be executed.  Note here that
\marginnote{Local scope}
we are declaring the local variable, {\tt s}, in this block.  Its
scope would extend only to the end of the block.  Note also that when
we use a compound statement, the compound statement itself is not
followed by a semicolon because it is already enclosed in braces.

A common programming error is to forget the braces around the compound
statement.  Merely indenting the statements following the if clause
doesn't alter the logic of the if statement.   For example, the
following if statement still has only one statement in its if clause:

\begin{jjjlisting}
\begin{lstlisting}
if (condition1)
  System.out.println("One");
  System.out.println("Two"); //Not part of if statement
\end{lstlisting}
\end{jjjlisting}

\noindent This segment will always print ``Two'' because the
second {\tt println()} is not part of the if statement. To include it
in the if statement, you must enclose both {\tt println()}
statements within braces:

\begin{jjjlisting}
\begin{lstlisting}
if (condition1)
{  System.out.println("One");
   System.out.println("Two");
}
\end{lstlisting}
\end{jjjlisting}

\JavaTIP{DEBUGGING TIP}{Indentation.}{Indentation can improve the
readability of a program but doesn't affect its logic.  Braces must be
used to group statements in the if clause.}

%%%RAM\subsection{The {\ttHtwo if-else} Statement}

\subsection{The {\tt if-else} Statement}
\label{sec-ifelse}
\label{pg-sec-ifelse}
\noindent A second version of the if statement incorporates an {\tt
else} clause into the structure.  This allows us to execute either of
two separate statements (simple or compound) as the result of one
boolean expression.  For example, the statement

\begin{jjjlisting}
\begin{lstlisting}
if (player == 1)
    System.out.println("Player One");
else
    System.out.println("Player Two");
\end{lstlisting}
\end{jjjlisting}

\noindent will print ``Player One'' if {\tt player == 1} is true.
Otherwise, it will print ``Player Two''.

\JavaRule{If-else Statement.}{The {\it if-else statement} has the
following syntax: \\
$
\matrix{
\hspace{1.1in}{\tt if} & ( \; {\it boolean\;expression } \; ) \cr
\hspace{1.1in}\;\;\;\;\;\;\; & {\it statement1 } \;  ;          \cr
\hspace{1.1in}\;\;{\tt else }  &                                  \cr
\hspace{1.1in}\;\;\;\;\;\;\; & {\it statement2 } \;  ;          \cr
}
$}

\noindent As in the case of the simple if statement, the
%\begin{marginalnote}\it If-else syntax\end{marginalnote}
\marginnote{If-else syntax}
keyword {\tt if} is followed by a parenthesized {\it boolean
expression}, which is followed by {\it statement1}, which may be
either simple or compound.  If {\it statement1} is a simple statement,
then it is followed by a semicolon.  The {\it else clause} follows
immediately after {\it statement1}. It begins with the keyword {\tt
else}, which is followed by {\it statement2}, which can also be either
a simple or compound statement. Note that there is no boolean
expression following the {\tt else} keyword. In an if-else statement,
the boolean expression following the keyword {\tt if} goes with both
the if and else clauses.

Semantically, the if-else statement\index{if-else statement} has the
%\begin{marginalnote}\it If-else semantics\end{marginalnote}
%\marginnote{If-else semantics}
following interpretation: If the {\it boolean expression} is true,
execute {\it statement1}; otherwise execute {\it statement2}.  This
\marginfig{chptr03/3f14.eps}{Flowchart of the {\tt if-else} statement.}
{fig-ifelsechart}
interpretation is shown in Figure~\ref{fig-ifelsechart}.

\subsection{The Nested if/else Multiway Selection Structure}
\label{sec-multi}
\label{pg-sec-multi}
\noindent The statements that one inserts in place of {\it statement1} and {\it
statement2} in the if-else statement can be any executable statement,
%\begin{marginalnote}\it Multiple alternatives\end{marginalnote}
%���for pagination\marginnote{Multiple alternatives}
including another if statement or if-else statement.  In other words,
it is possible to embed one or more if-else statements inside another
if-else statement, thereby creating a {\it nested\index{nested
if-else}} control structure.  As with most things, making a control
structure too complex isn't a good idea, but there is a standard
nested if-else control structure that is very useful.  It is known as
{\bf multiway selection\index{multiway selection}}. As shown in
Figure~\ref{fig-multiway}, the multiway structure is used when you
want to select one and only one option from several alternatives.

%\begin{figure}
\begin{figure}[tb]
\figaleft{chptr03/multi.eps}{Flowchart of a nested if-else statement.
%%\figa{chptr03/multi.eps}{Flowchart of a nested if-else statement.
\label{fig-multiway}}
\end{figure}
%\end{figure}

Suppose we have an {\tt int} variable {\tt num} that will contain one
of the values $1$, $2$, or $3$ unless there has been an error assigning
a value to it. Suppose that we want to write code that will write out
the English word for the value in {\tt num}. In the example shown in
Figure~\ref{fig-multiway} there are three alternatives plus an error
state.  Here is the Java code for this example:

\begin{jjjlisting}
\begin{lstlisting}
if (num == 1)
   System.out.println("One");
else if (num == 2)
   System.out.println("Two");
else if (num == 3)
   System.out.println("Three");
else
   System.out.println("Error: Unknown value");
\end{lstlisting}
\end{jjjlisting}

\noindent Note that the multiway structure has a single
entry point and that only one of the four possible alternatives is
\marginnote{Multiple alternatives}
executed.  The code will print exactly one of the strings.

We will have many occasions to use the if-else structure.  Although it
does not represent a significant change, we could rewrite our {\tt
takeStick()} method to make use of the if-else instead of the somewhat
obscure statement :

\begin{jjjlisting}
\begin{lstlisting}
player = 3 - player;
\end{lstlisting}
\end{jjjlisting}

\noindent to change the value of player from $1$ to $2$ or vice versa:

\begin{jjjlisting}
\begin{lstlisting}
public String takeSticks(int num)
{   nSticks = nSticks - num;
  if (player == 1)
    player = 2;   // From 1 to 2
  else
    player = 1;   // From 2 to 1
}
\end{lstlisting}
\end{jjjlisting}

\noindent In some respects this version of {\tt takeSticks()}
involves four lines of code instead of one but is simpler
to understand. The {\tt if}-statement tests whether the value of {\tt player}
is $1$.  If it is, the value is changed to $2$.  If the value of {\tt player}
is not $1$, then the value must be $2$ and so the value is changed to $1$.
Both versions of the code will give precisely the same result, a programmer
could choose to write the code either way.


\secEXRHone{Self-Study Exercises}
\begin{SSTUDY}
%<sec_flowchart:\thisCheadno>
%<pg_sec_flowchart:\thispageno>

\item  Consider the following method with
{\tt boolean} parameter.

\begin{jjjlisting}
\begin{lstlisting}
public String getStatus(boolean isDone)
{   if (isDone)
        return "Done";
    else
        return "Not Done";
}
\end{lstlisting}
\end{jjjlisting}

\noindent Draw a flowchart for the if-else version of the
%\begin{marginalnote}\it Flowchart symbols\end{marginalnote}
\marginnote{Flowchart symbols}
{\tt getStatus()} method, using the figures in this section as a
guide.  The if-else structure should be drawn exactly as shown in
Figure~3.11. It should have a single entry point that leads directly
to the top of a diamond-shaped box that contains a boolean condition.
There should be two branches coming out of the condition box.  The one
going to the right is the true case, and the one going to the left is
the false case.  Each of these branches should contain one rectangular
box, which contains the statements that would be executed in that
case.  The left and right branches should be connected by a circular
symbol that is aligned directly under the diamond box whose conditions
it connects.  There should be a single exit arrow pointing directly
down.


\item  Identify the error in the following statements:

\begin{jjjlisting}
\begin{lstlisting}
if (isHeavy == true)
    System.out.println("Heavy");
else ;
    System.out.println("Light");

if (isLong == true)
    System.out.println("Long")
else
    System.out.println("Short");
\end{lstlisting}
\end{jjjlisting}

\item  Suppose we have an {\tt  int} instance variable named {\tt player}
in some class describing a three person game.  Write a method named
{\tt getPlayerName()} that returns a {\tt String}.  It should return
``Ann'', ``Bill'', ``Cal'', or ``Error'' when the value of {\tt
player} is respectively 1, 2, 3, or any other value.

\item  How does a parameter for a primitive type differ from a parameter
for a reference type?

\end{SSTUDY}

%%%REW Subsection from chapter 6 on while inserted here
\subsection{The While Structure}

\noindent A {\bf repetition structure} 
is a control structure that repeats a statement or sequence of
statements in a controlled way.  Repetition structures are also
referred to as {\bf loop structures}.  Many types of programming tasks
require a repetition structure.  Consider some examples.

\begin{itemize}

\item You want to add up the squares of the numbers from 1 to 100.

\item You want to compute compound interest on an amount of money in a
savings account with a fixed interest rate if it is kept there for $30$ years.

\item A computer security employee wants to try every possible password in
order to break into an account of a suspected spy.

\item You want to have players input moves for a turn in a game until
the game is over. Our {\tt OneRowNim} is such an example.

\end{itemize}

We will study several different repetition structures of Java in depth in Chapter
6.
We will briefly consider the  while statement here so as to be able to
define methods that are more powerful and more interesting.
Let us write a method that solves a slight generalization
of the first problem above.  We will use the while statement to sum the
squares of integers from $1$ to a number specified as a parameter of the
method.  Thus, the method call {\tt sumSquares(3)} should return the value
$14$ since $1*1 + 2*2 + 3*3 = 1 + 4 + 9 = 14$.


\begin{jjjlisting}
\begin{lstlisting}
public int sumSquares(int max)
{ int num = 1;
  int sum = 0;
  while (num <=  max) { // While num <= max
    sum = sum + num*num; // Add square to sum
    num = num + 1;       // Add 1 to num
  } //while
  return sum;           // Return the sum
}
\end{lstlisting}
\end{jjjlisting}

\noindent Note that in this example, the variable {\tt num} gets
assigned an initial value of $1$ before the {\tt while} statement.
Note also that the {\tt boolean} expression {\tt num < max} in
parentheses after {\tt while} states the condition for which we wish
to continue summing squares.  Finally note that the last statement in
the block following the {\tt boolean} expression adds $1$ to {\tt
num}--that is, this variable is updated at the end of the block.

The {\bf while statement}\index{while statement} is a loop statement
in which the loop entry condition occurs before the loop body. It has
the following general form:

\JavaRule[false]{While Statement.}{The {\it while statement} has the
following syntax: \\
$
\matrix{
\hspace{1.1in}{\tt while} & ( \; {\it loop\;entry\;condition } \; ) \cr
\hspace{1.1in}\;\;\;\;\;\;\; & {\it loop body } \;  ;          \cr
}
$}

\noindent When the {\tt while} statement is executed, the loop entry condition
is evaluated and if this evaluates to {\tt false}, execution continues
at the statement immediately after the loop body.  If the loop entry condition
evaluates to {\tt true}, the loop body is executed  and then the
entry condition
is evaluated again.  The loop body continues to be executed until the
loop entry
condition evaluates to {\tt false}.


To have a {\tt while} statement accomplish a task, the variable or
variables in the loop entry condition must be initialized correctly before the
{\tt while} statement and these variables must be correctly updated at the
end of the loop body.  We can refer to the initializer statement
followed by a {\tt while} statement as a {\bf while structure}.
We can restate the above guidelines as a design principle:

\JavaTIP{EFFECTIVE DESIGN}{Loop Structure.}{A properly designed
while structure must include an initializer, a loop entry condition,
and an updater. The updater should guarantee that the loop entry
condition eventually fails, thereby allowing the loop to terminate.}

\noindent In pseudocode, the {\it while structure}\index{while structure} would
take the following form:


\begin{jjjlisting}
\begin{lstlisting}
InitializerStatements;         // Initializer
while (loop entry condition) { // Bound test
     Statements;                // Loop body
     UpdaterStatements;         // Updater
}
\end{lstlisting}
\end{jjjlisting}

\noindent As its form suggests, the while structure is designed
so that on some conditions the loop body will never be
executed. Because it tests for the loop entry condition before the loop
body, it is possible that the loop body is never executed.  We might
say that it is designed to perform 0 or more iterations.

For example, if the method call {\tt sumSquares(-3)} is executed, the
loop body will be skipped, because the loop entry condition {\tt num
<= max} is false to begin with. No iterations will be performed, and
the algorithm will simply return the value $0$.


\begin{figure}[tb]
\figaleft{chptr03/whileflow.eps}{Flowchart of the while statement and while
%%%\figa{chptr03/whileflow.eps}{Flowchart of the while statement and while
structure.
\label{fig-whileflow}}
\end{figure}


Note also that in the while statement the bound test is preceded by
initializer statements, and the loop body contains updater statements.
The semantics of the while structure are shown in Figure~\ref{fig-whileflow}.

\pagebreak
\secEXRHone{Self-Study Exercise}
\begin{SSTUDY}
\item  Modify the definition of the
{\tt sumSquares()} method to define a method
named {\tt sumCubes()} that sums the cubes of integers from a minimum
value up to a maximum value and returns that sum.  {\tt sumCubes()}
should have two parameters that will store the minimum and
maximum values.  Thus the method call {\tt sumCubes(2,3)} should return
$35$ since $2*2*2 + 3*3*3 = 8 + 27 = 35$.

\end{SSTUDY}


%%%\RHrightsec{3.7}{The Improved {\tt OneRowNim}}
\section{Testing an Improved {\tt OneRowNim}}

\noindent Let's use the control structures that we have discussed
to improve the definition of the {\tt takeSticks()} method
of {\tt OneRowNim}.  We noted earlier that our current definition
allows $4$ or more sticks to be removed from {\tt nSticks} even though
the rules of One Row Nim indicate that a player must take one, two, or
three sticks on a turn.  We can use {\tt if-else} statements
to make certain that no more than $3$ sticks get removed.

What should happen if the method {\tt takeSticks()} is called with an
argument that does not represent a legal number of sticks to remove?
In this case, it would probably make sense to remove no sticks at all
and to keep the value of {\tt player} the same so that the player whose turn
it is does not change. In addition, it would be nice if the method could
signal that an illegal move has been attempted.  This can  be accomplished if
we redefine {\tt takeSticks()} to return a {\tt boolean} value. Let's have
a return value of {\tt true} represent the case that a valid number of sticks
have been removed and the player to play next has been changed. A return
of {\tt false} will indicate that an illegal move has been attempted.  Making
these changes to the {\tt takeSticks()} method will yield a method definition
that looks like:

\begin{jjjlisting}
\begin{lstlisting}
public boolean takeSticks(int num) 
{   if (num < 1) {
       return false; // Error
    } else if ( num > 3) {
       return false; // Error
    } else {
       nSticks = nSticks - num;
       player = 3 - player;
       return true;
    } //else
} //takeSticks
\end{lstlisting}
\end{jjjlisting}

\noindent Notice that the new definition of the {\tt takeSticks()} method
has a boolean return type.  Also notice that the {\tt if/else} multiway
structure is used to handle the three cases of the parameter {\tt num}
being less than one, more than three, or a valid number.

Let us add one more method to the {\tt OneRowNim} class.  Let's define
a method called {\tt getWinner()}that will return the number of the
winning player if the game is over. Recall that the player who takes
the last stick loses, so after that last play, the player whose turn
it is to play next is the winner. However, we should be concerned
about what value to return if the game is not over when the method is
called. A common strategy is to have a method return a special value
to indicate that it is in a state in which it cannot return the value
requested.  Returning a $0$ value is a good way to indicate that the
game is not over so a winner cannot be identified.  With this
information, the {if/else} statement can be used in the definition of
{\tt getWinner()}.

\begin{jjjlisting}
\begin{lstlisting}
public int getWinner()
{   if (nSticks < 1)
        return player;
    else 
        return 0;
} // getWinner()
\end{lstlisting}
\end{jjjlisting}

We now have the final version (for this chapter) of the {\tt
OneRowNim} class whose implementation is given in
Figure~\ref{fig-ornclass}.  We have turned a very simple class into
one that contains quite a few elements.  Compared to our first version
(in Chapter~1), this Chapter's version of {\tt OneRowNim} presents an
interface (to other objects) that is easy and convenient to use.  The
constructor methods with parameters provide an easy way to create a
{\tt OneRowNim} instance with any number of sticks.  The use of {\tt
private} instance variables and a single, carefully designed mutator
method, {\tt takeSticks()}, prevents other objects from tampering with
the state of a {\tt OneRowNim} object's state.  The other methods
provide a flexible way to find out the state of a {\tt OneRowNim}
object.  The complete implementation of this {\tt OneRowNim} is shown
in Figure~\ref{fig-ornclass}.

\begin{figure}[p]
\jjjprogstart
\begin{jjjlisting}
\begin{lstlisting}
public class OneRowNim 
{   private int nSticks = 7;
    private int player = 1;

    public OneRowNim() 
    {
    } // OneRowNim() constructor

    public OneRowNim(int sticks) 
    {   nSticks = sticks;
    }  // OneRowNim() constructor2

    public OneRowNim(int sticks, int starter) 
    {   nSticks = sticks;
        player = starter;
    }  // OneRowNim() constructor3

    public boolean takeSticks(int num) 
    {   if (num < 1) return false;       // Error
        else if ( num > 3) return false; // Error
        else              // this is a valid move
        {   nSticks = nSticks - num;
            player = 3 - player;
            return true;
        } // else
    } // takeSticks()

    public int getSticks() 
    {   return nSticks;
    } // getSticks()

    public int getPlayer() 
    {   return player;
    } // getPlayer()

    public boolean gameOver()
    {   return (nSticks <= 0);
    } // gameOver()

    public int getWinner()
    {   if (nSticks < 1) return getPlayer();
        else return 0;  //game is not over
    } // getWinner()
    
    public void report()
    {   System.out.println("Number of sticks left: " + 
                                             getSticks());
        System.out.println("Next turn by player " + 
                                             getPlayer());
    }   // report()
} // OneRowNim class
\end{lstlisting}
\end{jjjlisting}
\jjjprogstop{The {\tt OneRowNim} class with improved methods.}
{fig-ornclass}
\end{figure}

Let's use a while statement to test the new methods of the class.  A
pseudocode description of how a game is played might look like:

\begin{jjjlisting}
\begin{lstlisting}
  Choose the initial number of sticks for the game
  while the game is not over
  {   Report the state of the game
      Process the next move
  }
  Report the state of the game
  Report who the winner is
\end{lstlisting}
\end{jjjlisting}

\noindent  Translating this pseudocode into Java code in a {\tt main()}
method in a separate class gives us the class shown in
Figure~\ref{fig-tornclass}.  We will use the {\tt Scanner} class 
\begin{figure}[hbt]
\jjjprogstart
\begin{jjjlisting}
\begin{lstlisting}
import java.util.Scanner;  
  
public class TestOneRowNim
{
  public static void main(String argv[])
  { Scanner sc = Scanner.create(System.in);
    OneRowNim game = new OneRowNim(11);
    while(game.gameOver() == false) 
    {   game.report();  // Prompt the user
      System.out.print("Input 1, 2, or 3: ");
      int sticks = sc.nextInt(); // Get move
      game.takeSticks(sticks);   // Do move
      System.out.println();
    } // while
    game.report();  // The game is now over
    System.out.print("Game won by player ");
    System.out.println(game.getWinner());
 } // main()
} // TestOneRowNim
\end{lstlisting}
\end{jjjlisting}
\jjjprogstop{The {\tt TestOneRowNim} class with a while loop.}
{fig-tornclass}
\end{figure}
introduced in the previous chapter to get moves from the keyboard
for both players.  Before each move {\tt game.report()} describes 
the state of the game before the user is prompted to input a move 
for one of the players.  A reader interested in
seeing the lengthy output to the console when the {TestOneRowNim}
class is run is encouraged to actually run the program.

Note that the return value of the {\tt takeSticks()} method is ignored
in this test program.  We will make use of the return value in test
programs in the next chapter when better user interfaces are developed
for {\tt OneRowNim}.  Note, however, that taken together, the public
methods for {\tt OneRowNim} provide other objects with an interface
that they can use to communicate with individual {\tt OneRowNim}
objects.

\JavaTIP[false]{EFFECTIVE DESIGN}{Interfaces.}{Well-designed objects provide a
useful public interface and protect the object's private elements from
other objects.}

\noindent To reiterate a point made at the outset, object-oriented programming
\marginnote{Object-oriented design}
is a process of constructing objects that will interact with each
other.  Object-oriented programs must ensure that the objects
themselves are well designed in terms of their ability to carry out
their designated functions.  Good design in this sense requires
careful selection of instance variables and careful design of methods
to ensure that the object can carry out its assigned tasks.  However,
equal care must be taken to ensure that the interactions that take
place among objects are constrained in ways that make sense for that
particular program.  This aspect of designing objects comes into play
in designing the methods---constructor, accessor, and mutator---that
make up the object's interface.

\section*{{\color{cyan}Special Topic:} Intelligent Agents}

{\color{cyan}Wouldn't it be nice} if we had a computer program that
could schedule appointments for us, remind us of meetings and
commitments, find information for us on the WWW, and manage our e-mail
messages for us?  Wouldn't it be nice to have a computerized personal
assistant?

Actually, such programs are called {\it intelligent agents}, which are
programs that are capable of acting autonomously to carry out certain
tasks.  Intelligent agent technology is becoming an important research
area in computer science.  Most agent programs incorporate some kind
of machine learning capability, so that their performance improves
over time.

As a typical agent activity, suppose I was able to tell my intelligent
agent to buy me a copy of a certain book that I just heard about.
Given a command like ``buy me a copy of X,'' the agent would perform a
search of online book sellers and come up with the best deal.  Once it
had found the best buy, the agent would communicate with a
computer-based agent representing the book seller.  My agent would
make the order and pay for it (assuming I gave it authority to do so),
and the book seller's agent would process the order.

As far-fetched as the capability may now seem, this is the direction
that research in this area is headed.  Researchers are developing
agent languages and describing protocols that agents can use to
exchange information in a reliable and trustworthy environment.
Obviously, you wouldn't want your agent to give your money to a
fraudulent book seller, so there are significant problems to solve in
this area that go well beyond the problem of simply exchanging
information between two agents.

The best way to learn more about this research area is to do a Web
search using the search string ``Intelligent Agent.''  There are
numerous research groups and companies that provide online
descriptions and demos of their products.

\section{{\bf From the Java Library} {\tt java.lang.Object}}
\label{sec-tostring}
\WWWjava

%<sec_object:\thisAheadno><pg_sec_object:\thispageno>
%%%%\tBOXseven{The most} general class in Java's class hierarchy is the
The most general class in Java's class hierarchy is the {\tt
java.lang.Object} class.  It is the superclass of all classes that
occur in Java programs.  By default, it is the direct superclass of
any class that does not explicitly specify a pedigree in its class
definition.

All subclasses of {\tt Object} {\bf inherit} the {\tt public} and {\tt
protected} methods contained in {\tt Object},
%\begin{marginalnote}\it Inheritance\end{marginalnote}
so all such methods can be thought of as belonging to the subclasses.
This means that all classes inherit the methods of the {\tt Object}
class, because every class is a subclass of it.  In this section,
let's look briefly at how we can use an inherited method and
also at how we can {\bf override} it--that is, redefine the method--if
it doesn't exactly suit our purposes.
%���for pagination\marginnote{Inheritance}

One of the most useful methods in the {\tt Object} class is the
{\tt toString()} method:

\begin{jjjlisting}
\begin{lstlisting}
public class Object
{   
    public String toString() ;
}
\end{lstlisting}
\end{jjjlisting}

\noindent The {\tt toString()}\index{toString() method} method
%\begin{figure}
%\begin{graphic}
%%%\marginfig{chptr03/p150f1.eps}{As a subclass of {\tt Object}, the {\tt OneRowNim} class
%%%inherits the {\tt toString()} method.
%%%\label{fig-tostring}
%%%\label{pg-fig-tostring}}
%\end{graphic}
%\end{figure}
returns a {\tt String} representation of its object. For
example, {\tt o1.toString()} will return a {\tt String} that in some
sense describes~{\tt o1}.

Because {\tt OneRowNim} is a subclass of {\tt Object}, it inherits the
{\tt toString()} method.  To illustrate the default behavior of {\tt
toString()}, let's use it with a {\tt OneRowNim} instance:

\begin{jjjlisting}
\begin{lstlisting}
  OneRowNim g1 = new OneRowNim(11);
  OneRowNim g2 = new OneRowNim(13);
  System.out.println(g1.toString());
  System.out.println(g2.toString());
\end{lstlisting}
\end{jjjlisting}

\noindent This code segment creates two {\tt OneRowNim} instances, one
named {\tt g1} and the other named {\tt g2}. The inherited {\tt
toString()} method is then invoked on each {\tt OneRowNim} instance, which
produces the following output:

\begin{jjjlisting}
\begin{lstlisting}
  OneRowNim@1dc6077b
  OneRowNim@1dc60776
\end{lstlisting}
\end{jjjlisting}

\noindent What this experiment shows is that the default
%\begin{marginalnote}\it Default {\tt toString()}\end{marginalnote}
%���for pagination\marginnote{\vspace{12pt}Default {\tt toString()}}
definition of {\tt toString()} returns some kind of internal
representation of its object.  It looks as if it returns the name of the
object's class concatenated with its memory address.  This may be
useful for some applications.   But for most objects we will want to
override the default definition to make the {\tt toString()}
method return a string that is more appropriate for {\tt OneRowNim}.

What {\tt String} should the {\tt g1.toString()} method return?
%\begin{marginalnote}\it Overriding a method\end{marginalnote}
%���for pagination\marginnote{Overriding a method}
Let's have it return a {\tt String} that reports the {\tt OneRowNim}
instances's current state, which are the values stored in the two
instance variables.  To override a method, you simply define a method
with the same signature in the subclass.  If you call {\tt toString()}
with an instance of the subclass, its version of the method will be
used. In this way, the subclass method overrides the superclass
version.  Thus, {\tt OneRowNim.toString()} will have the following
signature:

\begin{jjjlisting}
\begin{lstlisting}
public String toString();
\end{lstlisting}
\end{jjjlisting}

%\begin{figure}
%\begin{graphic}
%%%\marginfig{chptr03/p151f1.eps}{The {\tt OneRowNim} class overrides the {\tt toString()}
%%%method, which it inherits from the {\tt Object} class.
%%%\label{fig-override}
%%%\label{pg-fig-override}}
%\end{graphic}
%\end{figure}

\noindent Let us describe the state of a {\tt oneRowNim} instance very briefly
in the string returned by the {\tt toString()} method:

\begin{jjjlisting}
\begin{lstlisting}
public String toString()
{ return "nSticks = " + nSticks + ", player = " + player;
}
\end{lstlisting}
\end{jjjlisting}

\noindent If we add the {\tt toString()} method to the {\tt OneRowNim} class
and then run the program shown in Figure~\ref{fig-orntostring},
we get the following output:

\begin{jjjlisting}
\begin{lstlisting}
  nSticks = 9, player = 2
  nSticks = 13, player = 1
\end{lstlisting}
\end{jjjlisting}

%%% \JavaTIP{EFFECTIVE DESIGN}{Method Abstraction.}{Instead of duplicating
%%% code at several different places within a program, it is better to
%%% encapsulate the code in a method and call the method wherever the code
%%% is needed.  A single method is easier to maintain and extend, and the
%%% resulting program will be shorter and better structured.}

\begin{figure}[h]
\jjjprogstart
\begin{jjjlisting}
\begin{lstlisting}
public class TestToString
{
  public static void main(String argv[])
  { OneRowNim g1 = new OneRowNim(11);
    OneRowNim g2 = new OneRowNim(13);
    g1.takeSticks(2);
    System.out.println(g1.toString());
    System.out.println(g2.toString());
  } //main
} //TestToString
\end{lstlisting}
\end{jjjlisting}
\jjjprogstop{An application to test the overridden {\tt toString()} method.}
{fig-orntostring}
\end{figure}

\noindent While this new method may not play an important role in the
{\tt OneRowNim} class, it does provide a very brief, understandable
description of the state of the object.  This is the reason that the
{\tt toString()} method was included in the {\tt Object} class.

\section{Object-Oriented Design: Inheritance and Polymorphism}
\markright{{\color{cyan}SECTION\,\,\arabic{section}\,\,$\bullet$}\,\,OOD: Inheritance and Polymorphism}

\noindent This use of {\tt Object}'s {\tt toString()} method provides our first
\marginnote{Inheritance}
look at Java's inheritance mechanism and how it promotes the
generality and extensibility of the object-oriented approach.  As a
subclass of {\tt Object}, our {\tt OneRowNim} class automatically
inherits {\tt toString()} and any other {\tt public} or {\tt
protected} methods defined in {\tt Object}.  We can simply use these
methods as is, insofar as they are useful to us.  As we saw in this
case, the default version of {\tt toString()} wasn't very useful.  In
that case, we can override the method by defining a method in our
class with the exact same method signature.  The new version of {\tt
toString()} can be customized to do exactly what is most appropriate
for the subclass.

One of the great benefits of the object-oriented approach is the
ability to define a task, such as {\tt toString()}, at a very high
level in the class hierarchy and let the inheritance mechanism spread
that task throughout the rest of the hierarchy.  Because {\tt
toString()} is defined in {\tt Object}, you can invoke this method for
any Java object.  Moreover, if you override {\tt toString()} in the
classes you define, you will be contributing to its usefulness.  Two
important lessons from this example are

\JavaTIP{EFFECTIVE DESIGN}{Inheritance.}{The higher up in the class hierarchy
that a method is defined, the more widespread its use can be.}

\JavaTIP{EFFECTIVE DESIGN}{Overriding {\tt
toString()}.}{The {\tt toString()} method can be overridden in any
user defined Java class.  It is a useful thing to do in any class
where the state of an object can be defined briefly.}


\noindent Obviously there is much more that needs to be explained
about Java's inheritance mechanism.  Therefore, we will be revisiting
this topic on numerous occasions in subsequent chapters.
%\begin{figure}
%\begin{graphic}
%%%\marginfig{chptr03/p152f1.eps}{The {\tt toString()} method is polymorphic because its
%%%behavior depends on the object ({\tt Student} or {\tt OneRowNim})
%%%on which it is invoked.
%%%\label{fig-polymorph}
%%%\label{pg-fig-polymorph}}
%\end{graphic}
%\end{figure}

%%% %%%%%%%%%%%%%%%%%%%%%%%%%%%%%%%%%%%%%%%%%%%%%%%%%%%%%%%%%%%%%%%%%%%%%%%%%
%%%RAM We may need to sharpen this definition of polymorphism. Here are
%%% some definitions that Ralph has gathered:

%%% (2) Polymorphism - Selecting a method among several methods that
%%% have the same name on the basis of the actual types of the
%%% implicit parameters. Horstmann, Java Essentials ed 3, Glossary p. 830

%%% (3) Polymorphism - A feature of object-oriented languages in which the
%%% exact method to be invoked is determined at run time by the class
%%% of the receiving object. Arnow et. al. Into to Programming Using
%%% Java   ed 2 Glossary p. 665

%%% (4) Polymorphism is a programming language behavior where the same code
%%% expression can invoke different methods depending on the type of
%%% objects using the code. Cohoon and Davidson Java Program Design p. 455

%%%RAM The code segment shown here is an example of polymorphism, but the surrounding
%%% text doesn't emphasize the point that it's the selection of different
%%% methods that makes it so.
%%% %%%%%%%%%%%%%%%%%%%%%%%%%%%%%%%%%%%%%%%%%%%%%%%%%%%%%%%%%%%%%%%%%%%%%%%%%

Another important concept of object-oriented design is polymorphism.
The {\tt toString()} method is an example of a polymorphic method. The
term {\bf polymorphism} is from the Greek terms {\it poly}, which means
``many,'' and {\it morph}, which means ``form.'' The {\tt toString()}
method is polymorphic because it has different behavior when invoked
on different objects.

For example, suppose we design a class, {\tt Student}, as a subclass
of {\tt Object} and define its {\tt toString()} method to return the
student ID number.  Given this design, then {\tt obj.toString()} will
return a student ID if {\tt obj} is an instance of {\tt Student}, but
if it is an instance of {\tt OneRowNim}, it will return a the
description of its state that we defined above. The following code
segment illustrates this point:

\begin{jjjlisting}
\begin{lstlisting}
Object obj;           // obj can refer to any Object
obj = new Student("12345");// obj refers to a Student
System.out.println(obj.toString()); // Prints "12345"
obj = new OneRowNim(11); // obj refers to a OneRowNim
System.out.println(obj.toString());

// Prints: nSticks = 11, player = 1
\end{lstlisting}
\end{jjjlisting}

\noindent In this case, the variable {\tt obj} is used to refer to a
{\tt Student} and then to a {\tt OneRowNim} instance. This is okay
because both classes are subclasses of {\tt Object}. When {\tt
toString()} is invoked on {\tt obj}, Java will figure out what
subclass of {\tt Object} the instance belongs to and invoke the
appropriate {\tt toString()} method.
%\end{objectorienteddesign}

\section{Drawing Lines and Defining Graphical Methods (Optional)}
\markright{{\color{cyan}SECTION\,\,\arabic{section}\,\,$\bullet$}\,\,Drawing and Graphical Methods}

We used a  {\tt Graphics} object in the previous chapter to draw
rectangles and ovals in an applet window. The {\tt Graphics} class
also possesses a method for drawing a line segment. 
Problems involving drawing pictures in an applet window using a 
series of line segments can be a source of examples of defining 
useful methods and also of making good use of loops.

The {\tt Graphics} class has a public instance method with
the header:

\begin{jjjlisting}
\begin{lstlisting}
public  void  drawLine(int x1, int y1, int x2, int y2)
\end{lstlisting}
\end{jjjlisting}

\noindent The method call  {\tt g.drawLine(x1, y1, x2, y2)}
draws a line from the point $(x1,y1)$ to $(x2,y2)$ where $(x,y)$
refers to a point that is $x$ pixels from the left edge of
the area that {\tt g} is drawing in and $y$ pixels from the
top edge.  Thus {\tt g.drawLine(10, 10, 10, 60)} draws a
vertical line segment that is 50 pixels long and is 10 pixels
from the left edge of the drawing area, that is, a line
segment from the point $(10,10)$ to the point $(10,60)$.

Consider the problem of creating an applet program with a
method called {\tt drawSticks()} to draw any specified 
number of vertical line segments.
This method might be useful for an applet user interface
to the {\tt OneRowNim} game to draw the number of sticks 
at a given point in a game.  Suppose that this method must have an
{\tt int} parameter to specify the number of vertical lines to draw
and two {\tt int} parameters to specify the location of
the top endpoint of the left most line segment.  The  {\tt drawSticks()}
method will need to use a {\tt Graphics} object connected to the
applet window for drawing the line segment.  The only such 
{\tt Graphics} object available is the parameter in  the {\tt paint()}
method of an applet.  Thus the method must have a {\tt Graphics}
parameter and it will be called in the {\tt paint()} method
using the {\tt Graphics} object there
as an argument. Thus the header of the method should look like:

\begin{jjjlisting}
\begin{lstlisting}
public void drawSticks(Graphics g,int x,int y,int num)
\end{lstlisting}
\end{jjjlisting}

The length of the line segments and and the distance between 
them are not specified by parameters so we need to choose some 
fixed values for these quantities. Let us assume that the line 
segments are 10 pixels apart and 50 pixels long. We now have 
enough information to complete the definition of an applet
to solve this problem. Such a class definition is reproduced in 
Figure~\ref{fig-drawsticksprog}.

\begin{figure}[htb]
\jjjprogstart
\begin{jjjlisting}
\begin{lstlisting}
/** DrawLineApplet   demonstrates some graphics commands.
 *It draws a set of 12 vertical lines and a set of 7 lines.
 */
import java.awt.*;
import java.applet.*;

public class DrawSticksApplet extends Applet
  /** drawSticks(g,x,y,num) will draw num vertical line
   * segments.  The line segments are 10 pixels apart and
   * 50 pixels long. The top endpoint of the left most
   *line segment is at the point (x,y).
   */
  public void drawSticks(Graphics g, int x, int y, int num)
  { int k = 0;
      while (k < num)
      {   g.drawLine(x, y, x, y + 50);
          x = x + 10;
          k = k + 1;
      } // while
  } // drawSticks()

  public void paint(Graphics g)
  {   drawSticks(g, 25, 25, 12);
      g.setColor(Color.cyan);
      drawSticks(g, 25, 125, 7);
  } // paint()
} // DrawSticksApplet
\end{lstlisting}
\end{jjjlisting}
\jjjprogstop{An applet program with a method for drawing a set of sticks.}
{fig-drawsticksprog}
\end{figure}

Note that the body of {\tt drawSticks()} uses a while-loop to draw the lines
and declares and initializes a local variable to zero to use for counting
the number of lines drawn.  The statement {\tt g.drawLine(x, y, x, y + 50);}
draws a vertical line which is $50$ pixels long.  Increasing the value of 
$x$ by $10$ each time through the loop moves the next line $10$ pixels to the right.

The first call to {\tt drawSticks()} in the {\tt paint()} method draws $12$
lines with $(25,25)$ the top point of the left-most line.  The second call to
{\tt drawSticks()} will draw $7$ cyan sticks $100$ pixels lower.  Note that 
changing the color of {\tt g} before passing it as an argument to 
{\tt drawSticks()} changes the drawing color.

To run this applet, one needs the following HTML document, which specifies
the applet code as {\tt  DrawSticksApplet.class}:

\begin{jjjlisting}
\begin{lstlisting}
<HTML>
<HEAD>
<TITLE> Draw Sticks Web Page</TITLE>
</HEAD>
<BODY>
<H2> DrawSticksApplet will appear below.</H2>
<APPLET CODE = "DrawSticksApplet.class" 
              WIDTH = 400 HEIGHT = 200>
</APPLET>
</BODY>
</HTML>
\end{lstlisting}
\end{jjjlisting}

\noindent An image of the {\tt DrawSticksApplet} as it appears in a browser window
is shown in Figure~3.20.

\begin{figure}[tbh]
\figaleftscaled{chptr03/drawsticks.eps}{0.70}{The DrawSticksApplet as displayed in a
%%\figa{chptr03/drawsticks.eps}{The DrawSticksApplet as displayed in a
browser window.}
\label{fig-drawstickseps}
\end{figure}

As we have seen in this example, defining methods with parameters to
draw an object makes the code reusable and makes it possible to draw a
complex scene by calling a collection of simpler methods.  It is a
typical use of the divide-and-conquer principle.  The while-loop can
be useful in drawing almost any geometrically symmetric object.


\secSMHleft{Chapter Summary}
\secKTH{Technical Terms}
\begin{KT}
accessor method

class scope

formal parameter

if statement

if/else statement

inherit

local scope

loop structure

method overloading

method signature

mutator method

multiway selection

override

polymorphism

repetition structure

scope

selection

side effect 

while statement

while structure
\end{KT}

\secSMHtwo{Summary of Important Points}
\begin{SMBL}
\item  A {\it formal parameter\index{parameter}} is a variable in a method
declaration. It always consists of a type followed by a variable
identifier.  An {\it argument} is a value that is passed to a method
via a formal parameter when the method is invoked.   A method's
{\it parameters} constrain the type of information that can be passed to a
method.

\item  When an argument of primitive type is passed to a method, 
it cannot be modified within the method.  When an argument of
reference type is passed to a method, the object it refers to can be
modified within the method.

\item  Except for {\tt void} methods,
a {\it method invocation} or {\it method call} is an expression which
has a value of a certain type.  For example, {\tt nim.getSticks()}
returns a {\tt int} value.

\item  The {\it signature\index{signature}} of a method consists
of its name, and the number, types, and order of its formal
parameters.  A class may not contain more than one method with the same
signature.

\item  A {\it constructor\index{constructor}} is a method that is invoked
when an object is created.  If a class does not contain a
constructor method, the Java compiler supplies a {\it default
constructor}.

\item  Restricting access to certain portions of a class is a form of
{\it information hiding\index{Information Hiding Principle}}.
Generally, instance variables are hidden by declaring them
{\tt private}.  The class's {\tt public} methods make up its interface.

\item  The {\it if statement} executes a statement only if its boolean
condition is true.  The {\it if-else statement} executes one
or the other of its statements depending on the value of its boolean
condition.   {\it Multiway selection} allows one and only one of
several choices to be selected depending on the value of its boolean
condition.

\item  The {\tt while} statement is used for coding loop structures
that repeatedly execute a block of code while a boolean condition
is satisfied.

\end{SMBL}

\secANSH
\begin{ANS}
%{Exercise 3.1}
\item  A {\it method declaration} defines the method by specifying
its name, qualifiers, return type, formal parameters, and
its algorithm, thereby associating a name with a segment
of executable code.  A {\it method invocation} calls or
uses a defined method.


%{Exercise 3.2}
\item  A {\it formal parameter} is a variable in the method declaration,
whose purpose is to store a value while the method is running.  An
{\it argument} is a value that is passed to a method in place of a
formal parameter.


%{Exercise 3.3}
\item  The following code declares two instance variables for names
of players  and defines a {\tt setName()} method:

\begin{jjjlisting}
\begin{lstlisting}
private String nameOne = "Player One";
private String nameTwo = "Player Two";

public void setNames(String name1, String name2)
{    nameOne = name1;
     nameTwo = name2;
}
\end{lstlisting}
\end{jjjlisting}

\noindent Of course, there are many other appropriate names for
the variables and parameters and other initial assignments.

%{Exercise 3.4}
\item A method call that sets the names of the players of {\tt game1} is:

\begin{jjjlisting}
\begin{lstlisting}
game1.setNames("Xena","Yogi");
\end{lstlisting}
\end{jjjlisting}

%{Exercise 3.5}
\item  A constructor cannot have a return type, such as void.

%{Exercise 3.6}
\item  One definition for the method is:

\begin{jjjlisting}
\begin{lstlisting}
public OneRowNim(int sticks)
{    nSticks = sticks;
     player = 2;
}
\end{lstlisting}
\end{jjjlisting}

%{Exercise 3.7}
\item  The following would be displayed on the screen:

\begin{jjjlisting}
\begin{lstlisting}
   1
   20
   false
\end{lstlisting}
\end{jjjlisting}

%{Exercise 3.8}
\item  One definition for the method is:

\begin{jjjlisting}
\begin{lstlisting}
public int getMoves()
{   return nMoves;
}
\end{lstlisting}
\end{jjjlisting}

%{Exercise 3.9}
\item  One definition for the method is:

\begin{jjjlisting}
\begin{lstlisting}
public boolean playerOneIsNext()
{   return (player == 1);
}
\end{lstlisting}
\end{jjjlisting}

%{Exercise 3.10}
\item  See Figure~3.21.

%\begin{figure}
\marginfiglarge{chptr03/isdone.eps}{Flowchart of the if-else version of the
{\tt getStatus()} method.} {fig-getstateifelse}
%\end{figure}



%{Exercise 3.11}
\item  \verb| |

\begin{jjjlisting}
\begin{lstlisting}
if (isHeavy == true)
     System.out.println("Heavy") ;
else ;  // Error (remove this semicolon)
     System.out.println("Light");

if (isLong == true)
     System.out.println("Long") 
else   // Error (end line above with semicolon)
     System.out.println("Short");
\end{lstlisting}
\end{jjjlisting}


%{Exercise 3.12}
\item  \verb| |

\begin{jjjlisting}
\begin{lstlisting}
public String getPlayerName()
{    if (player == 1)
         return "Ann";
     else if (player == 2)
         return "Bill";
     else if (player == 3)
         return "Cal";
     else
         return "Error";
}
\end{lstlisting}
\end{jjjlisting}

%{Exercise 3.13}
\item  When passing an argument for a primitive type, a copy
of the argument's value is passed. The actual argument
cannot be changed inside the method. When passing a
reference to an object, the object can be changed within
the method.

%%{Exercise 3.14}
\item  \verb| |

\begin{jjjlisting}
\begin{lstlisting}
public int sumCubes(int min, int max)
{
    int num = min;
    int sum = 0;
    while (num <=  max) { // While num <= max
        sum = sum + num*num*num; // Add cube of num to sum
        num = num + 1;       // Add 1 to num
    } //while
    return sum;           // Return the sum
}
\end{lstlisting}
\end{jjjlisting}


\end{ANS}

\secEXRHtwo{Exercises}
\begin{EXRtwo}
\item  Fill in the blanks in each of the following sentences:

\begin{EXRtwoLL}
\baselineskip=14pt\item  When two different methods have the same name, this is
an example of \rule{30pt}{0.5pt}\,.
\item  Methods with the same name are distinguished by their
\rule{30pt}{0.5pt}\,.

\marginnote{\raggedright\vspace{9pt}{\bf Note:} For programming
exercises, {\bf first} draw a UML class diagram describing all classes
and their inheritance relationships and/or associations.}

\item  A method that is invoked when an object is created is
known as a \rule{30pt}{0.5pt} method.
\item  A method whose purpose is to provide access to an object's
instance variables is known as an \rule{30pt}{0.5pt} method.
\item  A {\tt boolean} value is an example of a \rule{30pt}{0.5pt} type.
\item  A {\tt OneRowNim} variable is an example of a \rule{30pt}{0.5pt} type.
\item  A method's parameters have \rule{30pt}{0.5pt} scope.
\item  A class's instance variables have \rule{30pt}{0.5pt} scope.
\item  Generally, a class's instance variables should have
\rule{30pt}{0.5pt} access.
\item  The methods that make up an object's interface should have
\rule{30pt}{0.5pt} access.
\item  A method that returns no value should be declared  \rule{30pt}{0.5pt}\,.
\item  Java's if statement and if-else statement are both examples of
\rule{30pt}{0.5pt}
control structures.
\item  An expression that evaluates to either {\tt true} or {\tt false} is
known as a  \rule{30pt}{0.5pt}\,.
\item  In an if-else statement, an else clause matches  \rule{30pt}{0.5pt}\,.
\item  The ability to use a superclass method in a subclass is due to Java's
  \rule{30pt}{0.5pt} mechanism.
\item  The process of redefining a superclass method in a subclass is
known as  \rule{30pt}{0.5pt} the method.
\end{EXRtwoLL}

\baselineskip=11pt\item  Explain the difference between the following pairs of
concepts:

\begin{EXRtwoLL}
\item  {\it Parameter} and {\it argument}.
\item  {\it Method definition} and {\it method invocation}.
\item  {\it Local scope} and {\it class scope}.
\item  {\it Primitive type} and {\it reference type}.
\item  {\it Access method} and {\it constructor method}.
\end{EXRtwoLL}


\item  Translate each of the following into Java code:
\begin{EXRtwoLL}
\item %
If b1 is true, then print ``one''; otherwise, print ``two''.
%
\item %
If b1 is false and if b2 is true,  then print ``one''; otherwise, print ``two''.
%
\item %
If b1 is false and if b2 is true,  then print ``one'';
otherwise, print ``two'', or print ``three''.
%
\end{EXRtwoLL}

\item  Identify and fix the syntax errors in each of the following:

\begin{EXRtwoLL}
\item \mbox{ }

\begin{jjjlisting}
\begin{lstlisting}
if (isWalking == true) ;
    System.out.println("Walking");
else
    System.out.println("Not walking");
\end{lstlisting}
\end{jjjlisting}

\item \mbox{ }

\begin{jjjlisting}
\begin{lstlisting}
if (isWalking)
     System.out.println("Walking")
else
     System.out.println("Not walking");
\end{lstlisting}
\end{jjjlisting}

\item \mbox{ }

\begin{jjjlisting}
\begin{lstlisting}
if (isWalking)
     System.out.println("Walking");
else
     System.out.println("Not walking")
\end{lstlisting}
\end{jjjlisting}

\item \mbox{ }

\begin{jjjlisting}
\begin{lstlisting}
if (isWalking = false)
     System.out.println("Walking");
else
     System.out.println("Not walking");
\end{lstlisting}
\end{jjjlisting}

\end{EXRtwoLL}

\item  For each of the following, suppose that {\tt isWalking}
is {\tt true} and {\tt isTalking} is {\tt false} (first
draw a flowchart for each statement and then determine
what would be printed by each statement):

\begin{EXRtwoLL}
\item \mbox{ }

\begin{jjjlisting}
\begin{lstlisting}
if (isWalking == false)
     System.out.println("One");
     System.out.println("Two");
\end{lstlisting}
\end{jjjlisting}

\item \mbox{ }

\begin{jjjlisting}
\begin{lstlisting}
if (isWalking == true)
     System.out.println("One");
     System.out.println("Two");
\end{lstlisting}
\end{jjjlisting}

\item \mbox{ }

\begin{jjjlisting}
\begin{lstlisting}
if (isWalking == false)
{
     System.out.println("One");
     System.out.println("Two");
}
\end{lstlisting}
\end{jjjlisting}

\item \mbox{ }

\begin{jjjlisting}
\begin{lstlisting}
if (isWalking == false)
     if (isTalking == true)
         System.out.println("One");
     else
         System.out.println("Two");
else
     System.out.println("Three");
\end{lstlisting}
\end{jjjlisting}

\end{EXRtwoLL}

\item  Show what the output would be if the following version of
{\tt main()}  were executed:

\begin{jjjlisting}
\begin{lstlisting}
public static void main(String argv[])
{
     System.out.println("main() is starting");
     OneRowNim game1;
     game1  = new OneRowNim(21);
     OneRowNim game2;
     game2 = new OneRowNim(8);
     game1.takeSticks(3);
     game2.takeSticks(2);
     game1.takeSticks(1);
     game1.report();
     game2.report();
     System.out.println("main() is finished");
}
\end{lstlisting}
\end{jjjlisting}

\item  Determine the output of the following program:

\begin{jjjlisting}
\begin{lstlisting}
public class Mystery
{
     public String myMethod(String s)
     {
         return("Hello" + s);
     }
     public static void main(String argv[])
     {
         Mystery mystery = new Mystery();
         System.out.println( mystery.myMethod(" dolly");
     }
}
\end{lstlisting}
\end{jjjlisting}

\item  Write a {\tt boolean} method---a method that returns
a {\tt boolean}---that takes an {\tt int} parameter and converts the
integers 0 and 1 into {\tt false} and {\tt true}, respectively.

\item  Define an {\tt int} method that takes a {\tt boolean}
parameter.  If the parameter's value is {\tt false}, the method
should return 0; otherwise, it should return~1.

\item  Define a {\tt void} method named {\tt hello} that takes a
single {\tt boolean} parameter.  The method should print ``Hello''
if its parameter is true; otherwise, it should print ``Goodbye''.


\item  Define a method named {\tt hello} that takes a
single {\tt boolean} parameter.  The method should return ``Hello'' if
its parameter is true; otherwise it should return ``Goodbye''.  Note the
difference between this method and the one in the previous
exercise.  This one returns a {\tt String}. That one was a {\tt void}
method.

\item  Write a  method named {\tt hello} that takes a
single {\tt String} parameter.  The method should return a {\tt String}
that consists of the word ``Hello'' concatenated with the value of its
parameter.  For example, if you call this method with the expression
{\tt hello("dolly")}, it should return ``hello dolly''. If you call it
with {\tt hello("young lovers wherever you are")}, it should return
``hello young lovers wherever you are''.

\item  Define a void method named {\tt day1} that prints
``a partridge in a pear tree''.

\item  Write a Java application program called {\tt TwelveDays}
that prints the Christmas carol ``Twelve Days of Christmas.'' For this
version, write a void method named {\tt intro()} that takes a single
{\tt String} parameter that gives the day of the verse and prints the
intro to the song.  For example, {\tt intro("first")} should print, ``On
the first day of Christmas my true love gave to me''. Then write
methods {\tt day1()}, {\tt day2()}, and so on, each of which prints
its version of the verse.  Then write a {\tt main()} method that calls
the other methods to print the whole song.

\item  Define a {\tt void} method named {\tt verse} that takes
two {\tt String} parameters and returns a verse of the Christmas
carol ``Twelve Days of Christmas.'' For example, if you call this
method with {\tt verse("first", "a\break partridge in a pear tree")}, it
should return, ``On the first day of Christmas my true love gave to
me, a partridge in a pear tree''.

\item  Define a {\tt void} method named {\tt permute},
which takes three {\tt String} parameters and prints out all possible
arrangements of the three strings.  For example, if you called
{\tt permute("a", "b", "c")}, it would produce the following output: abc,
acb, bac, bca, cab, cba, with each permutation on a separate line.

\item  Design a method that can produce limericks given a
bunch of rhyming words. That is, create a limerick template that will
take any five words or phrases and produce a limerick.  For example,
if you call

\begin{jjjlisting}
\begin{lstlisting}
limerick("Jones","stones","rained","pained","bones");
\end{lstlisting}
\end{jjjlisting}

\noindent your method might print (something better than)

\begin{jjjlisting}
\begin{lstlisting}
There once a person named Jones
Who had a great liking for stones,
But whenever it rained,
Jones' expression was pained,
Because stones weren't good for the bones.
\end{lstlisting}
\end{jjjlisting}

\noindent For each of the following exercises, write a complete
Java application program:

\item  Define a class named {\tt Donor} that  has two
instance variables, the donor's name and rating, both of which are
{\tt String}s.  The name can be any string, but the rating should be
one of the following values: ``high,'' ``medium,'' or ``none.''  Write
the following methods for this class: a constructor, {\tt
Donor(String,String)}, that allows you to set both the donor's name
and rating; and access methods to set and get both the name and rating
of a donor.

\item  {\bf Challenge.} Define a {\tt CopyMonitor} class that solves the
following problem.  A company needs a monitor program to keep track of
when a particular copy machine needs service.   The device has two
important (boolean) variables: its toner level (too low or not) and
whether it has printed more than 100,000 pages since its last
servicing (it either has or has not). The servicing rule that the
company uses is that service is needed when either 100,000 pages have
been printed or the toner is too low.  Your program should contain a
method that reports either ``service needed'' or ``service not needed''
based on the machine's state.  (Pretend that the machine has other
methods that keep track of toner level and page count.)

\item  {\bf Challenge.} Design and write an {\tt OldMacdonald} class that
sings several verses of ``Old MacDonald Had a Farm.''  Use methods to
generalize the verses.  For example, write a method named {\tt eieio()}
to ``sing'' the \mbox{``E I E I O''} part of the verse.  Write another method
with the signature {\tt hadAnX(String s)}, which sings the ``had a
duck'' part of the verse, and a method {\tt withA(String sound)} to
sing the ``with a quack quack here'' part of the verse.   Test your
class by writing a {\tt main()} method.

\secEXRHone{Additional Exercises}

\item Suppose you have an {\tt Object A}, with public methods
{\tt a()}, {\tt b()}, and private method {\tt c()}. And suppose
you have a subclass of {\tt A} named {\tt B} with methods named
{\tt b()}, {\tt c()} and {\tt d()}. Draw a UML diagram showing the
relationship between these two classes. Explain the inheritance
relationships between them and identify those methods that would
be considered polymorphic.

\item Consider the definition of the class {\tt C}.
Define a subclass of {\tt C} named {\tt B} that overrides method {\tt
m1()} so that it returns the difference between {\tt m} and {\tt n}
instead of their sum.
\begin{jjjlisting}
\begin{lstlisting}
     public class C {
         private int m;
         private int n;
         public C(int mIn, int nIn) {
             m = mIn;
             n = nIn;
         }
         public int m1() {
             return m+n;
         }
     }
\end{lstlisting}
\end{jjjlisting}

\end{EXRtwo}


%
%%%REW  BOLDFACE DEFINITIONS
%% accessor method
%% class scope
%% formal parameter
%% if statement
%% if/else statement
%% inherit
%% local scope
%% loop structure
%% method overloading
%% method signature
%% multiway selection
%% mutator method
%% override
%% polymorphism
%% repetition structure
%% scope
%% selection
%% side effect
%% while statement
%% while structure


% LocalWords:  CyberPet isEating isSleeping accessor BankAccount getVariable ch
% LocalWords:  getAccountNumber getAccountPassword setAccountBalance setName vs
% LocalWords:  setVariable getName proglist FormalParameterList str printStr
% LocalWords:  identifier's HelloWorld TestCyberPet UML noname Guinevere nStr
% LocalWords:  println setKind kStr ResultType MethodName int nonvoid myString
% LocalWords:  Fido
