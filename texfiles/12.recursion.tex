%%%  Chapter 12: Recursive Problem Solving
%%%  3rd Edition
\setcounter{SSTUDYcount}{1}
\setcounter{chapter}{11}
\chapter{Recursive Problem Solving}
\label{chapter-recursion}



\CObegin
\secCOBH{Objectives}

\noindent After studying this chapter, you will

\begin{COBL}
\item  Understand the concept of recursion.
\item  Know how to use recursive programming techniques.
\item  Have a better understanding of recursion as a
problem-solving technique.
\end{COBL}

\secCOLH{Outline}
{\bf \large Outline}

\begin{COL}
\item {Introduction}
\item {Recursive Definition}
\item {Recursive String Methods}
\item {Recursive Array Processing}
\item {Example: Drawing (Recursive) Fractals}
\item {Object-Oriented Design: Tail Recursion}
\item {Object-Oriented Design: Recursion or Iteration?}
\item[] {{\color{cyan}Special Topic:} Exploring the Mandelbrot Set}
\item {From the Java Library: {\tt javax.swing.JComboBox}}
\item[] {Java Language Summary}
\par\small\item[] {Chapter Summary}
\par\small\item[] {Solutions to Self-Study Exercises}
\par\small\item[] {Exercises}
\end{COL}
\COend


\section{Introduction}
\noindent The pattern in Figure~12.1 is known as the Sierpinski
gasket.  Its overall shape is that of an equilateral triangle.  But
notice how inside the outer triangle there are three smaller
triangles that are similar to the overall pattern.  And inside each of
those are three even smaller triangles, and so on.   The Sierpinski
gasket is known as a {\it fractal} because when you divide it up, you
end up with a smaller version of the overall pattern.  The overall
gasket pattern is repeated over and over, at smaller and smaller
scales, throughout the figure.
%\begin{figure}
%\begin{figure}[tb]
\marginfig{chptr12/sierpinski1.eps}{The Sierpinski gasket.}
  {fig-sierpinski1}
%\end{figure}
%\end{figure}


How would you draw this pattern?  If you try to use some kind of
nested loop structure, you'll find that it is very challenging.   It
can be done using loops, but it isn't easy.  On the other hand, if you
use an approach known as {\it recursion}, this problem is much easier
to solve.  It's a little bit like the representation issue we discussed
%%%REWin Chapter~\ref{chapter-models}. Your ability to solve a problem often
in Chapter~5. Your ability to solve a problem often
depends on how you represent the problem.  Recursion gives you another
way to approach problems that involve repetition, such as the problem
of drawing the Sierpinski gasket.

The main goal of this chapter is to introduce recursion as both a
problem-solving technique and as alternative to loops
%%%REW(Chapter~\ref{chapter-loops}) for implementing repetition.  We begin
(which we discussed in Chapter~6) for implementing repetition.  We
begin with the notion of a {\it recursive definition}, a concept used
widely in mathematics and computer science.  We then introduce the
idea of a {\it recursive method}, which is the way recursion is implemented
in a program.  

Recursion is a topic that is taken up in considerable detail in
upper-level computer science courses, so our goal here is mainly to
introduce the concept and give you some idea of its power as a
problem-solving approach.  To illustrate recursion, we use a number of
simple examples throughout the chapter.  One risk in using simple
examples, though, is that you might be tempted to think that recursion
is only good for ``toy problems.'' Nothing could be further from the
truth.  Recursion is often used for some of the most difficult
algorithms.  Some of the exercises at the end of the chapter are
examples of more challenging problems.


\subsection{Recursion as Repetition}
\noindent A {\bf recursive method}\index{recursive method} is a method that calls itself.  An {\bf
iterative method}\index{iterative method} is a method that uses a loop to repeat an action.  In
one sense, {\it recursion} is an alternative to the iterative
(looping) control structures we studied in
%%%REWChapter~\ref{chapter-loops}.  In this sense, recursion is just another
Chapter~6.  In this sense, recursion is just another
way to repeat an action.

For example, consider the following iterative method for
%\begin{marginalnote}\it Iterative method\end{marginalnote}
\marginnote{Iterative method}
saying ``Hello'' {\it N} times:

\begin{jjjlisting}
\begin{lstlisting}
public void hello(int N)  {
    for (int k = 0; k < N; k++)
        System.out.println("Hello");
} // hello()
\end{lstlisting}
\end{jjjlisting}

\noindent A recursive version of this method would be defined
%\begin{marginalnote}\it Recursive method\end{marginalnote}
\marginnote{Recursive method}
as follows:

\begin{jjjlisting}
\begin{lstlisting}
public void hello(int N)  {
    if (N > 0) \verb!{!
        System.out.println("Hello");
        hello(N - 1);                // Recursive call
} // hello()
\end{lstlisting}
\end{jjjlisting}

\noindent This method is recursive because it calls itself when {\it
N} is greater than 0. However, note that when it calls itself, it
passes $N-1$ as the value for its parameter.  If this method is
initially called with {\it N} equal to 5, the following is a trace of
what happens.  The indentations indicate each time the method calls
itself:

\begin{jjjlisting}
\begin{lstlisting}
hello(5)
    Print "Hello"
    hello(4)
        Print "Hello"
        hello(3)
            Print "Hello"
            hello(2)
                Print "Hello"
                hello(1)
                    Print "Hello"
                    hello(0)
\end{lstlisting}
\end{jjjlisting}

\noindent Thus, ``Hello'' will be printed five times, just as
it would be in the iterative version of this method.

So, in one sense, recursion is just an alternative to iteration.  In
fact, there are some programming languages, such as the original
versions of LISP and PROLOG, that do not have loop structures.  In
these languages, {\it all} repetition is done by recursion.  In
contrast, if a language contains loop structures, it can do without
recursion.  Anything that can be done iteratively can be done
recursively, and vice versa.

Moreover, it is much less efficient to call a method five times than to
repeat a for loop five times.  Method calls take up more memory than
loops and involve more {\bf computational overhead}\index{computational overhead}---for such tasks
%\begin{marginalnote}\it Computational overhead\end{marginalnote}
\marginnote{Computational overhead}
as passing parameters, allocating storage for the method's local
variables, and returning the method's results.   Therefore, because of
its reliance on repeated method calls, recursion is often less
efficient than iteration as a way to code a particular algorithm.


\JavaTIP{EFFECTIVE DESIGN}{Efficiency.}{Iterative algorithms and
methods are generally more efficient than recursive algorithms
that do the same thing.}

\pagebreak
\secEXRHone{Self-Study Exercises}
\begin{SSTUDY}
\item  What would be printed if we call the following method with
the expression {\tt mystery(0)}? 

\begin{jjjlisting}
\begin{lstlisting}
public void mystery(int N) {
    System.out.print(N + " ");
    if (N <= 5)
        mystery(N + 1);
} // mystery()
\end{lstlisting}
\end{jjjlisting}

\noindent What about {\tt mystery(100)}?

\item  What would be printed if we call the following method with
the expression {\tt mystery(5)}?

\begin{jjjlisting}
\begin{lstlisting}
public void mystery(int N) {
    System.out.print(N + " ");
    if (N <= 5)
        mystery(N - 1);
} // mystery()
\end{lstlisting}
\end{jjjlisting}

\end{SSTUDY}


\subsection{Recursion as a Problem-Solving Approach}
\noindent Given that recursion is not really necessary---if a
programming language has loops---and is not more efficient than loops,
why is it so important?  The answer is that, in a broader sense,
recursion is an effective approach to problem solving.  It is a way of
viewing a problem.  It is mostly in this sense that we want to study
recursion.

Recursion is based on two key problem-solving concepts: {\it
divide and conquer}\index{divide and conquer} and {\bf
self-similarity}\index{self-similarity}.  In recursive problem solving
we use the divide-and-conquer strategy repeatedly to break a big
problem into a sequence of smaller and smaller problems until we
arrive at a problem that is practically trivial to solve.

What allows us to create this series
%\begin{marginalnote}\it Subproblems\end{marginalnote}
\marginnote{Subproblems}
of subproblems is that each
subproblem is similar to the original problem---that is, each
subproblem is just a smaller version of the original problem.   Look
again at the task of saying ``Hello'' {\it N} times.  Solving this task
involves solving the similar task of saying ``Hello'' $N-1$ times,
which can be divided into the similar task~of saying ``Hello'' $N-2$
times.  And so on.

The ability to see a problem as being composed of smaller,
self-similar problems is at the heart of the recursive approach.  And
%\begin{marginalnote}\it Self-similarity\end{marginalnote}
\marginnote{Self-similarity}
although you might not have thought about this before, a surprising
number of programming problems have this self-similarity
characteristic.   Let's illustrate these ideas with some simple
examples.


\JavaTIP{PROGRAMMING TIP}{Divide and Conquer.}{Many programming problems
can be solved by dividing them into smaller, simpler problems.   For
recursive solutions, finding the key to the subproblem often holds
the solution to the original problem.}

\section{Recursive Definition}
\noindent One place you might have already seen recursion is in
mathematics.  A {\bf recursive definition}\index{recursive definition}
consists of two parts: a recursive part in which the {\it n}th value
is defined in terms of the $(n-1)$st value, and a nonrecursive, {\em
boundary} or {\bf base case}, which defines a limiting condition.

\subsection{Factorial: {\it N}!}
\noindent For example, consider the problem of calculating the
factorial\index{factorial} of {\it n}---that is, {\it n}! for $n
\geq 0$. As you may recall, {\it n}! is calculated as follows:

\begin{jjjlisting}
\begin{lstlisting}
n! = n * (n-1) * (n-2) * ... * 1, for n > 0
\end{lstlisting}
\end{jjjlisting}

\noindent In addition, 0! is defined as 1. Let's now look at some
examples for different values of {\it n}:

\begin{jjjlisting}
\begin{lstlisting}
4! = 4 * 3 * 2 * 1 = 24
3! = 3 * 2 * 1 = 6
2! = 2 * 1 = 2
1! = 1
0! = 1
\end{lstlisting}
\end{jjjlisting}

\noindent As these examples suggest, {\it n}! can always be calculated
in terms of $(n-1)$!  This relationship might be clearer if we rewrite
the previous calculations as follows:

\begin{jjjlisting}
\begin{lstlisting}
4! = 4 * 3 * 2 * 1 = 4 * 3! = 24
3! = 3 * 2 * 1     = 3 * 2! = 6
2! = 2 * 1         = 2 * 1! = 2
1!                 = 1 * 0! = 1
0!                 = 1
\end{lstlisting}
\end{jjjlisting}

\noindent The only case in which we can't calculate {\it n}! in terms of
$(n-1)$! is when {\it n} is 0. Otherwise, in each case we see that

\begin{jjjlisting}
\begin{lstlisting}
n! = n * (n-1)!
\end{lstlisting}
\end{jjjlisting}

\noindent This leads to the following recursive definition:

\begin{jjjlisting}
\begin{lstlisting}
n! = 1          if n = 0    // Boundary (or base) case
n! = n * (n-1)! if n > 0    // Recursive case
\end{lstlisting}
\end{jjjlisting}

\noindent   Note
that if we had omitted the base case, the recursion would have
continued to $(-1)$! and $(-2)$! and so on.


\JavaTIP[false]{DEBUGGING TIP}{Bounding the Repetition.}{An infinite repetition
will result if a recursive definition is not properly bounded.}

\noindent The recursive\index{recursion, recursive case} case uses
divide and conquer to break the problem into a smaller problem, but
the smaller problem is just a smaller version of the original
problem.  This combination of self-similarity and divide and conquer is
what characterizes recursion.  The base\index{recursion, base case}
case is used to stop or limit the recursion.

\JavaTIP{EFFECTIVE DESIGN}{Recursive Definition.}{For recursive
algorithms and definitions, the base case serves as the bound for the
algorithm.  The {\bf recursive case}\index{recursive case} defines the
{\it n}th case in terms of the $(n-1)$st case.}


\subsection{Drawing a Nested Pattern}
%%%\marginnote{Self-similarity}
\noindent As another example, consider the problem of drawing the nested boxes
pattern in Figure~12.2. The self-similarity occurs in the
fact that for this pattern, its parts resemble the
whole.  The basic shape involved is a square, which is repeated over
and over at an ever-smaller scale.  A recursive definition for this
\marginfig{chptr12/boxes.eps}{The nested squares pattern.}
{fig-boxes}
pattern would be

\begin{jjjlisting}
\begin{lstlisting}
Base case:      if side < 5 do nothing
Recursive case: if side >= 5
                  draw a square
                  decrease the side and draw a smaller 
                             pattern inside the square
\end{lstlisting}
\end{jjjlisting}

\noindent This definition uses the length of the square's side
to help define the pattern.  If the length of the side is greater than or
equal to 5, draw a square with dimensions $side \times side$. Then
decrease the length of the side and draw a smaller version of the
%%%\marginnote{Smaller subpattern}
pattern inside that square.  In this case, the {\it side} variable will
decrease at each level of the drawing.  When the length of the side
becomes less than 5, the recursion stops.  Thus, the length of the side
serves as the limit or bound for this algorithm.

You should note that the length of the side functions here like a
%\begin{marginalnote}\it The side as a parameter\end{marginalnote}
%%%\marginnote{\vspace{-12pt}The side as a parameter}
parameter in a method definition: It provides essential information
for the definition, just as a method parameter provides essential data
to the method.  Indeed, this is exactly the role that parameters play
in recursive methods.  They provide essential information that
determines the method's behavior.

\pagebreak
Figure~12.3 illustrates how we would apply the definition.
\marginfig{chptr12/nestedbox.eps}{A trace of the nested boxes
definition starting with a side of 20 and decreasing the
side by 5 each time.}
{fig-nestedbox2}
Suppose the side starts out at 20 and decreases by 5 at each level of
recursion.  Note that as you move from top to bottom across the four
patterns, each pattern contains the one below it.  A {\tt
nestedBoxes(20)} can be drawn by first drawing a $20 \times 20$ square
and then drawing a {\tt nestedBoxes(15)} pattern inside it.
Similarly, a {\tt nestedBoxes(15)} can be drawn by first drawing a $15
\times 15$ square and then drawing a {\tt nestedBoxes(10)} pattern
inside it.  And so on.

These examples illustrate the power of recursion\index{recursion} as a
%\begin{marginalnote}\it Recursive vs.  iterative bound\end{marginalnote}
%���\marginnote{\vspace{6pt}Recursive vs.  iterative bound}
problem-solving technique for situations that involve repetition.
Like the iterative (looping) control structures we studied in
%%%REWChapter~\ref{chapter-loops}, recursion is used to implement repetition
Chapter~6, recursion is used to implement repetition
within a bound.   For recursive algorithms, the bound is defined by the
base case, whereas for loops, the bound is defined by the loop's entry
condition.   In either case, repetition stops when the bound is
reached.

\secEXRHone{Self-Study Exercises}
\begin{SSTUDY}

\item  You can calculate $2^n$ by multiplying 2 by itself {\it n} times.
For example, $2^3$ is $2 \times 2 \times 2$. Note also
that $2^0=1$. Given these facts, write a recursive definition
for $2^n$, for $n \geq 0$.


\item  Generalize your solution to the previous exercise by giving
a recursive definition for $x^n$, where $x$ and $n$ are both
integers $\geq 0$.


\item  Is the recursive definition given earlier for the nested boxes
equivalent to the following recursive definition?  Explain.

\begin{jjjlisting}
\begin{lstlisting}
Draw a square.         // in every case
If side > 5
    draw a smaller nested boxes inside the square
\end{lstlisting}
\end{jjjlisting}

\noindent In this case, the base case $(side <= 5)$ is implicit.


\item  Write a recursive definition for the recursive pattern shown in
Figure~12.4.


\JavaTIP{DEBUGGING TIP}{Infinite Recursion.}{An unbounded or incorrectly
bounded recursive algorithm will lead to infinite repetition.  Care
must be taken to get the bound right.}

\end{SSTUDY}
%%\marginfigvspace{-80pt}{chptr12/circlebox.eps}{Write a


\section{Recursive String Methods}
\noindent Remember that a {\em recursive method} is a method that
calls itself.  Like recursive definitions, recursive methods are
designed around the divide-and-conquer and self-similarity principles.
Defining a recursive method involves a similar analysis to the one we
used in designing recursive definitions.  We identify a self-similar
subproblem of the original problem plus one or more limiting cases.

The idea of a method calling itself seems a bit strange at first.  It's
\marginfig{chptr12/circlebox.eps}{Write a
recursive definition for this pattern.}
{fig-exercise}
perhaps best understood in terms of a clone or a copy.  When a method
calls itself, it really calls a copy of itself, one that has a
slightly different internal state.  Usually the difference in state is
the result of a difference in the invoked method's parameters.
\marginnote{How can a method call itself?}

\subsection{Printing a String}
\noindent To illustrate the concept of a recursive method, let's define a
recursive method for printing a string.  This is not intended to be a
practical method---we already have the {\tt println()} method for
printing strings.   But pretend for a moment that you only have a
version of {\tt println()} that works for characters, and your task is
to write a version that can be used to print an entire string of
characters.

A little terminology will help us describe the algorithm.  Let's call
the first letter of a string the {\it head} of the string, and let's
refer to all the remaining letters in the string as the {\it tail} of
the string.  Then the problem of printing a string can be solved using
a {\bf head-and-tail algorithm}\index{head-and-tail algorithm}, 
\marginnote{Head-and-tail algorithm}
which consists of two parts: printing
the head of the string and recursively printing its tail.  The
limiting case here is when a string has no characters in it.  It's
trivial to print the empty string---there is nothing to print!  This
leads to the method definition shown in Figure~\ref{fig-printstring}.

\begin{figure}[htb]
\jjjprogstart
\begin{jjjlisting}
\begin{lstlisting}
 /**
  * printString() prints each character of the string s
  * Pre: s is initialized (non-null)
  * Post: none
  */
public void printString(String s) {
    if (s.length() == 0)
        return;                  // Base case: do nothing
    else {        // Recursive case: 
        System.out.print(s.charAt(0)); // Print head
        printString(s.substring(1));   // Print tail
    }
} // printString()
\end{lstlisting}
\end{jjjlisting}
\jjjprogstop{The recursive {\tt printString()} method.}
{fig-printstring}
\end{figure}

The base case here provides a limit and bounds the recursion
when the length of {\it s} is 0---that is, when the string is
empty.  The recursive case solves the problem of printing {\it s} by
solving the smaller, self-similar problem of printing a substring of
{\it s}.  Note that the recursive case makes progress toward the
limit.  On each recursion, the tail will get smaller and smaller until it
becomes the empty string.

\JavaTIP{EFFECTIVE DESIGN}{Recursive Progress.}{In a recursive algorithm,
each recursive call must make progress toward the bound, or base case.}

\noindent Let's now revisit the notion of a method\index{method call, recursive}
%\begin{marginalnote}\it Recursive call\end{marginalnote}
\marginnote{Recursive call}
calling itself.  Obviously this is what happens in the recursive case,
but what does it mean---what actions does this lead to in the
program?  Each recursive call to a method is really a call to a {\it
copy} of that method, and each copy has a slightly different internal
state.  We can define {\tt printString()}'s internal state completely
in terms of its recursion parameter, {\it s}, which is the string that's being
printed.  A {\bf recursion parameter}\index{recursion parameter} is a parameter whose value is
used to control the progress of the recursion.   In this case, if
{\it s} differs in each copy, then so will {\tt s.substring(1)} and
{\tt s.charAt(0)}.

\JavaTIP{EFFECTIVE DESIGN}{Recursion and Parameters.}{Recursive methods
use a recursion parameter to distinguish between self-similar
instances of the method call.  The parameter controls the progress of
the recursion toward its bound.}

Figure~\ref{fig-traceprint} illustrates the sequence of recursive
method calls and the output that results when
\begin{figure}[htb]
\figa{chptr12/traceprint.eps}{A recursive method call invokes a copy
of the method, each with a slightly different internal state.  As
this is done repeatedly, a stack of method calls is created.}
{fig-traceprint}
\end{figure}
{\tt printString("hello")} is invoked.  Each box represents a separate
instance of the {\tt printString()} method, with its own internal
\marginnote{Self-similar instances} state.  In this illustration, its
state is represented by its parameter, {\tt s}.  Because each instance
has a different parameter, behaviors differ slightly. Each box shows
the character that will be printed by that instance ({\tt
s.charAt(0)}), and the string that will be passed on to the next
instance ({\tt s.substring(1)}).

\JavaTIP[false]{DEBUGGING TIP}{Off-by-One Error.}{The expressions {\tt s.charAt(0)}
and {\tt s.substring(1)} will generate exceptions if {\it s} is the
empty string.}

\noindent The arrows represent the method calls and returns.  Note that the
first {\tt return} executed is the one in the base case.  Each
instance of the method must wait for the instance it called to return
before it can return.  That's why the instances ``pile up'' in a
cascade-like structure.  The arrowless lines trace the order in which
the output is produced.


Each instance of {\tt printString()} is similar to the next in that
each will print one character and then pass on a substring, but each
performs its duties on a different string.  Note how the string, the
recursion parameter in this case, gets smaller in each instance of
{\tt printString()}.  \marginnote{Progress toward the bound} This
represents progress toward the method's base case {\tt s.length() ==
0}.  When the empty string is passed as an argument, the recursion
will stop.  If the method does not make progress toward its bound in
this way, the result will be an infinite recursion.


\JavaTIP{EFFECTIVE DESIGN}{Bounding the Recursion.}{For recursive
algorithms, the  recursion parameter is used to express the
algorithm's bound or base case. In order for the algorithm to
terminate, each recursive call should make progress toward the bound.}

\noindent Note also the order in which things are done in this method.
First {\tt s.charAt(0)} is printed, and then {\tt s.substring(1)} is
%\begin{marginalnote}\it Self-similarity\end{marginalnote}
\marginnote{Self-similarity}
passed to {\tt printString()} in the recursion.  This is a typical
structure for a head-and-tail algorithm. What makes this work is
that the tail is a smaller, self-similar version of the original
structure.

\JavaTIP{EFFECTIVE DESIGN}{Head-and-Tail Algorithm.}{Many recursive solutions
involve breaking a sequential structure, such as a string or an array,
into its head and tail. An operation is performed on the
head, and the algorithm recurses on the tail.}

\pagebreak
\secEXRHone{Self-Study Exercise}
\begin{SSTUDY}

\item  What would be printed by the following 
version of the {\tt printString2()}
method, if it is called with {\tt printString2("hello")}?

\begin{jjjlisting}
\begin{lstlisting}
public void printString2(String s)  {
 if (s.length() == 1)
   System.out.print(s.charAt(0));         // Base case:
 else {
                           // Print last char
   System.out.print(s.charAt(s.length() - 1)); 
                          // Print rest of string
   printString2(s.substring(0, s.length() - 1));  
 }
} // printString2()
\end{lstlisting}
\end{jjjlisting}

\end{SSTUDY}

\subsection{Printing the String Backward}
\noindent What do you suppose would happen if we reversed the order of the
statements in the {\tt printString()} method? That is, what
if the recursive call came before {\tt s.charAt(0)} is printed, as
in the following method:

\begin{jjjlisting}
\begin{lstlisting}
 /**
  * printReverse() prints each character s in reverse order
  * Pre: s is initialized (non-null)
  * Post: none
  */
public void printReverse(String s) {
  if (s.length() > 0) {      // Recursive case:
     printReverse(s.substring(1));  // Print tail
     System.out.print(s.charAt(0)); // Print first char
  }
} // printReverse()
\end{lstlisting}
\end{jjjlisting}

\noindent As its name suggests, this method will print the string in
reverse order.  The trace in Figure~\ref{fig-tracerevprint} shows how
\begin{figure}[h]
\figaleft{chptr12/printreverse.eps}{A trace of {\tt print\-Reverse(s)}, which
prints its string argument in reverse order.
} {fig-tracerevprint}
\end{figure}
this works.  Before {\tt printReverse("hello")} can print {\it h}, it calls
{\tt printReverse("ello")} and must wait for that call to complete its
execution and return.  But {\tt printReverse("ello")} calls
{\tt printReverse("llo")} and so must wait for that call to complete its
execution and return.



This process continues until {\tt printReverse("")} is called.  While
the base case is executing, the other five instances of
{\tt printReverse()} must each wait for the instance that they called to
complete its execution.  It is only after the base case returns, that
{\tt printReverse("o")} can print its first character and return.   So
the letter {\it o} will be printed first.  After {\tt printReverse("o")}
has returned, then {\tt printReverse("lo")} can print its first
character.  So the letter {\it l} will be printed next, and so on, until the
original call to {\tt printReverse("hello")} is completed and
returns.  Thus, the string will be printed in reverse order.


Note that the method call and return structure in this example follows
%\begin{marginalnote}\it Last-in--first-out protocol\end{marginalnote}
a {\bf last-in--first-out (LIFO) protocol}\index{last-in--first-out (LIFO) protocol}. 
\marginnote{Last-in--first-out protocol}
That is, the last method
called is always the first method to return.  This is the protocol used
by all method calls, recursive or otherwise.

\JavaRule[false]{LIFO.}{All programming languages, including Java,
use a last-in-first-out protocol for procedure call and return.}

\noindent For example, compare the order in which things happen in
%\begin{marginalnote}\it Method call stack\end{marginalnote}
Figure~\ref{fig-tracerevprint} with the method stack trace in
%%%REWFigure~\ref{fig-methodstack}. The only real difference between the two
Figure~10.12. The only real difference between the two
figures is that here the method stack is represented as growing
%%%REWdownward, whereas in Figure~\ref{fig-methodstack} it grows upward.
downward, whereas in Figure~10.12 it grows upward.
As each method call is made, a representation of the method call is
placed on the {\em method call stack}. 
\marginnote{Method call stack}
When a method returns, its
block is removed from the top of the stack.  The only difference
between recursive and nonrecursive method calls is that recursive
methods call instances of the same method definition.   Of course, as
we've seen, the instances are all slightly different from each other.


\secEXRHone{Self-Study Exercises}
\begin{SSTUDY}

\item  Write a recursive method called {\tt countDown()} that takes
a single {\tt int} parameter, $N \geq 0$, and prints a countdown,
such as ``5, 4, 3, 2, 1, blastoff.'' In this case, the
method would be called with {\tt countDown(5)}.


\item  Revise the method in the previous exercise so that when it's
called with {\tt countDown(10)}, it will print ``10 8 6 4 2 blastoff'';
if it's called with {\tt countDown(9)}, it prints ``9 7 5 3 1 blastoff.''

\end{SSTUDY}

\subsection{Counting Characters in a String}
\noindent Suppose you're writing an encryption program and you need to count the
%\begin{marginalnote}\it Problem statement\end{marginalnote}
\marginnote{Problem statement}
frequencies of the letters of the alphabet.  Let's write a recursive
method for this task.

This method will have two parameters: a {\tt String} to store the
string that will be processed and a {\tt char} to store the
target character---the one we want to count.   The method should
return an {\tt int}, representing the number of occurrences of the
target character in the string:

\begin{jjjlisting}
\begin{lstlisting}
 // Goal: count the occurrences of ch in s
 public int countChar(String s, char ch) {
     ...
 }
\end{lstlisting}
\end{jjjlisting}

Here again our analysis must identify a recursive step that breaks the
problem into smaller, self-similar versions of itself, plus a base
case or limiting case that defines the end of the recursive process.
Because the empty string will contain no target characters, we can use
%\begin{marginalnote}\it Base case\end{marginalnote}
\marginnote{Base case}
it as our base case.  So, if it is passed the empty string,
 {\tt countChar()} should just return 0 as its result.

For the recursive case, we can divide the string into its head and
%\begin{marginalnote}\it Recursive case\end{marginalnote}
\marginnote{Recursive case}
tail.  If the head is the target character, then the number of
occurrences in the string is (1 + the number of occurrences in its
tail). If the head of the string is not the target character, then the
number of occurrences is (0 + the number of occurrences in its tail).  Of
course, we'll use recursion to calculate the number of occurrences in the
tail.


This analysis leads to the recursive method shown in
Figure~\ref{fig-countch}. Note that for both recursive cases the
\begin{figure}[htb]
\jjjprogstart
\begin{jjjlisting}
\begin{lstlisting}
 /**
  * Pre:  s is a non-null String, ch is any character
  * Post: countChar() == the number of occurrences of ch in str
  */
public int countChar(String s, char ch) {
 if (s.length() == 0)    // Base case: empty string
   return 0;
 else if (s.charAt(0) == ch)  // Recursive case 1
   return 1 + countChar(s.substring(1), ch); // Head = ch
 else                         // Recursive case 2
   return 0 + countChar(s.substring(1), ch); // Head != ch
} // countChar()
\end{lstlisting}
\end{jjjlisting}
\jjjprogstop{The recursive {\tt count\-Char()} method.}
{fig-countch}
\end{figure}
same recursive call is used.  In both cases we pass the tail of the
original string, plus the target character.  Note also how the
return statement is evaluated:

\begin{jjjlisting}
\begin{lstlisting}
return 1 + countChar(s.substring(1),ch); // Head = ch
\end{lstlisting}
\end{jjjlisting}

\noindent Before the method can return a value, it must receive the
result of calling {\tt countChar(s.substring(1),ch)} and add it to
%\begin{marginalnote}\it Evaluation order is crucial\end{marginalnote}
\marginnote{Evaluation order is crucial}
1. Only then can a result be returned to the calling method.   This
leads to the following evaluation sequence for
{\tt countChar("dad",'d')}:

\begin{jjjlisting}
\begin{lstlisting}
countChar("dad",'d');
1 + countChar("ad",'d');
1 + 0 + countChar("d",'d');
1 + 0 + 1 + countChar("",'d');
1 + 0 + 1 + 0 = 2              // Final result
\end{lstlisting}
\end{jjjlisting}

\noindent In this way, the final result of calling {\tt
countChar("dad",'d')} is built up recursively by adding together the
partial results from each separate instance of {\tt countChar()}. The
evaluation process is shown graphically in Figure~12.9.

\begin{figure}[htb]
\figaleft{chptr12/tracecount.eps}{A trace of
{\tt countChar("dad",'d')}, which
returns the value 2.
} {fig-tracecount}
\end{figure}

\JavaTIP[false]{DEBUGGING TIP}{Return Type.}{A common error with nonvoid
recursive algorithms is forgetting to make sure that those return
statements that contain a recursive call yield the correct data type.}


\secEXRHone{Self-Study Exercise}
\begin{SSTUDY}

\item  Here's a numerical problem.  Write a recursive method to compute
the sum of 1 to {\it N}, given {\it N} as a parameter.

\end{SSTUDY}

\subsection{Translating a String}
\noindent A widely used string-processing task is to convert one string into
another string by replacing one character with a substitute throughout
the string.  For example, suppose we want to convert a Unix path name,
%\begin{marginalnote}\it Problem statement\end{marginalnote}
which uses the forward slash ``/'' to separate one part of the path from
another, into a Windows path name, which uses the backslash character
\marginnote{Problem statement}
``$\backslash$'' as a separator.   For example, we want a method
that can translate the following two strings into one another:

\begin{jjjlisting}
\begin{lstlisting}
/unix_system/myfolder/java
\Windows_system\myfolder\java
\end{lstlisting}
\end{jjjlisting}

Thus, we want a method that takes three parameters: a {\tt String},
on which the conversion will be performed, and two {\tt char}
%\begin{marginalnote}\it Method design\end{marginalnote}
\marginnote{Method design}
variables, the first being the original character in the string and
the second being its substitute.   The precondition for this method is
simply that each of these three parameters has been properly
initialized with a value.  The postcondition is that all occurrences of
the first character have been replaced by the second character.

As in our previous
%\begin{marginalnote}\it Head-and-Tail Algorithm\end{marginalnote}
\marginnote{Head-and-tail algorithm}
string-processing methods, the limiting case in
this problem is the empty string, and the recursive case will divide
the string into its head and its tail.  If the head is the character we
want to replace, we concatenate its substitute with the result we
obtain by recursively converting its tail.


This analysis leads to the definition shown in
%%%REWFigure~\ref{fig-convert}. This method has more or less the same head
Figure~12.10. This method has more or less the same head
and tail structure as the preceding example.  The difference is that
here the operation we perform on the head of the string is
concatenation rather than addition.

\begin{figure}[htb]
\jjjprogstart
\begin{jjjlistingleft}[29pc]{-3pc}
\begin{lstlisting}
 /**
  * Pre:  str, ch1, ch2 have been initialized
  * Post: the result contains a ch2 everywhere that ch1 
  *       had occurred in str
  */
public static String convert(String str, char ch1, char ch2) {
 if (str.length() == 0)         // Base case: empty string
   return str;
 else if (str.charAt(0) == ch1) // Recursive 1: ch1 at head
                                // Replace ch1 with ch2
   return ch2 + convert(str.substring(1), ch1, ch2); 
 else                       // Recursive 2: ch1 not at head
   return str.charAt(0) + convert(str.substring(1), ch1, ch2);
} // convert()
\end{lstlisting}
\end{jjjlistingleft}
\jjjprogstop{The {\tt convert()} method replaces one character with another
in a string.}
{fig-convert}
\end{figure}

The base case is still the case in which {\tt str} is the empty
string.  The first recursive case occurs when the character being
replaced is the head of {\tt str}. In that case, 
its substitute ({\tt ch2}) is concatenated with the result of
converting the rest of the string and returned as the result.   The second recursive case occurs
when the head of the string is {\it not} the character being
replaced.  In this case, the head of the string is simply concatenated
with the result of converting the rest of the
string.  Figure~\ref{fig-traceconvert} shows an example of its
execution.

\begin{figure}[h!]
\figaleft{chptr12/traceconvert.eps}{A trace of {\tt convert("bad",'d','m')},
which returns ``bam.''
} {fig-traceconvert}
\end{figure}

\pagebreak
\secEXRHone{Self-Study Exercise}
\begin{SSTUDY}

\item Write a recursive method that changes
each blank in a string into two consecutive blanks, leaving the rest
of the string \mbox{unchanged.}

\end{SSTUDY}

\subsection{Printing all Possible Outcomes when Tossing N Coins}

\noindent Suppose that a student who is studying probability wishes to have a Java 
program that, for any positive integer $N$, will print out a list of all possible 
outcomes when $N$ coins are tossed.  For purposes of analyzing the problem, it is 
assumed that the coins are numbered $1$ through $N$ and that they are tossed one 
at a time.  An outcome will be represented by a string of {\tt H}s and {\tt T}s 
corresponding to heads and tails.  Thus, if $N = 2$, the string {\tt HT}  represents 
a head on the first  coin and a tail on the second coin.  What we need is a method 
\marginnote{A coin tossing experiment}
which, when given the number of coins, will print out all strings of this type.  
In case of two coins the output should be:

\begin{jjjlisting}
\begin{lstlisting}
HH
HT
TH
TT
\end{lstlisting}
\end{jjjlisting}

Let's devise a strategy, given any positive integer $N$, for printing the strings 
\marginnote{Designing a recursive algorithm}
that correspond to all possible outcomes when tossing $N$ coins. Clearly, for $N = 1$, 
the method needs to print an {\tt H} on one line and a {\tt T} on the next line.  For 
an arbitrary number of coins $N$, one way to generate all outcomes is to think of 
two kinds of strings---those that start with an {\tt H} and those that start with a {\tt T}.  
The strings that start with {\tt H} can be generated by inserting an {\tt H} in front of 
each of the outcomes that occur when $N-1$ coins are tossed.  The strings beginning with 
{\tt T} can be generated in a similar manner.  Thus, using the outcomes for two coins 
above, we know that the outcomes for three coins for which the first coin is {\tt H} are:

\begin{jjjlisting}
\begin{lstlisting}
HHH
HHT
HTH
HTT
\end{lstlisting}
\end{jjjlisting}

Using an argument similar to the one above, we can generalize this to
a description of the recursive case for an algorithm. We want an
algorithm that generates all those outcomes for $N$ coins where we are
given a string {\tt STR} representing one particular outcome for all
but the last $K$ coins where $0 < K <= N$.  To print out all such
outcomes, just print all 
outcomes with a fixed outcome corresponding to {\tt STR + "H"} for all
but the last $K - 1$ coins and then print all outcomes with the fixed
outcome {\tt STR + "T"} for all but the last $K - 1$ coins.  The base
case is the special case $K = 1$ when you just need {\tt STR
+ "H"} on one line and  {\tt STR + "T"} on the next.  If you
start the algorithm with {\tt STR = ""} and $K = N$, this algorithm
will print out all the outcomes for tossing $N$ coins.

To translate this into Java code we can create a class called CoinToss
which has a single static method called {\tt printOutcomes(String
str,int N)}. The above recursive description easily translates into
code for the method in Figure~\ref{fig-cointoss}.

\begin{figure}[htb] 
\jjjprogstart
\begin{jjjlisting}
\begin{lstlisting}
 /**
 * printOutcomes(str, N) prints out all possible outcomes  
 * beginning with str when N more coins are tossed.
 * Pre: str is a string of Hs and Ts.
 * Pre: N is a positive integer.
 * Post: none
 */
 public static void printOutcomes(String str,int N){
     if (N == 1){ // The base case
         System.out.println(str + "H");
         System.out.println(str + "T");
     } else {  // The recursive case
         printOutcomes(str + "H", N - 1);
         printOutcomes(str + "T", N - 1);
     } //else
}// printOutcomes()
\end{lstlisting}
\end{jjjlisting}
\jjjprogstop{The method {\tt printOutcomes()} prints all outcomes given the
results on some initial coins.}
{fig-cointoss}
\end{figure}

To print out all outcomes when tossing, say, seven coins, just make the 
method call {\tt CoinToss.printOutcomes("",7)}.  This particular call would 
generate the desired output:

\begin{jjjlisting}
\begin{lstlisting}
HHHHHHH
HHHHHHT
.......
TTTTTTH
TTTTTTT
\end{lstlisting}
\end{jjjlisting}

\noindent To better understand how the recursive method definition generates
its output, it might be helpful to trace the order of recursive calls and
output to {\tt System.out} that occurs when executing {\tt printOutcomes("",3)}
as shown in Figure~12.13.

Notice that the recursive case in the method implementation makes two
\marginfig{chptr12/cointoss.eps}{The order in which the recursive calls and 
output occur when {\tt printOutcomes("",3)} is executed.}
{fig-cointoss2}
calls to itself and as a result it is not so clear how this method
would be written using a loop instead of recursion.  This example is
more typical of the type of problem for which a recursive method is
shorter and clearer than a method that solves the same problem without
using recursion.

\secEXRHone{Self-Study Exercise}
\begin{SSTUDY}

\item  Modify the above  {\tt printOutcomes()} method 
so that it will print out all possible outcomes when a chess player plays a series 
of $N$ games.  The outcome of each game is to be represented by a {\tt W}, {\tt L}, 
or {\tt D} corresponding to the player winning, losing, or drawing the game.
\end{SSTUDY}  

\pagebreak
\section{Recursive Array Processing}

\noindent Like strings, arrays also have a recursive structure.  Just as each
substring of a string is similar to the string as a whole, each
portion of an array is similar to the array as a whole.  Similarly,
just as a string can be divided into a head and a tail, an array can
be divided into its {\it head}, the first element, and its {\it tail},
%\begin{figure}
%\begin{graphic}
\marginfig{chptr12/headtail.eps}{An array of {\tt int} is a recursive
structure whose tail is similar to the array as a whole.}
{fig-headtail}
%\end{graphic}
%\end{figure}
the rest of its elements (Fig.~12.14). Because the tail
of an array is itself an array, it satisfies the self-similarity
principle.  Therefore, arrays have all the appropriate characteristics
that make them excellent candidates for recursive processing.


\subsection{Recursive Sequential Search}
\noindent Let's start by developing a recursive version of the sequential
search algorithm that we discussed in Chapter~9.
Recall that the sequential search method takes two parameters:
%\begin{marginalnote}\it Method design\end{marginalnote}
\marginnote{Method design} the array being searched and the {\it key},
or target value, being searched for.  If the key is found in the
array, the method returns its index.  If the key is not found, the
method returns $-1$, thereby indicating that the key was not contained
in the array.  The iterative version of this method has the following
general form:

\begin{jjjlisting}
\begin{lstlisting}
 /**
  * Performs a sequential search of an integer array
  * @param arr is the array of integers
  * @param key is the element being searched for
  * @return the key's index is returned if the key is
  *  found otherwise -1 is returned
  * Pre:  arr is not null
  * Post: either -1 or the key's index is returned
  */
public int sequentialSearch(int arr[], int key) {
    return -1;           // failure if this is reached
}
\end{lstlisting}
\end{jjjlisting}

If we divide the array into its head and tail, then one way to
describe a recursive search algorithm is as follows:

\begin{jjjlisting}
\begin{lstlisting}
If the array is empty, return -1
If the array's head matches the key, return its index
If the array's head doesn't match the key,
 return the result of searching the tail of the array
\end{lstlisting}
\end{jjjlisting}

\noindent This algorithm clearly resembles the approach we used
in recursive string processing: Perform some operation on the head of
the array and recurse on the tail of the array.

The challenge in developing this algorithm is not so much knowing what
%\begin{marginalnote}\it How do we represent head-and-tail?\end{marginalnote}
\marginnote{How do we represent head and tail?}
to do but knowing how to represent concepts like the head and the tail of
the array.  For strings, we had methods such as {\tt s.charAt(0)} to
represent the head of the string and {\tt s.substring(1)} to represent
the string's tail.  For an array named {\tt arr}, the expression
{\tt arr[0]} represents the head of the array.  Unfortunately, we have no
method comparable to the {\tt substring()} method for strings that
lets us represent the tail of the array.

To help us out of this dilemma, we can use an integer parameter to
represent the head of the array.  Let's have the {\tt int} parameter,
{\it head}, represent the current head of the array
(Fig.~\ref{fig-headtail2}).  Then $head+1$ represents the start of the
tail, and {\tt arr.length-1} represents the end of the tail.  Our
method will always be passed a reference to the whole array, but it
will restrict the search to the portion of the array starting at {\it
head}. If we let {\it head} vary from 0 to {\tt arr.length} on each
recursive call, the method will recurse through the array in
head/tail fashion, searching for the key.  The method will stop when {\tt head =
arr.length}. 

\begin{figure}[htb]
\figaleft{chptr12/headtail2.eps}{A
parameter, {\it head}, can represent the head of some portion of the array.
} {fig-headtail2}
\end{figure}

\JavaTIP{PROGRAMMING TIP}{Subarray Parameter.}{For methods that take an
array argument, an {\tt int} parameter can be used to designate the
portion of the array that should be processed in the method.}

\begin{figure}[tb]
\jjjprogstart
\begin{jjjlisting}
\begin{lstlisting}
  /**
   * search(arr,head,key)---Recursively search arr for key
   *  starting at head
   * Pre:  arr != null and 0 <= head <= arr.length
   * Post: if arr[k] == key for k, 0 <= k < arr.length, 
   *       return k else return -1
   */
private int search(int arr[], int head, int key)  {
 if (head == arr.length) // Base: empty list - failure
   return -1;
 else if (arr[head] == key) // Base: key found---success
   return head;
 else               // Recursive: search the tail
   return search(arr, head + 1, key);
}
\end{lstlisting}
\end{jjjlisting}
\jjjprogstop{The recursive search method takes three parameters.  The
{\it head} parameter points to the begin\-ning of that portion of the
array that is being searched.}
{fig-rsearch}
\end{figure}


This leads to the definition for recursive search shown in
Figure~\ref{fig-rsearch}.  Note that the recursive search method takes
three parameters: the array to be searched, {\tt arr}, the {\tt key}
being sought, and an integer {\tt head} that gives the starting
location for the search.   The algorithm is bounded when {\tt head =
arr.length}. In effect, this is like saying the recursion should stop
when we have reached a tail that contains 0 elements.  This underscores
%\begin{marginalnote}\it Recursion parameter\end{marginalnote}
the point we made earlier about the importance of parameters in
designing recursive methods.  Here the {\it head} parameter serves as
the {\em recursion parameter}\index{recursion parameter}. 
\marginnote{Recursion parameter}
It controls the progress of the
recursion.

Note also that for the search algorithm we need two base cases.  One
represents the successful case, where the key is found in the array.
The other represents the unsuccessful case, which comes about after we
have looked at every possible head in the array and not found the
key.  This case will arise through exhaustion---that is, when we have
exhausted all possible locations for the key.

\JavaTIP[false]{DEBUGGING TIP}{Recursive Search.}{For the recursive search method
to work properly, it must be called with the correct value for the
{\it head} parameter.}

\subsection{Information Hiding}
\noindent Note that in order to use the {\tt search()} method, you would have to
know that you must supply a value of 0 as the argument for the {\it
head} parameter.  This is not only awkward but also impractical.
After all, if we want to search an array, we just want to pass two
arguments, the array and the key we're searching for.  It's
%\begin{marginalnote}\it Design issue\end{marginalnote}
\marginnote{Design issue}
unreasonable to expect users of a method to know that they also have
to pass 0 as the head in order to get the recursion started.  This
design is also prone to error, because it's quite easy for a mistake
to be made when the method is called.

For this reason, it is customary to provide a nonrecursive interface
to the recursive method.  The interface hides the fact that a
recursive algorithm is being used, but this is exactly the kind of
implementation detail that should be hidden from the user. This is an
example of the principle of information hiding that we introduced in
Chapter~0. A more
%\begin{marginalnote}\it Hide implementation details\end{marginalnote}
\marginnote{Hide implementation details}
appropriate design would make the recursive method a {\tt private}
method that's called by the public method, as shown
Figure~\ref{fig-searcher} and implemented in the {\tt Searcher} class
(Fig.~\ref{fig-searcherclass}).

\begin{figure}[htb]
\figaleft{chptr12/searcher.eps}{The public {\tt search()} method serves
as an interface to the private recursive method, {\tt search()}.
Note that the methods have different signatures.
} {fig-searcher}

\end{figure}

\begin{figure}[h!]
\jjjprogstart
\begin{jjjlistingleft}[29pc]{-3pc}
\begin{lstlisting}
public class Searcher {
  /**
   * search(arr,key) -- searches arr for key.
   * Pre:  arr != null and 0 <= head <= arr.length
   * Post: if arr[k] == key for k,  0 <= k < arr.length, 
   *       return k,  else return -1
   */
  public int search(int arr[], int key) {
    return search(arr, 0, key); // Call recursive search 
  }
  /**
   * search(arr, head, key) -- Recursively search arr for key
   *  starting at head
   * Pre:  arr != null and 0 <= head <= arr.length
   * Post: if arr[k] == key for k,  0 <= k < arr.length, return k
   *       else return -1
   */
  private int search(int arr[], int head, int key)  {
    if (head == arr.length)   // Base case: empty list - failure
      return -1;
    else if (arr[head] == key)// Base case: key found -- success
      return head;
    else                      // Recursive case: search the tail
        return search(arr, head + 1, key);
  } // search()
  public static void main(String args[]) {
    int numbers[] = {0, 2, 4, 6, 8, 10, 12, 14, 16, 18};
    Searcher searcher = new Searcher();
    for (int k = 0; k <= 20; k++) {
        int result = searcher.search(numbers, k);
        if (result != -1)
          System.out.println(k + " found at " + result);
        else
          System.out.println(k + " is not in the array ");
    } // for
  } // main()
} // Searcher
\end{lstlisting}
\end{jjjlistingleft}
\jjjprogstop{Information hiding principle: The {\tt public} {\tt search()}
method calls the {\tt private}, recursive {\tt search()}, thereby hiding the
fact that a recursive algorithm is used.}
{fig-searcherclass}
\end{figure}

\JavaTIP[false]{EFFECTIVE DESIGN}{Information Hiding.}{Unnecessary implementation
details, such as whether a method uses a recursive or iterative
algorithm, should be hidden within the class.  Users of a class or
method should be shown only those details that they need to know.}

\pagebreak
\secEXRHone{Self-Study Exercise}
\begin{SSTUDY}

\item  Write a {\tt main()} method for the {\tt Searcher} class to conduct
the following test of {\tt search()}. Create an {\tt int} array of ten
elements, initialize its elements to the even numbers from 0 to 18,
and then use a for loop to search the array for each of the numbers
from 0 to 20.

\end{SSTUDY}

\subsection{Recursive Selection Sort}
\noindent Next we want you to think back to Chapter~9, where we
introduced the selection sort algorithm. To review the concept,
suppose you have a deck of 52 cards.  Lay them out on a table, face
up, one card next to the other.  Starting at the last card, look
through the deck, from
%\begin{marginalnote}\it Sorting a deck of cards\end{marginalnote}
\marginnote{Sorting a deck of cards}
last to first, find the largest card and exchange it with the last
card.  Then go through the deck again starting at the next to the last
card, find the next largest card, and exchange it with the next to the
last card.  Go to the next card, and so on.  If you repeat this
process 51 times, the deck will be completely sorted.

\JavaTIP{DEBUGGING TIP}{Off-by-One Error.}{Sorting algorithms are
particularly susceptible to an off-by-one error.  To sort an array with
{\it N} elements, you generally need to make $N-1$ passes.}

Let's design a recursive version of this algorithm.  The algorithm we
just described is like a head-and-tail algorithm in reverse, where the
last card in the deck is like the head, and the cards before it are
like the tail.  After each pass or recursion, the last card will be in
its proper location, and the cards before it will represent the
unsorted portion of the deck.  If we use parameter to represent {\it
last}, then it will be moved one card to the left at each level of the
recursion.

\begin{figure}[b]
\figa{chptr12/selectsort.eps}{Selection sort: Using a head-and-tail
algorithm in reverse to sort an integer array.}
{fig-selectsort}
\end{figure}

Figure~\ref{fig-selectsort} illustrates this process for an array of
integers.  The base case is reached when the {\it last} parameter is
pointing to the first element in the array.  An array with one element
in it is already sorted.   It needs no rearranging.  The recursive case involves
searching an ever-smaller portion of the array.  This is represented in
our design by moving {\it last} down one element to the left.

Figure~\ref{fig-selectionsort} provides a partial implementation of
selection sort for an array of {\tt int}. In this definition, the
array is one parameter.  The second parameter, {\tt int last}, defines
that portion of the array, from right to left, that is yet to be
sorted.   On the first call to this method, {\it last} will be
{\tt arr.length $-$ 1}.  On the second, it will be {\tt arr.length $-$ 2}, and so
on.  When {\it last} gets to be 0, the array will be sorted.  Thus,
in terms of the card deck analogy, {\it last} represents the last
card in the unsorted portion of the deck.

\begin{figure}[tb]
\jjjprogstart
\begin{jjjlisting}
\begin{lstlisting}
 /**
  * selectionSort(arr,last)--Recursively sort arr starting 
  *   at last element
  * Pre:  arr != null and 0 <= last < arr.length
  * Post: arr will be arranged so that arr[j] <= arr[k], 
  *   for any j < k
  */
private void selectionSort(int arr[], int last) {
 if (last > 0) {
   int maxLoc = findMax (arr, last);// Find the largest
   swap(arr, last, maxLoc);         // Swap it with last
   selectionSort(arr, last - 1);    // Move down the array
 }
} // selectionSort()
\end{lstlisting}
\end{jjjlisting}
\jjjprogstop{The {\tt selection-} {\tt Sort()} method uses the {\tt findMax()}
and {\tt swap()} methods to help it sort an array.}
{fig-selectionsort}
\end{figure}

Note how simply the {\tt selectionSort()} method can be coded.   Of
course, this is because we have used separate methods to handle the
%\begin{marginalnote}\it Task decomposition\end{marginalnote}
\marginnote{Task decomposition}
tasks of finding the largest element and swapping the last element and
the largest.  This not only makes sense in terms of the
divide-and-conquer principle, but we also already defined a {\tt swap()}
%%%REWmethod in Chapter~\ref{chapter-arrays}. So this is a good example of
method in Chapter~9. So here is a good example of 
reusing code:

\begin{jjjlisting}
\begin{lstlisting}
 /**
  * swap(arr, el1 el2) swaps el1 and el2 in the array, arr
  * Pre: arr is non null, 0 <= el1 < arr.length, 
                          0 <= el2 < arr.length
  * Post: the locations of el1 and el2 are swapped in arr
  */
private void swap(int arr[], int el1, int el2)  {
 int temp = arr[el1]; // Assign the first element to temp
 arr[el1] = arr[el2]; // Overwrite first with second
 arr[el2] = temp;     // Overwrite second with first(temp)
} // swap()
\end{lstlisting}
\end{jjjlisting}

\noindent The definition of the {\tt findMax()} method is left as a
self-study exercise.

\JavaTIP[false]{PROGRAMMING TIP}{Method Decomposition.}{A task can be simplified
by breaking it up into simpler subtasks, especially if you already
have methods for solving one or more of the subtasks.}

%%
\secEXRHone{Self-Study Exercises}
\begin{SSTUDY}

\item As in the case of the {\tt search()}
method, we need to provide a public interface to the recursive {\tt
selectionSort()} method. We want to enable the user to sort an array
just by calling {\tt sort(arr)}, where {\tt arr} is the name of the
array to be sorted.  Define the {\tt sort()} method.


\item  Define an iterative version of the {\tt findMax(arr,N)} method that is used in
{\tt selectionSort()}. Its goal is to return the location (index) of
the largest integer between {\tt arr[0]} and {\tt arr[N]}.

\end{SSTUDY}

\section{Example: Drawing (Recursive) Fractals}
\label{sec-fractals}\label{pg-sec-fractals}
\noindent A {\bf fractal} is a geometric shape that exhibits a
recursive structure.  When it is divided into parts, each part is a
\marginnote{Fractal patterns}
smaller version of the whole.  Fractal patterns occur in many
situations and places. For example, if you look at a graph of the Dow
Jones Industrial Average (DJIA) over the past year, the graph for each
day is similar to the graph of each month, which is similar to the
graph of each year, and so on.  Each part is a reduced-scale version
of the whole.  Fractals also occur throughout nature. If you look at a
coastline from an airplane, the shape of each part of the coastline,
no matter how small the scale, resembles the shape of the whole
coastline.  If you look at a tree, each branch of the tree is similar
in shape to the whole tree.

So, fractal patterns are all around us.  Because of their
self-similarity and divisibility, fractals are well-suited for
recursive programming.  Drawing recursive patterns is also an
excellent way to illustrate how to use parameters to create generality
in method design.  In this section, we will develop two simple
patterns and incorporate them into a GUI.

\subsection{Nested Squares}
\noindent Earlier in this chapter, we developed a recursive definition
for drawing a nested squares pattern (Fig.~12.2).
Now let's develop a recursive method that actually draws the pattern.
For this pattern, the base case is the drawing of the square.  The
recursive case, if more divisions are desired, is the drawing of
smaller patterns within the square:

\begin{jjjlisting}
\begin{lstlisting}
Draw a square.
If more divisions are desired
    draw a smaller version of pattern within square.
\end{lstlisting}
\end{jjjlisting}

An important consideration for this algorithm is to specify precisely
what we mean by ``if more divisions are desired.'' In other words, how
%\begin{marginalnote}\it How should we represent the problem?\end{marginalnote}
\marginnote{How should we represent the\hfill\break problem?}
exactly do we control the recursion?  In our earlier definition of the
pattern, we used the length of the side to control the algorithm.  When
$side \geq 5$, we recursed.

Another more general way to do this is to describe the fractal
structure in terms of its {\it levels}. For nested squares, the
%\begin{marginalnote}\it Levels of recursion\end{marginalnote}
\marginnote{Levels of recursion}
level-zero pattern would be just the basic square shape
(Fig.~12.21). A level-one pattern would be the basic
\marginfig{chptr12/boxes2.eps}{Levels 0, 1, and 4 of the nested
squares pattern.}
{fig-boxes2}
square shape plus an inner square, and so on.   The higher the level,
the more subdividing we do.   Therefore, one way to control the
recursion is to use a {\it level} parameter as the {\it recursion
parameter}---as the parameter that controls the recursion:

\begin{jjjlisting}
\begin{lstlisting}
Draw a square.
If the level is greater than 0,
    draw a smaller version of pattern within square.
\end{lstlisting}
\end{jjjlisting}

What other parameters will we need for this method? If we're going to
draw a rectangle, we'll need parameters for its {\it x-} and {\it
y-}coordinates.  We'll also need a parameter for the length of sides
of the square.  Another issue we need to decide is how much the length
of the sides should change at each level.  Should length change by a
fixed amount, by a fixed ratio, or by some other factor? In order to
allow this kind of flexibility, let's use another parameter for this
value.

These design considerations suggest the method shown in
Figure~\ref{fig-drawboxes}. Note that we must also provide a
{\tt Graphics} parameter so the method can use the {\tt drawRect()} method
to draw the square.  As we decided, the {\tt level} parameter controls
the recursion.  Note that its value is decreased by 1 in the recursive
call.  This will ensure that {\tt level} will eventually reach 0, and
recursion will stop.

\begin{figure}[hb!]
\jjjprogstart
\begin{jjjlisting}[26.5pc]
\begin{lstlisting}
/**
 * drawBoxes()---recursively draws pattern of nested 
 *  squares with top left corner of outer square at 
 *  (locX, locY) and dimensions of length side.
 * level (>= 0) is the recursion parameter (base: = 0)
 * delta is used to adjust the length of the side.
 */
private void drawBoxes(Graphics g, int level,
           int locX, int locY, int side, int delta) {
  g.drawRect(locX, locY, side, side );
  if (level > 0) {
    int newLocX = locX + delta; int newLocY = locY + delta;
    drawBoxes(g, level - 1, newLocX, newLocY,
                                   side - 2 * delta, delta);
  } // if
} // drawBoxes()
\end{lstlisting}
\end{jjjlisting}
\jjjprogstop{The {\tt drawBoxes()} method.}
{fig-drawboxes}
\end{figure}

Finally, note the use of the {\tt delta} parameter, which is used to
change the length of the sides by a fixed amount, {\tt 2 * delta}, at
each level.  It is also used to calculate the {\it x-} and {\it
y-}coordinates for the location of the next level of boxes {\it (locX
+ delta, locY + delta)}. But {\tt delta}'s value remains constant
through all the levels.  This will lead to a pattern where the ``gap''
between nested squares is constant.


\JavaTIP{EFFECTIVE DESIGN}{Levels of Recursion.}{Many recursive algorithms
use a {\it level} parameter as the recursion parameter.}

\secEXRHone{Self-Study Exercises}
\begin{SSTUDY}

\item Trace through the {\tt drawBoxes()} method and draw the
level-four and level-five versions of the nested boxes pattern.
Assume that the initial values for {\tt side} and {\tt delta} are 100
and 5, respectively, and the initial coordinates for {\tt (locX,locY)}
are (20,20).

\item The pattern shown in Figure~12.23 can be drawn by using {\tt
delta} as a fixed ratio of the length of the side, for example, 10
percent.  Modify the {\tt drawBoxes()} method to use {\tt delta} in
this way.

\item Write an iterative version of the {\tt drawBoxes()} method.
({\it Hint}: On each iteration, you must change the {\it x-} and {\it
y-}coordinates of the square's location and the length of its side.)

\marginfiglarge{chptr12/boxes3.eps}{This version of nested boxes can be drawn
by using {\tt delta} as a fixed percentage of the length of the side.}
{fig-boxes3}

\end{SSTUDY}

\subsection{The Sierpinski Gasket}
\noindent Let's return now to the {\it Sierpinski gasket} pattern that we
introduced at the start of this chapter.  This is a much more
interesting fractal pattern (Fig.~\ref{fig-sierpinski}). The overall
shape of the pattern is that of a triangle, but notice how the outer
triangle is divided into three smaller triangles.  Then each of those
triangles are divided into three smaller triangles.  If you continue
this process of dividing and shrinking, you get the level-seven pattern
shown here.

\begin{figure}[hbt]
\figa{chptr12/sierpinski2.eps}{Levels 0, 1, and 7 of the Sierpinski gasket
fractal pattern.}
{fig-sierpinski}
\end{figure}

Let's develop a recursive method to draw this pattern.  If we follow
the same strategy we used in the nested squares example, we get the
following algorithm:

\begin{jjjlisting}
\begin{lstlisting}
Base case:       Draw a triangle.
Recursive Case:  If more divisions are desired,
     draw three smaller gaskets within the triangle.
\end{lstlisting}
\end{jjjlisting}

\noindent For this pattern the base case is the drawing of the
basic triangle.  The recursive cases, if more divisions are desired,
are the drawing of smaller gaskets within the triangle.  Again we will
use a {\tt level} parameter to control the depth of the recursion.  The
higher the level, the more divisions will be drawn.



If we're going to draw a triangle shape, we need the coordinates of
%\begin{marginalnote}\it What other parameters do we need?\end{marginalnote}
\marginnote{What other parameters do we need?}
its three vertices---that is, an {\it x-} and {\it y-}coordinate for each
vertex.  Taken
together, these design considerations suggest the method definition
shown in Figure~\ref{fig-gasket}.

\begin{figure}[tb]
\jjjprogstart
\begin{jjjlistingleft}[28pc]{-2pc}
\begin{lstlisting}
/**
 * drawGasket()---recursively draws the Sierpinski gasket
 *  pattern, with points (p1X, p1Y),(p2X, p2Y),(p3X, p3Y) 
 *  representing the vertices of its enclosing triangle.
 * level (>= 0) is the recursion parameter (base: = 0)
 */
private void drawGasket(Graphics g, int lev, int p1X, int p1Y,
                   int p2X, int p2Y, int p3X, int p3Y) {
  g.drawLine(p1X, p1Y, p2X, p2Y);  // Draw a triangle
  g.drawLine(p2X, p2Y, p3X, p3Y);
  g.drawLine(p3X, p3Y, p1X, p1Y);
  if (lev > 0) { // If more levels, draw 3 smaller gaskets
    int q1X = (p1X + p2X) / 2;    int q1Y = (p1Y + p2Y) / 2;
    int q2X = (p1X + p3X) / 2;    int q2Y = (p1Y + p3Y) / 2;
    int q3X = (p2X + p3X) / 2;    int q3Y = (p2Y + p3Y) / 2;
    drawGasket(g, lev - 1, p1X, p1Y, q1X, q1Y, q2X, q2Y);
    drawGasket(g, lev - 1, p2X, p2Y, q1X, q1Y, q3X, q3Y);
    drawGasket(g, lev - 1, p3X, p3Y, q2X, q2Y, q3X, q3Y);
  } // if
} // drawGasket()
\end{lstlisting}
\end{jjjlistingleft}
\jjjprogstop{The {\tt drawGasket()} method.}
{fig-gasket}
\end{figure}

As we described earlier, we use the {\tt level} parameter
as the recursion parameter for this method, which controls the
%\begin{marginalnote}\it Levels of recursion\end{marginalnote}
\marginnote{Levels of recursion}
recursion.  Note that each of the three recursive calls decreases the
{\tt level} by 1. This will ensure that eventually {\tt level} will
equal 0, and recursion will stop.

Note also how the three pairs of coordinates are used.  Drawing a
triangle is simple.  Just draw three lines from {\tt (p1X,p1Y)} to
{\tt (p2X,p2Y)}, from {\tt (p2X,p2Y)} to {\tt (p3X,p3Y)}, and from
{\tt (p3X,p3Y)} back to {\tt (p1X, p1Y)}. The most complicated part of
the method is calculating the vertices for the three inner gaskets.
If you look at Figure~\ref{fig-sierpinski} again, you'll notice that
each of the inner triangles uses one of the vertices of the main
triangle, plus the {\it midpoints} of the two adjacent sides.  Thus,
the triangle on the ``left'' uses the left vertex {\tt (p1X,p1Y)}, and
the midpoints of the other two lines from {\tt (p1X,p1Y)} to {\tt
(p2X,p2Y)} and from {\tt (p1X,p1Y)} to {\tt (p3X,p3Y)}.  As you might
remember from high school math, the formula for computing the midpoint
of the line segment $(x1,y1)$ to $(x2,y2)$ is
%\begin{marginalnote}\it Midpoint of a line\end{marginalnote}
\marginnote{Midpoint of a line}

\begin{jjjlisting}
\begin{lstlisting}
( (x1 + x2) / 2, (y1 + y2) / 2 )
\end{lstlisting}
\end{jjjlisting}

\noindent This formula is used repeatedly to calculate the
vertices of the three smaller gaskets.


\section{OBJECT-ORIENTED DESIGN:\\ Tail Recursion}
%\addcontentsline{toc}{section}{\S~~~ Object-Oriented Design: Tail Recursion}
\noindent Although the {\tt drawBoxes()} method is relatively simple to convert
into an iterative version (see Self-Study Exercise 12.18), the same cannot be said for the
{\tt drawGasket()} method.  It is clearly a case where the recursive
approach makes the problem easier to solve.

One difference between {\tt drawBoxes()} and {\tt drawGasket()} is
that {\tt drawBoxes()} is an example of a tail-recursive method.   A
method is {\bf tail recursive}\index{tail recursive} 
\marginnote{Tail recursion}
if all of its recursive calls occur as
the last action performed in the method.  You have to be a bit careful
%\begin{marginalnote}\it Tail recursion\end{marginalnote}
about this definition.  The recursive call in a tail-recursive method
has to be the last {\it executed} statement.  It needn't be the last
statement appearing in the method's definition.

For example, the following method will print ``Hello'' {\it N} times.   This
method is tail recursive even though its last statement is not a
recursive call:

\begin{jjjlisting}
\begin{lstlisting}
public void printHello(int N) {
 if (N > 1) {
   System.out.println("Hello");
   printHello(N - 1); // The last executed statement
 } else
   System.out.println("Hello");
} // printHello()
\end{lstlisting}
\end{jjjlisting}

\noindent This method is tail recursive because the last statement
that will be executed, in its recursive cases, is the recursive call.

A tail-recursive method is relatively easy to convert into an
iterative method.  The basic idea is to make the recursion parameter
into a loop variable, taking care to make sure the bounds are equivalent.
Thus, the following iterative method will print ``Hello'' {\it N} times:

\begin{jjjlisting}
\begin{lstlisting}
public void printHelloIterative(int N) {
    for (int k = N; k > 0; k--)
        System.out.println("Hello");
}
\end{lstlisting}
\end{jjjlisting}

\noindent In this case, we use the parameter {\it N} to set the
initial value of the loop variable, {\it k}, and we decrement {\it k}
on each iteration.  This is equivalent to what happens when we
decrement the recursion parameter in the recursive call.

\JavaTIP{EFFECTIVE DESIGN}{Tail Recursion.}{Tail-recursive algorithms
are relatively simple to convert into iterative algorithms that
do the same thing.}

As you can see, recursive methods that are not tail recursive are much
more complex.  Just compare the {\tt drawGasket()} and
{\tt drawBoxes()} methods.  Yet it is precisely for these nontail-recursive
algorithms that recursion turns out to be most useful.  As you might
expect, if you can't give a simple tail-recursive solution to a
problem, the problem probably doesn't have a simple iterative solution
either.   Thus, the problems where we most need recursion are those where
we can't give a simple tail-recursive or a simple iterative solution.
And there are a lot of such problems, especially when you get into
nonlinear data structures such as trees and graphs.

To gain some appreciation for this complexity, consider how difficult
it would be to draw the Sierpinski gasket using an iterative
approach.  We could start by developing an outer for loop to account
for the different levels in the pattern:

\begin{jjjlisting}
\begin{lstlisting}
for (int k = lev; k > 0; k--) {
 drawGasket(g, lev - 1, p1X, p1Y, q1X, q1Y, q2X, q2Y);
 drawGasket(g, lev - 1, p2X, p2Y, q1X, q1Y, q3X, q3Y);
 drawGasket(g, lev - 1, p3X, p3Y, q2X, q2Y, q3X, q3Y);
}
\end{lstlisting}
\end{jjjlisting}

\noindent But now each of the method calls within the body of this
loop would have to be replaced by very complex loops.  That would be a
daunting task.  So the lesson to be drawn from this observation is that
recursion is most useful as a problem-solving technique for problems
that don't yield to a simple iterative solution.

\JavaTIP{EFFECTIVE DESIGN}{Recursion or Iteration.}{If you have
difficulty designing an iterative solution to a problem, try
developing a recursive solution to it.}

\secEXRHone{Self-Study Exercises}
\begin{SSTUDY}

\item  Trace the {\tt drawGasket()} method for levels two and three.   Pick your
own values for the three vertices.

\item  Is the {\tt printReverse()} method, discussed earlier, tail recursive?
Explain.

\item  Is the {\tt countChar()} method, discussed earlier, tail recursive?
Explain.

\end{SSTUDY}

%\end{objectorienteddesign}

\markright{{\color{cyan}SECTION\,\,\arabic{section}\,\,$\bullet$}\,\, OOD: Recursion or Iteration}

\section{OBJECT-ORIENTED DESIGN:  \\Recursion or Iteration?}

\markright{{\color{cyan}SECTION\,\,\arabic{section}\,\,$\bullet$}\,\, OOD: Recursion or Iteration}

\noindent As we mentioned at the outset of this chapter, recursive
algorithms require more computational overhead than iterative
algorithms.  We're now in a good position to appreciate why this is
so.

A recursive algorithm incurs two kinds of overhead that are not
incurred by an iterative algorithm: memory and CPU time.  Both of these
are direct results of the fact that recursive algorithms do a lot of
method calling.

As we saw in our various traces, each time a method is called,
%\begin{marginalnote}\it Method call overhead\end{marginalnote}
\marginnote{Method call overhead}
a representation of the method call is placed on the {\it method call
stack}. These \mbox{representations} often take the form of a {\it block} of
memory locations, which can be quite large.  The block must contain
space for the method's local variables, including its
parameters.  Also, unless the method is {\tt void}, the block must
contain space for the method's return value.  In addition it must
contain a reference to the calling method, so it will know where to go
%\begin{marginalnote}\it Memory overhead\end{marginalnote}
\marginnote{Memory overhead}
when it is done.  Figure~\ref{fig-methodblock} shows what the method
call block would look like for the {\tt search()} method.

\begin{figure}[hb]
\figa{chptr12/methodblock.eps}{A more detailed picture of the
method call stack, showing two method blocks for {\tt search()} after
two levels of recursion.}
{fig-methodblock}
\end{figure}

In addition to the memory required, a method call also requires extra
%\begin{marginalnote}\it CPU overhead\end{marginalnote}
\marginnote{CPU overhead}
CPU time.   Each time a method is called, Java must create a method
call block, copy the method call arguments to the parameters in the
block, create initial values for any local variables that are used by
the method, and fill in the return address of the calling method.  All
of this takes time, and in the case of a recursive method, these steps
are repeated at each level of the recursion.

Compare these memory and CPU requirements with what normally
transpires for an iterative algorithm---an algorithm involving a
loop.  The loop structure usually occurs entirely within a method, so
it doesn't incur either the memory or CPU overhead involved in
recursion.   Therefore, iterative algorithms are generally more
efficient than recursive algorithms.  One useful guideline, then, is
when runtime performance and efficiency are of prime importance, you
should use iteration instead of recursion.

\JavaTIP{EFFECTIVE DESIGN}{Iteration or Recursion.}{Use an iterative
algorithm instead of a recursive algorithm whenever efficiency and
memory usage are important design factors.}

On the other hand, recursive algorithms are much easier to design than
the corresponding iterative algorithms for many problems.  We tried to
illustrate this point in our development of the Sierpinski gasket
algorithm, but there are many other examples that we could have used.
Given that programmer and designer time is the most expensive
%\begin{marginalnote}\it Efficiency of development\end{marginalnote}
\marginnote{Efficiency of development}
resource involved in software development, a recursive solution may be
easier to develop and maintain than a corresponding iterative
solution.  And given the great cost of software development, a less
efficient solution that is easier to develop, easier to understand,
and easier to maintain may be preferable to a highly efficient
algorithm that's difficult to understand.   For some problems then,
such as the Sierpinski gasket, a recursive algorithm may provide the
best solution.
%%\end{table}

\JavaTIP{EFFECTIVE DESIGN}{Keep It Simple.}{When all other factors
are equal, choose the algorithm (recursive or iterative)
that is easiest to understand, develop, and maintain.}

One final point that's worth making is that some {\it optimizing}
compilers are able to convert recursive methods into iterative
methods when they compile the program.   The algorithms for doing this
are well known.  They are often subjects for study in a data structures
%\begin{marginalnote}\it Optimizing compiler\end{marginalnote}
\marginnote{Optimizing compiler}
course, so we won't go into them here.  The resulting runtime programs
will be just as efficient, in CPU time and memory, as if you had
written iterative methods.  The point is that if you have such a
compiler, you really get the best of both worlds.  You get the
advantage of using recursion as a problem-solving and software
development approach, and the compiler takes care of producing an
efficient object program.
%\end{objectorienteddesign}

\section*{{\color{cyan}Special Topic:} Exploring the Mandelbrot Set}

\noindent The Mandelbrot set is one of the most fascinating fractals.
It is named after its discover, IBM mathematician Benoit Mandelbrot.
The Mandelbrot set itself is the black, heart-shaped image shown in
Figure~\ref{fig-mandel}.  What makes the Mandelbrot set so interesting is that
with the help of a Java GUI you can explore the set as if you were
taking a trip through outer space.  %%\\
%%\hspace*{1pc}
The most interesting regions to explore are those just along the
boundary of the set.  For example, notice that the boundary contains
numerous circular shapes, each of which is itself studded with
circular shapes.  This is an example of the scaled self-similarity that
we found to be so prevalent in recursive structures.  By continually
expanding the regions around the boundary, you'll find an infinite
recursion of fascinating images and shapes.  In some regions of the set
you'll even find miniature replications of the set itself.  %%\\
%%\hspace*{1pc}

%\begin{figure}
%\begin{graphic}
\marginfig{chptr12/mandel.eps}{The Mandelbrot set.}
{fig-mandel}
%\end{graphic}
%\end{figure}

The Mandelbrot set is generated by an {\it iterated function system}.
The mathematics underlying this fascinating object is quite
accessible, and there are a number of online tutorials that explain how
the set is generated and how the pictures are produced.  Many of the
Mandelbrot and fractal Web sites contain excellent Java applets that
let you explore the Mandelbrot set as well as related sets.   %%\\
%%\hspace*{1pc}
An excellent place to start your exploration would be David Joyce's
award-winning Web site,
%%%REWhttp://aleph0.clarku.edu/djoyce/julia/wwwrefs.html
\begin{jjjlisting}
\begin{lstlisting}[commentstyle=\color{black}]
http://aleph0.clarku.edu/~djoyce/julia/
\end{lstlisting}
\end{jjjlisting}
\newline which contains references to a number of other good sites.
For a tutorial on how the various Mandelbrot set-generating Java
programs work, see
\begin{jjjlisting}
\begin{lstlisting}[commentstyle=\color{black}]
http://storm.shodor.org/mteach/
\end{lstlisting}
\end{jjjlisting}
%\end{minipage}\bigskip
%%}


\section{From the Java Library: \\ javax.swing.JComboBox}

\tBOXseven{A} {\tt JComboBox} is a Swing component that combines a text field and
a drop-down list (Fig.~12.28).  It lets the user either
type in a selection or choose a selection from a list that appears
when the user requests it---a {\tt JComboBox}'s drop-down behavior is
somewhat similar to a {\tt java.awt.Choice} box.

%%\begin{figure}[tb]
%%\figaleft{chptr12/jcombobox.eps}{A {\tt JComboBox} responds to action events and item events.
%%\label{fig-jcombobox}\label{pg-fig-jcombobox}}
%%\end{figure}

\marginfig{chptr12/jcombobox.eps}{A {\tt JComboBox} responds to action events and item events.}
{fig-jcombobox}

A {\tt JComboBox} can be used to represent a {\it drop-down menu}.
When the user clicks on a {\tt JComboBox}, a list of options drops
down, and the user can select a particular option that is stored in
the box's internal state (Fig.~12.29).  The list of
options associated with a {\tt JComboBox} can be built beforehand and
inserted into the component in a constructor, or items can be inserted
one at a time by repeatedly using its {\tt addItem()} method.

As Figure~12.28 shows, either an array or a vector of items
can be passed to a constructor method to initialize the box's menu.
The items stored in a {\tt JComboBox} box are references to
{\tt Object}s, most commonly {\tt String}s that represent the name of the
menu item.  They are stored in the (zero indexed) order in which they
are added.   The {\tt addItem()} method is used to add an individual
{\tt Object} to a {\tt JComboBox}.  By default, the first item added
to a {\tt JComboBox} will be the {\it selected} item until the user
selects another item.

When the user makes a selection in a {\tt JComboBox}, the item
selected can be gotten either by its reference ({\tt
getSelectedItem()}) or by its position within the menu ({\tt
getSelectedIndex()}).  There are also methods to {\tt
set\-SelectedItem()} and {\tt setSelectedIndex()} that let you select
an individual item either by its reference or its position.  The {\tt
addItemListener()} method is used to designate some object as the
listener for the {\tt ItemEvent}s that are generated whenever the user
selects a menu option.  Alternatively, the \mbox{\tt
addActionListener()} method lets you handle action events, such as
when the user types a value into the box.

\subsection{A {\tt JComboBox} Example}

\noindent As a simple example, let's design an graphical interface that
can be used to display the fractal patterns we developed earlier.  We
want an interface that lets the user select from among the available
patterns---we'll use the Sierpinski gasket and nested boxes for
starters.  In addition, the user should also be able to select
different levels for the drawings, from 0 to 9. We want to present
these options in two menus, with one {\tt JComboBox} for each menu.

The first step is to declare and instantiate the {\tt JComboBox}es
as instance variables:

\begin{jjjlisting}
\begin{lstlisting}
private String items[] = 
    {"Sierpinski Gasket","Nested Boxes"};
private JComboBox patterns = new JComboBox(items);
private JComboBox levels = new JComboBox();
\end{lstlisting}
\end{jjjlisting}

\noindent Note that in this case we pass the constructor for the
\marginfig{chptr12/comboscreen.eps}{Using a {\tt JComboBox} box.}
{fig-comboscreen}
{\tt patterns} menu an entire array of items.  If we hadn't done it
this way, we would add individual items to the combo box in the
JFrame's constructor {\tt RecursivePatterns()}.  In fact, that's how we'll initialize
the {\tt levels} menu:

\begin{jjjlisting}
\begin{lstlisting}
for (int k=0; k < 10; k++)   // Add 10 levels
    levels.addItem(k + "" );
levels.setSelectedItem("4"); // Select default level
\end{lstlisting}
\end{jjjlisting}

\noindent This loop would be placed in the JFrame's constructor, {\tt
  RecursivePatterns()}.  
It adds strings representing levels 0 to 9 to the menu
and initializes the box so that level four is  showing as the
default option.

Our next step is to designate the JFrame as the {\tt ItemListener} for
both menus---that is, the JFrame is named as the object that will
handle the events that occur in the {\tt JComboBox}es.  Then we add the
{\tt JComboBox} component to the JFrame:

\begin{jjjlisting}
\begin{lstlisting}
controls.add(levels);    // Control panel for menus
controls.add(patterns);
                                // Add the controls
getContentPane().add(controls, "North"); 
                           // And the drawing panel
getContentPane().add(canvas, "Center");  
              // Register the menus with a listener
levels.addItemListener(this);  
patterns.addItemListener(this);
\end{lstlisting}
\end{jjjlisting}

\noindent Note that we use a separate {\tt controls} panel (a {\tt JPanel})
for the two menus and a {\tt canvas} panel (another {\tt JPanel}) for
the drawings.

The next step is to implement the {\tt itemStateChanged()} method to
handle the user's selections.  Whenever the user selects an item from
a {\tt JCombo\-Box} menu, an {\tt ItemEvent} is generated.  In order
to handle these events, the program must implement the {\tt
ItemListener} interface\index{ItemListener interface}, which consists
of the single method {\tt itemStateChanged()}. This method is invoked
automatically whenever the user selects an item from one of the {\tt
JComboBox}es:

\begin{jjjlisting}
\begin{lstlisting}
public void itemStateChanged(ItemEvent e) {
    canvas.setPattern(patterns.getSelectedIndex(), 
                         levels.getSelectedIndex());
    repaint();
}
\end{lstlisting}
\end{jjjlisting}

\noindent The {\tt itemStateChanged()} method has the same general
form as the {\tt actionPerformed()} method, except that its parameter
is an \mbox{\tt ItemEvent}.  For this example, the program uses the
{\tt get\-Selected\-Index()} method to get the selected pattern and the
selected level by their respective item numbers within the menus.  It
then passes these values along to the {\tt canvas} object, which takes
care of the drawing.  Finally, the method invokes the {\tt repaint()}
method.  Because the JFrame is a container, this will cause all of its
components to be repainted as well.  

Figure~\ref{fig-seqdiag} illustrates the sequence of events that
occurs when an item is selected from a {\tt JComboBox}.  The complete
implementation for the program is given in
Figure~\ref{fig-recursivepatterns}.

\begin{figure}[tb]
\figaright{chptr12/seqdiag.eps}{This UML sequence diagram shows the interaction between
the various objects included in the action of selecting an item
from a {\tt JComboBox}.
} {fig-seqdiag}

\end{figure}
\begin{figure}[h!]
\jjjprogstart
\begin{jjjlistingleft}[33pc]{-3pc}
\begin{lstlisting}
import java.awt.*;
import javax.swing.*;
import java.awt.event.*;

public class RecursivePatterns extends JFrame implements ItemListener  {
  private String choices[] = {"Sierpinski Gasket", "Nested Boxes"};
  private JComboBox patterns = new JComboBox(choices); // Pattern choices
  private JComboBox levels = new JComboBox();          // Level choices
  private Canvas canvas = new Canvas();                // Drawing panel
  private JPanel controls = new JPanel();

  public RecursivePatterns() {
    for (int k=0; k < 10; k++)            // Add 10 levels
      levels.addItem(k + "" );
    patterns.setSelectedItem(choices[0]); // Initialize menus
    levels.setSelectedItem("4");

    canvas.setBorder(BorderFactory.createTitledBorder("Drawing Canvas"));
    controls.add(levels);         // Control panel for menus
    controls.add(patterns);
    getContentPane().add(controls,"North"); // Add controls
    getContentPane().add(canvas,"Center");  // Add drawing panel
    levels.addItemListener(this);   // Register menus with listener
    patterns.addItemListener(this);
    setSize(canvas.WIDTH,canvas.HEIGHT+controls.getSize().width);
  } // init()

  public void itemStateChanged(ItemEvent e) {
    canvas.setPattern(patterns.getSelectedIndex(), 
                                     levels.getSelectedIndex());
    repaint();                             // Repaint the JFrame
  } // itemStateChanged()

   public static void main(String args[]) 
   {  
      JFrame f = new RecursivePatterns();
      f.setVisible(true);
   }
} // RecursivePatterns
\end{lstlisting}
\end{jjjlistingleft}
\jjjprogstop{The {\tt RecursivePatterns} program.}
{fig-recursivepatterns}
\end{figure}


\begin{figure}[tb]
\figaleft{chptr12/canvasnew.eps}{Design of a drawing {\tt Canvas} class.
} {fig-canvas}

\end{figure}

The actual drawing of the fractal patterns is handled by the {\tt
canvas JPanel} component, whose design is shown in
Figure~\ref{fig-canvas} and whose implementation is given in
Figure~\ref{fig-canvaspanel}.  All of the drawing is done in the {\tt
paintComponent()} method.  Because the {\tt canvas} is contained
within the JFrame, the {\tt paintComponent()} method is called
automatically whenever the JFrame repaints itself.  Notice how the
{\tt switch} statement uses the pattern that
\marginnote{Zero indexing}
the user chose to call the corresponding drawing method.   You can see
from this {\tt switch} statement that a {\tt JComboBox}'s items are
{\it zero indexed}.

\begin{figure}[h!]
\jjjprogstart
\begin{jjjlistingleft}[33pc]{-7pc}
\begin{lstlisting}
import javax.swing.*;
import java.awt.*;

public class Canvas extends JPanel {
  private static final int GASKET = 0, BOXES = 1;
  public static final int WIDTH=400, HEIGHT=400;
  private final int HBOX=10, VBOX=50, BOXSIDE=200, BOXDELTA=10;
  private final int gP1X = 10;  private final int gP1Y = 280; // Initial
  private final int gP2X = 290; private final int gP2Y = 280; // gasket
  private final int gP3X = 150; private final int gP3Y = 110; // points
  private int pattern = 0 ;                      // Current pattern
  private int level = 4;                         // Current level

  public Canvas() {
    setSize(WIDTH, HEIGHT);
  }

  public void setPattern(int pat, int lev) {
    pattern = pat;
    level = lev;
  }
\end{lstlisting}
\end{jjjlistingleft}
\jjjprogstop{The {\tt Canvas} class is a drawing panel, Part I.}
{fig-canvaspanel}
\end{figure}


\begin{figure}[p]
\addtocounter{figure}{-1}
\jjjprogstart
\begin{jjjlisting}[33pc]
\begin{lstlisting}
    public void paintComponent(Graphics g) {
        g.setColor(getBackground());   // Redraw the panel's background
        g.drawRect(0, 0, WIDTH, HEIGHT);
        g.setColor(getForeground());
        switch (pattern) {
        case GASKET:
            drawGasket(g, level, gP1X, gP1Y, gP2X, gP2Y, gP3X, gP3Y);
            break;
        case BOXES:
            drawBoxes(g, level, HBOX, VBOX, BOXSIDE, BOXDELTA );
            break;
        } // switch
    } // paintComponent()

   /** drawGasket()---recursively draws the Sierpinski 
    *  gasket pattern, with points (p1X, p1Y), (p2X, p2Y), (p3X, p3Y) 
    *  representing the vertices of its enclosing triangle.
    * level (>= 0) is the recursion parameter (base case: level  0)
    */
    private void drawGasket(Graphics g, int lev, int p1X, int p1Y,
                   int p2X, int p2Y, int p3X, int p3Y) {
        g.drawLine(p1X, p1Y, p2X, p2Y);  // Draw a triangle
        g.drawLine(p2X, p2Y, p3X, p3Y);
        g.drawLine(p3X, p3Y, p1X, p1Y);
        if (lev > 0) { // If more levels, draw 3 smaller gaskets
            int q1X = (p1X + p2X) / 2;    int q1Y = (p1Y + p2Y) / 2;
            int q2X = (p1X + p3X) / 2;    int q2Y = (p1Y + p3Y) / 2;
            int q3X = (p2X + p3X) / 2;    int q3Y = (p2Y + p3Y) / 2;
            drawGasket(g, lev - 1, p1X, p1Y, q1X, q1Y, q2X, q2Y);
            drawGasket(g, lev - 1, p2X, p2Y, q1X, q1Y, q3X, q3Y);
            drawGasket(g, lev - 1, p3X, p3Y, q2X, q2Y, q3X, q3Y);
        }
    } // drawGasket()
   /** drawBoxes()---recursively draws pattern of nested squares 
    *  with (locX, locY) the top left corner of outer the square and
    *  side being the length square's side.
    * level (>= 0) is the recursion parameter (base case: level  0)
    * delta is used to adjust the length of the side.
    */
    private void drawBoxes(Graphics g, int level,
           int locX, int locY, int side, int delta) {
        g.drawRect(locX, locY, side, side );
        if (level > 0) {
            int newLocX = locX + delta; int newLocY = locY + delta;
            drawBoxes(g, level - 1, newLocX, newLocY,
                                          side - 2 * delta, delta);
        }
    } // drawBoxes()
} // Canvas
\end{lstlisting}
\end{jjjlisting}
\jjjprogstop{The {\tt Canvas} class, Part II.}{fig:canvaspart2}
\end{figure}
\mbox{ }


\pagebreak
\secSMHleft{Chapter Summary}

\secKTH{Technical Terms}

\begin{KT}
base case

computational overhead

head-and-tail algorithm

iterative method

last-in-first-out (LIFO)

method call stack

recursion parameter

recursive case

recursive definition

recursive method

self-similarity

tail recursive

\end{KT}


\secSMHtwo{Summary of Important Points}

\begin{SMBL}
\item  A {\it recursive definition} is one that defines the
{\it n}th case of a concept in terms of the $(n-1)$st
case plus a limiting condition.   It is based on the idea of breaking a
problem up into smaller, self-similar problems.

\item  A {\it recursive\index{recursive method} method} is one that calls itself.
It is usually defined in terms of a {\it base case} or limiting case,
which stops the recursive process, and a recursive case, which breaks
the method into a smaller, self-similar copy of itself.   A {\it
recursion parameter} is generally used to control the recursion.

\item  An iterative algorithm is one that uses some kind of loop
as its control structure.  Any algorithm that can be done iteratively
can also be done recursively, and vice versa.


\item  Because method calling is relatively costly both in terms
of memory used and CPU time involved, a recursive algorithm
is generally less efficient than an iterative one that does the same
thing.

\item  In designing recursive algorithms, the {\it base case} defines
a limit.  Each level of recursion should make progress toward
the limit, and the algorithm should eventually reach the limit.
The limit is usually expressed in terms of the {\it recursion parameter}.

\item  A recursive method is {\it tail recursive} if and only if
each of its recursive calls is the last action executed by the
method.

\item  A Swing {\tt JComboBox} component is used to represent
a GUI drop-down menu.
\end{SMBL}

\secANSHleft

\begin{ANS}

\item The output produced by {\tt mystery(0)} would be 0 1 2 3 4 5 6.
The output produced by {\tt mystery(100)} would be 100.

\item  The output produced by {\tt mystery(5)} would be: 5 4 3, and so
on.  In other words, this is an infinite recursion.  

\item \mbox{ }
\begin{jjjlisting}
\begin{lstlisting}
Definition: twoToN(N), N >= 0
  1, if N == 0                   // Base case
  2 * twoToN(N - 1),  N > 0      // Recursive case
\end{lstlisting}
\end{jjjlisting}

\item  The function $x^n$ is known as the power function:

\begin{jjjlisting}
\begin{lstlisting}
Definition: power(X,N), N >= 0
  1, if N == 0                    // Base case
  X * power(X, N - 1),  N > 0     // Recursive case
\end{lstlisting}
\end{jjjlisting}

\item  Yes, the two definitions for nested boxes are equivalent.  Suppose
the square starts out with a side of 20. The definition given in
the exercise will also draw squares with sides of 20, 15, 10, 5.

\item  A recursive definition for the pattern in Figure~12.4:

\begin{jjjlisting}
\begin{lstlisting}
Draw a square with side, s.
Inscribe a circle with diameter, s.
If s > 5,
  Draw a smaller version of same pattern. // Recursive case
\end{lstlisting}
\end{jjjlisting}

\item  The {\tt printString2("hello")} method will print: ``olleh.''

\item  A definition for {\tt countDown()}:

\begin{jjjlisting}
\begin{lstlisting}
 /** countDown(N) recursively prints a countdown 
  *  beginning at N and ending at 1
  * @param N >= 1
  * Base case: N == 0
  */
void countDown(int N) {
    if (N == 0)                     // Base case
        System.out.println("blastoff");
    else {
        System.out.print(N + ", "); // Recursive case
        countDown(N - 1);
    }
} // countDown()
\end{lstlisting}
\end{jjjlisting}

\item  A revised definition for {\tt countDown()}:

\begin{jjjlisting}
\begin{lstlisting}
 /** countDown(N) recursively prints a countdown  
  *  beginning at N, counting every other number, 10 8 6 ...  
  *  and ending at "blastoff"
  * @param N >= 1
  * Base case: N <= 0
  */
void countDown(int N) {
    if (N <= 0)                     // Base case
        System.out.println("blastoff");
    else {
        System.out.print(N + ", "); // Recursive case
        countDown(N - 2 );
    }
} // countDown()
\end{lstlisting}
\end{jjjlisting}

\item  A method to sum the numbers from 1 to {\it N}.

\begin{jjjlisting}
\begin{lstlisting}
int sum(int N) {
    if (N == 0)
        return 0;
    else
        return N + sum(N-1);
}
\end{lstlisting}
\end{jjjlisting}

\item  A method to change each blank within a string
to two blanks.

\begin{jjjlisting}
\begin{lstlisting}
String addBlanks(String s) {
  if (s.length() == 0)
     return "";
  else if (s.charAt(0) == ' ')
     return ' ' + s.charAt(0) + addBlanks(s.substring(1));
  else
     return s.charAt(0) + addBlanks(s.substring(1));
}
\end{lstlisting}
\end{jjjlisting}

\item A method to print out all possible outcomes for a chess player
playing {\tt N} games. {\tt printOutcomes(str, N)} will print all
outcomes for the next {\tt N} games given that results for previous
games are stored in the string named {\tt str}. 
\begin{jjjlisting}
\begin{lstlisting}
 public static void printOutcomes(String str, int N){
     if (N = 1){ // Base case: win, lose, or draw one game
         System.out.println(str + "W");
         System.out.println(str + "L");
         System.out.println(str + "D");
     } else {  // Recursive case
         printOutcomes(str + "W", N - 1);
         printOutcomes(str + "L", N - 1);
         printOutcomes(str + "D", N - 1);
     } //else
 }// printOutcomes()
\end{lstlisting}
\end{jjjlisting}

\item  \mbox{ }
\begin{jjjlisting}
\begin{lstlisting}
public static void main(String args[]) {
  int numbers[] = {0, 2, 4, 6, 8, 10, 12, 14, 16, 18};
  Searcher searcher = new Searcher();
  for (int k = 0; k <= 20; k++) {
    int result = searcher.search(numbers, k);
    if (result != -1)
      System.out.println(k + " found at " + result);
    else
      System.out.println(k + " is not in the array ");
  } // for
} // main()
\end{lstlisting}
\end{jjjlisting}

\item  The {\tt sort()} method is used as a public interface to the
recursive {\tt selectionSort()} method:

\begin{jjjlisting}
\begin{lstlisting}
/** sort(arr) sorts the int array, arr
 *  Pre: arr is not null
 *  Post: arr will be arranged so that arr[j] <= arr[k] 
 *     for any j < k
 */
public void sort(int arr[]) {
    selectionSort(arr, arr.length - 1);  
             // Just call the recursive method
}
\end{lstlisting}
\end{jjjlisting}

\item  An iterative version of {\tt findMax()}:

\begin{jjjlisting}
\begin{lstlisting}
 /** findMax (arr,N) returns the index of the largest
  *  value between arr[0] and arr[N], N >= 0.
  *  Pre: 0 <= N <= arr.length -1
  *  Post: arr[findMax()]>=arr[k] for k between 0 and N.
  */
private int findMax(int arr[], int N) {
    int maxSoFar = 0;
    for (int k = 0; k <= N; k++)
        if (arr[k] > arr[maxSoFar])
            maxSoFar = k;
    return maxSoFar;
} // findMax()
\end{lstlisting}
\end{jjjlisting}
%\begin{figure}
%\begin{figure}[tb]
\marginfiglarge{chptr12/boxes45.eps}{Levels four and five of the nested boxes pattern.}
{fig-boxes45}
%\end{figure}
%\end{figure}

\item  Levels four and five of the nested boxes  pattern
are shown in Figure~12.34.



\item  The following method will reduce the length of the side by {\tt delta}
percent at each level of recursion.  The spacing between the boxes
will vary by a constantly decreasing amount.

\begin{jjjlisting}
\begin{lstlisting}
private void  drawBoxes(Graphics g, int level, int locX, 
                   int locY, int side, int delta) {
    g.drawRect(locX, locY, side, side );
    if (level > 0) {
      int dside = side * delta / 100; // Percent delta
      int newLocX = locX + dside;
      int newLocY = locY + dside;
      drawBoxes(g, level - 1, newLocX, newLocY, 
                         side - 2 * dside, delta);
    }
} // drawBoxes()
\end{lstlisting}
\end{jjjlisting}

\item  \mbox{ }
\begin{jjjlisting}
\begin{lstlisting}
private void drawBoxesIterative(Graphics g, int level,  
            int locX, int locY, int side, int delta) {
 for (int k = level; k >= 0; k--) {
  g.drawRect(locX, locY, side, side );  // Draw a square
  locX += delta;             // Calculate new location
  locY += delta;
  side -= 2 * delta;         // Calculate new side length
 }
} // drawBoxes()
\end{lstlisting}
\end{jjjlisting}

\item  The level two and three gaskets are shown in Figure~\ref{fig-sierpinski23}.

%\begin{figure}
\begin{figure}[tb]
\figa{chptr12/sierpinski23.eps}{Levels two and three of the Sierpinski gasket.}
{fig-sierpinski23}
\end{figure}
%\end{figure}

\item  The {\tt printReverse()} method is not tail recursive because in that
method the recursive call is not the last statement executed.

\item  The {\tt countChar()} method is tail recursive.  The recursive calls
are not the last statements in the method definition.  However, each of
the recursive calls would be the last statement executed by the
method.

\end{ANS}

\secEXRHtwoleft{Exercises}

\begin{EXRtwo}

\item  Explain the difference between the following pairs of terms:
\begin{EXRtwoLL}
\item  {\it Iteration} and {\it recursion}.
\item  {\it Recursive method} and {\it recursive definition}.
\item  {\it Base case} and {\it recursive case}.
\item  {\it Head} and {\it tail}.
\item  {\it Tail} and {\it nontail} recursive.
\end{EXRtwoLL}

\marginnote{\raggedright\vspace{9pt}{\bf Note:} For programming exercises, {\bf first} draw 
a UML class diagram describing all classes and
their inheritance relationships and/or associations.}
%\addcontentsline{toc}{section}{\S~~~ Exercises}

\item  Describe how the {\it method call stack}
is used during a method call and return.

\item  Why is a recursive algorithm generally less efficient
than an iterative algorithm?

\item  A tree, such as a maple tree or pine tree, has
a recursive structure.   Describe how a tree's structure
displays {\it self-similarity} and {\it divisibility}.

\item  Write a recursive method to print each element
of an array of {\tt double}.

\item  Write a recursive method to print each element
of an array of {\tt double} from the last to the first element.

\item  Write a recursive method that will concatenate
the elements of an array of {\tt String} into a
single {\tt String} delimited by blanks.

\item  Write a recursive method that is passed a single {\tt int}
parameter, $N \geq 0$, and prints all the odd numbers between 1 and {\it N}.

\item
Write a recursive method that takes a single {\tt int} parameter \mbox{$N \geq 0$}
and prints the sequence of even numbers between {\it N} down to 0.

\item
Write a recursive method that takes a single {\tt int} parameter $N \geq 0$
and prints the multiples of 10 between 0 and {\it N}.

\item  %%3
Write a recursive method to print the following geometric \mbox{pattern:}

\begin{jjjlisting}
\begin{lstlisting}
#
# #
# # #
# # # #
# # # # #
\end{lstlisting}
\end{jjjlisting}

\item  Write recursive methods to print each of the following
\mbox{patterns.}

\begin{jjjlisting}
\begin{lstlisting}
# # # # # # # #     # # # # # # # #
  # # # # # # #     # # # # # # #
    # # # # # #     # # # # # #
      # # # # #     # # # # #
        # # # #     # # # #
          # # #     # # #
            # #     # #
              #     #
\end{lstlisting}
\end{jjjlisting}

\item  Write a recursive method to print all multiples of {\it M} up
to \break{\it M} * {\it N}.

\item  Write a recursive method to compute the sum of grades stored in an array.


\item  Write a recursive method to count the occurrences of
a substring within a string.

\item  Write a recursive method to remove the HTML tags
from a string.

\item  Implement a recursive version of the {\tt Caesar.decode()}
method from Chapter~8.

\item  The Fibonacci\index{Fibonacci sequence} sequence
(named after the Italian mathematician Leonardo of Pisa, ca.  1200)
consists of the numbers {\tt 0,1,1,2,3,5,8,13,\dots }  in which each number
(except for the first two) is the sum of the two preceding
numbers.  Write a recursive method {\tt fibonacci(N)} that prints the
first {\it N} Fibonacci numbers.

\item  Write a recursive method to rotate a {\tt String} by {\it N} characters
to the right.  For example, {\tt rotateR("hello", 3)} should return
``llohe.''

\item  Write a recursive method to rotate a {\tt String} by {\it N} characters
to the left.  For example, {\tt rotateL("hello", 3)} should return
``lohel.''

\item  Write a recursive method to convert a {\tt String} representing
a binary number to its decimal equivalent.   For example,
{\tt binTodecimal("101011")} should return the {\tt int} value 43.

\item  A palindrome is a string that is equal to its reverse---``mom,'' ``i,'' 
``radar'' and ``able was i ere i saw elba.'' Write a recursive
{\tt boolean} method that determines whether its {\tt String}
parameter is a palindrome.


%\begin{figure}
%\begin{figure}[tb]
\marginfig{chptr12/btree.eps}{A level-four binary tree pattern.}
{fig-btree}
%\end{figure}
%\end{figure}

\item  {\bf Challenge: } Incorporate a {\tt drawBinaryTree()} method
into the {\tt RecursivePatterns} program.  A level-one binary tree has
two branches.  At each subsequent level, two smaller branches are grown
from the endpoints of every existing branch.  The geometry is easier if
you use 45-degree angles for the branches.  Figure~12.36
shows a level-four binary tree drawn upside down.

\item  {\bf Challenge: Towers of Hanoi.} According to
legend, some Buddhist monks were given the task of moving 64
golden disks from one diamond needle to another needle, using a third
needle as a backup.  To begin with, the disks were stacked one on top of
the other from largest to smallest (Fig.~12.37). The
rules were that only one disk can be moved at a time and that
a larger disk can never go on top of a smaller one.  The end of the
world was supposed to occur when the monks finished the task!

\hspace*{12pt}Write a recursive method,
{\tt move(int N, char A, char B, char C)}, that will print out
directions the monks can use to solve the towers of Hanoi problem.  For
example, here's what it should output for the three-disk case,
{\tt move(3, "A", "B", "C")}:
%\begin{figure}
%\begin{figure}[tb]
\marginfig{chptr12/hanoi.eps}{The towers of Hanoi problem.  Move all
the disks from needle A to needle B.~Only one disk can be moved at a
time, and a larger disk can never go on top of a smaller one.}
{fig-hanoi}
%\end{figure}
%\end{figure}

\begin{jjjlisting}
\begin{lstlisting}
Move 1 disk from A to B.
Move 1 disk from A to C.
Move 1 disk from B to C.
Move 1 disk from A to B.
Move 1 disk from C to A.
Move 1 disk from C to B.
Move 1 disk from A to B.
\end{lstlisting}
\end{jjjlisting}

\end{EXRtwo}
%
